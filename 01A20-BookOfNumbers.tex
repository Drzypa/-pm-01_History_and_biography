\documentclass[12pt]{article}
\usepackage{pmmeta}
\pmcanonicalname{BookOfNumbers}
\pmcreated{2013-03-22 16:38:22}
\pmmodified{2013-03-22 16:38:22}
\pmowner{PrimeFan}{13766}
\pmmodifier{PrimeFan}{13766}
\pmtitle{Book of Numbers}
\pmrecord{8}{38842}
\pmprivacy{1}
\pmauthor{PrimeFan}{13766}
\pmtype{Definition}
\pmcomment{trigger rebuild}
\pmclassification{msc}{01A20}

\endmetadata

% this is the default PlanetMath preamble.  as your knowledge
% of TeX increases, you will probably want to edit this, but
% it should be fine as is for beginners.

% almost certainly you want these
\usepackage{amssymb}
\usepackage{amsmath}
\usepackage{amsfonts}

% used for TeXing text within eps files
%\usepackage{psfrag}
% need this for including graphics (\includegraphics)
%\usepackage{graphicx}
% for neatly defining theorems and propositions
%\usepackage{amsthm}
% making logically defined graphics
%%%\usepackage{xypic}

% there are many more packages, add them here as you need them

% define commands here

\begin{document}
The {\em Book of Numbers} is the fourth book of the {\it Holy Bible}. The original Hebrew title is ``Ba-Midbar'', which means ``In the Desert.'' In the Greek translation of the Old Testament, the book was referred to as ``Arithmoi'', giving rise to its name in most modern languages. Chapter 1 lists the results of the Census taken after the Jews escaped from Egypt, ``every male from twenty years old and upward, all that were able to go forth to war,'' categorized by tribes (but the Levites were excluded from the Census). In the King James Version, the numbers are given rounded off by hundreds.

\begin{tabular}{|c|l|}
Reuben & 46500 \\
Simeon & 59300 \\
Gad & 5650 \\
Judah & 74600 \\
Issachar & 54400 \\
Zebulun & 57400 \\
Ephraim & 4500 \\
Manasseh & 32200 \\
Benjamin & 35400 \\
Dan & 62700 \\
Asher & 41500 \\
Naphtali & 53400 \\
Levi & not counted \\
\end{tabular}

Unlike other uses of numbers in the {\it Bible}, the numbers in the Book of Numbers are understood to be actual enumerations and not chosen for their symbolic value because of their mathematical or numerological properties (as is the case with the Book of Revelations).

The title {\it Book of Numbers} was used by John Conway and Richard Guy for their book on number theory. Title aside, it contains no biblical references.
%%%%%
%%%%%
\end{document}
