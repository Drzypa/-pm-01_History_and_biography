\documentclass[12pt]{article}
\usepackage{pmmeta}
\pmcanonicalname{EnricoBombieri}
\pmcreated{2013-03-22 16:25:30}
\pmmodified{2013-03-22 16:25:30}
\pmowner{PrimeFan}{13766}
\pmmodifier{PrimeFan}{13766}
\pmtitle{Enrico Bombieri}
\pmrecord{6}{38576}
\pmprivacy{1}
\pmauthor{PrimeFan}{13766}
\pmtype{Biography}
\pmcomment{trigger rebuild}
\pmclassification{msc}{01A61}
\pmclassification{msc}{01A60}
\pmclassification{msc}{01A65}

% this is the default PlanetMath preamble.  as your knowledge
% of TeX increases, you will probably want to edit this, but
% it should be fine as is for beginners.

% almost certainly you want these
\usepackage{amssymb}
\usepackage{amsmath}
\usepackage{amsfonts}

% used for TeXing text within eps files
%\usepackage{psfrag}
% need this for including graphics (\includegraphics)
%\usepackage{graphicx}
% for neatly defining theorems and propositions
%\usepackage{amsthm}
% making logically defined graphics
%%%\usepackage{xypic}

% there are many more packages, add them here as you need them

% define commands here

\begin{document}
\emph{Enrico Bombieri} (1940 - ) Italian mathematician, best known for the Bombieri-Vinogradov theorem regarding the measure of error in Dirichlet's theorem on arithmetic progressions, thought to be one possible direction towards proving the Riemann hypothesis (so much so that the Clay Mathematics Institute asked Bombieri to write the official description for the Millennium Problems). Bombieri was awarded the Fields medal in 1974.
%%%%%
%%%%%
\end{document}
