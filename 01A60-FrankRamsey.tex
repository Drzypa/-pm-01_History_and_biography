\documentclass[12pt]{article}
\usepackage{pmmeta}
\pmcanonicalname{FrankRamsey}
\pmcreated{2013-03-22 16:16:35}
\pmmodified{2013-03-22 16:16:35}
\pmowner{PrimeFan}{13766}
\pmmodifier{PrimeFan}{13766}
\pmtitle{Frank Ramsey}
\pmrecord{4}{38388}
\pmprivacy{1}
\pmauthor{PrimeFan}{13766}
\pmtype{Biography}
\pmcomment{trigger rebuild}
\pmclassification{msc}{01A60}
\pmsynonym{Frank Plumpton Ramsey}{FrankRamsey}
\pmsynonym{Frank P. Ramsey}{FrankRamsey}
\pmsynonym{F. P. Ramsey}{FrankRamsey}

% this is the default PlanetMath preamble.  as your knowledge
% of TeX increases, you will probably want to edit this, but
% it should be fine as is for beginners.

% almost certainly you want these
\usepackage{amssymb}
\usepackage{amsmath}
\usepackage{amsfonts}

% used for TeXing text within eps files
%\usepackage{psfrag}
% need this for including graphics (\includegraphics)
%\usepackage{graphicx}
% for neatly defining theorems and propositions
%\usepackage{amsthm}
% making logically defined graphics
%%%\usepackage{xypic}

% there are many more packages, add them here as you need them

% define commands here

\begin{document}
\emph{Frank Plumpton Ramsey} (1903 - 1930) English mathematician and economist. In mathematics he's best known for Ramsey numbers and his theorem on complete graphs, but he also studied combinatorics and logic. In economics he's best known for formulating a pricing model for monopolies that would theoretically maximize social welfare.

As Ramsey did important work on decision analysis, the Decision Analysis Society named its annual medal after him.
%%%%%
%%%%%
\end{document}
