\documentclass[12pt]{article}
\usepackage{pmmeta}
\pmcanonicalname{RomanNumerals}
\pmcreated{2013-03-22 12:57:31}
\pmmodified{2013-03-22 12:57:31}
\pmowner{Koro}{127}
\pmmodifier{Koro}{127}
\pmtitle{Roman numerals}
\pmrecord{8}{33321}
\pmprivacy{1}
\pmauthor{Koro}{127}
\pmtype{Definition}
\pmcomment{trigger rebuild}
\pmclassification{msc}{01A20}

\endmetadata

% this is the default PlanetMath preamble.  as your knowledge
% of TeX increases, you will probably want to edit this, but
% it should be fine as is for beginners.

% almost certainly you want these
\usepackage{amssymb}
\usepackage{amsmath}
\usepackage{amsfonts}

% used for TeXing text within eps files
%\usepackage{psfrag}
% need this for including graphics (\includegraphics)
%\usepackage{graphicx}
% for neatly defining theorems and propositions
%\usepackage{amsthm}
% making logically defined graphics
%%%\usepackage{xypic}

% there are many more packages, add them here as you need them

% define commands here
%\PMlinkescapeword{theory}
\begin{document}
\emph{Roman numerals} are a method of writing numbers employed primarily by the ancient Romans.  It place of digits, the Romans used letters to represent the numbers central to the system:

\begin{tabular}{cc}
$I$&$1$\\
$V$&$5$\\
$X$&$10$\\
$L$&$50$\\
$C$&$100$\\
$D$&$500$\\
$M$&$1000$
\end{tabular}

Larger numbers can be made by writing a bar over the letter, which means one thousand times as much.  For instance $\overline{V}$ is $5000$.

Other numbers were written by putting letters together.  For instance $II$ means $2$.  Larger letters go on the left, so $LII$ is $52$, but $IIL$ is not a valid Roman numeral.

One additional rule allows a letter to the left of a larger letter to signify subtracting the smaller from the larger.  For instance $IV$ is $4$.  This can only be done once; $3$ is written $III$, not $IIV$.  Also, it is generally required that the smaller letter be the one immediately smaller than the larger, so $1999$ is usually written $MCMXCIX$, not $MIM$.

It is worth noting that today it is usually considered incorrect to repeat a letter four times, so $IV$ is preferred to $IIII$.  However many older monuments do not use the subtraction rule at all, so $44$ was written $XXXXIIII$ instead of the now preferable $XLIV$.
%%%%%
%%%%%
\end{document}
