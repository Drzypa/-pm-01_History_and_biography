\documentclass[12pt]{article}
\usepackage{pmmeta}
\pmcanonicalname{HannaNeumann}
\pmcreated{2013-03-22 17:21:22}
\pmmodified{2013-03-22 17:21:22}
\pmowner{Mravinci}{12996}
\pmmodifier{Mravinci}{12996}
\pmtitle{Hanna Neumann}
\pmrecord{5}{39715}
\pmprivacy{1}
\pmauthor{Mravinci}{12996}
\pmtype{Biography}
\pmcomment{trigger rebuild}
\pmclassification{msc}{01A60}
\pmsynonym{Hanna von Caemmerer}{HannaNeumann}

\endmetadata

% this is the default PlanetMath preamble.  as your knowledge
% of TeX increases, you will probably want to edit this, but
% it should be fine as is for beginners.

% almost certainly you want these
\usepackage{amssymb}
\usepackage{amsmath}
\usepackage{amsfonts}

% used for TeXing text within eps files
%\usepackage{psfrag}
% need this for including graphics (\includegraphics)
%\usepackage{graphicx}
% for neatly defining theorems and propositions
%\usepackage{amsthm}
% making logically defined graphics
%%%\usepackage{xypic}

% there are many more packages, add them here as you need them

% define commands here

\begin{document}
\PMlinkescapeword{child}
\PMlinkescapeword{interest}
\PMlinkescapeword{focus}
\PMlinkescapeword{power}
\PMlinkescapeword{obvious}
\PMlinkescapeword{children}
\PMlinkescapeword{even}
\PMlinkescapeword{degree}

\emph{Hanna Neumann} n\`ee \emph{Hanna von Caemmerer} (1914 - 1971) German mathematician, best known for her work on varieties in group theory.

Her father died soon after her birth, and as a teen young Hanna supplemented the family's pension by tutoring at school. As a child she showed a strong interest in botany, but from her adolescence onwards, mathematics became her main focus. In 1932, she started studying at the University of Berlin where she met K\"ate Fenchel and Bernhard Neumann, a British Jewish mathematician born in Germany. Bernhard Neumann and Hanna vacationed in England in 1934 and engaged to be married, but as Adolf Hitler rose to power in Germany, it became obvious that the Aryan Hanna's marriage to the British Jew would have to wait, or even seeing each other. Hanna was openly critical of the Nazis, and this would be an obstacle to her obtaining a doctorate degree from a German university. It wasn't until 1938 that Bernhard and Hanna married in England; she took his name and bore him five children. Bernhard was drafted into the Royal Army and Hanna completed her doctorate at Oxford. After the war, Hanna Neumann taught in the United States, in Australia and in Canada.

\begin{thebibliography}{5}
\bibitem{gn} M. F. Newman ``Hanna Neumann'' in {\it Women of Mathematics: A Bibliographic Sourcebook} L. Grinstein, P. Campbell, eds. New York: Greenwood Press (1987): 156 - 160
\end{thebibliography}
%%%%%
%%%%%
\end{document}
