\documentclass[12pt]{article}
\usepackage{pmmeta}
\pmcanonicalname{JudyHoldener}
\pmcreated{2013-03-22 16:58:54}
\pmmodified{2013-03-22 16:58:54}
\pmowner{Mravinci}{12996}
\pmmodifier{Mravinci}{12996}
\pmtitle{Judy Holdener}
\pmrecord{5}{39258}
\pmprivacy{1}
\pmauthor{Mravinci}{12996}
\pmtype{Definition}
\pmcomment{trigger rebuild}
\pmclassification{msc}{01A65}
\pmclassification{msc}{01A61}
\pmclassification{msc}{01A60}

\endmetadata

% this is the default PlanetMath preamble.  as your knowledge
% of TeX increases, you will probably want to edit this, but
% it should be fine as is for beginners.

% almost certainly you want these
\usepackage{amssymb}
\usepackage{amsmath}
\usepackage{amsfonts}

% used for TeXing text within eps files
%\usepackage{psfrag}
% need this for including graphics (\includegraphics)
%\usepackage{graphicx}
% for neatly defining theorems and propositions
%\usepackage{amsthm}
% making logically defined graphics
%%%\usepackage{xypic}

% there are many more packages, add them here as you need them

% define commands here

\begin{document}
\PMlinkescapeword{Force}

\emph{Judy Holdener} (? - ) American mathematician and educator. Her primary interest is the study of perfect numbers (and specifically the search for odd perfect numbers), but she has also worked in biomathematics and child psychology.

Holdener studied at Kent State University and the University of Illinois-Urbana. After a year at Wolfram Research in Urbana, she began teaching at her alma mater there, followed by a stint teaching at the U.S. Air Force Academy in Colorado, which gave her the 1995 Tony M. Johnson Excellence in Teaching Award. Holdener joined the faculty of Kenyon College in 2003, where she teaches a course on mathematical models for biology, three sections of calculus, the number theory seminar and two sections of abstract algebra. The National Science Foundation gave her a grant to further her biomathematical research at Kenyon.

% Since Holdener chooses for coauthors Kenyon students, she has no Erdos number.
%%%%%
%%%%%
\end{document}
