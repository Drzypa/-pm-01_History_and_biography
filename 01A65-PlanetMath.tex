\documentclass[12pt]{article}
\usepackage{pmmeta}
\pmcanonicalname{PlanetMath}
\pmcreated{2013-03-22 16:45:07}
\pmmodified{2013-03-22 16:45:07}
\pmowner{PrimeFan}{13766}
\pmmodifier{PrimeFan}{13766}
\pmtitle{PlanetMath}
\pmrecord{13}{38977}
\pmprivacy{1}
\pmauthor{PrimeFan}{13766}
\pmtype{Definition}
\pmcomment{trigger rebuild}
\pmclassification{msc}{01A65}
\pmclassification{msc}{01A61}
%\pmkeywords{PM outline}
%\pmkeywords{No\"osphere}

% this is the default PlanetMath preamble.  as your knowledge
% of TeX increases, you will probably want to edit this, but
% it should be fine as is for beginners.

% almost certainly you want these
\usepackage{amssymb}
\usepackage{amsmath}
\usepackage{amsfonts}

% used for TeXing text within eps files
%\usepackage{psfrag}
% need this for including graphics (\includegraphics)
%\usepackage{graphicx}
% for neatly defining theorems and propositions
%\usepackage{amsthm}
% making logically defined graphics
%%%\usepackage{xypic}

% there are many more packages, add them here as you need them

% define commands here

\begin{document}
\PMlinkescapeword{adapted}
\PMlinkescapeword{categories}
\PMlinkescapeword{cover}
\PMlinkescapeword{groups}
\PMlinkescapeword{project}
\PMlinkescapeword{rights}
\PMlinkescapeword{running}

{\em PlanetMath} is a free, collaborative, online mathematics encyclopedia. The emphasis is on peer review, rigor, openness, pedagogy, real-time content, interlinked content, and community. Intended to be comprehensive, the project is located at the Digital Library Research Lab at Virginia Tech.

PlanetMath was started when the popular free online mathematics encyclopedia MathWorld was taken offline by a court injunction as a result of the CRC Press lawsuit against the Wolfram Research company and its employee (and MathWorld's author) Eric Weisstein.

PlanetMath uses the same copyleft as Wikipedia: the \PMlinkexternal{GNU Free Documentation License}{http://www.gnu.org/copyleft/fdl.html}. An author who starts a new article becomes the owner of that article; he or she may then choose to grant editing rights to other individuals or groups. All textual content and mathematical formulas are written in \LaTeX{}, a typesetting system that requires some learning but is popular among mathematicians because of its support of the technical needs of mathematical typesetting and its high-quality output. The user can explicitly create links to other articles, and the system also automatically turns certain words into links to the defining articles. For more details on the automatic linking, see the collaboration on \PMlinkexternal{PlanetMath automatic reference linking}{http://planetmath.org/?op=getobj&from=collab&id=32}. For more details on controlling the linking of an article, see the collaboration on \PMlinkexternal{controlling linking}{http://planetmath.org/?op=getobj&from=collab&id=33} The topic area of every article is classified by the \PMlinkexternal{Mathematics Subject Classification}{http://planetmath.org/?op=mscbrowse} of the American Mathematical Society. Users may attach addenda, errata, and discussions to articles.

The most common method of public communication within PlanetMath is posts. Users can add posts in the \PMlinkexternal{forums}{http://planetmath.org/?op=forums} as well as attach posts to articles, \PMlinkexternal{corrections}{http://planetmath.org/?op=globalcors}, \PMlinkexternal{collaborations}{http://planetmath.org/?op=collab}, \PMlinkexternal{requests for new articles}{http://planetmath.org/?op=reqlist}, and other posts. A system for private messaging among users is also in place.

Users who are new to PlanetMath are highly encouraged to read the following collaborations:

\begin{itemize}
\item \PMlinkexternal{New user guide}{http://planetmath.org/?op=getobj&from=collab&id=36}
\item \PMlinkexternal{The PlanetMath FAQ}{http://planetmath.org/?op=getobj&from=collab&id=35}
\item \PMlinkexternal{PlanetMath Content Committee `Guidelines'-Draft v.}{http://planetmath.org/?op=getobj&from=collab&id=113}
\item \PMlinkexternal{PlanetMath content and style guide}{http://planetmath.org/?op=getobj&from=collab&id=28}
\item \PMlinkexternal{Forum policy}{http://planetmath.org/?op=getobj&from=collab&id=55}
\end{itemize}

The software running PlanetMath is written in Perl and runs on Linux and the Apache Web server. It is known as 
$No\"{o}sphere$ and has been released under the \PMlinkexternal{free BSD License}{http://www.opensource.org/licenses/bsd-license.html}.

Most of the very most basic topics are covered, though PlanetMath is striving to improve coverage of elementary and intermediate topics. Due to the increasing popularity of the package PSTricks, more members of PlanetMath are able to incorporate graphics into their articles. This has enabled PlanetMath to cover many elementary and intermediate topics in geometry that were once lacking. There are several methods of creating graphics on PlanetMath. For more details on creating graphics on PlanetMath, see the collaboration on \PMlinkexternal{graphics and PlanetMath}{http://planetmath.org/?op=getobj&from=collab&id=53}.

PlanetMath also has entries on highly advanced and specialized topics. PlanetMath has entries on the integers \PMlinkname{42}{FortyTwo} and \PMlinkname{666}{NumberOfTheBeast}. The following top-level Mathematics Subject Classification categories have only one or two topic entries at PlanetMath:

\begin{itemize}
\item 74-XX, mechanics of deformable solids;
\item 76-XX, fluid mechanics;
\item 85-XX, astronomy and astrophysics.
\end{itemize}

The Wikipedia:WikiProject Mathematics/PlanetMath Exchange project assists in content exchange between PlanetMath and Wikipedia.

{\it This entry was adapted from the Wikipedia article \PMlinkexternal{PlanetMath}{http://en.wikipedia.org/wiki/PlanetMath} as of February 24, 2007.}

%%%%%
%%%%%
\end{document}
