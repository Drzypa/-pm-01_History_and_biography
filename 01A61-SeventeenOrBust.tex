\documentclass[12pt]{article}
\usepackage{pmmeta}
\pmcanonicalname{SeventeenOrBust}
\pmcreated{2013-03-22 17:20:32}
\pmmodified{2013-03-22 17:20:32}
\pmowner{PrimeFan}{13766}
\pmmodifier{PrimeFan}{13766}
\pmtitle{Seventeen or Bust}
\pmrecord{6}{39697}
\pmprivacy{1}
\pmauthor{PrimeFan}{13766}
\pmtype{Definition}
\pmcomment{trigger rebuild}
\pmclassification{msc}{01A61}
\pmclassification{msc}{01A65}

\endmetadata

% this is the default PlanetMath preamble.  as your knowledge
% of TeX increases, you will probably want to edit this, but
% it should be fine as is for beginners.

% almost certainly you want these
\usepackage{amssymb}
\usepackage{amsmath}
\usepackage{amsfonts}

% used for TeXing text within eps files
%\usepackage{psfrag}
% need this for including graphics (\includegraphics)
%\usepackage{graphicx}
% for neatly defining theorems and propositions
%\usepackage{amsthm}
% making logically defined graphics
%%%\usepackage{xypic}

% there are many more packages, add them here as you need them

% define commands here

\begin{document}
{\em Seventeen or Bust} is a distributed computing project aimed at proving that 78557 is the smallest \PMlinkname{Sierpi\'nski numbers}{SierpinskiNumbers}. The Seventeen or Bust project has discovered some of the largest non-Mersenne primes, the most recent being $19249 \times 2^{13018586} + 1$ discovered by Konstantin Agafonov. Over the past five years, it is estimated that the summatory of the frequencies of all the computer processors participating in the project is 47 teraHertz spread across almost a hundred thousand different computers operated by ten thousand different users.

There were seventeen different $k$ to test when the project started; as of June 2007, only eight are left. The smallest $k$ still being tested by the project is 10223, for which more than six hundred tests have been performed, while the largest $k$ remaining is 69109.

\subsection{External links}
\PMlinkexternal{Official Website}{http://www.seventeenorbust.com/}
%%%%%
%%%%%
\end{document}
