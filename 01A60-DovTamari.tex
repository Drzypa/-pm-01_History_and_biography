\documentclass[12pt]{article}
\usepackage{pmmeta}
\pmcanonicalname{DovTamari}
\pmcreated{2013-03-22 16:49:11}
\pmmodified{2013-03-22 16:49:11}
\pmowner{Mravinci}{12996}
\pmmodifier{Mravinci}{12996}
\pmtitle{Dov Tamari}
\pmrecord{7}{39057}
\pmprivacy{1}
\pmauthor{Mravinci}{12996}
\pmtype{Biography}
\pmcomment{trigger rebuild}
\pmclassification{msc}{01A60}

% this is the default PlanetMath preamble.  as your knowledge
% of TeX increases, you will probably want to edit this, but
% it should be fine as is for beginners.

% almost certainly you want these
\usepackage{amssymb}
\usepackage{amsmath}
\usepackage{amsfonts}

% used for TeXing text within eps files
%\usepackage{psfrag}
% need this for including graphics (\includegraphics)
%\usepackage{graphicx}
% for neatly defining theorems and propositions
%\usepackage{amsthm}
% making logically defined graphics
%%%\usepackage{xypic}

% there are many more packages, add them here as you need them

% define commands here

\begin{document}
\emph{Dov Tamari} (193? - ) French mathematician, best known for the Tamari lattice. He earned a doctorate of science from the Universit\'e de Paris. His students include Carlton Maxson and Kevin Osondu. As of 1990, Tamari was living in New York.

Tamari has \PMlinkname{Erd\H{o}s number}{ErdHosNumber} 2 because he coauthored with Abraham Ginzburg ``Representation of multiplicative systems by families of binary relations'' in the {\it Journal of the London Mathematical Society} {\bf 37} in 1962.

Search engine results for ``Dov Tamari'' turn up the first Chief Intelligence Officer of the Israeli Corps, who is someone other than the mathematician.

\begin{thebibliography}{1}
\bibitem{imu} International Mathematical Union, {\it World Directory of Mathematicians 1990}, 9th Ed. Bombay: Tata Institute of Fundamental Research
\end{thebibliography}
%%%%%
%%%%%
\end{document}
