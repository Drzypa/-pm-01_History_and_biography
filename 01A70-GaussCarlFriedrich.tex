\documentclass[12pt]{article}
\usepackage{pmmeta}
\pmcanonicalname{GaussCarlFriedrich}
\pmcreated{2013-03-22 14:10:12}
\pmmodified{2013-03-22 14:10:12}
\pmowner{mathwizard}{128}
\pmmodifier{mathwizard}{128}
\pmtitle{Gauss, Carl Friedrich}
\pmrecord{13}{35594}
\pmprivacy{1}
\pmauthor{mathwizard}{128}
\pmtype{Biography}
\pmcomment{trigger rebuild}
\pmclassification{msc}{01A70}
\pmclassification{msc}{01A55}
\pmclassification{msc}{01A50}
\pmsynonym{Gauss}{GaussCarlFriedrich}
\pmsynonym{Gau\ss}{GaussCarlFriedrich}

% this is the default PlanetMath preamble.  as your knowledge
% of TeX increases, you will probably want to edit this, but
% it should be fine as is for beginners.

% almost certainly you want these

\usepackage{amssymb}
\usepackage{amsmath}
\usepackage{amsfonts}

% used for TeXing text within eps files
%\usepackage{psfrag}
% need this for including graphics (\includegraphics)
%\usepackage{graphicx}
% for neatly defining theorems and propositions
%\usepackage{amsthm}
% making logically defined graphics
%%%\usepackage{xypic}

% there are many more packages, add them here as you need them

% define commands here
\begin{document}
Carl Friedrich Gauss was born in Braunschweig on April the 30th 1777 as the son of a poor worker. Already in his youth he was interested in mathematics. It is reported that when Gauss was a student at elementary school his teacher asked the students to add up all natural numbers from 1 to 100, hoping to keep his students busy for some time. Gauss however found the correct answer within a few minutes by cleverly rearranging the summands.

From 1792 to 1795 he was a student at the \textit{Collegium Carolinum} in Braunschweig, which was made possible by a scholarship by the duke of Braunschweig. From 1795 to 1798 he studied at the university of G\"ottingen. 1807 he became professor for astronomy in G\"ottingen and director of the observatory. He remained in this position until his death in 1855.

During his time as a student in G\"ottingen he met the Hungarian mathematician Wolfgang Bolyai and they swore each other ``brotherhood under the banner of truth''.

The first important discovery he made was the construction of the 17-gon by circle and ruler in 1796. In 1799 he finished his dissertation, in which he proved the fundamental theorem of algebra, adding complex number notation to find all n roots of n degree equations. In 1801 his main work, the \textit{Disquisitiones Arithmeticae} (DA) was published. Initially rejected by the Paris Academy the DA formally created congruence notation for the purposes of presenting a proof of the fundamental theorem of arithmetic, and formalizing parametric solutions to indeterminate equations. Dunnington as one of Gauss' biographers included library check out card references as an appendix showing Gauss' broad interests as a graduate student.

In the year 1801 the Italian astronomer Joseph Piazzi discovered the planetoid Ceres, but could only watch it for a few days. Gauss predicted correctly the position at which it could be found again. It was rediscovered by Zach at 31st of December 1801  in Gotha and one day later by Olbers in Bremen. Zach said about this: ``Ohne die scharfsinnigen Bem\"uhungen und Berechnungen des Doktor Gauss h\"atten wir vielleicht die Ceres nicht wiedergefunden'' (Without the intelligent work and calculations of doctor Gauss we might not have found Ceres again.)
By this and the discovery of the planetoid Pallas by Olbers in 1802 Gauss worked on a theory of the motion of planetoids disturbed by large planets. This work was published in 1809 under the name \textit{Theoria motus corporum coelestium in sectionibus conicis solem ambientum} (Theory of motion of the celestial bodies moving in conic sections around the sun).

In 1816 he is asked to do land surveying in the kingdom of Hannover, which gave rise to several important scientific discoveries. In that time he developed the method of least squares independently of Legendre and the curvature of the earth made him think about non-Euclidean geometries. Gauss also tried to measure whether earth's surface was euclidean, by measuring the sum of angles in a triangle formed by three mountains (Brocken, Inselberg and Hohen Hagen) but found that it was $180^\circ$. He never published his works on geometry though. Later his friend Wolfgang Bolyai sent him his son's works about non-Euclidean geometry Gauss said that he was unable to praise that work for he'd then have to praise himself, meaning that he had already found these things himself. But though Gauss wrote to Gerling that he considered the young Bolyai to be a genius, he never praised him in public and the young Bolyai turned away from mathematics.

Together with the physicist Wilhelm Weber he built the first electromagnetic telegraph in 1833, which connected the observatory with the institute for physics in G\"ottingen. Gauss also worked on the theory of magnetism and found a representation for the unit of magnetism by the units of mass, length and time. Inspired by Alexander von Humboldt he was interested in earth's magnetism. He ordered a magnetic observatory to be built in the garden of the observatory and founded the \textit{magnetischer Verein} (magnetic club) together with Wilhelm Weber and this club supported measurements of earth's magnetic field in many regions of the world.

His private life however was not so happy. In 1809 died his first wife Johanna. Gauss mourned over her very much and it was more due to a feeling of duty and sympathy that he took his wife's best friend Minna as his second wife. From his first wife he had one daughter, Minna, whom he liked very much, and one son, Joseph, named after the discoverer of Ceres. 
From his second wife he had two sons and one daughter. He did not get on well with his children except for Minna, who was very much like her mother Johanna.

Gauss did not very much like to give public lectures. He hated it when he had to give lectures at university and it is said that he only attended a single scientific conference, which was in Berlin in 1828 when he was a guest of Humboldt's. He also did not have mathematical students.

However in other natural sciences he had good cooperation with many people for example with Olbers in Bremen, Schumacher in Altona (today Hamburg), Gerling in Marburg and Bessel in Koenigsberg.
\subsubsection*{References}
\begin{itemize}
\item Hermann Heimpell, Theodor Heuss, Benno Reifenberg (editors): Die gro\ss en Deutschen volume 3, Ullstein Verlag Berlin, 1956
\item Lexikon der Naturwissenschaftler, Spektrum Akademischer Verlag Heidelberg, 2000
\item \PMlinkexternal{An online biography}{http://www-gap.dcs.st-and.ac.uk/~history/Mathematicians/Gauss.html}
\end{itemize}
\PMlinkescapeword{theory}
\PMlinkescapeword{name}
\PMlinkescapeword{bodies}
\PMlinkescapeword{connected}
\PMlinkescapeword{representation}
\PMlinkescapeword{unit}
\PMlinkescapeword{club}
\PMlinkescapeword{field}
\PMlinkescapeword{regions}
\PMlinkescapeword{children}
\PMlinkescapeword{natural}
\PMlinkescapeword{connection}
\PMlinkescapeword{measure}
\PMlinkescapeword{units}
%%%%%
%%%%%
\end{document}
