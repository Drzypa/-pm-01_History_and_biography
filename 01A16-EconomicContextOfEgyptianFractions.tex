\documentclass[12pt]{article}
\usepackage{pmmeta}
\pmcanonicalname{EconomicContextOfEgyptianFractions}
\pmcreated{2015-02-04 6:22:57}
\pmmodified{2015-02-04 6:22:57}
\pmowner{milogardner}{13112}
\pmmodifier{milogardner}{13112}
\pmtitle{economic context of Egyptian fractions}
\pmrecord{210}{40545}
\pmprivacy{1}
\pmauthor{milogardner}{13112}
\pmtype{Definition}
\pmcomment{trigger rebuild}
\pmclassification{msc}{01A16}

% this is the default PlanetMath preamble.  as your knowledge
% of TeX increases, you will probably want to edit this, but
% it should be fine as is for beginners.

% almost certainly you want these
\usepackage{amssymb}
\usepackage{amsmath}
\usepackage{amsfonts}

% used for TeXing text within eps files
%\usepackage{psfrag}
% need this for including graphics (\includegraphics)
%\usepackage{graphicx}
% for neatly defining theorems and propositions
%\usepackage{amsthm}
% making logically defined graphics
%%%\usepackage{xypic}

% there are many more packages, add them here as you need them

% define commands here

\begin{document}
Egyptian Old Kingdom (OK) weights and measures rounded  rational numbers to truncated binary weights and measures units to six binary fractions, threw away smaller than 1/64 units before 2050 BCE. Binary balance beam units were at the center of the OK centrally controlled Egyptian economic system. 

Ahmes in 1650 BCE scaled 50 Egyptian fraction entries in a 2/n table that established theoretical economic units 400 years earlier . Rational numbers n/p were scaled by least common multiple (LCM) m to mn/mp and then to concise unit fraction series. The units scaled hekat and and related sub-units. The scaled numerator mn was often written as additive 'red auxiliary' series before a concise unit fraction series was recorded. 

Ahmes in RMP 36 scaled 3/53 hekat by LCM 20 to 60/1030 hekat = (53 + 4 + 2 + 1)/1060 hekat as a theoretical unit that recorded exact portions of grain contained in products as practical units.

The methodology established value used for trade and wage payments for hundreds of years.

Scribes scaled rational numbers by LCMs to red auxiliary numbers, 2/n tables, algebra, geometry and weights and measures that jump-started Middle Kingdom finite arithmetic and a partially decentralized economy. Scribes created finite quotient and exact remainders combining methods to measure 1/320 portions of grain in finite arithmetic. Scribes converted rational numbers to concise unit fraction series in the AWT, MMP, and RMP that scaled a hekat unity, (64/64), by quotients (Q/64) and remainders (5R/n) obtained 1/320 (ro) units to value products, as well as (4-hekat) and (4-ro) in the MMP.

During the Middle Kingdom (MK) (2050 BCE to 1550 BCE) a decentralized Egyptian economy encoded rational numbers into exact weights and measures units based in inverse square root and pesu methologies. The replacement economic system allowed absentee landlords to pay wages in grain units as well as pay 1/3 of 1/3 of gross profits taxes to Pharoah. The economic and math systems stressed exact volume units (hekat, hin, ro and other units) that replaced the OK rounded-off binary units in secular life. 

MK scribes corrected OK rounded off errors by including theoretical and practical Egyptian fraction hekat units that exactly scaled commodities and paid commodity based wages. The corrected MK economic system was finite and considered prime numbers for arithmetic operations. Egyptian MK scribes scaled rational numbers to unit fraction series within a system that formaly lasted until 1454 AD and informally to the early 1600s (to the time of \PMlinkexternal{Galileo}{http://www.ams.org/samplings/feature-column/fc-2013-05})  in Europe and 1637 AD in the Arab world. One MK system established a grain volume unit (named hekat, about 4800 ccm translated into modern metrics) that served as a unified monetary unit.

The Egyptian fraction numeration system assisted Pharaoh and elites in controlling granary outputs and decentralized productions of bread, beer and other grain based products. Scribal algebraic geometry formulas created cubit-cubits, khar, and hekat weight and measures units. A cubit-cubit contained 3/2 khar. A khar contained 20 hekat. A hekat contained 4800 ccm. Theoretical and practical measurements of cubit-cubits, khar, 400-hekat, 100-hekat, 4-hekat, 2-hekat, 1-hekat, 4-ro, 2-ro and 1-ro reported additional hekat sub-divisions. Smaller hekat units appear in \PMlinkexternal{RMP 47}{http://mathforum.org/kb/message.jspa?messageID=7212156&tstart=0}, \PMlinkexternal{RMP 83}{http://planetmath.org/encyclopedia/AhmesBirdFeedingRateMethod.html}, and the \PMlinkexternal{Kahun Papyrus}{http://planetmath.org/encyclopedia/KahunPapyrusAndArithmeticProgressions.html}. 

Overall scribal recorded weights and measures units in double-entry book book keeping systems making scribes first accountants and second mathematicians. One inventory control method was discussed in one Rhind Mathematical Papyrus problem. RMP 38 exposed as an aspect of trading units by multiplying one hekat, 320 ro, by 7/22 reporting (101 + 9/11)1-ro. Ahmes proved the accuracy of the answer by inverting 7/22 and multiplying (101 + 9/11) by 22/7 returning 320 1-ro, adding a comment that an exact hekat had been found.

Inside the partially decentralized economy \PMlinkexternal{the Heqanakht Papers}{http://www.reshafim.org.il/ad/egypt/texts/heqanakht.htm} discuss two absentee landlords' family and estate production and profit concerns in four letters. \PMlinkexternal{Accounting for Private Estates and the Household in the 20th Century BC Middle Kingdom}{http://www.blackwell-synergy.com/doi/abs/10.1111/1467-6281.00107} by Mahmoud Ezzamel appeared in the journal Abacus, Vol. 38, No. 2 (2002), pp.235-262. Ezzamel, an accountant, shows that absentee landlords relied on theoretical commodities and metals units that summed to a monetary system. Practical measurements issued payments to workers and implemented other management controls within rational number remainders written in Egyptian fractions. An abstract of Ezzamel's article follows: \PMlinkexternal{2,000 BCE Accounting Article}{http://www.blackwell-synergy.com/doi/abs/10.1111/1467-6281.00107}.

The Rhind Mathematical Papyrus (RMP) and the Moscow Mathematical Papyrus(MMP) precisely scaled grain to bread, beer and other products. Pesu and sub-units (i.e besha, des-jugs) further scaled grain products for practical distributions as wages. Gillings showed that alternando and dividendo, modern proportions, and a harmonic mean was used in bread and beer recipes. RMP and MMP problems reported 5 hekats of grain each producing 200 loaves of bread. The balance of hekats, 10 hekats in the RMP and 11 hekats in the MMP produced beer. Each hekat produced one, two, and three types of beer, labeled from 8/3 pesu to 6 pesu, denoted initial grain content by an inverse to the final product. A product with 3/8 hekat of grain was reported as 8/3 pesu and written as 2 2/3 pesu. 

\PMlinkexternal{Marianne Michel}{https://www.academia.edu/8899140/Les_mathématiques_de_lÉgypte_ancienne._Numération_métrologie_arithmétique_géométrie_et_autres_problèmes_Safran_2014_  } showed in 2014 that the MMP partitioned 5 (4-hekat), 10 (4-hekat), 20 (4-hekat) and 40 (4-hekat) by 2/3. For example 20 (4-hekat) mulltiplied by 2/3 = (16 1/4 + 1/16 + 1/64)(4-hekat) and (1 + 2/3) (4-ro), the same partition  that 20 (1-hekat) x 2/3 = (16 + 1/4 + 1/16 + 1/64)(1-hekat) + (1 + 2/3)(1-ro) reveals. That is, scribal partitioning of small and large hekat volumes followed the same binary quotient and ro remainder intellectual process.

As additional background it is well known that before 2050 BCE Egyptians and Babylonians both used cursive algorithms to record numbers. After 2050 BCE Babylonian numbers and math remained algorithmic. It is lesser known that Egyptian numeration and math became non-algorithmic after 2050 BCE. It is interesting to note that Egyptian cubic-cubits and khar trading units may have extended to Babylon and Larsa in perfume trade reported by Robert Middeke Conlin in \PMlinkexternal{2010}{http://yale.academia.edu/RobertMiddekeConlin/Papers}. The wider regional trading is included in \PMlinkexternal{Robert Middeke-Conlin}{http://yale.academia.edu/RobertMiddekeConlin/StatusUpdates/46225/}'s PHD thesis. The paper focuses upon perfume trade controlled from \PMlinkexternal{Larsa}{http://en.wikipedia.org/wiki/Larsa} during the Egyptian Middle Kingdom period, 500 years before \PMlinkexternal{Egyptian Demotic script}{http://en.wikipedia.org/wiki/Demotic} merged aspects of Egyptian and Babylonian business, science and mathematics, and 1000 years before Greek Demotic came into use.

Equally important \PMlinkexternal{James P. Allen}{http://eh.net/bookreviews/library/0868} published in 2002 
\PMlinkexternal{The Heqanakht Papers}{http://en.wikipedia.org/wiki/Heqanakht_papyri} mentioning a political compromise that encouraged absentee landlords to accumulate wealth. Egyptology and economic historians debate the implications of the Heqanakht Papers often without stressing a central role of Egyptian fractions. Intellectually Egyptian fractions unified the newly decentralized Egyptian economy that created verifiable weights and measures units for elite participants. Morris Silver, Professor Emeritus, Department of Economics, City College of the City University of New York, reviewed Allen's book, adding several discussion points. \PMlinkexternal{Professor Silver}{http://members.tripod.com/~sondmor/index-24.html} also reviewed "The Invention of Coinage and the Monetization of Ancient Greece" by David M. Schaps in the same manner, adding meta and micro economic considerations. For example, \PMlinkexternal{Morris Silver}{http://www.amazon.com/Economic-Structures-of-Antiquity/dp/B000PY3KRQ} aptly cited Babylonian (and Egyptian) monetary systems that spread across the Ancient Near East using 'bags of coins' well before Lydia and Greeks were reported to have created coinage.

In conclusion economic considerations motivated Pharaoh and elite absentee landlords to control commodity inventories after 2050 BCE by improving rounded off Old Kingdom weights and measures units. Middle Kingdom fraction weights and measure were unitized in a finite arithmetic system. Weights and measures units created double entry accounting entries that compared expected to actual usages of inventories. Scribal managers resolved differences by considering expected (theoretical) daily usages by substituting (64/64) and 320 ro for 1- hekat  and 4-hekat unities and rational two-part binary quotients and 1-ro and 4-ro remainders. The hekat unit was one of several keys to the Egyptian fraction system. Wages were paid in equivalent grain units of bread and other products was another key. The Heqanakht Papers documents absentee landlords, politically supported by Pharaoh, to gain profits and reduce payments to workers during flood years to avoid losses.

\begin{thebibliography}{9}
\bibitem{1}  A.B. Chace, Bull, L, Manning, H.P., Archibald, R.C., \emph{The Rhind Mathematical Papyrus}, Mathematical Association of Amnerica, Vol I, 1927. NCTM reprints available.
\bibitem{2} Milo Gardner, \emph{An Ancient Egyptian Problem and its Innovative Solution, Ganita Bharati}, MD Publications Pvt Ltd, 2006.
\bibitem{3}Richard Gillings, \emph{Mathematics in the Time of the Pharaohs}, Dover Books, 1992.
\bibitem{4} Oystein Ore, \emph{Number Theory and its History}, McGraw-Hill Books, 1948, Dover reprints available.
\bibitem{5} T.E. Peet, \emph{Arithmetic in the Middle Kingdom}, Journal Egyptian Archeology, 1923.
\bibitem{6} Tanja Pommerening, \emph{"Altagyptische Holmasse Metrologish neu Interpretiert" and relevant phramaceutical and medical knowledge, an abstract,  Phillips-Universtat, Marburg, 8-11-2004, taken from "Die Altagyptschen Hohlmass}, Buske-Verlag, 2005.
\bibitem{7} Gay Robins, and Charles Shute \emph{Rhind Mathematical Papyrus}, British Museum Press, Dover reprint, 1987.
\bibitem{8} L.E. Sigler, \emph{Fibonacci's Liber Abaci: Leonardo Pisano's Book of Calculation}, Springer, 2002.
\bibitem{9} Hana Vymazalova, \emph{The Wooden Tablets from Cairo:The Use of the Grain Unit HK3T in Ancient Egypt, Archiv Orientalai}, Charles U Prague, 2002.
\end{thebibliography}




Linked Reference 1:\PMlinkexternal{The_Arithmetic_used_to_Solve_an_Ancient_Horus-Eye_Proble}{http://independent.academia.edu/MiloGardner/Papers/163573/The_Arithmetic_used_to_Solve_an_Ancient_Horus-Eye_Problem} (published 2006)

Linked Reference 2:\PMlinkexternal{New York Times}{http://www.nytimes.com/2010/12/07/science/07first.html?_r=1&ref=science} (Dec. 6, 2010)

Linked Reference 3\PMlinkexternal{Egyptian_Fractions_Unit_Fractions_Hekats_and_Wages_-_an_Update}{http://independent.academia.edu/MiloGardner/Papers/623827/Egyptian_Fractions_Unit_Fractions_Hekats_and_Wages_-_an_Update}(submitted for publication May 2011)

%%%%%
%%%%%
\end{document}
