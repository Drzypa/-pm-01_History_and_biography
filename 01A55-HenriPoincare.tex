\documentclass[12pt]{article}
\usepackage{pmmeta}
\pmcanonicalname{HenriPoincare}
\pmcreated{2013-03-22 13:17:40}
\pmmodified{2013-03-22 13:17:40}
\pmowner{Daume}{40}
\pmmodifier{Daume}{40}
\pmtitle{Henri Poincar\'e}
\pmrecord{17}{33793}
\pmprivacy{1}
\pmauthor{Daume}{40}
\pmtype{Biography}
\pmcomment{trigger rebuild}
\pmclassification{msc}{01A55}
\pmsynonym{Poincar\'e}{HenriPoincare}
\pmsynonym{Jules Henri Poincar\'e}{HenriPoincare}

\endmetadata

% this is the default PlanetMath preamble.  as your knowledge
% of TeX increases, you will probably want to edit this, but
% it should be fine as is for beginners.

% almost certainly you want these
\usepackage{amssymb}
\usepackage{amsmath}
\usepackage{amsfonts}

% used for TeXing text within eps files
%\usepackage{psfrag}
% need this for including graphics (\includegraphics)
\usepackage{graphicx}
% for neatly defining theorems and propositions
%\usepackage{amsthm}
% making logically defined graphics
%%%\usepackage{xypic} 

% there are many more packages, add them here as you need them

% define commands here
\begin{document}
%=======================================================[[Picture of Poincare]]
\begin{figure}
\begin{center}
\includegraphics[scale=0.4]{poincare.eps}\\
\itshape Jules Henri Poincar\'e (1854 - 1912) (the above photograph is from \cite{WHP})\footnote{Photograph from the frontpiece of the 1913 edition of `Last thoughts' and therefore public domain in the U.S. and other contries.}
\end{center}
\end{figure}

%=======================================================[[Biography begins here]]

Jules Henri Poincar\'e was born on April $29^{th}$ 1854 in Cit\'e Ducale\cite{BA} a \PMlinkescapetext{neighborhood} in Nancy, a city in France. He was the son of Dr. L\'eon Poincar\'e \textsl{(1828-1892)} who was a professor at the University of Nancy in the faculty of medicine.\cite{SJ}  His mother, Eug\'enie Launois \textsl{(1830-1897)} was described as a ``gifted mother''\cite{EB} who gave special instruction to her son.  She was 24 and his father 26 years of age when Henri was born\cite{MT}.  Two years after the birth of Henri they gave birth to his sister Aline.\cite{EB}

In 1862 Henri entered the Lyc\'ee of Nancy which is today, called in his honor, the Lyc\'ee Henri Poincar\'e. In fact the University of Nancy is also named in his honor.  He graduated from the Lyc\'ee in 1871 with a bachelors \PMlinkescapetext{degree} in letters and sciences.  Henri was the top of class in almost all subjects, he did not have much success in music and was described as ``\PMlinkescapetext{average} at best'' in any physical activities.\cite{MT}  This could be blamed on his poor eyesight and absentmindedness.\cite{BC}  Later in 1873, Poincar\'e entered l'Ecole Polytechnique where he performed better in mathematics than all the other students. He published his first paper at 20 years of age, titled \emph{D\'emonstration nouvelle des propri\'et\'es de l'indicatrice d'une surface}.\cite{BR}  He graduated from the institution in 1876.  The same year he decided to attend l'Ecole des Mines and graduated in 1879 with a degree in mining engineering.\cite{SJ}  After his graduation he was appointed as an ordinary engineer in charge of the mining services in Vesoul.  At the same time he was preparing for his doctorate in sciences \textsl{(not surprisingly)}, in mathematics under the supervision of Charles Hermite. Some of Charles Hermite's most famous contributions to mathematics are:  Hermite's polynomials, Hermite's differential equation, Hermite's formula of interpolation and Hermitian matrices.\cite{MT}  Poincar\'e, as expected graduated from the University of Paris in 1879, with a thesis relating to differential equations.  He then became a teacher at the University of Caen, where he taught analysis.  He remained there until 1881. He then was appointed as the ``ma\^itre de conf\'erences d'analyse''\cite{SJ} (professor in charge of analysis conferences) at the University of Paris.  Also in that same year he married Miss Poulain d'Andecy.  Together they had four \PMlinkescapetext{children}: Jeanne born in 1887, Yvonne born in 1889, Henriette born in 1891, and finally L\'eon born in 1893.  He had now returned to work at the Ministry of Public Services as an engineer. He was responsible for the development of the northern railway. He held that position from 1881 to 1885. This was the last job he held in administration for the government of France. In 1893 he was awarded the title of head engineer in charge of the mines.  After that his career awards and position continuously escalated in greatness and quantity.  He died two years before the war on July $17^{th}$ 1912 of an embolism at the age of 58.  Interestingly, at the beginning of World War I, his cousin Raymond Poincar\'e was the president of the French Republic.

%=======================================================[[Method of Working begins here]]
Poincar\'e's work habits have been compared to a bee flying from flower to flower.  Poincar\'e was interested in the way his mind worked, he studied his habits. He gave a talk about his observations in 1908 at the Institute of General Psychology in Paris.  He linked his way of thinking to how he made several discoveries.  His mental organization was not only interesting to him but also to Toulouse, a psychologist of the Psychology Laboratory of the School of Higher Studies in Paris.  Toulouse wrote a book called \emph{Henri Poincar\'e} which was published in 1910.  He discussed Poincar\'e's \PMlinkescapetext{regular} schedule: he worked during the same times each day in short \PMlinkescapetext{periods} of time.  He never spent a long time on a problem since he believed that the subconscious would continue working on the problem while he worked on another problem.  Toulouse also noted that Poincar\'e also had an exceptional memory.  In addition he stated that most mathematicians worked from principle already established while Poincar\'e was the \PMlinkescapetext{type} that started from basic principle each time.\cite{MT}  His method of thinking is well summarized as:
\begin{quote}
Habitu\'e \`a n\'egliger les d\'etails et \`a ne regarder que les cimes, il passait de l'une \`a l'autre avec une promptitude surprenante et les faits qu'il d\'ecouvrait se groupant d'eux-m\^emes autour de leur \PMlinkescapetext{centre} \'etaient instantan\'emant et automatiquement class\'e dans sa m\'emoire. (He neglected details and jumped from idea to idea, the facts gathered from each idea would then come together and solve the problem) \cite{BA}
\end{quote}
The mathematician Darboux claimed he was ``un intuitif''(intuitive)\cite{BA}, arguing that this is demonstrated by the fact that he worked so often by visual \PMlinkescapetext{representation}.  He did not care about being rigorous and disliked \PMlinkescapetext{logic}. He believed that logic was not a way to invent but a way to \PMlinkescapetext{structure} ideas and that logic \PMlinkescapetext{limits} ideas.  

%=======================================================[[Philosophy & logic begins here]]
Poincar\'e had the opposite philosophical views of Bertrand Russell and Gottlob Fredge who believed that mathematics were a \PMlinkescapetext{branch} of logic.  Poincar\'e strongly disagreed, claiming that intuition was the life of mathematics.  Poincar\'e gives an interesting point of view in his book \emph{Science and Hypothesis}:
\begin{quote}
For a superficial observer, scientific truth is beyond the possibility of doubt; the logic of science is infallible, and if the scientists are sometimes mistaken, this is only from their mistaking its rule. \cite{PHSH}
\end{quote}
Poincar\'e believed that arithmetic is a synthetic science.  He argued that Peano's axioms cannot be proven non-circularly with the principle of induction.\cite{MM}  Therefore concluding that arithmetic is a priori synthetic and not analytic.  Poincar\'e then went on to say that mathematics can not be a deduced from logic since it is not analytic.  It is important to note that even today Poincar\'e has not been proven wrong in his argumentation. His views were the same as those of Kant\cite{KD}. However Poincar\'e did not share Kantian views in all branches of philosophy and mathematics.  For example in geometry Poincar\'e believed that the structure of non-Euclidean space can be known analytically. He wrote 3 books that made his philosophies known: \emph{Science and Hypothesis}, \emph{The Value of Science} and \emph{Science and Method}.

%=======================================================[[fuchsian function begins here]]
Poincar\'e's first area of interest in mathematics was the Fuchsian function that he named after the mathematician Lazarus Fuch because Fuch was known for being a good teacher and done alot of research in differential equations and in the \PMlinkescapetext{theory} of functions.  The functions did not keep the \PMlinkescapetext{name} fuchsian and are today called automorphic.  Poincar\'e actually developed the concept of those functions as part of his doctoral thesis.\cite{MT}  An automorphic function is a function $f(z)$ where $z\in \mathbb{C}$ which is analytic under its domain and which is invariant under a denumerable infinite group of linear fractional transformations, they are the generalizations of trigonometric functions and elliptic functions.  Below Poincar\'e explains how he discovered Fuchsian functions:
\begin{quote}
\PMlinkescapetext{For fifteen days I strove to prove that there could not be any functions like those I have since called Fuchsian functions.  I was then very ignorant; every day I seated myself at my work table, stayed an hour or two, tried a great number of combinations and reached no results.  One evening, contrary to my custom, I drank black coffee and could not sleep.  Ideas rose in crowds; I felt them collide until pairs interlocked, so to speak, making a stable combination.  By the next morning I had established the existence of a class of Fuchsian functions, those which come from the hypergeometric series; I had only to write out the results, which took but a few hours.} \cite{PHSM}
\end{quote}
This is a clear indication Henri Poincar\'e brilliance. Poincar\'e communicated a lot with Klein another mathematician working on Fuchsian functions. They were able to discuss and further the theory of automorphic\textsl{(Fuchsian)} functions.  Apparently Klein became jealous of Poincar\'e's high opinion of Fuch's work and ended their relationship on bad \PMlinkescapetext{terms}.  

%=======================================================[[algebraic topology & homotopy theory begins here]]
Poincar\'e contributed to the field of algebraic topology and published \emph{Analysis situs} in 1895 which was the first \PMlinkescapetext{real} systematic look at topology.  He acquired most of his knowledge from his work on differential equations.  He also formulated the Poincar\'e conjecture, one of the great unsolved mathematics problems. It is currently one of the ``Millennium Prize Problems''. The problem is stated as:
\begin{quote}
Consider a compact 3-dimensional manifold V without boundary.  Is it possible that the fundamental group V could be trivial, even though V is not homeomorphic to the 3-dimensional sphere? \cite{CMI}
\end{quote}
The problem has been attacked by many mathematicians such as Henry Whitehead in 1934, but without success.  Later in the 50's and 60's progress was made and it was discovered that for higher-dimension manifolds the problem was easier. \textsl{(Theorems have been stated for those higher dimensions by Stephe Smale, John Stallings, Andrew Wallace, and many more)} \cite{CMI}  Poincar\'e also studied homotopy theory, which is the study of topology reduced to various groups that are algebraically invariant.\cite{MT}  He introduced the fundamental group in a paper in 1894, and later stated his infamous conjecture.  He also did work in analytic functions, algebraic geometry, and Diophantine problems where he made important contributions not unlike most of the areas he studied in.

%=======================================================[[three body problem & celectial mechanics begins here]]
In 1887, Oscar II, King of Sweden and Norway held a competition to celebrate his sixtieth birthday and to promote higher learning.\cite{EE}  The King wanted a contest that would be of interest so he decided to hold a mathematics competition.  Poincar\'e entered the competition submitting a memoir on the three \PMlinkescapetext{body} problem which he describes  as:
\begin{quote}
Le but final de la M\'ecanique c\'eleste est de r\'esoudre cette grande question de savoir si la loi de Newton explique \`a elle seule tous les ph\'enom\`enes astronomiques; le seul moyen d'y parvenir est de faire des observation aussi pr\'ecises que possible et de les comparer ensuite aux r\'esultats du calcul.  (The goal of celestial mechanics is to answer the great question of whether Newtonian mechanics explains all astronomical phenomenons. The only way this can be proven is by taking the most precise observation and comparing it to the theoretical calculations.) \cite{PHMC}
\end{quote}
Poincar\'e did in fact win the competition.  In his memoir he described new mathematical ideas such as homoclinic points. The memoir was about to be published in \emph{Acta Mathematica} when an error was found by the editor.  This error in fact led to the discovery of chaos theory. The memoir was published later in 1890.\cite{MT}  In addition Poincar\'e proved that the determinism and predictability were disjoint problems. He also found that the solution of the three body problem would change drastically with small change on the initial conditions.  This area of research was neglected until 1963 when Edward Lorenz discovered the famous a chaotic deterministic system using a \PMlinkescapetext{simple model} of the atmosphere.\cite{MM} 

%=======================================================[[relativity and einstein]]

Henri Poincar\'e and Albert Einstein had an interesting 
relationship concerning their work on relativity \textit{(one might actually
describe it as a lack of a
relationship)}.  Their interaction begins in 1905 when Poincar\'e 
published his first paper on relativity.  The topic of the paper 
was ``partly kinematic, partly dynamic''\cite{PA} which included the 
correction of Lorentz's proof related to the Lorentz transformation 
\textit{(actually named by Poincar\'e)}. About a month later 
Einstein published his first paper on relativity.  Both continued 
publishing work about relativity, but neither of them would reference 
each others work.  Not only did Einstein not reference
Poincar\'e's work but claimed never to have read it\textit{(and it is not known if he eventually read
Poincar\'e papers)}\cite{PA}.  On one occasion Einstein referenced 
Poincar\'e aknowledging his work on relativity in the text of a 
lecture in 1921 called `Geometrie und Erahrung'.\cite{PA}  Although 
later in Einstein's life, he would comment on Poincar\'e as being 
one of the pioneers of relativity. Before Einstein's death, Einstein said:
\begin{quote}
Lorentz had already recognized that the transformation named after him is
essential for the analysis of Maxwell's equations, and Poincar\'e deepened
this insight still further...\cite{PA}
\end{quote}



%=======================================================[[applied mathematics begins here]]
Poincar\'e made many contributions to different fields of applied mathematics such as: celestial mechanics, fluid mechanics, optics, electricity, telegraphy, capillarity, elasticity, thermodynamics, potential theory, quantum theory, theory of relativity and cosmology.  In the field of differential equations Poincar\'e has given many results that are critical for the qualitative theory of differential equations, for example the Poincar\'e sphere and the Poincar\'e \PMlinkescapetext{map}.

It is that intuition that led him to discover and study so many areas of science.  Poincar\'e is considered to be the next universalist after Gauss.  After Gauss's death in 1855 people generally believed that there would be no one else that could master all branches of mathematics. However they were wrong because Poincar\'e took all areas of mathematics as ``his province''\cite{BC}.
%=======================================================[[Appendix]]
\section{Appendix}
\subsection{Awards:}
\begin{itemize}
\item In 1901 was awarded the 
\PMlinkexternal{Royal Society Sylvester Medal}{http://www.royalsoc.ac.uk/}.
\item The 1911 \PMlinkexternal{Astronomical Society of the Pacific}{http://www.astrosociety.org/}, \PMlinkexternal{Bruce Medalist}{http://www.phys-astro.sonoma.edu/BruceMedalists/Poincare/index.html}.
\end{itemize}
\subsection{Fellowships:}
\begin{itemize}
\item In 1894 was elected fellow of \PMlinkexternal{The Royal Society}{http://www.royalsoc.ac.uk/}.
\item In 1985 was elected fellow of \PMlinkexternal{The Royal Society of Edinburgh}{http://www.royalsoced.org.uk/}.
\end{itemize}

%=======================================================[[Bibliography begins here]]\%
\begin{thebibliography}{99}
\bibitem[EE]{EE} The 1911 Edition Encyclop{\ae}dia: \textsl{Oscar II of Sweden and Norway}, [online], \PMlinkexternal{http://63.1911encyclopedia.org/}{http://63.1911encyclopedia.org/}
\bibitem[BA]{BA} Belliver, Andr\'e: \textsl{Henri Poincar\'e ou la vocation souveraine}, Gallimard, 1956.
\bibitem[BR]{BR} Bour P-E., Rebuschi M.: \textsl{Serveur W3 des Archives H. Poincar\'e} [online] \PMlinkexternal{http://www.univ-nancy2.fr/ACERHP/}{http://www.univ-nancy2.fr/ACERHP/}
\bibitem[BC]{BC} Boyer B. Carl: \textsl{A History of Mathematics: Henri Poincar\'e}, John Wiley \& Sons, inc., Toronto, 1968.
\bibitem[CMI]{CMI} Clay Mathematics Institute: \textsl{Millennium Prize Problems}, 2000, [online] \PMlinkexternal{http://www.claymath.org/prizeproblems/}{http://www.claymath.org/prizeproblems/}.
\bibitem[EB]{EB} Encyclop{\ae}dia Britannica: \textsl{Biography of Jules Henri Poincar\'e}.
\bibitem[MM]{MM} Murz, Mauro: \textsl{Jules Henri Poincar\'e [Internet Encyclopedia of Philosophy]}, [online] \PMlinkexternal{http://www.utm.edu/research/iep/p/poincare.htm}{http://www.utm.edu/research/iep/p/poincare.htm}, 2001.
\bibitem[KD]{KD} Kolak, Daniel: \textsl{Lovers of Wisdom (second edition)}, Wadsworth, Belmont, 2001.
\bibitem[MT]{MT} O'Connor, J. John \& Robertson, F. Edmund: \textsl{The MacTutor History of Mathematics Archive}, [online] \PMlinkexternal{http://www-gap.dcs.st-and.ac.uk/~history/}{http://www-gap.dcs.st-and.ac.uk/~history/}, 2002.
\bibitem[OHP]{OHP} Oeuvres de Henri Poincar\'e: Tome XI, Gauthier-Villard, Paris, 1956.
\bibitem[PA]{PA} Pais, Abraham:  Subtle is the Lord..., Oxford University Press, New York, 1982.
\bibitem[PHSM]{PHSM} Poincar\'e, Henri: \textsl{Science and Method}; The Foundations of Science, The Science Press, Lancaster, 1946.
\bibitem[PHSH]{PHSH} Poincar\'e, Henri: \textsl{Science and Hypothesis}; The Foundations of Science, The Science Press, Lancaster, 1946.
\bibitem[PHMC]{PHMC} Poincar\'e, Henri: \textsl{Les m\'ethodes nouvelles de la m\'ecanique celeste}, Dover Publications, Inc. New York, 1957.
\bibitem[SJ]{SJ} Sageret, Jules: \textsl{Henri Poincar\'e}, Mercvre de France, Paris, 1911.
\bibitem[WHP]{WHP} Wikipedia: Henri, Poincar\'e. [online] \PMlinkexternal{http://en.wikipedia.org/wiki/Henri_Poincare}{http://en.wikipedia.org/wiki/Henri_Poincare}, 2004.
\end{thebibliography}

\subsection*{See also}
\begin{itemize}
\item The MacTutor History of Mathematics Archive, \PMlinkexternal{Jules Henri Poincar\'e}{http://www-gap.dcs.st-and.ac.uk/~history/Mathematicians/Poincare.html}
\item Internet Encyclopedia of Philosophy, \PMlinkexternal{Jules Henri Poincar\'e}{http://www.utm.edu/research/iep/p/poincare.htm}
\item Wikiquote, \PMlinkexternal{Henri Poincar\'e}{http://en.wikiquote.org/wiki/Henri_Poincare}
\item Mathpages, \PMlinkexternal{Poincar\'e and the Copernican Alternative}{http://www.mathpages.com/home/kmath305/kmath305.htm}
\end{itemize}


%=======================================================[[Document ends here]]
%%%%%
%%%%%
\end{document}
