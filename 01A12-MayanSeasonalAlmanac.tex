\documentclass[12pt]{article}
\usepackage{pmmeta}
\pmcanonicalname{MayanSeasonalAlmanac}
\pmcreated{2015-12-18 5:13:35}
\pmmodified{2015-12-18 5:13:35}
\pmowner{milogardner}{13112}
\pmmodifier{milogardner}{13112}
\pmtitle{Mayan Seasonal Almanac}
\pmrecord{403}{42509}
\pmprivacy{1}
\pmauthor{milogardner}{13112}
\pmtype{Definition}
\pmcomment{trigger rebuild}
\pmclassification{msc}{01A12}
\pmsynonym{Mayan math}{MayanSeasonalAlmanac}
%\pmkeywords{Mayan mathematics}
\pmdefines{Mayan astronomy}

% this is the default PlanetMath preamble.  as your knowledge
% of TeX increases, you will probably want to edit this, but
% it should be fine as is for beginners.

% almost certainly you want these
\usepackage{amssymb}
\usepackage{amsmath}
\usepackage{amsfonts}

% used for TeXing text within eps files
%\usepackage{psfrag}
% need this for including graphics (\includegraphics)
%\usepackage{graphicx}
% for neatly defining theorems and propositions
%\usepackage{amsthm}
% making logically defined graphics
%%%\usepackage{xypic}

% there are many more packages, add them here as you need them

% define commands here

\begin{document}
\PMlinkexternal{Advanced cultures around the globe}{http://www.math.okstate.edu/~wrightd/crypt/lecnotes/node21.html} have created abstract mathematics. The math modeled the \PMlinkexternal{cycles of planets}{http://en.wikipedia.org/wiki/Naked-eye_planet} and a wide array of astronomical events.  Chinese math, for example, modeled a "string of pearls" event such that planetary alignments focused on Feb. 1, 1951 BCE and other times when fewer planets aligned.  \PMlinkexternal{Chinese}{http://en.wikipedia.org/wiki/List_of_Chinese_discoveries} mathematical astronomers birthed the \PMlinkexternal{Chinese remainder theorem}{http://en.wikipedia.org/wiki/Chinese_remainder_theorem} a generalized indeterminate equation method aligned calendars and predicted astronomical events including solar and lunar eclipses. 

Several advanced cultures, \PMlinkexternal{including our own}{http://www.universetoday.com/34076/planetary-alignment/}, have predicted planetary cycles and events in exacting ways. One aspect of this paper introduces \PMlinkexternal{LCM}{http://www.calculatorsoup.com/calculators/math/lcm.php#.UbCo5JyTs68} math that was independently developed by Mayans in base 13 and base 20. Mayans aligned calendars in n-calendar rounds by \PMlinkexternal{LCM}{http://mathworld.wolfram.com/LeastCommonMultiple.html} tests in multiple planetary and lunar cycles. The modular math tests followed implicit features of CRT-like math paired, tripled and quadrupled 117, 260, 360, 364,365, 584, 585, and 780 cycles, and other planetary cycles and groupings scaled to n-calendar rounds(CR). One CR =  18980 days.

Following Mayan LCM math via the sun, equinoxes and solstices across the horizon, as  Carlos Berrera clearly decodes the \PMlinkexternal{Dresden Venus Table}{https://www.academia.edu/4220988/The_Dresden_Codex_Master_Structure_Illustrated_-_English_Version}. The Mayan year began at the summer solstice. Nominal lunar and planetary  cycles aligned solar and lunar cycles and cycles of nearby our planets Mercury, Venus, Mars, Jupiter and Saturn.  

In 2012 \PMlinkexternal{four super-numbers}{http://planetmath.org/mayansupernumberarithmetic} were dated to 419 AD. The first long count reported the LCM (584, 585) = 341640 = (8)(365)(117) = (2)(9)(73)(260) = (13)(73)(360) =(73)(4680)= (13)(72)(365)= (5)(9)(13)(584) =(8)(73)(585)=  (6)(73)(780)= (3)(6)(18980) = 18-CR.  

Aveni and other scholars linked the remaining three  long count super-numbers to Mercury, earth, Venus and Mars LCM identities scaled to 63-CR, 93-CR and 129-CR, respectively.  Mayans may have intended these super numbers to be scaled to 21-(3-CR), 31-(3-CR) and 43-(3-CR), as parsed by: 

2. 1195740 = (4)(7)(365)(117) = (3)(3)(7)(73)(260) = (3)(3)(5)(73)(364)= (21)(156)(365)= (4)(7)(73)(585)= (21)(73)(780)= (3)(21)(18940) = 21(56940) meant LCM (260, 364, 365, 585) = LCM (364, 365, 585,780)= LCM (260, 364, 365, 585, 780) = LCM(364, 365, 584, 585).

3. 1765140 = (31)(219)(260) = (31)(156)(365) = (31)(73)(780)= (3)(31)(18980) = 31(56940)= LCM (260, 365, 780, 2263) 

4. 2448420  = (43)(219)(260) = (43)(156)(365) = (43)(73)(780) = (3)(43)(18980)= 43(56940) = LCM (260, 365, 780, 3139)

\PMlinkexternal{The Dresden Venus Almanac}{http://www.bibliotecapleyades.net/ciencia/dresden/dresdencodex04.htm} also breaks  down 9.9.16.0.0 = 1,366,560 = 2340(584) = 72(18980)= 36(37960) = 24(3CR) = (8)(365)(468) into 10 sub-almanacs. For example, sub-almanac VI breaks down (65)(584) = (104)(365) into (236 + 90 + 250 + 8) = 584. Lounsbury discussed the 236 term by the diophantine equation 37960x - 2340y = N to find x = 4 and y = 61 and N = 9100. The result offered an exact alignment of the long count to our modern Julian Day calendar was published in "A Solution for the Number 1.5.5.0 of the Mayan Venus Table".  Floyd Lounsbury did not offer Diophantine approaches to solve the 90, 250 and 8 Venus terms, a three-part analysis that would confirm the 236 relationship to the 1.5.5.0 conclusion.

In addition, another four-part synodic analysis of Venus (2920 days) was discussed by Mayans that improved Venus-Mars 18-CR super-number tables (with no explicit almanac for historians to compare) to Dresden 72-CR Mercury-Venus almanac data. At the same time the 419 AD Xultun lunar calendar built on 177, 178 cycles (7972 - 7795) = 177 days was more accurate than the Dresden 405-lunation (119580 built upon 6-moon(177 days) and five-moon (148 days), as reported by Stuart (Unearthing the Heavens: Classic Mayan murals and Astronomical Tables at Xultun, Guatemala by Zender, Skidmore). 

The linear door to Mayan time encoded four long count dates as least common multiple (\PMlinkexternal{LCM}{http://www.calculatorsoup.com/calculators/math/lcm.php#.UbCo5JyTs68})s.  Related classes of quotients and remainder appeared 800 years later in Dresden Codex almanacs.  

The Dresden codex scaled families of almanacs to four modular and linear synodic and sidereal cycles. A short list scaled Mercury (117), 9-moons (260), 405-moons (11960), earth (360, 364, 365), Venus (584, 585), and Mars (780) introduces this topic. Other planetary cycles included LCMs 2340, 2920, 11960, 18980, and 37960. At times nominal sidereal cycles were discussed super-number and serpent-numbers. Two of 13 serpent-numbers were parsed by Mayans into super-numbers and planetary cycles were mentioned on pages 61-64 in the Dresden Codex.

For example 4.6.1.9.15.0 = 12,394,740 = 60(167)(1237), 

and 

12,466,942 = 34156(365) + 2 days

opens a topic explained on pages 65-69.

To introduce pages 65-68 data that Barbara Tedlock reported synodic and sidereal lunar cycles in "The Sky in Mayan Literature" (1992), The book was edited by A. Aveni in "The Road of light: Theory and Practice opf Mayan Sky Watching". Tedlock described overlapping A, B, C, and D circular 65 day and 82 day lunar cycles defined one aspect of the modular Mayan 260 day lunar cycle. Per Tedlock, "Simultaneous with a sidereal rhythm these same visits contain a synodic rhythm. For any two successive mountaintop shrines A and B, the phase of the observed at the opening of A will repeat in 147 days later at the closing of B, and yet again when B is opened 178 after it was closed, a total of 325 days after the opening of A. ... Summarizing the arithmetic, we find 147 = 65 + 82 and 325 = 147 + 178 = 4(65) + 65...". Scholars tend to discuss this data in linear ways. 

The seasonal almanac documents how and why Mayans thought of calendarrounds. The four lines cited cyclical mod 20 quotients and day remainders as binary pairs. A new second level reports 1820 = 7(260) = 5(364).  When quotients and remainders of all four lines were scaled on level three by an additional 20: line 1 = 2 calendar rounds, line 2 = 2 calendar rounds, line 3 =2 calendar rounds + 260 days, and line 4 = 2 calendar arounds + 520 days. 

Mayans integrated older \PMlinkexternal{Olmec}{http://www.andaman.org/BOOK/chapter54/text-Olmec/text-Olmec.htm} \PMlinkexternal{long count}{http://en.wikipedia.org/wiki/Mesoamerican_Long_Count_calendar} dating and numeration systems within classes of LCMs and nominal modular planetary cycles. For example, the lunar 260 day cycle is a divisor of 341640 reported quotient 1314 with zero remainder. 

In the 1200 AD Dresden Codex seasonal almanac numerals were recorded mod 13 data in four part (stative words), and four colors representing the four directions. Long count mpd 18, 20, 360, 7200 and 144000 mentioned paired six-digit and longer serpent-numbers. Mayan scribes encoded four line of seasonal almanac mod 13 written across four pages. The quotient(20) and day remainder system referenced black and red serpent-numbers on pages 61-64. Planetary synodic data was recorded on pages 65-68 of the Dresden Codex.

A \PMlinkexternal{1988 paper}{http://adsabs.harvard.edu/full/1988JHAS...19....1B} and a \PMlinkexternal{2011 year book}{http://www.amazon.com/Astronomy-Codices-Memoirs-American-Philosophical/dp/0871692651} written by Victoria Bricker and Harvey Bricker decodes a first level of scribal math reported on pages 65-68 of the Dresden Code. The raw almanac data reported linear and possibly modular eclipse, solstice and equinox cycles are suggested to be valid for 800 years (per 15 reference dates cited on pages 61-64), and longer by serpent-numbers.

Second and third levels of the seasonal almanac are reported by Bruce Friedman. Friedman, like the Brickers, sums the four rows to 1898, 1898, 1924 and 1911 day totals on pages 65-68.  The Brickers' 1988 paper reported the same data by a linear series: 9 5 1 10 6 2 11 7 3 12 8 4 13 an from the [third visible line] bottom of the codex pages. 

Modern translations into base 13 define black numbers as a + 9 additive sequence, or s times 9 multiplication sequence.
    
    Lines 1 and 4 black numbers 
    
    
    9 5 1 10 6 2 11 7 3 12 8 4 13 
    
    seen  as + 9 addition says ( 9 + 9 = 18, base 13) = 5 base 13 and so forth.
    
    seen as a 9 table reportys the same squence:
    
    (1 2 3 4 5 6 7 8 9 10 11 12 13) by 9 report base 13 remainders 
    
    examples:  1 x 9 = 9, 2 x 9 = (18 - 13) = 5, 3 x 9 = (27 - 26) = 1

Lines 2 and 3 also sum to 91, a simple sum of 1 + 2 + 3 + 4 + 5 + 6 + 7 + 8 + 9 + 10 + 11 + 12 +13

and base 13 checksum. 

The rule used to reorder lines 2 and 3 black numbers has not been decoded.

The second and third levels do not appear in the Seasonal Table. Reconstruction of the damaged data represents even entries of the missing data. 

Reconstructions suggest that the entire top line  read: 

9 12 5 4 1 5 10 2 6 8 2 10 11 8 7 2 3 5 12 4 8 12 4 3 13 3

Following the Bricker and older reconstructions  11 13 11 1 8 6 4 2 13 6 6 8 2 offers entries from the [second visible line] middle of the Codex.

Secondly, 1 1 12 13 8 1 5 7 7 13 6 1 3 shows proto-tzolkin coefficietnwas comprised of QUOTIENT entries of the [first visible line] top of the CdX pages. Figure 6's description is a schematic of the upper half of the CdX. To review this in greater detail,  one other minor item on Page S36, the near top paragraph,  begins with "The sky band" has a misplaced sentence that begins with "appears twice".

The reconstructed first line (summed to 1898) have been replaced on pages 65-68 as one data set:

9 12 5 4 1 5 10 2 6 8 2 10 11 8 7 2 3 5 12 4 8 12 4 3 13 

11 1 13 1 11 12 1 13 8 8 6 13 4 5 2 7 13 7 6 13 6 6 8 1

11 11 13 11 11 9 1 10 8 5 6 11 4 2 2 4 13 4 6 10 6 3 8 11

9 9 5 1 1 2 10 12 6 5 2 7 11 5 7 12 3 2 12 1 8 9 4 13

Unpacking the Mayan Dresden Codex seasonal almanac of four lines A, B, C, D of alternative 13 black and 13 red base 13 numbers offers a fuzzy glimpse into 1200 AD Mayan data on lines A, D by four patterns , pattern (1): black numbers were a mod 13 multiplication by 9 table (as discussed earlier) , pattern (2) alternative 13 black and 13 red numbers sum: red + black = red (mod 13) endlessly, an additive property maintained in rows B , C example from line A: 12 + 5 = (17 - 13) = 4, 4 + 1 = 5, 5 + 10 = (15 -13) = 2 and so forth;  "pattern, (3) the 26 black and red entries subtracted one row from an adjoining line (A -B ) and lines ( D-C ) creates two new 26 term series; pattern(4) [(A-B) - ((D-C)] forms an interesting pattern ... that Mayans may have used to double, and triple check the table's entries:

7 0 7 0 7 0 7 0 7 0 7 0 7 0 7 0 7 0 7 0 7 0 7 0 7

Odd members in each row represent the black numbers from the Seasonal Table. All the rows black numbers sum to 91. When scaled by 20 all four quotients sum to 1820 = 7(260) = 5(364). Even members, Red number sums for the rows are: 78, 78, 104, 91. Total of all rows black is 364 and red is 351. The inherent astronomical viability of these numbers allows for treatment in different modular bases. Example 20 * 364 is 7280 and 20 * 351 is 7020. One 260 day Tzolkin separates these values which themselves are [for 7280] 20 Earth counting years and also 10 synodic periods of Mars less 2 Tzolkins and [for 7020] 10 synodic periods of Mars less its retrograde periods and also 12 synodic periods of Venus. Other multipliers will often yield other astronomical "coincidences." More importantly, there may be modular and algorithmic generators for the central rows B and C as there clearly are for the outer rows A and D.

The second level offers 9 12 = 9(20) + 12 = 192 such that  four new 13-term series were repeated 20 times reveal 37960 four times  on level three:

192 104 25 202 128 60 228 142 65 244 172 63 263 = 1898(20)=146(260)= 104(365)

221 261 232 33 168 121 85 47 267 133 126 161 43=1898(20)= 146(260) = 65(584)

231 271 229 30 165 131 82 44 264 130 123  171 53= 1924(20) = 148(260) = 65(592)

189 101 22 212 125 47 225 152 62 241 169 93 273 = 1911(20)= 147(260)=49(780)

        TOTAL                 8 CALENDAR ROUNDS + 780 and              260(584) + 780
 
In summary, the \PMlinkexternal{new data} {http://www.famsi.org/mayawriting/codices/pdf/3_madrid_rosny_bb_pp57-78.pdf} scaled raw ST data by 20^2 defines four series, proposed information that requires five explanations. 

First, the Dresden Codex seasonal almanac was scaled by \PMlinkexternal{LCM}{http://www.calculatorsoup.com/calculators/math/lcm.php#.UbCo5JyTs68}s to two calendar rounds on a third level. The CR data aligned lunar, earth and Venus cycles. The scaled data computed LCM (260, 360) = 4680 = 18(260) = 13(360) = 6(780), LCM (260, 364) = 1820 = 7(260) = 5(364), LCM (260, 365) = 1 CR, LCM (260, 584) = 2 CR and LCM (260, 585) = 2340 = 9(260) = 4(585).


Second, the \PMlinkexternal{Madrid Codex}{http://www.famsi.org/mayawriting/codices/pdf/3_madrid_rosny_bb_pp57-78.pdf} has long been known as a calendar round text. \PMlinkexternal{Cyrus Thomas}{http://www.doaks.org/library-archives/rare-book-collection/rare-books/ancient-future/mayan-calendar-systems} in 1900 scaled certain quotients by 360, ie. 13.13.13.13.13.13.13.13 in alterative black (quotients) and red (remainders)colors = [13(360) + 13(20)] x 4 = 18720 = 72(260) = 52(360) = 32(585), 260 days from a calendar round.  

Third,  Madrid distance numbers were scaled to lunar, earth, Venus, Mars and often Jupiter cycles (399, 1432) per long count numbers:

4.4.4.4 =  30824 = 76(399) = 7(4332)

5.5.5.5 = 37905 = 95(399) = [54(702)-3]

6.6.6.6 =  45486 = 114(399) 

7.7.7.7 = 53067 = 133(399)

8.8.8.8 =  61648= 152(399) = 14(4332)

11.11.11.11 = 83391 = 209(399)

13.13.13.13 = 98553 = 247(399)

Note, since 1.1.1.1  = (19)(399), the differnce between most adjactent lines is 19(399).

An independent analysis of 120 LCM combinations of nominal Mayan  planetary cycles 260, 360, 364, 365, 584, 585 and 780 yields 18-CR 28 times includes three additional long count LCM identity relationships, many of which are directly mentioned on the wall's four long count super numbers in ways that later Mayans improved upon by extending to 72-CR. 

In addition, quotients of 260, 365, 780 and 56940 validates a \PMlinkexternal{Mars}{http://www.pnas.org/content/98/4/2107.long} focus per  73(160) = 11680 rather than 11679 cited by Aveni and the Brickers. Note 20 Venus synodic cycles approximates 11680, off by one day. \PMlinkexternal{Powell "New View of Mayan Astronomy"}{http://mayaexploration.com/themes_archaeo.php}  detail Mayan LCM methods that expose Mars, Jupiter and Saturn nominal cycles and  methods that coincide with the LCM premise of  this paper.

Concerning 5.5.5.5  54(702) = [Mars Synodic less retrograde] - 3 days = 100(379) [Saturn Synodic] + 5 days, basically  95 Jupiter roughly equal to 100 Saturn;  NASA data compares 95(398.88) = 37893.6 and 100(378.09) = 37809.

Iimplicit 3-calendar and 7-calendar rounds were computed by \PMlinkexternal{LCM}{http://www.calculatorsoup.com/calculators/math/lcm.php#.UbCo5JyTs68} (a, b) per: LCM (260, 360) = 4680 = 18(260) = 13(360) = 6(780), LCM (365, 364) = 7 CR, LCM (365, 260) = 1 CR, LCM (365, 584) = 2920 = 8(365) = 5(584), LCM (365, 780) = 3 CR, and LCM (365, 585) = 42705 = 117(365) = 73(585). Had LCM(4680, 365) = 18-CR, LCM (468, 3650)= 45-CR, and LCM (4680, 3650) = 90-CR been considered  a longer Madrid set of tables would have discussed.
 
Third, numerical symmetries fill  the four ST lines ,  each 13-term terms on level two. Reference number 178 ”appears” where 177 is ”expected” without 148 being mentioned related to eclipse cycles. Line two referred an anomaly (namely line 2 on p.65 where 13,11 appears to break a pattern) of 10 day mod 13 additions to alternate entries in all of lines one and two. Clearly 13,10 was ”expected” related to the b pattern and would have made the 178 ”appears” as 177. 

Madrid-type scalings of level one of the Dresden seasonal table by 360 offers other level two patterns: 18(1898) + 12(13) days- twice, 18(1924) + 16(13) days and 18(1911) + 14(13) days such that LCM (520, 260, 67) = 34840, LCM (260, 390, 780, 176) =34320, LCM (260, 364, 76) = 34580. 

In contrast, the Madrid itself offers several line
four lines of 9-term  trecena remainders with differentials of five. Interesting patterns.   
 
Fourth, on level three of the ST and the Madrid, \PMlinkexternal{LCM}{http://www.calculatorsoup.com/calculators/math/lcm.php#.UbCo5JyTs68} (a, b, c) combinations of 260, 360, 364, 365, 584, 585 and 780 offers an interesting distribution:
2340 (1), 4680 (2), 5460 (1), 16380 (2), 26280 (1), 32760 (3), 2-CR (1),  3-CR (1), 6-CR (2), 7-CR (1), 9-CR (2), 14-CR (2), 18-CR (6), 21-CR (1), 42-CR (2), 63-CR (1),  and 126-CR (3).
 
Fifth, on level three of the Dresden and Madrid, considering all 120 \PMlinkexternal{LCM}{http://www.calculatorsoup.com/calculators/math/lcm.php#.UbCo5JyTs68}s of 260, 360, 364, 365, 584, 585, and 780  by unique combinations: LCM (a,b), LCM ( a, b, c), LCM (a, b, c, d),  LCM (b, c, d, e,), LCM(a, b, c, d, e , f), and  LCM (a, b, c, d, e, f, g) computes 86 in multiples of 18, 980 days  = 1 calendar round (CR) per:
 
1-CR (1), 2-CR (2), 3-CR (2),  6-CR (4), 7-CR (2), 9-CR (3),  14-CR (4),  18-CR (28),  21-CR (2),  42-CR (4), 63-CR (4) and 126-CR (30)
 
attests to Mayan calendar round  LCM math.

RELATED DATA

Mixtec, Aztec and cultures nearby the Maya also used calendar round based time keeping that seemed to be absent almanacs that double checked the cycles of nearby planets with lunar and earth cycles.

Four 2012 reported super-numbers were recorded on a wall near near Tikal, Guatemala around 419 AD. The long count distance numbers represented \PMlinkexternal{lunar, Mars and Mercury calendars}{http://articles.latimes.com/2012/may/10/science/la-sci-ancient-mayan-calendar-20120511} data that may double check great year and universe modeling origins built-in to the Dresden Codex.

Mayan astronomers independently worked from the mid-year season, outwards, denoted by feet icons going forward and back, and year bearers as the Brickers report. The matrix data double checked and proved almanac and other data by several techniques, the details of which are under investigation.

A set of least common multiples created and double checked multiple calendars. The most interesting is the 260 day tzolkin calendar; 260 [20*13] days is not a good estimate for planetary periods but, i.e. 3 tzolkins is an excellent estimate for the synodic period of Mars; 9 tzolkins is good for 4 synodic periods of Venus; 16 tzolkins is good for 11 synodic periods of Saturn; 17 tzolkins estimates one sidereal Jupiter period (less 12 days), 22 tzolkins is an excellent estimate for 65 sidereal periods of Mercury; 23 tzolkins is great for 52 synodic Mercury periods. Note non-Mayan but interesting
91 tzolkins is only 2 days over 29 sidereal periods of Neptune and 59 tzolkins is only ~4 hours over 42 solar earth years and 91 tzolkins is only 2 days over 64 sidereal periods of Uranus}.

\PMlinkexternal{FOOTNOTE}{http://en.wikipedia.org/wiki/Mayan_languages}: Statives and positional words

In Mayan languages, words were usually viewed as belonging to one of four classes: verbs, statives, adjectives, and nouns.

Stative numbers were  predicates of four-part astronomical cycles. The syntactic properties fell  between verbs and adjectives in Indo-European languages. Like verbs, statives words and numbers were sometimes be inflected by persons but normally lacked inflections for tense, aspect and other purely verbal categories. This is, very similar to the so-called Japanese "adjectives". Statives words and numbers were adjectives, and  positional numerals.

Positional words were a class of root characteristics of, if not unique to, the Mayan languages, form stative adjectives and verbs (usually with the help of suffixes) with meanings related to the position or shape of an object or person. Mayan languages have between 250 and 500 distinct positional roots.

\begin{thebibliography}{5}

\bibitem{1} Asger Aaboe, \emph{"Remarks on the theoretical treatment of eclipses in antiquity"},Journal for the History of Astronomy (Cambridge), 1972.
\bibitem{2} Anthony F. Aveni editor, \emph{The Sky in Mayan Literature},Oxford University Press, 1992 
\bibitem{3} Victoria Bricker and Harvey Bricker, \emph{"The Seasonal Table of the Dresden Codex and Related Almanacs"},ARCHAAEASTRONOMY (supplement to the JOURNAL OF THE HISTORY OF ASTRONOMY 19 12:S1-S62), 1988.
\bibitem{4} Albert Kroeber, \emph{"Handbook of California Indians"},pages 876-877, 1920.
\bibitem{5} Cyrus Thomas, \emph{"Mayan Calendar Systems"  (Smithsonian Institution, Bureau of  American Ethnology, Nineteenth Annual Report} (1884-1898), Washington, D.C. Government Printing Office, part 2, pp. 693- 955 (817), 1900

\end{thebibliography}

%%%%%
%%%%%
\end{document}
