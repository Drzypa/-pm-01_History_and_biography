\documentclass[12pt]{article}
\usepackage{pmmeta}
\pmcanonicalname{BlaisePascal}
\pmcreated{2013-03-22 16:51:57}
\pmmodified{2013-03-22 16:51:57}
\pmowner{PrimeFan}{13766}
\pmmodifier{PrimeFan}{13766}
\pmtitle{Blaise Pascal}
\pmrecord{5}{39115}
\pmprivacy{1}
\pmauthor{PrimeFan}{13766}
\pmtype{Biography}
\pmcomment{trigger rebuild}
\pmclassification{msc}{01A45}

% this is the default PlanetMath preamble.  as your knowledge
% of TeX increases, you will probably want to edit this, but
% it should be fine as is for beginners.

% almost certainly you want these
\usepackage{amssymb}
\usepackage{amsmath}
\usepackage{amsfonts}

% used for TeXing text within eps files
%\usepackage{psfrag}
% need this for including graphics (\includegraphics)
%\usepackage{graphicx}
% for neatly defining theorems and propositions
%\usepackage{amsthm}
% making logically defined graphics
%%%\usepackage{xypic}

% there are many more packages, add them here as you need them

% define commands here

\begin{document}
\emph{Blaise Pascal} (1623 - 1662) French mathematician, best known for Pascal's triangle and Pascal's theorem.

The son of a judge, young Blaise was at first discouraged from studying mathematics despite showing talent for it. The father finally relented when the son came up with his own proof of the Pythagorean theorem. Barely 18-years-old, Blaise invented one of the first mechanical calculators, the Pascaline.

As an adult, Pascal became interested in the mathematics of gambling. To study probabilities, he studied a triangular arrangement of binomial coefficients he called an ``arithmetic triangle.'' The triangle had been studied long before by the ancient Chinese, but Pascal's treatise on it was so influential to the modern study of mathematics that his name became attached to the triangle.

In 1658, Pascal caused controversy when he published a solution to a prize problem of his (the quadrature of a cycloid) under a pseudonym so as to not have to pay out the prize. Pascal made significant contributions to physics, and the metric unit of pressure is now called the pascal.

Towards the end of his life Pascal became increasingly interested in religion, and published a theological treatise.

The PASCAL programming language is named after him.
%%%%%
%%%%%
\end{document}
