\documentclass[12pt]{article}
\usepackage{pmmeta}
\pmcanonicalname{RuthBari}
\pmcreated{2013-03-22 16:38:44}
\pmmodified{2013-03-22 16:38:44}
\pmowner{Mravinci}{12996}
\pmmodifier{Mravinci}{12996}
\pmtitle{Ruth Bari}
\pmrecord{4}{38849}
\pmprivacy{1}
\pmauthor{Mravinci}{12996}
\pmtype{Biography}
\pmcomment{trigger rebuild}
\pmclassification{msc}{01A60}
\pmclassification{msc}{01A61}
\pmsynonym{Ruth Aaronson Bari}{RuthBari}

% this is the default PlanetMath preamble.  as your knowledge
% of TeX increases, you will probably want to edit this, but
% it should be fine as is for beginners.

% almost certainly you want these
\usepackage{amssymb}
\usepackage{amsmath}
\usepackage{amsfonts}

% used for TeXing text within eps files
%\usepackage{psfrag}
% need this for including graphics (\includegraphics)
%\usepackage{graphicx}
% for neatly defining theorems and propositions
%\usepackage{amsthm}
% making logically defined graphics
%%%\usepackage{xypic}

% there are many more packages, add them here as you need them

% define commands here

\begin{document}
\emph{Ruth Aaronson Bari} (1917 - 2005) American mathematician renowned for her work in graph theory. Almost 50, she was appointed professor of mathematics at George Washington University. Not directly related to Nina Bari.

\begin{thebibliography}{1}
\bibitem{ff} F. Fasanelli ``Ruth Aaronson Bari'' in {\it Women of Mathematics: A Bibliographic Sourcebook} L. Grinstein, P. Cambpell, ed.s New York: Greenwood Press (1987): 13 - 20
\end{thebibliography}
%%%%%
%%%%%
\end{document}
