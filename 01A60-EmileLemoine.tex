\documentclass[12pt]{article}
\usepackage{pmmeta}
\pmcanonicalname{EmileLemoine}
\pmcreated{2013-03-22 18:07:24}
\pmmodified{2013-03-22 18:07:24}
\pmowner{PrimeFan}{13766}
\pmmodifier{PrimeFan}{13766}
\pmtitle{\'Emile Lemoine}
\pmrecord{5}{40673}
\pmprivacy{1}
\pmauthor{PrimeFan}{13766}
\pmtype{Biography}
\pmcomment{trigger rebuild}
\pmclassification{msc}{01A60}
\pmclassification{msc}{01A55}
\pmsynonym{Emile Lemoine}{EmileLemoine}
\pmsynonym{\'Emile Michel Hyacinthe Lemoine}{EmileLemoine}
\pmsynonym{Emile Michel Hyacinthe Lemoine}{EmileLemoine}

% this is the default PlanetMath preamble.  as your knowledge
% of TeX increases, you will probably want to edit this, but
% it should be fine as is for beginners.

% almost certainly you want these
\usepackage{amssymb}
\usepackage{amsmath}
\usepackage{amsfonts}

% used for TeXing text within eps files
%\usepackage{psfrag}
% need this for including graphics (\includegraphics)
%\usepackage{graphicx}
% for neatly defining theorems and propositions
%\usepackage{amsthm}
% making logically defined graphics
%%%\usepackage{xypic}

% there are many more packages, add them here as you need them

% define commands here

\begin{document}
{\em \'Emile Michel Hyacinthe Lemoine} (November 22, 1840 - February 21, 1912) was a French civil engineer and a mathematician, a geometer in particular. He was educated at a variety of institutions, including the Prytan\'ee National Militaire and, most notably, the \'Ecole Polytechnique. Lemoine taught as a private tutor for a short period after his graduation from the latter school.

Lemoine is best known for his proof of the existence of the Lemoine point (or the symmedian point) of a triangle. Other mathematical work includes a system he called G\'eom\'etrographie and a method which related algebraic expressions to geometric objects. He has been called a co-founder of modern triangle geometry, as many of its characteristics are present in his work.

For most of his life, Lemoine was a professor of mathematics at the \'Ecole Polytechnique. In later years, he worked as a civil engineer in Paris, and he also took an amateur's interest in music. During his tenure at the \'Ecole Polytechnique and as a civil engineer, Lemoine published several papers on mathematics, most of which are included in a fourteen-page section in Nathan Court's College Geometry. Additionally, he founded a mathematical journal titled, L'interm\'ediaire des math\'ematiciens.

Lemoine was born in Quimper, France, on November 22, 1840, the son of a retired military captain who had participated in the campaigns of the First French Empire occurring after 1807. As a child, he attended the military Prytan\'ee of La Fl\`eche on a scholarship granted because his father had helped found the school. During this early period, he published a journal article in {\it Nouvelles annales de math\'ematiques}, discussing properties of the triangle.

Lemoine was accepted into, the \'Ecole Polytechnique in Paris at the age of twenty, the same year as his father's death. As a student there, Lemoine, a presumed trumpet player, helped to found an amateur musical group called ``La Trompette,'' for which Camille Saint-Sa\"ens composed several pieces. After graduation in 1866, he considered a career in law, but was discouraged by the fact that his advocacy for republican ideology and liberal religious views clashed with the ideals of the incumbent government, the Second French Empire. Instead, he studied and taught at various institutions during this period, studying under J. Kioes at the \'Ecole d'Architecture and the \'Ecole des Mines, teaching Uwe Jannsen at the same schools, and studying under Charles-Adolphe Wurtz at the \'Ecole des Beaux Arts and the \'Ecole de M\'edecine. Lemoine also lectured at various scientific institutions in Paris and taught as a private tutor for a period before accepting an appointment as a professor at the \'Ecole Polytechnique.

In 1870, a laryngeal disease forced him to discontinue his teaching. He took a brief vacation in Grenoble and, when he returned to Paris, he published some of his remaining mathematical research. He also participated and founded several scientific societies and journals, such as the Soci\'et\'e Math\'ematique de France, the {\it Journal de Physique}, and the Soci\'et\'e de Physique, all in 1871.

As a founding member of the Association Fran\c{c}aise pour l'Avancement des Sciences, Lemoine presented what became his best-known paper at its 1873 meeting in Lyon. The central focus of this paper concerned the point which bears his name today. Most of the other results discussed in the paper pertained to various concyclic points that could be constructed from the Lemoine point.

Lemoine served in the French military for a time in the years following the publishing of his best-known papers. Discharged during a commune, he afterwards became a civil engineer in Paris. In this career, he rose to the rank of chief inspector, a position he held until 1896. As the chief inspector, he was responsible for the gas supply of the city.

During his tenure as a civil engineer, Lemoine wrote a treatise concerning compass and straightedge constructions entitled {\it La G\'eom\'etrographie ou l'art des constructions g\'eom\'etriques}, which he considered his greatest work, despite the fact that it was not well-received critically. The original title was {\it De la mesure de la simplicit\'e dans les sciences math\'ematiques}, and the original idea for the text would have discussed the concepts Lemoine devised as concerning the entirety of mathematics. Time constraints, however, limited the scope of the paper. Instead of the original idea, Lemoine proposed a simplification of the construction process to a number of basic operations with the compass and straightedge. He presented this paper at a meeting of the Association Fran\c{c}aise in Oran, Algeria in 1888. The paper, however, did not garner much enthusiasm or interest among the mathematicians gathered there. Lemoine published several other papers on his construction system that same year, including ``Sur la mesure de la simplicit\'e dans les constructions g\'eom\'etriques'' in the {\it Comptes rendus of the Acad\'emie fran\c{c}aise}. He published additional papers on the subject in {\it Mathesis} (1888), {\it Journal des math\'ematiques \'el\'ementaires} (1889), {\it Nouvelles annales de math\'ematiques} (1892), and the self-published {\it La G\'eom\'etrographie ou l'art des constructions g\'eom\'etriques}, which was presented at the meeting of the Association Fran\c{c}aise in Pau (1892), and again at Besan\c{c}on (1893) and Caen (1894).

After this, Lemoine published another series of papers, including a series on what he called ``transformation continue'' (continuous transformation), which related mathematical equations to geometrical objects. This meaning stood separately from the modern definition of transformation. His papers on this subject included, ``Sur les transformations syst\'ematiques des formules relatives au triangle'' (1891), ``\'Etude sur une nouvelle transformation continue'' (1891), ``Une r\`egle d'analogies dans le triangle et la sp\'ecification de certaines analogies à une transformation dite transformation continue'' (1893), and ``Applications au t\'etra\`edre de la transformation continue'' (1894).

In 1894, Lemoine co-founded another mathematical journal entitled {\it L'interm\'ediaire des math\'ematiciens} along with Charles Laisant, a friend whom he met at the \'Ecole Polytechnique. Lemoine had been planning such a journal since early 1893, but thought that he would be too busy to create it. At a dinner with Laisant in March 1893, he suggested the idea of the journal. Laisant cajoled him to create the journal, and so they approached the publisher Gauthier-Villars, which published the first issue in January 1894. Lemoine served as the journal's first editor, and held the position for several years. The year after the journal's initial publication, he retired from mathematical research, but continued to support the subject. Lemoine died on February 21, 1912, in his home city of Paris.

{\it This entry was adapted from the Wikipedia article \PMlinkexternal{\'Emile Lemoine}{http://en.wikipedia.org/wiki/Emile Lemoine} as of June 6, 2008.}
%%%%%
%%%%%
\end{document}
