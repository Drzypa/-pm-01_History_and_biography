\documentclass[12pt]{article}
\usepackage{pmmeta}
\pmcanonicalname{KalleVaisala}
\pmcreated{2013-03-22 17:52:42}
\pmmodified{2013-03-22 17:52:42}
\pmowner{pahio}{2872}
\pmmodifier{pahio}{2872}
\pmtitle{Kalle V\"ais\"al\"a}
\pmrecord{8}{40359}
\pmprivacy{1}
\pmauthor{pahio}{2872}
\pmtype{Biography}
\pmcomment{trigger rebuild}
\pmclassification{msc}{01A60}
\pmsynonym{K. V\"ais\"al\"a}{KalleVaisala}
\pmsynonym{Kalle Vaisala}{KalleVaisala}
\pmrelated{EulersDerivationOfTheQuarticFormula}
\pmrelated{InverseLaplaceTransformOfMeromorphicFunction}
\pmrelated{VectorPotential}
\pmrelated{ProductOfNegativeNumbers}
\pmrelated{AlgebraicallySolvable}
\pmrelated{DyadProduct}
\pmrelated{FourierSeriesInComplexFormAndFourierIntegral}
\pmrelated{TruncatedCone}
\pmrelated{BombellisMethodOfComputingSquare}

% this is the default PlanetMath preamble.  as your knowledge
% of TeX increases, you will probably want to edit this, but
% it should be fine as is for beginners.

% almost certainly you want these
\usepackage{amssymb}
\usepackage{amsmath}
\usepackage{amsfonts}

% used for TeXing text within eps files
%\usepackage{psfrag}
% need this for including graphics (\includegraphics)
%\usepackage{graphicx}
% for neatly defining theorems and propositions
 \usepackage{amsthm}
% making logically defined graphics
%%%\usepackage{xypic}

% there are many more packages, add them here as you need them

% define commands here

\theoremstyle{definition}
\newtheorem*{thmplain}{Theorem}

\begin{document}
Kalle V\"ais\"al\"a (1893-8-19 \`a 1968-9-16), a Finnish mathematician and pedagogue, was born in Kontiolahti, Finland.  Having a function-theoretic background, his his PhD thesis 1916 was anyway {\em \"Uber die algebraisch aufl\"osbaren Gleichungen f\"unften Grades} (`On algebraically solvable equations of degree five').  Professor of mathematics in University of Tartu (Estonia), University of Turku (Finland) and Helsinki University of Technology.

V\"ais\"al\"a was known as an excellent teacher and author of textbooks of mathematics.  For schools he made textbooks of algebra and geometry, which was printed altogether 800,000 copies.  He abandoned the Euclidean systematics and extended the school algebra to the differential and integral calculus.  For orientating himself in the problematics he teached on the \PMlinkescapetext{side} in a school.\, V\"ais\"al\"a made also textbooks for university-level, the most known are {\em Lukuteorian ja korkeamman algebran alkeet} (`Elements of number theory and higher algebra'), {\em Vektorianalyysi} and {\em Laplace-muunnos} (`Laplace transform').
%%%%%
%%%%%
\end{document}
