\documentclass[12pt]{article}
\usepackage{pmmeta}
\pmcanonicalname{IngridDaubechies}
\pmcreated{2013-03-22 16:49:41}
\pmmodified{2013-03-22 16:49:41}
\pmowner{PrimeFan}{13766}
\pmmodifier{PrimeFan}{13766}
\pmtitle{Ingrid Daubechies}
\pmrecord{7}{39067}
\pmprivacy{1}
\pmauthor{PrimeFan}{13766}
\pmtype{Biography}
\pmcomment{trigger rebuild}
\pmclassification{msc}{01A60}
\pmclassification{msc}{01A61}
\pmclassification{msc}{01A65}

% this is the default PlanetMath preamble.  as your knowledge
% of TeX increases, you will probably want to edit this, but
% it should be fine as is for beginners.

% almost certainly you want these
\usepackage{amssymb}
\usepackage{amsmath}
\usepackage{amsfonts}

% used for TeXing text within eps files
%\usepackage{psfrag}
% need this for including graphics (\includegraphics)
%\usepackage{graphicx}
% for neatly defining theorems and propositions
%\usepackage{amsthm}
% making logically defined graphics
%%%\usepackage{xypic}

% there are many more packages, add them here as you need them

% define commands here

\begin{document}
\PMlinkescapeword{image}
\PMlinkescapeword{schemes}
\PMlinkescapeword{child}
\PMlinkescapeword{base}
\PMlinkescapeword{representation}
\PMlinkescapeword{degree}

\emph{Ingrid Daubechies} (1954 - ) Belgian mathematician and physicist, best known for Daubechies wavelets used in some image compression schemes.

The daughter of a civil engineer and a criminologist, Ingrid was as a child interested in the mathematics of base 10 representation of numbers and the workings of mechanical devices. After earning a degree in physics from the Vrije Universiteit Brussel, she stayed there teaching and researching until 1984. That year she won the Louis Empain Prize for Physics, a sort of Fields Medal for Belgian physicists.

In 1987, her study of compression prompted her to move to America, specifically to work at the AT \& T Bell Labs in New Jersey where she fell in love with mathematician Robert Calderbank. The two married that year and later both wound up teaching at Princeton. In 1994, the American Mathematical Society awarded her the Steele Prize for an expository paper on wavelets. In 1998, Daubechies coauthored with her husband a paper on ``Wavelet transforms that map integers to integers'' in {\it Appl. Comput. Harmon. Anal.} {\bf 5}, giving her \PMlinkname{Erd\H{o}s number}{ErdHosNumber} 3.

Daubechies has taught at Princeton since 1993.
%%%%%
%%%%%
\end{document}
