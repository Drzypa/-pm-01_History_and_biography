\documentclass[12pt]{article}
\usepackage{pmmeta}
\pmcanonicalname{MathematicsSubjectClassification}
\pmcreated{2013-03-22 16:45:16}
\pmmodified{2013-03-22 16:45:16}
\pmowner{PrimeFan}{13766}
\pmmodifier{PrimeFan}{13766}
\pmtitle{Mathematics Subject Classification}
\pmrecord{4}{38980}
\pmprivacy{1}
\pmauthor{PrimeFan}{13766}
\pmtype{Definition}
\pmcomment{trigger rebuild}
\pmclassification{msc}{01A60}
\pmclassification{msc}{01A61}
\pmclassification{msc}{01A65}

\endmetadata

% this is the default PlanetMath preamble.  as your knowledge
% of TeX increases, you will probably want to edit this, but
% it should be fine as is for beginners.

% almost certainly you want these
\usepackage{amssymb}
\usepackage{amsmath}
\usepackage{amsfonts}

% used for TeXing text within eps files
%\usepackage{psfrag}
% need this for including graphics (\includegraphics)
%\usepackage{graphicx}
% for neatly defining theorems and propositions
%\usepackage{amsthm}
% making logically defined graphics
%%%\usepackage{xypic}

% there are many more packages, add them here as you need them

% define commands here

\begin{document}
The {\em Mathematics Subject Classification} is a system of classifying mathematical papers published in peer-reviewed journals. The system was devised by the American Mathematical Society and is also used by PlanetMath to classify its content, and to a lesser extent, the mathematical content of Wikipedia.

The codes consist of a 2-digit base 10 number (zero-padded when less than 10), followed by a letter of the Roman alphabet or a dash, followed by another 2-digit base 10 number. The first two characters refer to the most general level (e.g., history, combinatorics, number theory, general topology, etc.), with the following character narrowing it down and the last two numbers being the most specific level. The letter X is used as a kind of wildcard to denote a general level that has not been narrowed down. For example, 81-XX refers to quantum theory, 81PXX refers to the foundational axioms, 81P68 refers to quantum computation and quantum cryptography.

The MSC system has been around almost as long as the AMS. The most important changes to the top-level categories began in 1962, when 09-XX was discontinued. Various other reorganizations have occurred since then, such as 04-XX, set theory, being merged into 03-XX, logic. The most recent version dates from 2000, and has room in its 01-XX, history, category for four more centuries.
%%%%%
%%%%%
\end{document}
