\documentclass[12pt]{article}
\usepackage{pmmeta}
\pmcanonicalname{JosephLiouville}
\pmcreated{2013-03-22 17:16:52}
\pmmodified{2013-03-22 17:16:52}
\pmowner{PrimeFan}{13766}
\pmmodifier{PrimeFan}{13766}
\pmtitle{Joseph Liouville}
\pmrecord{6}{39622}
\pmprivacy{1}
\pmauthor{PrimeFan}{13766}
\pmtype{Biography}
\pmcomment{trigger rebuild}
\pmclassification{msc}{01A55}
\pmrelated{DecimalExpansion}

\endmetadata

% this is the default PlanetMath preamble.  as your knowledge
% of TeX increases, you will probably want to edit this, but
% it should be fine as is for beginners.

% almost certainly you want these
\usepackage{amssymb}
\usepackage{amsmath}
\usepackage{amsfonts}

% used for TeXing text within eps files
%\usepackage{psfrag}
% need this for including graphics (\includegraphics)
%\usepackage{graphicx}
% for neatly defining theorems and propositions
%\usepackage{amsthm}
% making logically defined graphics
%%%\usepackage{xypic}

% there are many more packages, add them here as you need them

% define commands here

\begin{document}
\PMlinkescapeword{places}
\PMlinkescapeword{near}
\PMlinkescapeword{succeed}

{\em Joseph Liouville} (1809 -- 1882) was a French mathematician, editor and author who proved that $$\sum_{i = 1}^\infty \frac{1}{10^{i!}}$$ is a transcendental number (that is 0.11000100000000000000\ldots\, and sometimes called the ``basic Liouville number'').

After graduating from the \'Ecole Polytechnique, Liouville taught at several places before getting a job teaching at his old {\em alma mater}.  He was pen pals with Christian Goldbach and Daniel Bernoulli. Liouville tried to prove that the natural log base $e$ is transcendental.  Though he did not succeed in this, he did prove in 1844 that transcendental numbers exist, and gave the aforementioned Liouville number as an example. This paved the way for other mathematicians to prove the transcendality of $e$ and other important irrational constants. Liouville also introduced the Liouville function $\lambda(n) = (-1)^{\Omega(n)}$ (equal in value to the M\"obius function $\mu(n)$ when $n$ is squarefree).

Liouville wrote many papers as well as edited the writings of \'Evariste Galois and started the {\it Journal de Math\'ematiques Pures et Appliqu\'ees} that is still published today. A lunar crater near Dubyago is named after him.
%%%%%
%%%%%
\end{document}
