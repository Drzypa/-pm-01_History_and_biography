\documentclass[12pt]{article}
\usepackage{pmmeta}
\pmcanonicalname{AmericanMathematicalSociety}
\pmcreated{2013-03-22 16:42:04}
\pmmodified{2013-03-22 16:42:04}
\pmowner{PrimeFan}{13766}
\pmmodifier{PrimeFan}{13766}
\pmtitle{American Mathematical Society}
\pmrecord{5}{38915}
\pmprivacy{1}
\pmauthor{PrimeFan}{13766}
\pmtype{Definition}
\pmcomment{trigger rebuild}
\pmclassification{msc}{01A55}
\pmclassification{msc}{01A60}
\pmclassification{msc}{01A61}
\pmclassification{msc}{01A65}

% this is the default PlanetMath preamble.  as your knowledge
% of TeX increases, you will probably want to edit this, but
% it should be fine as is for beginners.

% almost certainly you want these
\usepackage{amssymb}
\usepackage{amsmath}
\usepackage{amsfonts}

% used for TeXing text within eps files
%\usepackage{psfrag}
% need this for including graphics (\includegraphics)
%\usepackage{graphicx}
% for neatly defining theorems and propositions
%\usepackage{amsthm}
% making logically defined graphics
%%%\usepackage{xypic}

% there are many more packages, add them here as you need them

% define commands here

\begin{document}
\PMlinkescapeword{support}

The {\em American Mathematical Society} (AMS) is an association of professional mathematicians in the United States. Best known as the inventor of the Mathematics Subject Classification codes used by many math journals and PlanetMath, the AMS is a staunch advocate of \TeX{} and \LaTeX{}. The AMS publishes {\it Mathematical Reviews}.

Originally called the New York Mathematical Society, the AMS was founded in 1888 by Thomas Fiske upon returning to the New York after attending a meeting of the London Mathematical Society. In 1894 the name was changed to the American Mathematical Society and later on the headquarters were moved to Providence, Rhode Island, and offices were added in Ann Arbor, Michigan, and Washington D.C.

Today, the AMS has more than 550 institutional members and almost thirty thousand individual members, and does much to support young people studying mathematics.

One of four partners of the Joint Policy Board for Mathematics, the AMS should not be confused with the Mathematical Association of America. The official website of the AMS is \PMlinkexternal{www.ams.org}{http://www.ams.org/}.
%%%%%
%%%%%
\end{document}
