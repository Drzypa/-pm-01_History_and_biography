\documentclass[12pt]{article}
\usepackage{pmmeta}
\pmcanonicalname{RMP535455}
\pmcreated{2013-03-22 18:57:25}
\pmmodified{2013-03-22 18:57:25}
\pmowner{milogardner}{13112}
\pmmodifier{milogardner}{13112}
\pmtitle{RMP 53, 54, 55}
\pmrecord{57}{41814}
\pmprivacy{1}
\pmauthor{milogardner}{13112}
\pmtype{Definition}
\pmcomment{trigger rebuild}
\pmclassification{msc}{01A16}

\endmetadata

% this is the default PlanetMath preamble.  as your knowledge
% of TeX increases, you will probably want to edit this, but
% it should be fine as is for beginners.

% almost certainly you want these
\usepackage{amssymb}
\usepackage{amsmath}
\usepackage{amsfonts}

% used for TeXing text within eps files
%\usepackage{psfrag}
% need this for including graphics (\includegraphics)
%\usepackage{graphicx}
% for neatly defining theorems and propositions
%\usepackage{amsthm}
% making logically defined graphics
%%%\usepackage{xypic}

% there are many more packages, add them here as you need them

% define commands here

\begin{document}
Early in the 21 century AD an innovative set of decoding keys update translations of hieratic math texts. The keys add-back missing initial and intermediate mathematical facts that scribal shorthand had omitted. Scholars in the 19th and 20th centuries, i.e. Peet, published additive hieratic math texts by mentioning the need for \PMlinkexternal{'ab initio'}{http://en.wikipedia.org/wiki/Ab_initio} studies.

First level decoding keys consider two sides of scribal theoretical and practical math statements of longer story lines. For example, first level keys translate RMP 53, 54 and 55 raw data in ways that expose scribal theoretical and practical geometry recorded in setats.

Second level keys add-back theoretical aspects of RMP 53, 54 and 55 setat statements that connect to 2/n tables. As background a 2008 study reported RMP and the Kahun Papyrus 2/n tables were parsed within an aliquot part method that was suggested by F. Hultsch in 1895.The theoretical approach was confirmed by E.M. Bruins in 1944. The second level view of RMP 53, 54 and 55 arithmetic fragments shows that numerators and divisors of 2/n table answers were recorded in either an additive or a subtraction context.

1. The additive context reports 2/n = (2/n)(m/m) = 2m/mn, the context that Ahmes tended to record. The primary method structured RMP 53-54-55 cubit as rational number statements. Ahmes demonstrate three conversion methods that converted rational numbers 2/3 to 2/101 to concise unit fraction series in the \PMlinkexternal{2/n table}{http://rmprectotable.blogspot.com/}. 

RMP 53, 54 and 55 also used the 2/m table scaling method per: 

a. 2/53*(30/30) = 60/1590 = (53 + 5 + 2)/1590 = 1/30 + 1/318 + 1/795

b. 2/73*(60/60) = 120/(60*73) = (73 + 20+ 15+ 12)/(60*73) = 1/60 + 1/219 + 1/292 + 1/365

When 2/n and n/p could not be scaled by one (m/m) Ahmes in RMP 31 (28/97) and RMP 35 (30/53) solved n/p by solving two separate problems 

n/p = (n -2)/p and 2/p, 

and summed the unit fractions into one series. In the case of 28/97 Ahmes solved 26/97 by (4/4) to 104/388, and 2/97 by (56/56) to 112/5432. In the case of 30/53 solved 28/53 by (2/2) to 56/106 and 2/53 by 30/30 to 60/1590.     

Middle Kingdom scribes like Ahmes may have used the subtraction context in limited ways, such as:

c. (2/43 - 1/42) = (84-43)/1806 =  (21 + 14 + 6)/1806 = 1/42 + 1/86 + 1/129 + 1/301  (2/n table)
d. (26/97 - 1/4) = (104 - 97)/388 = (97 + 4 + 2 + 1)/388 = 1/4 + 1/97 + 1/194 + 1/388 (RMP 31)

and two EMLR-like series scaled by  by m/m = 72/72 in RMP 37, an Ahmes fragment. 

2. The subtraction context was reported by Arabs, and hinted at by Ahmes. Arabs demonstrated the subtraction method that Fibonacci used to scale rational numbers to awkward unit fraction series per:

(2/n - 1/m) = (2m - n)/mn, with (2m -n) set to unity in 1202 AD as often as possible, one of three medieval rational number notations.

When unity could not be reached two-separate problems were solved. For example Fibonacci reported by Sigler's as a 7th distinction converted 4/13 by LCM 4, finding a remainder 3/52, and solved 3/53 by LCM 18 such that 

4/13 = 1/4 + 1/18 + 1/468. 

Greeks may have created the Arab notation, one of three that were reported in the Liber Abaci. It is possible that Ahmes himself used the subtraction notation from time to time. RMP 37 reports a scaling of 1/4 and 1/8 by LCM 72 to EMLR-like unit fraction series, a story-line that goes beyond the RMP 53-54-55, the issues of this PM entry.

Third level keys test the practical contents of 5MP 53, 54 and 55 against theoretical methods. In his 2/n table Ahmes' red auxiliary numbers pointed out additive numerators written within a multiplication context. In RMP 21-23 additional background facts shows that Ahmes selected LCM multipliers closer to the first method, 2/n =2/n*(m/m) = 2/mn.

Additive 20th century scholars tended to muddle hieratic math texts. Transliterations misread scribal math contents by stressing additive paradigms and minimizing potential meta scribal math issues. Finding meta scribal mathematical methods by reading texts from three different points of view birthed the three decoding keys that include hints of scribal theoretical and practical thinking, one double checking the other.

Hints of theoretical scribal math date to the 114 year old Hultsch-Bruin suggestion of aliquot parts. In the 2/43 case the aliquot parts of 42 considered divisors (42, 21, 14, 7, 6, 3, 2, and 1). Ahmes selected additive divisors in red ink from the aliquot parts of 42. When n/p could not be scaled by one m/m to mn/mp could not be found record a concise unit fraction series Ahmes used a second method. The second method replaced n/p with (n -2)/p (m1/m1) + 2/p(m2/m2). The second scaling method did not work every time. Fibonacci used a closely related two-phase conversion method in a 7th distinctions in tbat (Liber Abaci) tbat emulated Ahmes two methods,  one than one included Ahmes third method reported in RMP 36, unity (1) = 53/53 = (2/53)(30/30) + (3/53)(20/20) + (5/53)(12/12) + (15/53)(4/4) +  (28/53)(2/2). 

Conclusions: Translating  RMP 53, 54, and 55 by three keys report Ahmes' raw data included theoretical cubit,  khet units, and setat and mh areas. One substitution of one setat by two LCM multipliers, 4 and 2, facilitated the partition, the conversion, of a setat into 1/8 setat and mh units as 2/n table members were partitioned/converted.

The LCM data reported cubit and khet units written in setat areas, 100 cubit by 100 cubits. Setats were sub-divided into 1/100 setat strips, or mh units. A first reading of RMP 54 included Ahmes' implicit use of the LCM 2/n table conversion method. Ahmes scaled a setat unit to (4/4) and (2/2) before multiplying by 7/10, 14/10 and 28/10, respectively, such that:

1. (7/10)*(4/4) setat = 28/40 setat = (25 + 3)/40 setat

allowed the 2/n table LCM scaling method to confirm that

2. 5/8 setat + 300/40 mh = 5/8 setat + 7 1/2 mh.

described the scibal method and correct answer.

The first keys read RMP 55 impliy that Ahmes computed in mh units, as 5 setat times 3/5 was solved and written out.

Second level RMP 53, 54 and 55 keys consider closely related theoretical contents of the Akhmim Wooden Tablet, as reported by Hana Vymazalova in 2002. Vymazalova reported that one hekat was divided by 3, 7, 10, 11 and 13, was exactly returned to (64/64) (as Daressy had not seen in 1906). The RMP also reported the AWT method as an implicit initial substitution over 36 times. As one theoretical method, the AWT and RMP replaced a hekat by (64/64), a hekat unity, to allow the division by n, limited to 1/64 < n < 64 over 40 times.

The theoretical AWT substitution method appeared in RMP 47. Ahmes reported 100 hekat written as 6400/64, divided by 70 to compute a quotient 91/64 and a remainder 30/(70*64), scaled to a ro unit, writing (150/70)*1/320 as the answer. An intermediate step included:

[(64 + 16 + 8 + 2 + 1)/64]hekat + (2 + 1/7)*1/320, or

Ahmes' answer recorded:

[1 + 1/4 + 1/8 + 1/32 + 1/64]hekat + [2 + 1/7]ro

A second theoretical substitution method replaced one hekat with 320 ro. This method may have replaced the "awkward" 6400/64 substitution of 100 hekat method reported in RMP 47.

In RMP 38, the division of one hekat, written as 320 ro, was multiplied by 7/22, and returned by 320 ro when multiplied by 22/7. A confirmation of the RMP 38 method is provided by RMP 35. RMP 35 dividing 10 hekat of fat, written as 3200 ro, by 365. Ahmes reported a rate (8 + 280/365)ro, which was returned to 3200 ro by multiplying the answer by 365.

Third level RMP 53, 54 and 55 keys conclude that Ahmes' division and multiplication methods were inverse operations, a feature of modern multiplication and division operations.

Another view of cubit partitions is found in a fragmented text, which may have found a portion of a cubit defined by 1/3 of a palm: Given 1/14 a cubit is 1/2 a palm [aka 2 fingers]. Operate on 1/2 palm to find 1/3 palm to find 2/3 of 1/2 palm = 2/6 = 1/3 palm and find 2/3 of 1/14 = 2/42 = 1/21 cubits.

SUMMARY: Three level decoding keys update translations of RMP 53, 54, and 53 by adding-back missing theoretical facts that connect to 2/n tables and other meta scribal rules. Confirming keys follow Occam's Razor considerations, the simplest version was the historical method. The keys translate hieratic arithmetic, algebra, geometry and weights and measures methods by adding back theoretical facts that scribes omitted. The updated translations of the \PMlinkexternal{RMP}{http://ahmespapyrus.blogspot.com/2009/01/ahmes-papyrus-new-and-old.html} correct 20th century additive scholars had not directly considered theoretical aspects of 2/n tables, and geometry calculations.

\begin{thebibliography}{10}
\bibitem{1}  A.B. Chace, Bull, L, Manning, H.P., Archibald, R.C., \emph{The Rhind Mathematical Papyrus}, Mathematical Association of Amnerica, Vol I, 1927. NCTM reprints available.
\bibitem{2} Mahmoud Ezzamel, \emph{Accounting for Private Estates and the Household in the 20th Century BC Middle Kingdom}, Abacus Vol 38 pp 235-263, 2002
\bibitem{3} Milo Gardner, \emph{An Ancient Egyptian Problem and its Innovative Solution, Ganita Bharati}, MD Publications Pvt Ltd, 2006.
\bibitem{4}Richard Gillings, \emph{Mathematics in the Time of the Pharaohs}, Dover Books, 1992.
\bibitem{5} Oystein Ore, \emph{Number Theory and its History}, McGraw-Hill Books, 1948, Dover reprints available.
\bibitem{6} T.E. Peet, \emph{Arithmetic in the Middle Kingdom}, Journal Egyptian Archeology, 1923.
\bibitem{7} Tanja Pommerening, \emph{"Altagyptische Holmasse Metrologish neu Interpretiert" and relevant phramaceutical and medical knowledge, an abstract,  Phillips-Universtat, Marburg, 8-11-2004, taken from "Die Altagyptschen Hohlmass}, Buske-Verlag, 2005.
\bibitem{8} Gay Robins, and Charles Shute \emph{Rhind Mathematical Papyrus}, British Museum Press, Dover reprint, 1987.
\bibitem{9} L.E. Sigler, \emph{Fibonacci's Liber Abaci: Leonardo Pisano's Book of Calculation}, Springer, 2002.
\bibitem{10} Hana Vymazalova, \emph{The Wooden Tablets from Cairo:The Use of the Grain Unit HK3T in Ancient Egypt, Archiv Orientalai}, Charles U Prague, 2002.
\end{thebibliography}




%%%%%
%%%%%
\end{document}
