\documentclass[12pt]{article}
\usepackage{pmmeta}
\pmcanonicalname{AkhmimWoodenTablet1}
\pmcreated{2013-07-10 6:52:29}
\pmmodified{2013-07-10 6:52:29}
\pmowner{milogardner}{13112}
\pmmodifier{milogardner}{13112}
\pmtitle{Akhmim Wooden Tablet}
\pmrecord{30}{42279}
\pmprivacy{1}
\pmauthor{milogardner}{13112}
\pmtype{Definition}
\pmcomment{trigger rebuild}
\pmclassification{msc}{01A16}
\pmsynonym{hekat division}{AkhmimWoodenTablet1}
%\pmkeywords{hekat unity}
\pmdefines{Egyptian math}

\endmetadata

% this is the default PlanetMath preamble.  as your knowledge
% of TeX increases, you will probably want to edit this, but
% it should be fine as is for beginners.

% almost certainly you want these
\usepackage{amssymb}
\usepackage{amsmath}
\usepackage{amsfonts}

% used for TeXing text within eps files
%\usepackage{psfrag}
% need this for including graphics (\includegraphics)
%\usepackage{graphicx}
% for neatly defining theorems and propositions
%\usepackage{amsthm}
% making logically defined graphics
%%%\usepackage{xypic}

% there are many more packages, add them here as you need them

% define commands here

\begin{document}
\PMlinkexternal{Akhmim Wooden Tablet (AWT)}{http://akhmimwoodentablet.blogspot.com/}

INTRODUCTION

The Akhmim Wooden Tablet dates to the 12th dynasty, circa 1950 BCE. The wooden tablet is 46.5 x 26 cm and is housed in the main Cairo Museum. The text mentions 27 servant names, an unknown king's name (citing the 8th year of his reign), and five binary multiplications of a hekat unity (a volume unit) by 1/3, 1/7, 1/11, 1/11, 1/13 and by implication 1/n.  One [Q/64 + 5R/n) calculation 1/3 was repeated four times. The remaining four 1/n calculations that scaled (64/64) were repeated at least once. Proofs returned the five two-part [Q/64 + (5R/n)ro] answers to the initial (64/64) scaling parameter by multiplying each answer by the initial 3, 7, 10, 11 or 13 integer.

The document was reported in 1901 b y Georges Daressy. Daressy analyzed and published in 1906 by indicating five multiplications of a hekat by 1/3, 1/7, 1/10, 1/11 and 1/13. Unit fraction answers were recorded in binary quotients and 1/320 of a hekat (ro) remainders. Daressy validated exact scribal proofs for 1/3, 1/7 and 1/10, but failed to prove exactness for the 1/11 and 1/13 cases.  

Daressy discussed scribal typos and scribal errors. Daressy muddled the 1/11 and 1/13 multiplications and proofs in acubit context though suspected all five multiplications were intended to be exact.  Daressy's data stressed cubits rather than hekats a concern for Peet (1923). Daressy's views were consistent with an unscaled binary fraction remainder. Peet did identify the ro remainder aspect of the AWT but not in the precise scribal scaled context. Daressy's approach analyzed AWT binary fractions and scaled remainders as single statements a point of view that is accepted today.

Gillings in 1972, and others, published 1/10th of a hekat (a volume unit) named hin, hinu, or henu scaled a linear cubit to a cubit-cubit-cubit. 

Exactness of all five scribal answers and proofs were published in 2001 by Hana Vymazalova, a Charles U., Prague, graduate student within hekat unity (64/64) context.

Gillings, Peet, Chace and other 20th century schgolars had not mentioned a possible hekat unity (64/64) multiplied by 1/10 such that 1/320 of a hekat became a ro remainder by these steps, facts implied (but not cited by Vymazalova):

(64/64)hekat times (1/10) = (6/64 + 4/640)hekat =

(4 + 2)/64 hekat + 20/10 ro =

(1/16 + 1/32)hekat + 2ro

Note that the (4/640) hekat step became a ro unit by multiplying (4/640) by (5/5) = 20/10 ro

That is, 1/10 of a hekat was equivalent to 1 hin. The 1/10 of a hekat fact was discussed by Ahmes in 1650 BCE in RMP 81.

Ahmes, in RMP 81 converted 29 binary hekat quotients + scaled ro remainders to hin units that followed the AWT's theoretical statement:

(64/64) times 1/n = Q/64 + (5R/n)ro

with Q = Quotient, R = Remainder, and

divisor n was limited to the range 1/64 < n < 64. 

The RMP included one part hekat statements, a point stressed by Peet. For example: 10/n hin meant a hekat was scaled to a 1/10 unit. Peet incompletely discussed the AWT's binary fractions statements as a one part statement, which it surely was not. Peet reported 1/3, 1/7, 1/10, 1/11, 1/13 multiplication aspects of the problems and stressed the 1/320 ro unit by discussing one part statements. 

Peet did not report the (64/64) hekat unity that allowed five binary quotient plus a 1/320 remainder answers and five proofs to be recorded. Peet under reported the scribal context. 

Updated translations of the AWT published in 2006 and 2011 correct Daressy' 1/11 and 1/11 cases as division problems by complete beginning, middle and ending hekat statements that merge into 1/320 remainder answers: 

a. (64/64)/11 hekat = (5/64 + 9/704)hekat = (4 + 1)/64 hekat + (45/11)ro = (1/16 + 1/64)hekat + (4 + 1/11)ro 

b. (64/64)/13 hekat = (4/64 + 12/832)hekat = 4/64 hekat + (60/13)ro = 1/16 hekat + (4 + 8/13)ro 
with 8/13 scaled by LCM 2 to 16/26 = 

(13 + 2 + 1)/26 = 1/2 + 1/13 + 1/26 recorded as: (64/64)/13 hekat = 1/16 hekat + (4 + 1/2 + 1/13 + 1/26)ro. 

It took 95 years for Vymazalova to correctly report the proof context of Daressy's 1906 AWT hekat partitioning story. Hana Vymazalova reported the proof side of the five two-part statements that returned (64/64) five times. Vymazalova did not challenge Peet's view of the five AWT binary quotient and 1/320 remainder. Vymazalova only published scribal duplation calculations without mentioning how and where the (64/64) hekat unity was first introduced by the scribe.

Daressy's 1906 review of the AWT's data surely garbled the 1/11 and 1/13 proofs by not identifying the (64/64) aspect of the text. In 2002 Hana Vymazalova corrected Daressy's 1/11 and 1/13 oversights by pointing out that (64/64) was returned five times in proofs. Vymazalova's two corrections were updated and published in 2006 and 2011 that showed a larger scribal context,  that all five initial calculations began with the hekat unity (64/64).

The AWT reported a well-defined system of weights and measures arithmetic from an Old Kingdom inexact system to an exact Middle Kingdom rational numberscaled by LCMs system.

The meta context in which Peet properly identified the 1/320 ro aspect of the AWT was missed until Hana Vyamazalova's 2002 paper. Scholars for almost 100 years did not connect the AWT partition method to RMP 81 and 29 data points, and over 30 additional two-part partitions of a (64/64) hekat unity discussed in the RMP.


CONTENTS OF THE AWT

Translated to our modern base 10, the AWT simply states that unity (64/64th) was divided by 3, 7, 10, 11 and 13 following a general rule of division, as is clearly read by:

(64/64)/n = Q/64 + (5R/n)ro

with Q = quotient, and R = Remainder

writing the initial problem in modern base 10 fractions.

Middle Kingdom scribes used this pattern, by easily writing the quotient term into a Horus-Eye series, for example (64/64)/3 = 21/64 (Q). Several scholars have seen this portion, but become foggy with respect to the remainder, one (1) in the case of n = 3.

The 2nd portion, the handling of the remainder, has been muddled and misreported by scholars. One reason can be excused since R/64n) was 'encoded' by scribes that replaced 1/64 of a hekat with equivalent 1/320 of a hekat remainders in an unusual style. The style allowed student scribes to create quotient (Q) and  scaled reminders (R) as Q/64 + (5R/n)ro answers as one number.

Scholars did not see the scaled aspect of the (5R/n)ro remainders, nor the vivid two-part number aspect of the notation that consistently reported 

(64/64) times 1/n = [Q/64 + (5R/n)ro]

A few scholars openly pondered scribal issues when reporting the AWT and RMP two-part quotient and remainder data. Two 1987 scholars, Robins-Shute reported 100 hekat times 1/70 (one of ten RMP 47 problems) without commenting on the two-part [Q/64 + (5R/70)ro] notation in a  RMP book. Robins-Shute fairly reported products and remainders contained in the scribal intermediate data, a form of scratching of heads that did not report the beginning of the problem (6400/64) times 1/70, nor the final form of the answer [(91/64)hekat + (150/70)ro]. 

Robins-Shute reported a personalized version of (64 + 16 + 8 + 2 + 1)/64 + (150/70)ro answer by reporting 

[(1 + 1/4 + 1/8 + 1/32 + 1/64)hekat + (2 + 1/7)ro.

Ahmes, for fun reported the remainder as [2 + (1/7)(6/6)]ro = (2 + (6/42))ro = 2 + (3 + 2 + 1)/42 = (2 + 1/14 + 1/21 + 1/42)ro, an answer that the EMLR scribe included on line 14 300 years earlier. Robins-Shute saw the correct scaled ro remainder but wished to report the simpler (2 + 1/7)ro version for a personal reason.       

Chace, writing in a classic 1927 RMP book reported RMP 83 that discussed three related data sets. Today it is clear that (64/64) was divided by n = 6, 20, and 40. The bird-feeding rates made little sense to Chace, suggesting that Ahmes had garbled the data. Chace mentioned that Ahmes left no clues to on this matter, a personal point that was obviously incorrect, when the AWT is considered. All Middle Kingdom scribes used the methods. The Kahun Papyrus and Moscow Mathematical Papyrus also used the notation.

Middle Kingdom scribes created and applied

[Q/64 +(5R/n)ro] statements, over 60 times that followed the same theoretical notation.

For none mathematicians, it may be best to refer to a modern base 10 version of the 4,000 year old arithmetic notation (shorthand), and ponder the information for few days. After sleeping on the data, attempt to write in the ancient 2-part notation. That is, think in modern base 10 for a few days, before trying to think and write as a 4,000 year old scribe.

There are four trees being discussed here, each suggesting confusion, unless care is taken to locate the forest of formulas at work. The tree are: (1) the Horus-Eye notation, followed by a hekat, written in the first half of the expression, (2) Hieratic Egyptian fraction notion, followed by ro in the second half of the expression, (3) the Egyptian fraction series represented only the rational number (5*R/n), noting n in the denominator, allowed the remainder to grow to any size, thereby allowed an exact computation, and (4) the word ro, as 1/320th was factored from the remainder term.

Clearly ro was an element in a larger expression. Ro was a common divisor that was used to add the Horus-Eye and Egyptian fraction series together, that also allowed a proof to be quickly recorded. Five proofs were cited in the AWT. Each proof returned (64/64), meant that (64/64) was the initial term in the problem.

It is recommended that novice readers allow one or two of the four trees that you meet to be confused. Clear up the confusion by looking for a formula that define a forest that includes all four trees. Then and only read and work with the AWT data. 

The ancient scribal notation began with (64/64) times 1/n problems. Each answer was proved by multiplying two-part answers by n (inverse of 1/n, thereby defining our modern invert and multiply division rule).  Proofs ended by returning (64/64), after which a Q.E.D. could have been written (has a Latin scribe completed the task).

Have patience. Each of the four ideas (trees) are simple. Summing them up to one ancient tool may cause problems for certain readers. Please avoid shortcuts, especially the ones that Ahmes himself practiced. Only practice Ahmes' shortcuts when you have discovered the formula hidden in this class of Middle Kingdom arithmetic.

Peet, Gillings and other scholars tended not to take the needed time. Scholars often jumped into reading scribal shortcuts without understanding certain hidden formulas, in this case (64/64)/n = [Q/64 + (5R/n)ro].

Begin at the beginning of each of the five AWT problems, as any scribal problem must be parsed. Compare the beginning, middle and ending of each problem. For example to obtain 45 additional AWT examples, consider RMP 47, 81, and 83. Work each of the 45 problems by following any of the five AWT patterns. Do your own work, every step of the way.

You will be rewarded. Spend the necessary time to work through a dozen or more of this class of problem. You will be rewarded with a scribal skill that serious scholars in the 1920s, Robins-Shute (1987) and Clagett (1999) did not understand.

BACKGROUND

Thomas E. Peet, 1923, partially repeated an analysis of the AWT by showing several connections between Egyptian math and the practical experiences of an ancient Egyptian scribe that used two numeration systems, Horus-Eye and the Egyptian fractions (cited in the RMP). Peet muddled where one system ended and where the other system began by only detailing additive aspects of the two numeration systems, missing the exact Egyptian division features using 5ro as a partitioning idea, using numerators and denominators = 320 in an interesting way. Peet also prematurely concluded that the Egyptian division was only an inverse of the Egyptian multiplication operation.

Peet did not directly discuss Egyptian division, as a general operation, as confirmed by the AWT examples. However, contrary to Gillings and Robins-Shute, Peet did seem to compute with 5ro, 4ro, 3ro, 2 ro and ro, but only from a limited view of the AWT student. Peet was slightly myopic, asking few meta questions, such as: were all of the student's divisions required to be exact? More importantly, no comparisons of Peet's view of the AWT were made to the RMP and its 84 problems. At least ten RMP problems, 36-43, and 81-82, have been misread with respect to ro, suggesting it was a weights and measures unit. Ro was actually connected to a generalized partitioning role, as closely related to other exact partitioning methods cited in the RMP, and other Middle Kingdom mathematical texts.

Scholars are  free to explore Egyptian mathematics that meet a range of standards. One of the Egyptian math topics is hekat division standard reported in the AWT.

Peet, Gillings, 1972, and Robin-Shute, 1987 show that all the three scholars prematurely concluded in independent analyses that MK ro data (from the RMP and AWT) only meant 1/320 of a hekat, and no more. Not one of the three scholars grasped Ahmes central fact, that quotients and scaled ro remainders defined a weights and measures unit by beginning with a hekat unity (64/64), and dividing by any rational number less than 64.

Returning to Peet, and his analysis will be shown in the next few paragraphs. He apparently made serious errors with respect to ro and its relationship to Egyptian division, as was vividly declared in the AWT hekat and 1/64 divided by 1/n context. Peet did not see ro's actual association with remainders, though he mentioned remainders from time to time. It is clear that 64 times 5, an early form of mod 5 in the R/3 term, included the use of numerators and denominators, or, by example, let the divisor of 64/64 be n, then

Q + R/3

appeared in two shorthand forms, the first being

(1) (64/64)/n = Q/64 + R/(64n).

(2) (64/64)/n = Q/64 + (5*R/n)* 1/320, with ro = 1/320

Ahmes used the second form. The first form is the manner in which modern mathematicians read this type of information.

One of the most exciting aspects of the AWT is that the rational number (5*R/n) was easily converted to an Egyptian fraction series. It not known if this was the
first generalized used of Egyptian fractions.

Silverman's 1975 point is plausible. Egyptian fractions were found in the Old Kingdom. Silverman's proposal may mean that the balance beam problem was solved by an Old Kingdom scribe, noting:

R/(64n)

with the Egyptian fractions series being either R/n or R/(64n).

Thanks to Middle Kingdom scribes for leaving red flags raised by the AWT, RMP, MMP and the KP so that profound hekat division problems are now understood. 

In the modern era, Daressy in 1906 incompletely discussed the AWT Egyptian fraction series data. Daressy saw exact divisions which seemed to be in cubit units. In 2002 Vymazalova added binary fractions and a scaled remainder that summed to (64/64). Today is is clear that the AWT method corrected and Old Kingdom cubit-cubit-cubit recorded as a hekat division method, that was proven by the AWT scribe by returning binary quotient and scaled remainder answers to its initial numerical values, (64/64).

\begin{thebibliography}{8}
\bibitem{1} Mahmoud Ezzamel, \emph{Accounting for Private Estates and the Household in the 20th Century BC Middle Kingdom}, Abacus Vol 38 pp 235-263, 2002
\bibitem{2} Milo Gardner, \emph{The Egyptian Mathematical Leather Roll Attested Short Term and Long Term, History of Mathematical Sciences}, Hindustan Book Company, 2002.
\bibitem{3} Milo Gardner, \emph{An Ancient Egyptian Problem and its Innovative Solution, Ganita Bharati}, MD Publications Pvt Ltd, 2006.
\bibitem{4}Richard Gillings, \emph{Mathematics in the Time of the Pharaohs}, Dover Books, 1992.
\bibitem{5} T.E. Peet, \emph{Arithmetic in the Middle Kingdom}, Journal Egyptian Archeology, 1923.
\bibitem{6} Tanja Pommerening, \emph{"Altagyptische Holmasse Metrologish neu Interpretiert" and relevant phramaceutical and medical knowledge, an abstract,  Phillips-Universtat, Marburg, 8-11-2004, taken from "Die Altagyptschen Hohlmass}, Buske-Verlag, 2005.
\bibitem{7} L.E. Sigler, \emph{Fibonacci's Liber Abaci: Leonardo Pisano's Book of Calculation}, Springer, 2002.
\bibitem{8} Hana Vymazalova, \emph{The Wooden Tablets from Cairo:The Use of the Grain Unit HK3T in Ancient Egypt, Archiv Orientalai}, Charles U Prague, 2002.

\end{thebibliography}

%%%%%
%%%%%
\end{document}
