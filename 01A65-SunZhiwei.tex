\documentclass[12pt]{article}
\usepackage{pmmeta}
\pmcanonicalname{SunZhiwei}
\pmcreated{2013-03-22 18:04:12}
\pmmodified{2013-03-22 18:04:12}
\pmowner{Mravinci}{12996}
\pmmodifier{Mravinci}{12996}
\pmtitle{Sun Zhiwei}
\pmrecord{5}{40603}
\pmprivacy{1}
\pmauthor{Mravinci}{12996}
\pmtype{Biography}
\pmcomment{trigger rebuild}
\pmclassification{msc}{01A65}
\pmclassification{msc}{01A61}
\pmclassification{msc}{01A60}
\pmsynonym{Chihwei Sun}{SunZhiwei}
\pmsynonym{Sun Chihwei}{SunZhiwei}
\pmsynonym{Sun Zhi-Wei}{SunZhiwei}
\pmsynonym{Zhi-Wei Sun}{SunZhiwei}

% this is the default PlanetMath preamble.  as your knowledge
% of TeX increases, you will probably want to edit this, but
% it should be fine as is for beginners.

% almost certainly you want these
\usepackage{amssymb}
\usepackage{amsmath}
\usepackage{amsfonts}

% used for TeXing text within eps files
%\usepackage{psfrag}
% need this for including graphics (\includegraphics)
%\usepackage{graphicx}
% for neatly defining theorems and propositions
%\usepackage{amsthm}
% making logically defined graphics
%%%\usepackage{xypic}

% there are many more packages, add them here as you need them

% define commands here

\begin{document}
\PMlinkescapeword{degrees}
\PMlinkescapeword{Ph}

\emph{Sun Zhiwei} (or \emph{Chihwei Sun}, depending on romanization, generally given as \emph{Zhi-Wei Sun} in journals in English) (born 1965) Chinese mathematician and educator, Editor-in-Chief of the {\it International Journal of Modern Mathematics}.

Born in Lianshui, China, Sun earned degrees in mathematics at Nanjing University and started teaching number theory, combinatorics, logic and set theory soon after earning his Ph.D. in 1992. The next year he joined the American Mathematical Society. Since 1999, he supervises Ph.D. students at Nanjing.

Zhi-Wei's twin brother Zhihong is also interested in mathematics; the two of them with Donald Dines Wall searched for counterexamples to Fermat's last theorem prior to Andrew Wiles proving it in 1994. More recently, Zhi-Wei Sun has been researching sums of a prime number to figurate numbers.

In 1996, Sun co-authored with Andrew Granville a paper on ``Values of Bernoulli polynomials'' for the {\it Pacific Journal of Mathematics}. Years before, Granville had co-authored with Erd\H{o}s a paper ``On the normal behavior of the iterates of some arithmetic functions'' in {\it Progress in Mathematics}, giving Sun \PMlinkname{Erd\H{o}s number}{ErdHosNumber} 2.
%%%%%
%%%%%
\end{document}
