\documentclass[12pt]{article}
\usepackage{pmmeta}
\pmcanonicalname{PietroMengoli}
\pmcreated{2013-03-22 18:05:11}
\pmmodified{2013-03-22 18:05:11}
\pmowner{Mravinci}{12996}
\pmmodifier{Mravinci}{12996}
\pmtitle{Pietro Mengoli}
\pmrecord{4}{40624}
\pmprivacy{1}
\pmauthor{Mravinci}{12996}
\pmtype{Biography}
\pmcomment{trigger rebuild}
\pmclassification{msc}{01A45}
\pmclassification{msc}{01A50}

\endmetadata

% this is the default PlanetMath preamble.  as your knowledge
% of TeX increases, you will probably want to edit this, but
% it should be fine as is for beginners.

% almost certainly you want these
\usepackage{amssymb}
\usepackage{amsmath}
\usepackage{amsfonts}

% used for TeXing text within eps files
%\usepackage{psfrag}
% need this for including graphics (\includegraphics)
%\usepackage{graphicx}
% for neatly defining theorems and propositions
%\usepackage{amsthm}
% making logically defined graphics
%%%\usepackage{xypic}

% there are many more packages, add them here as you need them

% define commands here

\begin{document}
\emph{Pietro Mengoli} (1625 or 1626 - 1686) Italian mathematician, lawyer, priest, author and educator.

Baptized in the Roman Catholic faith, Mengoli was ordained a priest in 1660 and said Mass in Bologna until his death. Taught by Bonaventura Cavalieri, Mengoli succeeded him as chair of the University of Bologna. In 1644 he posed the Basel problem, which remained unsolved until Euler tackled it in the next century, unaware when he confirmed the correctness of the Wallis formulae that $\pi$ was also involved in it, nor did his successful work on the sums of the reciprocals of the triangular numbers help him solve the problem.

Unlike Pietro Cataldi, Pietro Mengoli preferred to teach and write in Latin rather than in a modern language. Besides mathematical topics, Mengoli also wrote on atmospheric refraction and musical harmony.
%%%%%
%%%%%
\end{document}
