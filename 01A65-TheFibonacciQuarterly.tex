\documentclass[12pt]{article}
\usepackage{pmmeta}
\pmcanonicalname{TheFibonacciQuarterly}
\pmcreated{2013-03-22 17:21:32}
\pmmodified{2013-03-22 17:21:32}
\pmowner{PrimeFan}{13766}
\pmmodifier{PrimeFan}{13766}
\pmtitle{The Fibonacci Quarterly}
\pmrecord{5}{39718}
\pmprivacy{1}
\pmauthor{PrimeFan}{13766}
\pmtype{Definition}
\pmcomment{trigger rebuild}
\pmclassification{msc}{01A65}
\pmclassification{msc}{01A61}
\pmclassification{msc}{01A60}
\pmsynonym{Fibonacci Quarterly}{TheFibonacciQuarterly}

\endmetadata

% this is the default PlanetMath preamble.  as your knowledge
% of TeX increases, you will probably want to edit this, but
% it should be fine as is for beginners.

% almost certainly you want these
\usepackage{amssymb}
\usepackage{amsmath}
\usepackage{amsfonts}

% used for TeXing text within eps files
%\usepackage{psfrag}
% need this for including graphics (\includegraphics)
%\usepackage{graphicx}
% for neatly defining theorems and propositions
%\usepackage{amsthm}
% making logically defined graphics
%%%\usepackage{xypic}

% there are many more packages, add them here as you need them

% define commands here

\begin{document}
{\em The Fibonacci Quarterly} (sometimes {\em Fib. Quart.} in bibliographies) is the official publication of the Fibonacci Association, intended ``to serve as a focal point for interest in Fibonacci numbers and related questions, especially with respect to new results, research proposals, challenging problems, and innovative proofs of old ideas.'' Published since 1962, it is available in hard copy from most libraries as well as online.

The journal has its own editorial board, but the board of directors of the Fibonacci Association also helps. Additionally, there is a rotating group of additional referees consisting of ``mathematicians, engineers and physicists''; for example, the referees for {\bf 42} included Ron Graham, Jud McCranie, Carl Pomerance, and Samuel Wagstaff Jr, among a couple dozen others.

Starting with the February 2004 issue, the journal invites ``well known mathematicians and scientists'' to write articles  on how they ``use Fibonacci numbers, or similar recurrence sequences, in their research and writing.'' The first such article was by George Andrews on the Fibonacci numbers and the Rogers-Ramanujan identities. Articles in the journal touch on various topics of different obviousness of relation to the Fibonacci numbers, such as the Fibonacci fractions, the Collatz sequence, cryptography, etc.

\PMlinkexternal{Fibonacci Quarterly Homepage}{http://www.engineering.sdstate.edu/~fib/}

%%%%%
%%%%%
\end{document}
