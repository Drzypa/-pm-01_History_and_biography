\documentclass[12pt]{article}
\usepackage{pmmeta}
\pmcanonicalname{SofiaKovalevskaya}
\pmcreated{2013-03-22 17:05:55}
\pmmodified{2013-03-22 17:05:55}
\pmowner{Mravinci}{12996}
\pmmodifier{Mravinci}{12996}
\pmtitle{Sofia Kovalevskaya}
\pmrecord{5}{39394}
\pmprivacy{1}
\pmauthor{Mravinci}{12996}
\pmtype{Biography}
\pmcomment{trigger rebuild}
\pmclassification{msc}{01A55}
\pmsynonym{Sofia Vasilyevna Kovalevskaya}{SofiaKovalevskaya}
\pmsynonym{Sofia Vasilevna Kovalevskaia}{SofiaKovalevskaya}
\pmsynonym{Sofia Vasilevna Korvin-Krukovskaya}{SofiaKovalevskaya}

\endmetadata

% this is the default PlanetMath preamble.  as your knowledge
% of TeX increases, you will probably want to edit this, but
% it should be fine as is for beginners.

% almost certainly you want these
\usepackage{amssymb}
\usepackage{amsmath}
\usepackage{amsfonts}

% used for TeXing text within eps files
%\usepackage{psfrag}
% need this for including graphics (\includegraphics)
%\usepackage{graphicx}
% for neatly defining theorems and propositions
%\usepackage{amsthm}
% making logically defined graphics
%%%\usepackage{xypic}

% there are many more packages, add them here as you need them

% define commands here

\begin{document}
\PMlinkescapeword{children}
\PMlinkescapeword{degree}
\PMlinkescapeword{degrees}

\emph{Sofia Vasilyevna Kovalevskaya} (sometimes transliterated \emph{Sofia Vasilevna Kovalevskaia}) n\'ee \emph{Sofia Vasilevna Korvin-Krukovskaya} (1850 - 1891) Russian mathematician, women's \PMlinkescapetext{rights} activist, novelist and playwright.

Born in czarist Russia of an artillery officer and a German scholar, young Sofia was the second of three children. Her father nurtured her interest in mathematics and hired Strannoliubskii to tutor her in calculus. However, at the time she could not get an university degree except by going to Europe with the permission of her father or her husband. Thus, she entered a marriage of convenience with paleontologist Vladimir Kovalevskii, who gave her permission to go to Germany to study at the universities in Heidelberg and Berlin. Whereas Heidelberg allowed her to study as long as the professors involved approved, at Berlin she was not allowed to study at all in the university. However, private tutoring with Karl Weierstrass was very formative. For G\"ottingen University Kovalevskaya prepared three different doctoral dissertations before settling on a fourth one that earned her a degree summa cum laude, making her the first woman in Europe to earn a doctorate in mathematics. The return of the Kovalevskys to Russia was futile, no university would hire them with their European degrees. Returning to Germany, they consummated their marriage leading to the birth of a daughter, Sofia ``Fufa.'' When the girl turned one year old, Kovalevskaya resumed her work in mathematics. After Kovalevsky's suicide in 1883, Kovalevskaya was offered a position at Stockholm. The next year she was on tenure-track and began editing {\it Acta Mathematica}. She had never been offered a professorship in Russia, but had received other honors from her homeland when she died of pneumonia at forty-one.

\begin{thebibliography}{1}
\bibitem{ak} A. H. Koblitz ``Sofia Vasilevna Kovalevskaia'' in {\it Women of Mathematics: A Bibliographic Sourcebook} L. Grinstein, P. Cambpell, ed.s New York: Greenwood Press (1987): 103 - 113
\end{thebibliography}
%%%%%
%%%%%
\end{document}
