\documentclass[12pt]{article}
\usepackage{pmmeta}
\pmcanonicalname{GiuseppePeano}
\pmcreated{2013-03-22 16:32:35}
\pmmodified{2013-03-22 16:32:35}
\pmowner{PrimeFan}{13766}
\pmmodifier{PrimeFan}{13766}
\pmtitle{Giuseppe Peano}
\pmrecord{4}{38725}
\pmprivacy{1}
\pmauthor{PrimeFan}{13766}
\pmtype{Biography}
\pmcomment{trigger rebuild}
\pmclassification{msc}{01A55}
\pmclassification{msc}{01A60}

% this is the default PlanetMath preamble.  as your knowledge
% of TeX increases, you will probably want to edit this, but
% it should be fine as is for beginners.

% almost certainly you want these
\usepackage{amssymb}
\usepackage{amsmath}
\usepackage{amsfonts}

% used for TeXing text within eps files
%\usepackage{psfrag}
% need this for including graphics (\includegraphics)
%\usepackage{graphicx}
% for neatly defining theorems and propositions
%\usepackage{amsthm}
% making logically defined graphics
%%%\usepackage{xypic}

% there are many more packages, add them here as you need them

% define commands here

\begin{document}
\emph{Giuseppe Peano} (1858 - 1932) Italian mathematician, linguist and author, perhaps best known for the Peano curve.

Born in Piedmont, Peano studied at the university in Turin. After graduating, Peano ghostwrote a calculus textbook for Angelo Genocchi. In his first book published under his own name, Peano introduced our modern symbols for set union and set intersection. Work on a {\it Formulario}, a comprehensive encyclopedia of mathematical formulas, led him to invent a new language in which he sought to do for Latin what Ludwig Zamenhof was trying to do with Esperanto. The final edition of {\it Formulario} was written in the language he invented.
%%%%%
%%%%%
\end{document}
