\documentclass[12pt]{article}
\usepackage{pmmeta}
\pmcanonicalname{Hypatia}
\pmcreated{2013-03-22 17:00:47}
\pmmodified{2013-03-22 17:00:47}
\pmowner{Mravinci}{12996}
\pmmodifier{Mravinci}{12996}
\pmtitle{Hypatia}
\pmrecord{5}{39295}
\pmprivacy{1}
\pmauthor{Mravinci}{12996}
\pmtype{Biography}
\pmcomment{trigger rebuild}
\pmclassification{msc}{01A20}
\pmclassification{msc}{01A16}
\pmsynonym{Hypatia of Alexandria}{Hypatia}

\endmetadata

% this is the default PlanetMath preamble.  as your knowledge
% of TeX increases, you will probably want to edit this, but
% it should be fine as is for beginners.

% almost certainly you want these
\usepackage{amssymb}
\usepackage{amsmath}
\usepackage{amsfonts}

% used for TeXing text within eps files
%\usepackage{psfrag}
% need this for including graphics (\includegraphics)
%\usepackage{graphicx}
% for neatly defining theorems and propositions
%\usepackage{amsthm}
% making logically defined graphics
%%%\usepackage{xypic}

% there are many more packages, add them here as you need them

% define commands here

\begin{document}
\emph{Hypatia} of Alexandria (350 - 415) Greek mathematician and educator.

The daughter of Theon of Alexandria, Hypatia was allowed to study subjects most contemporary women were not allowed to. Theon wrote commentaries on the works of Diophantus, Ptolemy, and Apollonius. Some historians believe some of these may have been written by Hypatia, but there has been no proof of this. Hypatia was murdered in an intrigue involving a skirmish between Jews and Christians. The one woman in Raphael's painting {\it The School of Athens} is said to be Hypatia.

\begin{thebibliography}{1}
\bibitem{eh} I. Mueller ``Hypatia'' in {\it Women of Mathematics: A Bibliographic Sourcebook} L. Grinstein, P. Cambpell, ed.s New York: Greenwood Press (1987): 74 - 79
\end{thebibliography}
%%%%%
%%%%%
\end{document}
