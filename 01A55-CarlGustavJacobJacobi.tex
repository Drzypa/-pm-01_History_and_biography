\documentclass[12pt]{article}
\usepackage{pmmeta}
\pmcanonicalname{CarlGustavJacobJacobi}
\pmcreated{2013-03-22 16:49:51}
\pmmodified{2013-03-22 16:49:51}
\pmowner{PrimeFan}{13766}
\pmmodifier{PrimeFan}{13766}
\pmtitle{Carl Gustav Jacob Jacobi}
\pmrecord{6}{39071}
\pmprivacy{1}
\pmauthor{PrimeFan}{13766}
\pmtype{Biography}
\pmcomment{trigger rebuild}
\pmclassification{msc}{01A55}
\pmsynonym{Jacques Simon Jacobi}{CarlGustavJacobJacobi}

\endmetadata

% this is the default PlanetMath preamble.  as your knowledge
% of TeX increases, you will probably want to edit this, but
% it should be fine as is for beginners.

% almost certainly you want these
\usepackage{amssymb}
\usepackage{amsmath}
\usepackage{amsfonts}

% used for TeXing text within eps files
%\usepackage{psfrag}
% need this for including graphics (\includegraphics)
%\usepackage{graphicx}
% for neatly defining theorems and propositions
%\usepackage{amsthm}
% making logically defined graphics
%%%\usepackage{xypic}

% there are many more packages, add them here as you need them

% define commands here

\begin{document}
\emph{Carl Gustav Jacob Jacobi} (1804 - 1851) Jewish-German mathematician best known for the Jacobian matrix and the Jacobi symbol.

The second son of a successful banker, young Carl was home-schooled until the age of 12, when he entered the Potsdam high school. Four years after earning a degree from Berlin University in 1825 and converting to Catholicism, Jacobi became a mathematics professor there and taught for a dozen years. He read Greek and Latin fluently, and was quite familiar with the work of \PMlinkname{Leonhard Euler}{EulerLeonhard}. In the late 1820s, Jacobi did significant work on elliptic functions in relation to fractions, attracting the interest and praise of \PMlinkname{Carl Friedrich Gauss}{CarlFriedrichGauss} and Adrien-Marie Legendre. In 1843, Jacobi went on a vacation to Italy, beginning his retirement.

A lunar crater is named after Jacobi.
%%%%%
%%%%%
\end{document}
