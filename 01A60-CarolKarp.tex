\documentclass[12pt]{article}
\usepackage{pmmeta}
\pmcanonicalname{CarolKarp}
\pmcreated{2013-03-22 17:05:31}
\pmmodified{2013-03-22 17:05:31}
\pmowner{Mravinci}{12996}
\pmmodifier{Mravinci}{12996}
\pmtitle{Carol Karp}
\pmrecord{5}{39386}
\pmprivacy{1}
\pmauthor{Mravinci}{12996}
\pmtype{Biography}
\pmcomment{trigger rebuild}
\pmclassification{msc}{01A60}
\pmsynonym{Carol Ruth Vander Velde}{CarolKarp}

\endmetadata

% this is the default PlanetMath preamble.  as your knowledge
% of TeX increases, you will probably want to edit this, but
% it should be fine as is for beginners.

% almost certainly you want these
\usepackage{amssymb}
\usepackage{amsmath}
\usepackage{amsfonts}

% used for TeXing text within eps files
%\usepackage{psfrag}
% need this for including graphics (\includegraphics)
%\usepackage{graphicx}
% for neatly defining theorems and propositions
%\usepackage{amsthm}
% making logically defined graphics
%%%\usepackage{xypic}

% there are many more packages, add them here as you need them

% define commands here

\begin{document}
\emph{Carol Karp} n\'ee \emph{Carol Ruth Vander Velde} (1926 - 1972) American mathematician of Dutch ancestry. Best known for her work on infinitary logic, she also played viola in an all-women orchestra.

Born in Michigan to a farming supply store manager and a housewife, Carol and her siblings graduated from high school in Ohio. After that, she graduated from Manchester College and went back to Michigan to study at Michigan~State~University (then called Michigan State College). In 1951 she married Arthur Karp and took his last name. Carol Karp earned a \PMlinkescapetext{Ph}.D. in California while teaching in New Mexico. At the University of Maryland she helped enlarge the mathematical logic department in professors and graduate students, and obtained a National Science Foundation grant for undergraduate research. In 1969 she was diagnosed with breast cancer but remained active until her death three years later.

\begin{thebibliography}{1}
\bibitem{jg} J. Green ``Carol Karp'' in {\it Women of Mathematics: A Bibliographic Sourcebook} L. Grinstein, P. Cambpell, ed.s New York: Greenwood Press (1987): 86 - 91
\end{thebibliography}
%%%%%
%%%%%
\end{document}
