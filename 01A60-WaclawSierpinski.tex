\documentclass[12pt]{article}
\usepackage{pmmeta}
\pmcanonicalname{WaclawSierpinski}
\pmcreated{2013-03-22 17:03:45}
\pmmodified{2013-03-22 17:03:45}
\pmowner{PrimeFan}{13766}
\pmmodifier{PrimeFan}{13766}
\pmtitle{Wac\l{}aw Sierpi\'nski}
\pmrecord{8}{39354}
\pmprivacy{1}
\pmauthor{PrimeFan}{13766}
\pmtype{Biography}
\pmcomment{trigger rebuild}
\pmclassification{msc}{01A60}
\pmclassification{msc}{01A55}
\pmsynonym{Wac\l{}aw Franciszek Sierpi\'nski}{WaclawSierpinski}
\pmsynonym{Waclaw Sierpinski}{WaclawSierpinski}
\pmsynonym{Vaclav Sierpinski}{WaclawSierpinski}
\pmsynonym{Wac\l{}aw Franti\v{s}ek Sierpi\'nski}{WaclawSierpinski}

% this is the default PlanetMath preamble.  as your knowledge
% of TeX increases, you will probably want to edit this, but
% it should be fine as is for beginners.

% almost certainly you want these
\usepackage{amssymb}
\usepackage{amsmath}
\usepackage{amsfonts}

% used for TeXing text within eps files
%\usepackage{psfrag}
% need this for including graphics (\includegraphics)
%\usepackage{graphicx}
% for neatly defining theorems and propositions
%\usepackage{amsthm}
% making logically defined graphics
%%%\usepackage{xypic}

% there are many more packages, add them here as you need them

% define commands here

\begin{document}
\emph{Wac\l{}aw Franciszek Sierpi\'nski} (1882 - 1969) Polish mathematician, best known for the \PMlinkname{Sierpi\'nski triangle}{SierpinskiGasket} and the Sierpi\'nski numbers.

Sierpi\'nski studied physics and mathematics at the University of Warsaw, graduating in 1903, then went for a doctorate at university in Krak\'ow. In 1908 he started teaching at the University of Lw\'ow. A prolific author of books and papers, Sierpi\'nski wrote on many subjects but focused on irrational numbers and set theory. During World War I, Sierpi\'nski was stuck in Russia, but the experience proved useful when he returned to Poland and helped his comrades break Russian cryptography. During this time he also founded and edited the journal {\it Fundamenta Mathematicae}. From 1929 to the end of his life, he received various honors at home and abroad. In 1964, he was elected to the London Mathematical Society.

In 1954, Sierpi\'nski published a French paper on some properties of Euler's totient function and the sum of divisors function in {\it Bull. Acad. Polon. Sci. Cl. III} {\bf 2} with Andrzej Schinzel, who a few years later co-authored with Erd\H{o}s a paper on the ``Distributions of the values of some arithmetical functions'' in {\it Acta Arithmetica} {\bf 6}, giving Sierpi\'nski an \PMlinkname{Erd\H{o}s number}{ErdHosNumber} of 2.
%%%%%
%%%%%
\end{document}
