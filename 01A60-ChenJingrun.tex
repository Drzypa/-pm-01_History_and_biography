\documentclass[12pt]{article}
\usepackage{pmmeta}
\pmcanonicalname{ChenJingrun}
\pmcreated{2013-03-22 16:19:16}
\pmmodified{2013-03-22 16:19:16}
\pmowner{PrimeFan}{13766}
\pmmodifier{PrimeFan}{13766}
\pmtitle{Chen Jingrun}
\pmrecord{7}{38448}
\pmprivacy{1}
\pmauthor{PrimeFan}{13766}
\pmtype{Biography}
\pmcomment{trigger rebuild}
\pmclassification{msc}{01A60}
\pmsynonym{Jingrun Chen}{ChenJingrun}
\pmsynonym{Jing Run Chen}{ChenJingrun}
\pmsynonym{Chen Jing Run}{ChenJingrun}

% this is the default PlanetMath preamble.  as your knowledge
% of TeX increases, you will probably want to edit this, but
% it should be fine as is for beginners.

% almost certainly you want these
\usepackage{amssymb}
\usepackage{amsmath}
\usepackage{amsfonts}

% used for TeXing text within eps files
%\usepackage{psfrag}
% need this for including graphics (\includegraphics)
%\usepackage{graphicx}
% for neatly defining theorems and propositions
%\usepackage{amsthm}
% making logically defined graphics
%%%\usepackage{xypic}

% there are many more packages, add them here as you need them

% define commands here

\begin{document}
\emph{Chen Jingrun} (1933 - 1996) Chinese mathematician, best known for his theorem ``on the representation of a large even integer as the sum of a prime and the product of at most two primes'' (Chen's theorem). In connection with this theorem, the Chen primes are named after him. Chen also worked on the twin prime conjecture, Goldbach's conjecture, Legendre's conjecture and Waring's problem.

In 1999, China issued an 80-yuan postage stamp with a silhouette of Chen and the inequality $$P_x(1, 2) \ge \frac{0.67xC_x}{(\log x)^2}.$$
%%%%%
%%%%%
\end{document}
