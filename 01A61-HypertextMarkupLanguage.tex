\documentclass[12pt]{article}
\usepackage{pmmeta}
\pmcanonicalname{HypertextMarkupLanguage}
\pmcreated{2013-03-22 17:06:18}
\pmmodified{2013-03-22 17:06:18}
\pmowner{PrimeFan}{13766}
\pmmodifier{PrimeFan}{13766}
\pmtitle{Hypertext Markup Language}
\pmrecord{9}{39403}
\pmprivacy{1}
\pmauthor{PrimeFan}{13766}
\pmtype{Definition}
\pmcomment{trigger rebuild}
\pmclassification{msc}{01A61}
\pmclassification{msc}{01A65}
\pmclassification{msc}{01A60}
\pmsynonym{HTML}{HypertextMarkupLanguage}
%\pmkeywords{HTML}
\pmdefines{HTML}

% this is the default PlanetMath preamble.  as your knowledge
% of TeX increases, you will probably want to edit this, but
% it should be fine as is for beginners.

% almost certainly you want these
\usepackage{amssymb}
\usepackage{amsmath}
\usepackage{amsfonts}

% used for TeXing text within eps files
%\usepackage{psfrag}
% need this for including graphics (\includegraphics)
%\usepackage{graphicx}
% for neatly defining theorems and propositions
%\usepackage{amsthm}
% making logically defined graphics
%%%\usepackage{xypic}

% there are many more packages, add them here as you need them

% define commands here

\begin{document}
{\em Hypertext Markup Language} (HTML) is an application of SGML for creating documents to exchange over the Internet using the Hypertext Transfer Protocol (HTTP). Created by Tim Berners-Lee in the early 1990s, early HTML was meant to describe document structure rather than document formatting. Berners-Lee intended HTML to enable scientists (physicists, biologists, doctors, etc.) to publish research results and stay up to date on relevant advances in their fields more quickly. As such, HTML provided for almost all the document publishing needs of scientists: six levels of headings, subscripts, superscripts, the letters of the Greek alphabet available by character entities, etc. What was missing was the ability to include diagrams inline, which was solved by the addition of the image tag, first proposed by Marc Andreessen in 1993.

For mathematics, and for some of the more theoretical branches of physics, HTML did not provide enough: Simple superscripts and subscripts are insufficient for power towers (e.g., $2^{2^x} - 1$); tables are at best a kludge for iterated sum and product notation; the providing of the radical sign falls short for overlapping radicals such as in $\sqrt{1 + \sqrt{1 + \sqrt{1 + \sqrt{1 + \sqrt{1 + \ldots}}}}}$; etc. To address these deficiencies, MathML was created in 1999, but it has yet to catch on even among mathematicians.

The HTML basics can be easily learned in about `5 minutes'; automatic conversions from plaintext or Microsoft Word documents to HTML are readily available.

For math resources on the Web HTML remains the preferred format. The two major strategies for dealing with HTML's mathematical formatting deficiencies are: 1) Writing formulas in the way they would appear in some computer programming language, or in \TeX{} (Steve Cheng and others have developed Web browser plug-ins that parse and render \TeX{} found on HTML pages); this is the strategy adopted by the OEIS. 2) Turning mathematical formulas into bitmap images, sometimes providing the underlying \TeX{} command as a tooltip (this helps computerized searchability, but presents somewhat of an obstacle for cutting and pasting for citations); this is the strategy adopted by MathWorld and PlanetMath (though PlanetMath also offers the option of viewing an entire document as a bitmap or as \TeX{} source).

%%%%%
%%%%%
\end{document}
