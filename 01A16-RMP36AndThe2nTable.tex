\documentclass[12pt]{article}
\usepackage{pmmeta}
\pmcanonicalname{RMP36AndThe2nTable}
\pmcreated{2013-03-22 19:04:40}
\pmmodified{2013-03-22 19:04:40}
\pmowner{milogardner}{13112}
\pmmodifier{milogardner}{13112}
\pmtitle{RMP 36 and the 2/n table}
\pmrecord{73}{41965}
\pmprivacy{1}
\pmauthor{milogardner}{13112}
\pmtype{Definition}
\pmcomment{trigger rebuild}
\pmclassification{msc}{01A16}

% this is the default PlanetMath preamble.  as your knowledge
% of TeX increases, you will probably want to edit this, but
% it should be fine as is for beginners.

% almost certainly you want these
\usepackage{amssymb}
\usepackage{amsmath}
\usepackage{amsfonts}

% used for TeXing text within eps files
%\usepackage{psfrag}
% need this for including graphics (\includegraphics)
%\usepackage{graphicx}
% for neatly defining theorems and propositions
%\usepackage{amsthm}
% making logically defined graphics
%%%\usepackage{xypic}

% there are many more packages, add them here as you need them

% define commands here

\begin{document}
The Rhind Mathematical Papyrus (RMP), a 1650 BCE hieratic text, began with a 2/n table.  The table of unit fraction series included shorthand calculations that took up 1/3 of a papyrus. The RMP also stated and solved 87 problems. The 2/n table contained 51 representations of rational numbers 2/n that were scaled to concise unit fraction series. A trivial 2/3 = 1/3 + 1/3 series and 50 non-trivial series was written down by Ahmes, the RMP scribe. Ahmes converted more difficult rational numbers 2/5 to 2/101 to concise unit fraction series by proto-number theory methods.  

Ahmes scaled 2/n by optimized least common multiples (LCMs) within proto-number theory methods that were often misunderstood by 20th century AD scholars. Aspects of RMP 36 and RMP 37 hekat (volume) problems were also misreported by 20th century scholars. Scholars understated Middle Kingdom proto-number theory in 2/n tables and hekat problems, raw data facts that scribes like Ahmes considered p and q as prime numbers and LCMs as scaling factors. 

Ahmes raw unit fraction data reported in RMP 36 and 37 reported important non-additive patterns. In the 21st century AD three classes of scaled rational number conversion methods have been decoded.  Rational numbers n/p were scaled by m/m to mn/mp and inspected the best divisors of mp that summed to mn before writing out concise unit fraction series.  

The first 2/n table conversion method scaled rational number 2/n by LCM m to 2m/mn. Ahmes selected 'red auxiliary' divisors of denominator mn that summed to numerator 2m that allowed concise unit fraction series to be recorded. The first conversion method converted 2/3, 2/5, 2/7, ..., 2/101 to concise unit fraction series.

The second conversion method shows that Ahmes could not solve 20/53 by one LCM m. The second rational number conversion method replaced n/p by (n-2)/p + 2/p as demonstrated in RMP 36 and RMP 31.

In RMP 36 Ahmnes did not solve 30/53 by one LCM m. None could be found. Ahmes solved 28/53 + 2/53 by scaling 28/53 by LCM 2 and 2/53 by LCM 30, two easy to find LCMs. 

Note that the second conversion method solved 30/53 and 28/97 by selecting two least common multiples (LCM) m to solve 30/53 = 28/53 + 2/53 and 28/97 = 26/97 + 2/97 by extending the first conversion method. Ahmes converted 26/97 by LCM 4 by considering 104/388 before writing a final unit fraction series.  In RMP 31 Ahmes had solved 28/97 by selecting two LCM m scaling factors. Ahmes solved 28/97 that scaled (26/97 + 2/97) that scaled 26/97 by LCM 4 and scaled 2/97 by LCM 56. Note that 2/97 was solved as the 2/n table reported. Hence LCM 56 was a well thought out scaling factor when Ahmes scaled 2/97.

Ahmes converted 2/97 by LCM 56 in the 2/n table and RMP 31 that considered 112/5432 such that divisors (97 + 8 + 7) were recorded in red and summed to numerator 112 meant:

2/97 = 112/5432 = (97 + 8 + 7)/5432 = 1/56 + 1/678 + 1/776

a complete 2/n table series sentence.

The same class of LCM m sentence scaled 28/53 by LCM 4 to 104/212 and 2/53 by LCM 56 to 112/2968 before 'red auxilinary numbers' selected the best divisors of mn. The best divisors were summed to numerator 2m and calculated concise unit fraction series. 

Ahmes' second conversion method solved difficult n/p conversions by substituting (n-2)/p + 2/p by solving two conversion problems.  

Ahmes solved 87 problems applying three 2/n table conversion methods. Modern arithmetic operations, algebra, arithmetic proportions and geometric methods were discussed by Ahmes. The first two conversion methods lead to the third method that found the best divisors of denominators mp that summed to numerators mp related to scaling n/p by LCM m to mn/mp. Red numbers divisors of denominator mp were summed to numerator mn in RMP 36 reporting:

2/53 + 3/53 + 5/53 +  15/53 + 28/53 = 53/53 = 1

Ahmes could have chosen any set of n/53 partitions that summed to one, such as:

51/53 + 2/53 = 1
50/53 + 3/53 = 1
49/53 + 4/53 = 1
46/43 + 5/53 = 1

as well as any n/p table at any other time, thereby exposing a third conversion method in the context of the hekat, the weights and measures unit used to pay \PMlinkexternal{wages}{http://www.nytimes.com/2010/12/07/science/07first.html?_r=2&ref=science} and make other business transactions.

Selections of LCM m encoded Ahmes' 2/n table. a fact missed by scholars for over 100 years. The raw 2/n table data is published on-line in: \PMlinkexternal{2008}{http://rmprectotable.blogspot.com/} resolved this issue. The RMP's 87 problems reported fragmented initial, intermediate, final, and proof information, the central reason for the scholarly delay. The correct information had been decoded and translated into modern arithmetic statements by adding back missing initial and intermediate steps and facts. The corrected RMP translations were posted online in \PMlinkexternal{2009 and 2010}{http://ahmespapyrus.blogspot.com/2009/01/ahmes-papyrus-new-and-old.html}.

The corrected RMP and Kahun Papyrus(KP) problems have been decoded by opening new doors to MK arithmetic. Once muddled RMP and KP texts that were misunderstood by \PMlinkexternal{Marshall Clagett, Ancient Egyptian Science, Vol III have 1999}{http://books.google.com/books?id=8c10QYoGa4UC&pg=PA469&dq=Ancient+Egyptian+Science} and other 20th century scholars are corrected by new decoding methods.

Red auxiliary numbers have long been noted by RMP scholars. However, frequent scribal uses of red numbers were not pinned down in Ahmes' 87 problems, the KP, and other texts as scribes understood their use until 2008. 

The best context to decode red auxiliary numbers is RMP 36. RMP 36 offers scribal evidence of red number details that scaled rational numbers n/p to mn/mp that calculated the \PMlinkexternal{2/n table series}{http://rmprectotable.blogspot.com/} by selecting optimized LCMs m.

What were the 2/n table series details cited in Ahmes' 87 problems? To answer those questions, the methods reported in RMP will be discussed. One RMP problem, RMP 36, established Ahmes red auxiliary number method that also calculated 2/n table unit fraction series data reported in Ahmes 87 problems.

RMP 36 solved 

3x + (1/3)x + 1/5(x) = 1 (hekat)

a simple algebra problem that Ahmes solved in a weights and measures context.

A duplation proof of the problem considered LCM 15 by solving:

((45 + 5 + 3)x)/15 = 1

such that

53x/15 = 1

x = 15/53

Ahmes converted 15/53 to a unit fraction series by thinking: 

(15/53)*(4/4) = 60/212

and writing:

(53 + 4 + 2 + 1)/212= (1/4 + 1/53 + 1/106 + 1/212)hekat

with 4 + 2 + 1 implicitly recorded in red,

Ahmes also considered 

106 times 15/53 = 30 

With 30 the LCM that converted 2/53 to a unit fraction series

and,

30 times 53/15 = 106

the GCD 106 was found by scaling 28/53 by LCM 2 translated as 2/2.

Ahmes' primary division method inverted 15/53 to 53/15, a property of modern division, an arithmetic operational fact also reported in RMP 38. Scholars writing on this topic in the 19th and 20th century falsely concluded that 'single false position', a medieval method for finding roots, was Ahmes' primary division operation. 

Ahmes converted 30/53 as 2/53 + 28/53. RMP 36 explicitly converted 3/53 to unit fraction series by a 2/n table red auxiliary number proof.

1 2/n table proofs included scaled 2/53 + 3/53 +  5/15 +  15/53 +  28/53 = 53/53  unit fraction calculations:

a. 2/53 = 2/53*(30/30) = 60/1590 = (53 + 5 + 2)/1590 = 1/30 + 1/318 + 1/79

b. 3/53 = 3/53(20/20) = 60/1060 = (53 + 4 + 2 + 1)/1060 = 1/20 + 1/265 + 1/530 + 1/1060 

c. 5/53 = (5/53)*(12/12) = 60/636 = (53 + 4 + 2 + 1)/(12*53) = 1/12 + 1/159 + 1/318 + 1/636

d. 15/53 = 15/53*(4/4) = 60/212= (53 + 4 + 2 + 1)/212 = 1/4 + 1/53 + 1/106 + 1/212

e. 28/73 = (28/53)*(2/2) = 56/106 = (53 + 2 + 1)/106 = 1/53 + 1/318 + 795 + 1/2 + 1/53 + 1/106

f. such that: (2/53 + 3/53 + 5/53 + 15/53 + 28/53)hekat = 53/53 hekat = one hekat (unity)


Note the repeated used of red numbers (53 + 4 + 2 + 1) that converted 3/53, 5/53, and 15/53, eie,.

 3/53 = (3/53)*(20/20) = 60/1060 = (53 + 4 + 2 + 1)/1060

The proof listed duplation proofs, followed by unit fraction series calculations that considered

15/53 = (1/4 + 1/53 + 1/106 + 1/212)hekat that reported:

1. 15/53*(4/4) = (60/212 - 1/4) = 7/212 = (4 + 2 + 1)/212 with 4 + 2 + 1 recorded in red ink.

2. 28/53*(2/2) = (56/106 - 1/2) =  7/106 = (4 + 2 + 1)/106 with 4 + 2 + 1 recorded in red ink

3. 5/53*(12/12) = (60/636 - 1/12) = 7/53 = (4 + 2 + 1)/53 with 4 + 2 + 1 recorded in red ink

4. 3/53*(20/20)=  60/1060 - 1/20 =7/1060 = (4 + 2 + 1)/1060 with 4 + 2 + 1 recorded in red ink
   
Note that the step introduced a a subtraction statement. the only context used by Fibonacci 2850 years later to work the same class of problems in the Liber Abaci.

Ahmes' arithmetic also scaled (30/53)+ (15/53) + (5/53)+ (3/53) = 1

solving 30/53 = 28/53 + 2/53 by applying a red auxiliary method that was implicitly used in the 2/n table. Note that 30/53 nor 28/97 in RMP 31 could not be solved by an LCM, hence the substitution of 28/53 + 2/53 for 30/53 and 28/97 + 2/97 for 28/97.  

Concerning the RMP's 87 problems Ahmes rarely calculated beginning, intermediate, answers, and duplation proofs as modern mathematicians understand Middle Kingdom arithmetic and mathematics. To fairly parse Middle Kingdom arithmetic explicit details are required.

In RMP 36 one scribal proof was fairly outlined by Marshall Clagett as an ancient multiplication operation. However, meta 2/n table construction details were unexposed by scholars until published in 2008. A majority of 19th and 20th century Egyptologists identified fragmented duplation details without fairly describing scribal calculations that created concise unit fraction series. Missing from academic discussions were precise meanings of red auxiliary numbers and the context in which scribal calculations took place. The red numbers and economic aspects were implicitly used in 2/n tables and scribal ways that caused scholars to badly guess and muddle the historical record. 

In 2008 RMP 36 and related 2/n table methods were explicitly spelled out by solving a critical problem. Rational numbers n/p were scaled to mn/mp by selecting optimized, but not optimal LCM m. Ahmes calculated several red number examples in RMP 36 that expose his thinking. Ahmes considered the divisors of mp, and selected the best set of divisors that added to numerator mn, by denoting selections in red.

Related Egyptian fraction method also 'healed' an Old Kingdom "Eye of Horus" binary numeration problem that wrote:

1 = 1/2 + 1/4 + 1/8 + 1/16 + 1/32 + 1/64 + ...

by writing

1 = 1/2 + 1/4 + 1/8 + 1/16 + 1/32 + 10 ro

Middle Kingdom scribes reported the remainder 2/64 as 10/320 in hekat problems The rational number 1/320 was named ro
in about 40 RMP problems.

In other texts, scribes scaled 2/64 to 10/320 and (8 + 2)/320 = 1/40 + 1/160 writing a complete statement

1 = 1/2 + 1/4+ 1/16 + 1/32 + 1/40 + 1/160 

and in RMP 36 by one hekat (unity)= 28/53 + 15/53 + 5/53 + 3/53 + 2/53 = 53/53

as Middle Kingdom scribes wrote scaled arithmetic statements and solving once impossible "Eye of Horus" arithmetic problems created from binary balance beam weights used in the Old Kingdom.

Conclusion: RMP 36 exposed three rational number conversion methods. The first conversion method scaled 2/53 by LCM 30 to 60/1590. A unit fraction series was obtained by red numbers 53 + 4 + 2 + 1 equal to numerator 60. The red numbers were "best" divisors of denominator 1590. Taken together red number divisors and concise unit fraction series exposed the two of the three 2/n table construction methods. Red number divisors were used to solve an impossible rational number 30/53 by substituting 28/53 + 2/43 Ahmes solved 28/53 by LCM 4 and 2/53 by LCM 30. In RMP 31 Ahmes solved another impossible rational number conversion. Ahmes solved 28/97 by substituting 26/97 scaled by LCM 4 and 2/97 scaled by LCM 56. The n/p = (n -2)/p + 2/p substitution defined Ahmes second conversion method. A third method created virtual n/p tables suggested by identity sums: 2/53 + 3/53 + 5/53 + 15/53 + 28/53 = 53/53, a method that Greeks, Coptics, Arabs and medieval scribes stressed. It is important to note that RMP 36 was written in an economic context. The main purpose of unit fraction arithmetic scaled hekats (of grain) values of beer, bread, and other  products for use in wage payments. 

References:

\PMlinkexternal{RMP data/Clagett, Chace}{http://rmp36.blogspot.com/2010/04/rmp-36-and-2n-table.html}

\PMlinkexternal{Ahmes Papyrus}{http://ahmespapyrus.blogspot.com/2009/01/ahmes-papyrus-new-and-old.html}

\PMlinkexternal{New York Times "Science" 12/7/10}{http://www.nytimes.com/2010/12/07/science/07first.html?_r=1&ref=science}

%%%%%
%%%%%
\end{document}
