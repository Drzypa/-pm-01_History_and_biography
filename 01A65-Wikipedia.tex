\documentclass[12pt]{article}
\usepackage{pmmeta}
\pmcanonicalname{Wikipedia}
\pmcreated{2013-03-22 16:45:13}
\pmmodified{2013-03-22 16:45:13}
\pmowner{PrimeFan}{13766}
\pmmodifier{PrimeFan}{13766}
\pmtitle{Wikipedia}
\pmrecord{12}{38979}
\pmprivacy{1}
\pmauthor{PrimeFan}{13766}
\pmtype{Definition}
\pmcomment{trigger rebuild}
\pmclassification{msc}{01A65}
\pmclassification{msc}{01A61}

\endmetadata

% this is the default PlanetMath preamble.  as your knowledge
% of TeX increases, you will probably want to edit this, but
% it should be fine as is for beginners.

% almost certainly you want these
\usepackage{amssymb}
\usepackage{amsmath}
\usepackage{amsfonts}

% used for TeXing text within eps files
%\usepackage{psfrag}
% need this for including graphics (\includegraphics)
%\usepackage{graphicx}
% for neatly defining theorems and propositions
%\usepackage{amsthm}
% making logically defined graphics
%%%\usepackage{xypic}

% there are many more packages, add them here as you need them

% define commands here

\begin{document}
{\em Wikipedia} is a free online encyclopedia that anyone may edit. Its coverage of elementary mathematical topics is almost \PMlinkescapetext{complete}, and its coverage of intermediate and advanced topics is adequate. Wikipedia has articles on the integers in the \PMlinkescapetext{range} $-2 < n < 201$ as well as 220, 284, 496, 666, 720, 1138, 1729, 69105 and certain powers of ten. Its biographical coverage of the most famous mathematicians and physicists (Albert Einstein, Gauss, Ramanujan, Wiles, etc.) is on par with that of the best paper encyclopedias, and there is often at least a couple of \PMlinkescapetext{lines of information} on the more obscure mathematicians.

The Wikipedia articles on mathematical (and other) themes are not only in English, but also in many other languages; the available language alternatives of a given article are found in a box in the left margin.\, The contents of distinct language versions are not identical, and the English version is not always the most comprehensive. Although Wikipedia can usually be relied on for definitions of mathematical terms, it is by design not meant for mathematical  proofs; Wikipedia has only the most famous and elementary proofs (such as Euclid's proof that there are infinitely many primes).

Because anyone may edit Wikipedia and introduce intentional or unintentional computational errors, it is advisable to double-check or recalculate any number from Wikipedia before using it in any intensive calculations. Unlike PlanetMath, Wikipedia does not use the Mathematics Subject Classification codes to organize its entries on mathematical topics; however, the PlanetMath Exchange does use the MSC codes to coordinate exchange between PlanetMath and Wikipedia. This would seem to reveal the same \PMlinkescapetext{deficiencies} as PlanetMath, but Wikipedia has articles on those topics from a more physics-oriented standpoint.
%%%%%
%%%%%
\end{document}
