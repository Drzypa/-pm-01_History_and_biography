\documentclass[12pt]{article}
\usepackage{pmmeta}
\pmcanonicalname{RMP35To38PlusRMP66}
\pmcreated{2015-12-19 6:17:44}
\pmmodified{2015-12-19 6:17:44}
\pmowner{milogardner}{13112}
\pmmodifier{milogardner}{13112}
\pmtitle{RMP 35 to 38 plus RMP 66}
\pmrecord{93}{41838}
\pmprivacy{1}
\pmauthor{milogardner}{13112}
\pmtype{Definition}
\pmcomment{trigger rebuild}
\pmclassification{msc}{01A16}

% this is the default PlanetMath preamble.  as your knowledge
% of TeX increases, you will probably want to edit this, but
% it should be fine as is for beginners.

% almost certainly you want these
\usepackage{amssymb}
\usepackage{amsmath}
\usepackage{amsfonts}

% used for TeXing text within eps files
%\usepackage{psfrag}
% need this for including graphics (\includegraphics)
%\usepackage{graphicx}
% for neatly defining theorems and propositions
%\usepackage{amsthm}
% making logically defined graphics
%%%\usepackage{xypic}

% there are many more packages, add them here as you need them

% define commands here

\begin{document}
The Rhind Mathematical Papyrus (RMP) in RMP 35-38 and RMP 66 report two classes of \PMlinkexternal{hekat (volume) units}{http://planetmath.org/encyclopedia/HekatDivisionsEgyptianWeightsAndMeasures.html}. The first replaced one hekat by a  64/64, hence forth a hekat unity. The second replaced a hekat scaled by (64/64) x (5/5) = 320/320 = 320 ro. RMP 36  discussed 320 ro. One tenth of a hekat, 10 hin, were discussed in RMP 83.  Another unit, 64 dja, was the focus of the Ebers Papyrus medical formulas. 
 
The first method was recorded five times in Akhmim Wooden Tablet problems, as well in RMP 47, 82, and 83 problems. 

The second method converted 2/53, 3/53, 5/53, 15/53, 28/53, and 30/53 to unit fraction series in RMP 36. Initial and intermediate calculations infer the 320 ro substitution for 64/64. In \PMlinkexternal{RMP 35-38 and RMP 66}{http://ahmespapyrus.blogspot.com/2009/01/ahmes-papyrus-new-and-old.html}, summarized by:

A. RMP 35: Find 3/10 of one hekat in ro units

1. 320 ro * 3/10 = 96 ro

2. The 96 ro data created a unity from

a. 320*3/10 = 96 ro 

b. 320*6/10 = 192 ro 

c. 320*1/10 = 32 ro 

d. Unity sum 320 ro = 1 hekat

B. RMP 36 solved 3x + (1/3)x + 1/5(x) = 1 hekat 

(45x + 5x + 3x)/15 = 1

(53/15)x = 1 

53x = 15  and 

x = 15/53 hekat

was solved much as rational number and algebra problems were solved by Greek, Arab, and medieval scribes. Ahmes used modern-like multiplication and division operations. To Ahmes, the division of a rational number by another rational number inverted the divisor, and multiplied. Ahmes did not use the 1920s 'single false position' division method. With division and multiplication answers in-hand, Ahmes applied a duplation multiplication operation in his proof.  

The conversions of 2/53, 3/53, 5/53, and 15/53 were scaled to numerator 60 by selecting LCM m values of 30, 20, 12 and 4, respectively. Each numerator was assigned a denominators 53m. A fifth rational number 28/53 was scaled to numerator 56 and assigned a denominator 106. Considering all five rational number conversions Ahmes scaled n/53 to mn/53m, with numerator mn additively parsed by selecting the best divisors of m. The selected divisors of m were denoted in red ink. A required conversion of 30/53 to a unit fraction series was obtained by the substitution of 28/53 + 2/53.

Fibonacci used a related two-step method to convert otherwise impossible rational numbers like 4/13. Fibonacci used the same LCM 4 in a subtraction contact (as Ahmes used in RMP 37).

Step one reported (4/13 - 1/4) = (16 - 13)/52 = 3/52

Ahmes solved 3/52 by inspecting the divisors of 52, finding (2 + 1) = 3

3/52 = (2 + 1)/52 = 1/26 + 1/52
meant 4/13 = 1/4 + 1/26 + 1/52

Fibonacci's second step solved 3/52 with LCM 18 per

(3/52 - 1/18) = (54 = 52)/936 = 2/936 = 1/468
meant 4/13 = 1/4 + 1/18 + 1/468

In RMP 18-23, the scribe worked "completion to 1 algebra problems" that practiced the selection of LCMs. In RMP 24 - 34, algebra lessons were worked that led up to scribe to obtain x = 15/53 hekat in RMP 36. 

Ahmes converted 15/53 to a unit fraction series by considering:
 
(15/53)*(4/4) = 60/212 = (53 + 4 + 2 + 1)/212= (1/4 + 1/53 + 1/106 + 1/212) hekat. 

Ahmes converted 2/53, 3/53, 5/53, 28/53, and 30/53 to unit fraction series by following 2/n table \PMlinkexternal{red auxiliary}{http://rmprectotable.blogspot.com/} rules within two proofs.

1. The first \PMlinkexternal{2/n table}{http://rmprectotable.blogspot.com/} proof considered:

a. 15/53*(4/4) = 60/212= (53 + 4 + 2 + 1)/212 = 1/4 + 1/53 + 1/106 + 1/212

b. 30/53 = 2/53 + 28/53= (2/53)*(30/30) + (28/53)*(2/2) = 1/30 + 1/318 + 795 + 1/2 + 1/53 + 1/106

c. 5/53 = (5/53)*(12/12) = (53 + 4 + 2 + 1)/636= 1/12 + 1/159 + 1/318 + 1/636

d. 3/53 = (3/53)*(20/20) = (53 + 4 + 2 + 1)/1060)= 1/20 + 265 + 1/530 + 1/1060

e. sum: 2/53 + 3/53 + 5/53 + 15/53 + 28/63 = 53/53 = one (hekat unity)

2. The second proof considered 2/53, 3/53, 5/53, 15/53, 28/53 and 30/53) as parts of a hekat in terms of \PMlinkexternal{red auxiliary}{http://rmprectotable.blogspot.com/} numbers, and other issued per:


15/53 = 15/53 (4/4) = 60/212= (53 + 4 + 2 + 1 )/212] = 1/4 + 1/53 + 1/106 + 1/212

b. (35 + 1/3) + (3 + 1/3) + (1 + 2/3) + 20 + 10 = 70 scaled 28/53 + 2/53 = 30/53  

c. (88 + 1/3) + (6 + 2/3) + (3 + 1/3) + (1+ 2/3)= 100 scaled 3/53 = 60/1060 = (53 + 4 + 2 + 1)

d. 53 + 4 + 2 + 1 scaled (3/53) = (3/53)*(20/20)= 60/1060 = (53 + 4 + 2 + 1)/1060

e. Each part of 15/53 = (1/4 + 1/53 + 1/106 + 1/212)hekat  3/53 and 5/53 are multiples of 15/53.

Conclusion: Proofs converted 2/53, 3/53, 15/5, 28/53, and 30/53, with 

30/53 = 2/53 + 28/53 

and 

2/53 + 3/53 + 5/53 + 15/53 + 28/53 = 53/53 = one hekat (unity)

Unity aspects were mentioned by Peet within 45/53 + 5/53 + 3/53 = 1 hekat. Ahmes proofs contained proto-number theory that were not mentioned by Peet, Chace or \PMlinkexternal{Marshall Clagett, Ancient Egyptian Science, Vol III have 1999}{http://books.google.com/books?id=8c10QYoGa4UC&pg=PA469&dq=Ancient+Egyptian+Science}. 

For example, red auxiliary numbers were incorrectly parsed by Peet, Chace and Clagett in RMP 36. RMP 36 generally converted one difficult n/p, 30/53, by solving for (n -2)/p + 2/p, a method that was used in RMP 31  to convert 28/97 = 2/97 + 26/97. 

A second set of RMP 36 facts showed that Ahmes' used multiplication and division as inverse operations (as discussed in RMP 24-34) as well as converting 7/212, 7/106, 7/53 and 3/53 by red auxiliary proofs that summed to 265, showing that the parts of 1060 are 530 + 265 + 265.
 
C. RMP 37: Find 1/90 of a hekat in ro units from:

1. 320 ro*(1/90) = 3 + 1/2 + 1/18 = 64/18

2. Ahmes playfully reported four unity sum methods, the first being:

a. 320*(1/180) = 64/36

b. 320*(1/360) = 64/72

c. 320*(1/720) = 64/144

d. 320*(1/1440) = 64/288

e. 320*(1/2880) = 64/576

f. unity sum (b + e) = 64/72 + 64/576 = 1

Three inverse red number calculations included \PMlinkexternal{EMLR}{http://en.wikipedia.org/wiki/Egyptian_Mathematical_Leather_Roll}-like conversion of 1/4 = 72/288 = (9 + 18 + 24 + 3 + 8 + 1 + 8 + 1)/288, with additive numerators recorded in red ink. Ahmes aligned red numbers (9 + 18 + 24 + 3 + 8 + 1 + 8 + 1) below a non-optimal(1/32 + 1/16 + 1/12 + 1/96 + 1/36 + 1/288 + 1/36 + 1/288) series. The paired lines meant that red integers were inverses of unit fractions. Ahmes recorded 1/8 as 72/576 with (8 + 36 + 18 + 9 + 1) recorded in red, again below (1/72 + 1/16 + 1/32 + 1/64 + 1/576), ending the playful problem.

D. RMP 38 multiply one hekat by 7/22, written in ro units.

1. 320*(35/11)*(1/10) = 320*(7/22)= (101 + 9/11)ro

2. Three implications of the proof are:

a. (101 + 9/11)*(22/7) = 320 ro = 1 hekat

b. A complete hekat was returned as Ahmes by inverting the divisor 7/22 to 22/7. This may have meant that 22/7 was a better approximation for pi than 256/81. A second aspect of the proof revealed scribal multiplication and division as inverse to one another, a property of modern arithmetic overlooked by 20th century scholars.
that multiplication and division were ssen as inverse operations. ***

c. A geometry implication considered the traditional hekat that used $$pi = 256/81$$. The traditional pi approximation overstated inventory volumes. To attempt to correct for inventory losses a practical $$22/7$$ approximation was implemented by Ahmes.

E. RMP 66: divided 10 hekats of fat by 365 days reporting a daily usage rate

a. 3200/365 = 8 + 2/3 + 1/10 + 1/2190 = 8 + 280/365 was given as the answer. A fragmented calculation was implied.

Ahmes divided 3200 by 365. Fragmented calculations reported

(246 + 1/3) + (36  + 1/2) + 1/6 = 280, meant that

3200/365 = 8 +  280/365 = 8 + 56/73 in modern arithmetic

Ahmes scaled 8 + (280/365 by 6/6) raising 1/6 to a unit such that:

8 + 1680/2190 was parsed by (1460 + 219 + 1)/2160 = 8 + 2/3 + 1/10 + 1/219

completed the calculation.

b. \PMlinkexternal{Marshall Clagett, Ancient Egyptian Science, Vol III have 1999}{http://books.google.com/books?id=8c10QYoGa4UC&pg=PA469&dq=Ancient+Egyptian+Science} summarized RMP 66 in an inappropriate additive manner only suggesting that Ahmes' duplation proof: 

2.        730

4.        1460

8 (a).    2920

2/3(b)   243 1/3

1/10(c)   36 1/2 

1/2160(d)  1/2160

was Ahmes; calculation, which it was not. Ahmes had the 8 + 2/3 + 1/10 + 1/2190 answer in hand, and applied the traditional Old Kingdom duplation proof, confirming each unit fraction in the answer.

F. Reference (1): A.B. Chace, Bull, L.
, Manning, H.P. and Archibald, R.C., The Rhind Mathematical Papyrus, Mathematical Association of America, Vol 1, 1927, vol 2, 1929, and reprint 1979 (NCTM) transliterated the scribal unit fraction shorthand. To translate Ahmes' shorthand to modern arithmetic statements initial calculations and other arithmetic steps are parsed and added back, as noted above. \PMlinkexternal{Milo Gardner}{http://en.wikipedia.org/wiki/User:Milogardner} and \PMlinkexternal{Bruce Friedman}{http://www.mathorigins.com/} collaborated on this project.

G. Reference (2) Marshall Clagett,  1999,  Egyptian Science and Mathematics (Volume III) includes Chace's 1927 views of the RMP, as well as transliterations of the Kahun Papyrus, the Moscow Mathematical Papyrus. The valid transliterations should not be considered complete translations. Missing initial and intermediate arithmetic steps were not parsed, and inserted, as well as the other mathematics, i.e. the attested arithmetic operations used by all Middle Kingdom scribes, info required to prepare complete translations.

H. Reference (3) Joran Friberg, 2005, "Unexpected links to Egyptian and Babylonian Mathematics" includes the outdated Gillings, Chace, and Peet 1920's transliterations contrasted to Babylonian mathematics. Again, Friberg did not add back the missing initial and intermediate Egyptian fraction statements, or discuss the Akhmim Wooden Tablet in a serious way to create valid translations to modern mathematics.

%%%%%
%%%%%
\end{document}
