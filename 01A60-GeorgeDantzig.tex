\documentclass[12pt]{article}
\usepackage{pmmeta}
\pmcanonicalname{GeorgeDantzig}
\pmcreated{2013-03-22 16:41:29}
\pmmodified{2013-03-22 16:41:29}
\pmowner{PrimeFan}{13766}
\pmmodifier{PrimeFan}{13766}
\pmtitle{George Dantzig}
\pmrecord{4}{38903}
\pmprivacy{1}
\pmauthor{PrimeFan}{13766}
\pmtype{Biography}
\pmcomment{trigger rebuild}
\pmclassification{msc}{01A60}
\pmclassification{msc}{01A61}
\pmsynonym{George Bernard Dantzig}{GeorgeDantzig}

\endmetadata

% this is the default PlanetMath preamble.  as your knowledge
% of TeX increases, you will probably want to edit this, but
% it should be fine as is for beginners.

% almost certainly you want these
\usepackage{amssymb}
\usepackage{amsmath}
\usepackage{amsfonts}

% used for TeXing text within eps files
%\usepackage{psfrag}
% need this for including graphics (\includegraphics)
%\usepackage{graphicx}
% for neatly defining theorems and propositions
%\usepackage{amsthm}
% making logically defined graphics
%%%\usepackage{xypic}

% there are many more packages, add them here as you need them

% define commands here

\begin{document}
\emph{George Bernard Dantzig} (1914 - 2005) American mathematician and programmer, the ``father of linear programming.''

The son of Tobias Dantzig, George began his formal studies at the University of Maryland. As the United States became involved in World War II, Dantzig was drafted by the U. S. Air Force to help with supply logistics. After the war, he obtained master's degrees from the University of Michigan and the University of California at Berkeley, where he famously solved two specially difficult problems believing them to be run-of-the-mill homework assignments.
%%%%%
%%%%%
\end{document}
