\documentclass[12pt]{article}
\usepackage{pmmeta}
\pmcanonicalname{MillenniumProblems}
\pmcreated{2013-03-22 16:32:06}
\pmmodified{2013-03-22 16:32:06}
\pmowner{Mravinci}{12996}
\pmmodifier{Mravinci}{12996}
\pmtitle{Millennium Problems}
\pmrecord{8}{38716}
\pmprivacy{1}
\pmauthor{Mravinci}{12996}
\pmtype{Definition}
\pmcomment{trigger rebuild}
\pmclassification{msc}{01A65}
\pmclassification{msc}{01A61}
\pmsynonym{Millenium Prize Problems}{MillenniumProblems}
\pmrelated{FamousAndInfamousOpenQuestionsInMathematics}

\endmetadata

% this is the default PlanetMath preamble.  as your knowledge
% of TeX increases, you will probably want to edit this, but
% it should be fine as is for beginners.

% almost certainly you want these
\usepackage{amssymb}
\usepackage{amsmath}
\usepackage{amsfonts}

% used for TeXing text within eps files
%\usepackage{psfrag}
% need this for including graphics (\includegraphics)
%\usepackage{graphicx}
% for neatly defining theorems and propositions
%\usepackage{amsthm}
% making logically defined graphics
%%%\usepackage{xypic}

% there are many more packages, add them here as you need them

% define commands here

\begin{document}
The \emph{Millennium Problems} are seven problems for the solution of which the Clay Mathematics Institute (CMI) is offering a prize of \$7 million. Someone who can solve just one of these problems receives \$1 million. For each problem, the CMI had a professional mathematician write up an official statement of the problem which will be the main standard by which a given solution will be measured against. The prize offer was announced in May 2004.

The seven problems are:

\begin{enumerate}
\item The Birch and Swinnerton-Dyer conjecture. It concerns elliptic curves over rational numbers and the L-series attached to those curves. The official statement of the problem was given by Andrew Wiles.
\item The Hodge conjecture. It concerns Hodge cycles and their linear combinations. The official statement of the problem was given by Pierre Deligne.
\item ``Existence and smoothness of the The Navier-Stokes equation.'' The Navier-Stokes equation calculates momemtum in fluids (specifically, liquids and gases). The official statement of the problem was given by Charles Fefferman.
\item ``P versus NP.'' It concerns the difference between verification and computation in polynomial time. The official statement of the problem was given by Stephen Cook.
\item The Poincar\'e conjecture. It is believed that every simply-connected compact 3-manifold is homeomorphic to $S^3$. The official statement of the problem was given by John Milnor.
\item Quantum Yang-Mills theory. Just as Hilbert's problems a hundred years before included a problem relating to the mathematics of physics, so does this 2000 collection. The official statement of the problem was given by Arthur Jaffe and Edward Witten. % a link to PlanetPhysics might be appropriate here, eh?
\item The Riemann hypothesis. It concerns the distribution of primes. This one was 8 of 23 in Hilbert's problems. The official statement of the problem was given by Enrico Bombieri.
\end{enumerate}

\begin{thebibliography}{1}
\bibitem{kd} Keith Devlin, {\it The Millennium Problems: The Seven Greatest Unsolved Mathematical Puzzles of Our Time} New York: Perseus Books Group (2002)
\end{thebibliography}

\section{External link}

\PMlinkexternal{CMI page on Millennium Problems}{http://www.claymath.org/millennium/}
%%%%%
%%%%%
\end{document}
