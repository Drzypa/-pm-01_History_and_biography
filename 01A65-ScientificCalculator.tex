\documentclass[12pt]{article}
\usepackage{pmmeta}
\pmcanonicalname{ScientificCalculator}
\pmcreated{2013-03-22 16:39:19}
\pmmodified{2013-03-22 16:39:19}
\pmowner{PrimeFan}{13766}
\pmmodifier{PrimeFan}{13766}
\pmtitle{scientific calculator}
\pmrecord{5}{38860}
\pmprivacy{1}
\pmauthor{PrimeFan}{13766}
\pmtype{Definition}
\pmcomment{trigger rebuild}
\pmclassification{msc}{01A65}
\pmclassification{msc}{00A05}

\endmetadata

% this is the default PlanetMath preamble.  as your knowledge
% of TeX increases, you will probably want to edit this, but
% it should be fine as is for beginners.

% almost certainly you want these
\usepackage{amssymb}
\usepackage{amsmath}
\usepackage{amsfonts}

% used for TeXing text within eps files
%\usepackage{psfrag}
% need this for including graphics (\includegraphics)
%\usepackage{graphicx}
% for neatly defining theorems and propositions
%\usepackage{amsthm}
% making logically defined graphics
%%%\usepackage{xypic}

% there are many more packages, add them here as you need them

% define commands here

\begin{document}
A {\em scientific calculator} is a calculator with all the arithmetic capabilities of a basic calculator plus trigonometric, statistical, logarithmic, binary logic, etc. which are useful in various scientific applications. A typical scientific calculator has some forty to fifty keys and is capable of performing about twice as many operations than that (necessitating a ``Shift'' or ``2nd'' key). They are often capable of displaying results in binary, octal and hexadecimal, but are usually limited to integers in those bases.

The calculator trigonometric functions are $\sin$, $\cos$, and $\tan$, which have their own keys. The arc versions of these are usually obtainable by using the ``Shift'' or ``2nd'' key.

A scientific calculator may have a single memory register with the associated keys M+, MR and MC, but they often have several more registers to hold various values for use in statistical computations.

For values greater than about $10^{12}$ or smaller than $10^{-12}$ (slightly more or less depending on the model) hardware scientific calculators must switch to scientific notation.
%%%%%
%%%%%
\end{document}
