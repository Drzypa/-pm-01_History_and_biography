\documentclass[12pt]{article}
\usepackage{pmmeta}
\pmcanonicalname{WorldRecordsInMathematics}
\pmcreated{2013-03-22 17:09:26}
\pmmodified{2013-03-22 17:09:26}
\pmowner{PrimeFan}{13766}
\pmmodifier{PrimeFan}{13766}
\pmtitle{world records in mathematics}
\pmrecord{9}{39468}
\pmprivacy{1}
\pmauthor{PrimeFan}{13766}
\pmtype{Feature}
\pmcomment{trigger rebuild}
\pmclassification{msc}{01A07}

\endmetadata

% this is the default PlanetMath preamble.  as your knowledge
% of TeX increases, you will probably want to edit this, but
% it should be fine as is for beginners.

% almost certainly you want these
\usepackage{amssymb}
\usepackage{amsmath}
\usepackage{amsfonts}

% used for TeXing text within eps files
%\usepackage{psfrag}
% need this for including graphics (\includegraphics)
%\usepackage{graphicx}
% for neatly defining theorems and propositions
%\usepackage{amsthm}
% making logically defined graphics
%%%\usepackage{xypic}

% there are many more packages, add them here as you need them

% define commands here

\begin{document}
In the sciences and in sports there are world records for achievements and discoveries. There are {\em world records in mathematics}, too.

\section{Numbers}

\subsection{Largest numbers}

{\bf Largest named number} In a standard abridged dictionary of the  English language, the largest named number is the centillion, $10^{600}$. Given a googol $10^{100}$, a googolplex $10^{10^{100}}$ is clearly much larger than a centillion; these words may be found in more recent unabridged dictionaries and certainly in mathematics dictionaries. According to the {\it Guinness Book of World Records 1991}, ``the highest number ever used in a mathematical proof is a bounding value published in 1977 and known as Graham's number. It concerns bichromatic hypercubes and is inexpressible without the  special `arrow' notation, devised by Knuth in 1976, extended to 64 layers.'' (McFarlan, 1990)

{\bf Largest number factored} The largest composite number factored (for which the factoring team did not know the answer beforehand) is RSA-200, 3532461934402770121272604978198464368671197400197625023649303468776121253679423200058547956528088349 times 7925869954478333033347085841480059687737975857364219960734330341455767872818152135381409304740185467, which was factored by a four-man team in 2005 using the general number field sieve. This record could be beat by the factorization of a Fermat number (beyond the known Fermat primes, and some partially factored Fermat numbers, the primality of most of these numbers remains an open question).

{\bf Largest known prime} According to the Prime Pages, the largest known prime number is usually a Mersenne prime, currently $2^{32582657} - 1$, discovered by GIMPS last year. The largest known non-Mersenne prime, seventh overall, is currently $19249 \times 2^{13018586} + 1$, discovered by Konstantin Agafonov earlier this month using SoBSieve and other programs.

{\bf Largest known perfect number} The largest known perfect number is of course the largest known Mersenne prime times the nearest power of two less than the Mersenne prime, in this case, $(2^{32582657} - 1)2^{32582656}$. No odd perfect numbers are known, and the current lower bound for an odd perfect number is significantly smaller.

{\bf Newest constant} As of 1990, the newest mathematical constant was Feigenbaum's constant, approximately 4.6692016010299, according to {\it Guinness}.

% \bf Most places of pi

\section{Theorems, proofs, puzzles, etc.}

{\bf Most-proved theorem} According to the {\it Guinness}, Pythagoras' theorem has been proved the most often. A book of over a thousand proofs of the theorem includes an 1876 proof by then-Congressman James Garfield (PlanetMath has a \PMlinkname{proof with a square}{ProofOfPythagoreasTheorem}, a \PMlinkname{proof splitting a triangle into two smaller triangles}{ProofOfPythagoreanTheorem}, a \PMlinkname{proof with two triangles inside a square}{ProofOfPythagoreanTheorem2} and Garfield's proof of Pythagorean theorem). Many people have authored proofs that there are infinitely many primes, however, most of these use either factorials or primorials and thus don't count as distinct proofs.

% {\bf Longest proof} proof of classif. finite simple groups says Guinness

{\bf Largest prize} Paul Wolfskehl's prize for a proof of Fermat's last theorem was 100000 Deutsche Marks; at the time it was offered, it would've been equivalent to about two million American dollars today, but because of inflation in Germany, it was only about sixty thousand dollars when Andrew Wiles received it. In 1993, Andrew Beal offered US\$100000 for a proof of Beal's conjecture; this remains the largest prize offered by an individual. In 2001, the Clay Mathematics Institute offered US\$7000000 for solutions of its seven Millennium Problems, or US\$1000000 for a solution of one of the problems.

% \bf Longest computation for a yes-no answer

\section{People}

{\bf Longest-lived professional mathematician} Austrian topologist Leopold Vietoris was born on June 4, 1891 and died two months short of his $111^{\mathrm{th}}$ birthday on April 9, 2002.

% Oldest living professional mathematician ???

{\bf Most prolific collaborator} Paul Erd\H{o}s collaborated with 509 other mathematicians on papers on a wide variety of mathematical topics, giving rise to the idea of the  \PMlinkname{Erd\H{o}s number}{ErdHosNumber}.

% Most distinct paths to Erdos Neil Sloane has Erdos number 2 by at least ten different paths, incl. by way of Ronald Graham

{\bf Highest documented Erd\H{o}s number} Michael Hones has Erd\H{o}s number 8 (Styer, 2005)

\begin{thebibliography}{2}
\bibitem{dm} D. McFarlan, editor. {\it The Guinness Book of World Records 1991} New York: Guinness Publishing Limited (1990): 75 - 76
\bibitem{rs} R. Styer \PMlinkexternal{``Erd\"os numbers at Villanova}{http://www41.homepage.villanova.edu/robert.styer/ErdosNumber.htm} http://www41.homepage.villanova.edu/robert.styer/ErdosNumber.htm Last updated July 22, 2005, accessed May 25, 2007
\end{thebibliography}

%%%%%
%%%%%
\end{document}
