\documentclass[12pt]{article}
\usepackage{pmmeta}
\pmcanonicalname{ChristineLaddFranklin}
\pmcreated{2013-03-22 17:15:10}
\pmmodified{2013-03-22 17:15:10}
\pmowner{Mravinci}{12996}
\pmmodifier{Mravinci}{12996}
\pmtitle{Christine Ladd-Franklin}
\pmrecord{4}{39587}
\pmprivacy{1}
\pmauthor{Mravinci}{12996}
\pmtype{Biography}
\pmcomment{trigger rebuild}
\pmclassification{msc}{01A55}
\pmclassification{msc}{01A60}
\pmsynonym{Christine Ladd}{ChristineLaddFranklin}
\pmsynonym{Christine Franklin}{ChristineLaddFranklin}
\pmsynonym{Mrs. Fabian Franklin}{ChristineLaddFranklin}

\endmetadata

% this is the default PlanetMath preamble.  as your knowledge
% of TeX increases, you will probably want to edit this, but
% it should be fine as is for beginners.

% almost certainly you want these
\usepackage{amssymb}
\usepackage{amsmath}
\usepackage{amsfonts}

% used for TeXing text within eps files
%\usepackage{psfrag}
% need this for including graphics (\includegraphics)
%\usepackage{graphicx}
% for neatly defining theorems and propositions
%\usepackage{amsthm}
% making logically defined graphics
%%%\usepackage{xypic}

% there are many more packages, add them here as you need them

% define commands here

\begin{document}
\emph{Christine Ladd-Franklin} n\`ee \emph{Christine Ladd} (1847 - 1930) American mathematician and educator.

Born in Connecticut of a merchant and the daughter of officeholders, young Christine started her studies at Portsmouth and then Wesleyan in Massachusetts, and was among the first students of the newly opened Vassar College. There she focused on linguistics and physics. After graduation she was not surprised to find that women were not granted much access to labs and observatories, so ``she took up, as the next best subject, mathematics, which could be carried on without any apparatus.'' Ladd taught and published articles on various topics. Thanks to James Joseph Sylvester, Ladd was able to study at John Hopkins, where she married a mathematics professor who taught there, Fabian Franklin, and bore him two children. The son did not survive infancy, but the daughter, Margaret Ladd Franklin grew up to be a major scholar in the women's suffrage movement. At the end of the 19th Century, Ladd-Franklin wrote extensively on the mathematics of optics. In 1926, decades after her doctoral work at John Hopkins, she was finally given the \PMlinkescapetext{Ph.D.} she had earned. In the time prior, she helped many women go on to grad work.

\begin{thebibliography}{1}
\bibitem{jg} J. Green ``Christine Ladd-Franklin'' in {\it Women of Mathematics: A Bibliographic Sourcebook} L. Grinstein, P. Cambpell, ed.s New York: Greenwood Press (1987): 121 - 128
\end{thebibliography}
%%%%%
%%%%%
\end{document}
