\documentclass[12pt]{article}
\usepackage{pmmeta}
\pmcanonicalname{MathematicalAssociationOfAmerica}
\pmcreated{2013-03-22 16:42:02}
\pmmodified{2013-03-22 16:42:02}
\pmowner{PrimeFan}{13766}
\pmmodifier{PrimeFan}{13766}
\pmtitle{Mathematical Association of America}
\pmrecord{5}{38914}
\pmprivacy{1}
\pmauthor{PrimeFan}{13766}
\pmtype{Definition}
\pmcomment{trigger rebuild}
\pmclassification{msc}{01A65}

% this is the default PlanetMath preamble.  as your knowledge
% of TeX increases, you will probably want to edit this, but
% it should be fine as is for beginners.

% almost certainly you want these
\usepackage{amssymb}
\usepackage{amsmath}
\usepackage{amsfonts}

% used for TeXing text within eps files
%\usepackage{psfrag}
% need this for including graphics (\includegraphics)
%\usepackage{graphicx}
% for neatly defining theorems and propositions
%\usepackage{amsthm}
% making logically defined graphics
%%%\usepackage{xypic}

% there are many more packages, add them here as you need them

% define commands here

\begin{document}
\PMlinkescapeword{development}

The {\em Mathematical Association of America} (MAA) is a society of professional mathematics educators as well as mathematics students in the United States. The publisher of {\it American Mathematical Monthly}, the MAA holds the annual Putnam Competition and awards the Chauvenet Prize. Through various programs, the MAA seeks to advance education, research, professional development, public policy and public appreciation of mathematics.

One of four partners of the Joint Policy Board for Mathematics, the MAA should not be confused with the American Mathematical Society. The MAA is headquartered in Washington, D.C. The official website of the MAA is \PMlinkexternal{www.maa.org}{http://www.maa.org/}.
%%%%%
%%%%%
\end{document}
