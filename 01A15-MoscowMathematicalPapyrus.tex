\documentclass[12pt]{article}
\usepackage{pmmeta}
\pmcanonicalname{MoscowMathematicalPapyrus}
\pmcreated{2014-06-10 6:55:05}
\pmmodified{2014-06-10 6:55:05}
\pmowner{milogardner}{13112}
\pmmodifier{milogardner}{13112}
\pmtitle{Moscow Mathematical Papyrus}
\pmrecord{87}{41017}
\pmprivacy{1}
\pmauthor{milogardner}{13112}
\pmtype{Definition}
\pmcomment{trigger rebuild}
\pmclassification{msc}{01A15}
\pmsynonym{geometry}{MoscowMathematicalPapyrus}
\pmrelated{area}
\pmrelated{volume}

% this is the default PlanetMath preamble.  as your knowledge
% of TeX increases, you will probably want to edit this, but
% it should be fine as is for beginners.

% almost certainly you want these
\usepackage{amssymb}
\usepackage{amsmath}
\usepackage{amsfonts}

% used for TeXing text within eps files
%\usepackage{psfrag}
% need this for including graphics (\includegraphics)
%\usepackage{graphicx}
% for neatly defining theorems and propositions
%\usepackage{amsthm}
% making logically defined graphics
%%%\usepackage{xypic}

% there are many more packages, add them here as you need them

% define commands here

\begin{document}
From Wikipedia

The Moscow Mathematical Papyrus (MMP),also known as the Golenischev Mathematical Papyrus, was once owned by Egyptologist Vladimir Golenidenov. Today the document is housed in the Pushkin State Museum of Fine Arts in Moscow. Based on the palaeography of the hieratic text dates to the Eleventh dynasty of Egypt. Approximately 18 feet long and varying between 1 1/2 and 3 inches wide, its format was divided into 25 problems with tentative solutions by the Soviet Orientalist Vasily Vasilievich Struve in 1930. It is one of a half dozen well-known Mathematical Papyri. Along with the Rhind Mathematical Papyrus(RMP), the Kahun Papyrus(KP), the Berlin Papyrus (KP), the Egyptian Mathematical Papyrus (EMLR), the Akhmim Wooden Tablet(AWT), and the Ebers Papyrus(EB). The MMP is about the same age as the AWT, BP, KP, and the EMLR, and about 250 years older than the RMP and EB.

The 10th problems of the MMP calculated the area of a hemisphere, a 1/2 slice of a cylinder. Gillings dedicated a chapter to a diameter (D/2) times (D/2) times pi = 256/81 as the area of a circle, cubit x cubit, a topic repeated in RMP 41,42, 43, 44, 45, and 46 by adding height to compute volume, cubit x cubit x cubit.

The MMP began with the modern area of the circle $$ A = (pi)r^2$$ and replaced pi with 256/81 and r with D/2 and considered:

1. A = (256/81)(D/2)(D/2)
2. A = (64/81)(D)(D)
3. A = (8/9)(8/9)(D)(D)

The scribe input D = 9 and found the area of a semi-circle

4. A = ((9 - 9/9) = [(8)(8)]/2 = 64/2 = 32 (the cubit squared unit was omitted)

in RMP 41 Ahmes input D = 9

5. A = (9 - 9/9) = (8)(8) = 64 cubits squared

In RMP 42 Ahmes input D = 10

6. A = (10 - 10/9) = (80/9)(80/9) = 6400/81 = (79 + 1/81)cubits squared

and input H =10

7. V = (6400/81)(10) = 64000/81 = 790 + 10/81 = 790 + 20/162 = [790 + (18 + 2)/162 = 790 + 1/18 + 1/81]

and converted to khar units by:

8. V = [790 +10/81] + 1/2[790 + 10/81] = [1185 + 15/81]khar 

leaving rational number conversions to unit fraction series calculations to the reader.    

B. In RMP 43 Ahmes into D = 8 and H = 6 in line 2 as the Kahun Papyrus input D = 12 and H = 8 using a volume of a circular granary formula in khar units that used height (H) and further scaled V = (3/2) (H)(8/9)(8/9)(D)(D) by 3/2, obtaining (3/2)V = (H)(3/2)(3/2)(8/9)(8/9)(D)(D) = (H)(4/3)(4/3)(D)(D) and scaling both sides by 2/3

9. V = [(2/3)H(4/3)(D)(4/3)(D)]khar

10. RMP 43 data:(2/3)(6)(4/3)(8)((4/3)(8) = (4)(32/3)(32/9) = 4096/9 = (455 + 1/9)khar

A khar contained 10 hekats in the Middle Kingdom and 16 hekats in the New Kingdom.

11. Kahun Papyrus(KP) data: (2/3)(8)(4/3)(12)(4/3)(12)= (16/3)(16)(16) = 4096/3 = (1365 + 1/3) Khar

The KP scribe formula used the MMP 10 area of a circle formula, and RMP 43'S volume of a cylinder formula. The scribe also considered r = D/2 and pi = 256/81 as the MMP and RMP scribes had done
and added \PMlinkexternal{arithmetic progressions}{http://planetmath.org/kahunpapyrusandarithmeticprogressions} in a generalized formula.

As a related point is important to comment that RMP 38 improved scribal uses of pi = 256/81 to 22/7 in the context of correcting granary losses, large with an implied correction. RMP 38 apparently corrected for granary losses such that 22/7 was used for pi in area and circumference calculations. Indirect proof is suggested by the \PMlinkexternal{Berlin Papyrus}{http://planetmath.org/berlinpapyrusandseconddegreeequations}. 

Problem 14: Volume of frustum of square pyramid. The problem may be the most difficult to decode following scribal rules. The scribe calculated the volume of a frustum. This is the only ancient example finding the volume of a frustum of a pyramid or cone[4]. There are no known examples of a volume calculation of a complete pyramid or cone.f However, it is expected that an infinite series of the formula was known in the Old Kingdom. The MMP scribe would have corrected the Old Kingdom round off errors in the older formula by writing exact unit fraction series by following 2/n table rational number conversion rules.
 
Similarly, in Mesopotamia interest seems to have been in finding the volumes of frustums rather than complete pyramids or cones. The Babylonian mathematical tablet BM 85194, for example, sets out the calculation for the volume of a trapezium-sectioned fortification wall.

The problem states that a pyramid has been truncated in such a way that the top area is a square of length 2 units, the bottom a square of length 4 units, and the height 6 units, as shown. The volume is found to be 56 cubic units, which is correct.

The text of the example runs like this: "If you are told: a truncated pyramid of 6 for the vertical height by 4 on the base by 2 on the top: You are to square the 4; result 16. You are to double 4; result 8. You are to square this 2; result 4. You are to add the 16 and the 8 and the 4; result 28. You are to take 1/3 of 6; result 2. You are to take 28 twice; result 56. See, it is of 56. You will find (it) right" [5]

This describes the correct calculation:

    $$V = 1/3*6(4^2 + 4 \times 2 +2^2)$$

which indicates that the Egyptians knew the correct formula for obtaining the volume of a truncated pyramid:

    $$V = 1/3*h(a^2 + ab + b^2)$$.

The origins on how and why Egyptians arrived at the formula for the volume of a frustum are unknown. Regionally, the Babylonians, about the same time, used an incorrect approach by averaging the area of base and top and multiplying by height.[6]

Touraeff, the first commentator, strangely saw Problem 14 as describing the more general formula for the volume of any frustum[7], a formula that was not derived for another 3000 years. He was not alone in this view.[8]

    $$V = 1/3*h(A+\sqrt{A B}+B)$$

as discussed by \PMlinkexternal{Allen}{http://www.math.tamu.edu/~dallen/history/egypt/node4.html}, and \PMlinkexternal{Gardner}{http://moscowmathematicalpapyrus.blogspot.com/2009/12/moscow-mathematical-papyrus.html}.

MMP 12 and 16 are beer formula problems. MMP 12 scaled 13 hekats of grain producing 8 des-jubs of date-beer. The scribe asked how many times 13 hekat of grain produces 13/6 (pesu) beer? Answer 6 times. On its surface beer (and bread) problems were simple. However, on the distribution side a deeper economic context of Egyptian fractions has been exposed, one that connects proportional mixing of beer ingredients using modern algebraic ideas of 

1. alternando (y/x) = (q/p) 

and,

2. dividendo (y -x))/x = (q-p)/p

and, in MMP 21 a harmonic mean mixed sacrificial bread, reported by Gillings with confirming RMP 76 information.

Gillings 40 years ago summarized (on pages 134 and 135 in "Mathematics in the Time of the Pharaohs") algebraic connections to proportional mixes of bread and beer ingredients rather than invverse proportional pesu units. Gillings did not suspect that the trading of beer and bread products with the pesu, a secondary monetary scaling unit to him, was more important. Gillings read RMP 72 that traded  45 pesu bread for 10 pesu bread, a class of problem that was missing in the MMP. The hekat was a primary monetary unit with many commodities scaled against is value.

The Kahum Papyrus and arithmetic progressions show that simple proportions were well understood.

There were several bread and beer production and distribution problems that included raw materials and labor. Bread produced 200 loaves from 5 hekats of grain were scaled to 20 pesu. From 13 to 16 hekats of grain were split between bread and beer productions in a manner that allowed accurate distributions. One hekat of grain, 1/4 to 1/8 of a worker's salary, scaled into pesu units, facilitated bread, beer, and other product distributions. RMP 72, 73 and 75 confirm the MMP conclusion.

In MMP 18, the 20 setat area of a triagle, with a base of (1/3 + 1/5) khet, solved for the altitude by considering the 40 setat area of a rectangle, reckoning 2/5 of 1 to be (2 1/2), to reach the correct altitude of 4 khet. 

In MMP 20, 1/5 of 2/3 of a 1000 hekat was shown to be 133 1/3 hekat, with a final answer written in Akhmim Wooden Tablet (\PMlinkexternal{AWT)}{http://en.wikipedia.org/wiki/Akhmim_Wooden_Tablet} form (133 + 1/4 + 1/16 + 1/64)hekat + (1 + 2/3)ro, a division of a hekat unity (64/64) that Ahmes used 30 plus times in the RMP. 

Related Middle Kingdom area and volume problems are discussed in the Rhind Mathematical Papyrus and transliterations of other texts cited by Marshall Clagett, 1999, Ancient Egyptian Science" VOL III, beginning on page 210, as well as by several sections of Richard J. Gillings "Mathematics in the Time of the Pharaohs". The Clagett and Gillings treatises  are available on-line thanks to google books.

%%%%%
%%%%%
\end{document}
