\documentclass[12pt]{article}
\usepackage{pmmeta}
\pmcanonicalname{MathematicsInEgyptMathematicalLeatherRoll2014Update}
\pmcreated{2014-03-12 16:47:25}
\pmmodified{2014-03-12 16:47:25}
\pmowner{milogardner}{13112}
\pmmodifier{milogardner}{13112}
\pmtitle{Mathematics in Egypt: Mathematical Leather Roll (2014 Update)}
\pmrecord{52}{88023}
\pmprivacy{1}
\pmauthor{milogardner}{13112}
\pmtype{Definition}
\pmclassification{msc}{01A16}

\endmetadata

% this is the default PlanetMath preamble.  as your knowledge
% of TeX increases, you will probably want to edit this, but
% it should be fine as is for beginners.

% almost certainly you want these
\usepackage{amssymb}
\usepackage{amsmath}
\usepackage{amsfonts}

% need this for including graphics (\includegraphics)
\usepackage{graphicx}
% for neatly defining theorems and propositions
\usepackage{amsthm}

% making logically defined graphics
%\usepackage{xypic}
% used for TeXing text within eps files
%\usepackage{psfrag}

% there are many more packages, add them here as you need them

% define commands here

\begin{document}
      Mathematics in Egypt: Mathematical Leather Roll (2014 Update)

Henry Rhind purchased a 10″ × 17″ leather roll (Egyptian Mathematical Leather Roll, EMLR) and a mathematical papyrus (Rhind Mathematical Papyrus, RMP) in 1858 on the streets of Cairo, Egypt. Both texts date to the Middle Kingdom (2050 BCE to 1550 BCE). Upon Henry Rhind’s unexpected death, both texts were gifted to the British Museum in 1864 (Seyf, Hall, 1927). The papyrus was published in 1879 in Germany, and reported as translated in the USA (Chace, 1927). In 1927 the EMLR was chemically softened, unrolled, and analyzed (Seyf,  Hall, 1927). 

The Middle Kingdom hieratic script was written right to left. There were 26 rational numbers listed in a right column, followed by a series of equivalent unit fractions. There were ten binary rational numbers: 1/2, 1/4 (twice), 1/8 (thrice), 1/16 (twice), 1/32, and 1/64. There were seven other even rational numbers: 1/6 (twice—but wrong once), 1/10, 1/12, 1/14, 1/20, and 1/30. There were also nine odd rational numbers: 2/3, 1/3 (twice), 1/5, 1/7, 1/9, 1/11, 1/13, and 1/15.

British Museum examiners found no description of how or why the equivalent unit fraction series were recorded (Gillings, 1972). There was a trivial scribal error associated with the 1/15 unit fraction series. The scribe mistakenly listed 1/6. A serious scribal error was associated with the 1/13 line, a problem that the examiners did not resolve. Seyf and Hall (1927) naively reported in The British Museum Quarterly the chemical analysis was more interesting than the Egyptian mathematical leather roll's apparent singular focus on additive unit fraction relationships.

Nearby Babylonian scribes used an infinite series base 60 system 1,000 years before finite Egyptian numeration statements appeared. Babylonians gained numerical accuracy by minimizing round-offs to 1/3,600 for two terms, and 1/216,000 for.  Babylonian inverse prime number tables used for dividing fractions by fractions rounded off and degraded an intended high accuracy (Campbell-Kelly, et al, 2003). 

The Egyptian binary fractions were restatements of and improvements to the Old Kingdom Horus-Eye infinite series system. The Egyptian system asked a “decimal fraction” numeration question: How can a Horus-Eye  representation for the number one (1)—and all other numbers—that rounded off a seventh 1/64 term be represented by an exact unit fraction series?

 
Properties of infinite and finite Egyptian arithmetic were similar to those contained in our modern decimal system (Ore, 1948). Modern researchers, since 1927, minimize the EMLR's significance as a teaching document by overlooking non-additive aspects of the student test paper. Horus-Eye round off errors were implicitly corrected in the EMLR. But did the EMLR demonstrate that unit fractions were always intended to be recorded as exact unit fraction series?

Probing the intent of the text, EMLR examiners suggested that the 26 lines only contained simple additive information.  Five modern arithmetic categories freshly parsed the EMLR's 26 unit fraction series in 2004.  The first four additive categories were understood in 1927. However, a non-additive fifth category, proposed to have been an algebraic identity, included single and double least common multiples (LCM) that implicitly scaled unit fractions to smaller unit fraction series (Gardner, 2004).

Today it is clear that the fifth 2004 category was not intended to be an algebraic identity. The EMLR student scribe actually scaled 1/8 by two single-LCMs: (3/3) and (5/5), and one double-LCM: (25/25), (6/6). The three EMLR alternative 1/8 series did not include shorthand notes. A modern translation of the scribal data therefore adds-back scribal intermediate steps that are consistent with scribal shorthand used in the AWT and RMP.

1.  1/8 (3/3) = 3/24 = (2 + 1)/24 = 1/12 + 1/24

2.	1/8(5/5) = 5/40 = (4 + 1)/40 = 1/10 + 1/40

3.	1/8(25/25) = 25/200 = (8 + 17)/200 = 1/25 + 17/200 such that

17/200(6/6) = 102/1200 = (80 + 16 + 6)/1200  meant

 1/8 = 1/25 + 1/15 + 1/75 + 1/200 
 
The third 2-LCM method could not have been built by a 1-LCM since

a. (1/8 - 1/25) = (25 - 8)/200 = 17/200 = (10 + 5 + 2)/200= 

1/8 = 1/25 + 1/20 + 1/40 + 1/100

b. (1/8 - 1/25) = (25 - 8)/200 = 17/200 = (8 + 5 + 4)/200 =

1/8 = 1/25 + 1/25 + 1/40 + 1/50 must solve 

2/25(3/3) = 6/100 = (5 + 1)/100 = 1/25 + 1/75 + 1/100

1/8 = 1/15 + 1/40 + 1/50 + 1/75 

The sequence of the actual EMLR 1/8 unit fraction series was first scaled by (25/25) in the third two 2-LCM method has been confirmed by 60 RMP examples and five Akhmim Wooden Tablet (AWT) examples. 

The AWT is housed in the main Cairo, Egypt museum scaled a hekat (a volume unit) by double-LCMs to record 1/64 quotents and 1/320 of a hekat sub-unit. That is, RMP and AWT hekat quotients scaled a hekat unity (64/64) when multiplied by 1/3, 1/7, 1/10, 1/11, 1/13 and 1/n. The (64/64)/n remainder was scaled by LCM 5/5 to 1/320th of a hekat unit was called ro (Vymazalova, 2002; Gardner, 2006). The RMP and AWT scribal shorthand was subtle, and therefore was also misread for over 110 years. Two corrected translations of the 1/3 and 1/10 scribal longhand cases makes subtle points:

1.	(64/64) (1/3) = 21/64 + (1/192)(5/5)  = (16 + 4 + 1)/64 + 5/3(1/320) =

1/4 + 1/16 + 1/64 + (1 + 2/3) ro

2.	(64/64)(1/10) = 6/64 + 1/640(5/5) = (4 + 2)/64 + 20/10(1/320) = 

1/16 + 1/32 + 2 ro

Specific common EMLR and RMP rational number conversion methods were not accurately parsed in a scribal manner during the 20th century. Scholars reported scribal division as "false position” guesses (Chace, 1927, Eves, 1961). Scholarly citations of "false position” showed only that scholars guessed.
No scribal or scholarly form of "false position" guessing has been implicitly or explicitly connected to the EMLR, RMP, or the AWT.

Given that the EMLR recorded no intermediate calculations, no implicit proof of “false position” division guessing can be found. Concerning the 65 RMP and AWT examples not one explicit or implicit scribal use of "false position" was found. What was found: the division of fractions 1/n by a second fraction 1/p first scaled 1/n by (m/m) to m/mn so that m/mnp was recorded 65 times to concise unit fraction series (that summed the best divisors of denominator mnp often by red auxiliary numbers to numerator mm in ways that 19th and 20th scholars failed to decode). In six cases, all five AWT problems and RMP 38, proofs inverted divisor 1/p to p/1 such that m/mnp times p/1 returned 1/n (Gardner, 2006, M. Gardner, 2008). 

In summary, the EMLR’s 26 lines of text responded to the same infinite series question, found an exact unit fraction series for each unit fraction, when needed. Concerning details, the EMLR student scaled Horus Eye (binary), composite and prime unit fractions by single and double LCMs (m/m) to concise and awkward exact unit fraction series. EMLR data reported finite series without 1920s scholarly induced “false position” guesses that were misreported with respect to the AWT, the RMP and other texts. That is, the EMLR implied when contrasted to the AWT and RMP that scribal division and proofs inverted divisors and multiplied in the modern arithmetic operation manner .

References

1. Campbell-Kelly, M., Croarken, M., Flood, R., and E. Robson, E. (Eds.). (2003). The history of mathematical tables from Sumer to spreadsheets. Oxford: Oxford University Press.

2. Chace, A.B., et al, The Rhind Mathematical Papyrus, 2 volumes, Oberlin, Math Association of America, 1927.

3. Eves, H. (1961). An introduction to the history of mathematics. New York: Holt, Rinehart and Winston.

4. Gardner, M. (2004). The Egyptian Mathematical Leather Roll, attested short term and long term. In I. Grattan-Guiness, and B. C. Yadav (Eds.), History of the mathematical sciences (pp. 119–134). New Dehli: Hindustan Book Agency.

5. Gardner, M. (2006). The Arithmetic used to solve an ancient Horus-Eye problem. Ganita Bharati: Bulletin for the Indian Society for the History of Mathematics, David Pingree .  Volume, 28, 157–173.

6. M. Gardner. (2008, July 21). Breaking the RMP 2/n table code. (Web log comment. Retrieved
	From http://remprectotable.blogspot.com/)

7. Gillings, R. J. (1972). “The Egyptian Mathematical Leather Roll”, Mathematics in the Time of Pharaohs, 89-104.

8. Ore, O. (1948).  Number Theory and its History. New York: McGraw Hill.

9. Seyf A., Hall, H. H. (1927). Laboratory Notes: Egyptian Mathematical Leather Roll of the Seventeenth Century BC. British Museum Quarterly 2, 56–57.

10. Vymazalova, H. (2002). The Wooden Tablets from Cairo: The Use of the Grain Unit HK3T in ancient Egypt." Archiv Orientalni, 70(1), 27–42.


\end{document}
