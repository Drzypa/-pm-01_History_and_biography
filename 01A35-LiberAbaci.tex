\documentclass[12pt]{article}
\usepackage{pmmeta}
\pmcanonicalname{LiberAbaci}
\pmcreated{2015-04-06 17:28:27}
\pmmodified{2015-04-06 17:28:27}
\pmowner{milogardner}{13112}
\pmmodifier{milogardner}{13112}
\pmtitle{Liber Abaci}
\pmrecord{72}{40116}
\pmprivacy{1}
\pmauthor{milogardner}{13112}
\pmtype{Definition}
\pmcomment{trigger rebuild}
\pmclassification{msc}{01A35}
\pmsynonym{rational numbers}{LiberAbaci}
\pmdefines{Egyptian fractions}

% this is the default PlanetMath preamble.  as your knowledge
% of TeX increases, you will probably want to edit this, but
% it should be fine as is for beginners.

% almost certainly you want these
\usepackage{amssymb}
\usepackage{amsmath}
\usepackage{amsfonts}

% used for TeXing text within eps files
%\usepackage{psfrag}
% need this for including graphics (\includegraphics)
%\usepackage{graphicx}
% for neatly defining theorems and propositions
%\usepackage{amsthm}
% making logically defined graphics
%%%\usepackage{xypic}

% there are many more packages, add them here as you need them

% define commands here

\begin{document}
Abstract (2015 Update):
The “Liber Abaci”, a 1202 AD Latin text, primarily scaled vulgar fractions n/p to unit fraction by seven distinctions in a subtraction context. The Hindu-Arabic numeral notation began in 800 AD scaled rational numbers n/p by LCM m per: (n/p - 1/m) = (mn -p)/mp, replaced Greek alphabet ciphered rational numbers encoded by multiplication. Fibonacci set numerator (mn -p) to unity (1) as often as possible to define 2-term series. In the seventh distinction 3-term series were calculated by subtracting a second 1/m. The first of three Arab arithmetic notations solved arithmetic, algebra, geometry and modular arithmetic problems, such as an inverse proportion square root method used by Archimedes, pertinent to the medieval era.

INTRODUCTION
The “Liber Abaci” (Book of Calculation) was written by Leonardo Pisano. Fibonacci was Leonardo's Latin name. Leonard was the son of a Pisa merchant, and often went with his father to Arab ports and other trading locations. On these trips Leonardo wrote down practical and abstract sides Hindu-Arabic arithmetic operations that were used in the regional weights and measures system of the Mediterranean world. As a historical consequence, Fibonacci documentation of medieval trading units indirectly and directly reported  aspects s of very old theoretical and practical Egyptian fraction arithmetic that were  passed down to medieval  traders.

The “Liber Abaci” was a very successful book. It was used in the medieval period for 250 years to teach arithmetic and higher mathematics in European Latin schools. Today, five copies of the book exist. The 800 year old book had been partially translated by 20th century scholars.  Scholars tended to take single arithmetic subjects, often omitted related arithmetic subjects linked to older unit fraction arithmetic. 

Modern base 10 decimal themes were built by an algorithm flowered after 1585 CE. The foundations of our decimal system retained 3,200 year older finite Egyptian fraction arithmetic, but implicitly hid the older finite number theory, arithmetic, and algebraic foundations that stressed algorithms as questions. Singular 20th century themes oddly limited “Liber Abaci” translations to base 10 decimals foundations related to algorithms. For example, theory of equation and number theory algorithm focused books drew upon medieval algebra but omitted large aspects of medieval number theory. 

Sigler’s 2002 translation is 500 pages. The first 125 pages began to expose medieval arithmetic and hints of pertinent aspects of the 3,200 year older Egyptian fraction arithmetic. Earlier scholars footnoted and discussed “Liber Abaci “math topics without decoding arithmetic and algebra building blocks taken from older arithmetic foundations. The Sigler translation was a great improvement over the earlier partial translations.

 But even Sigler failed to expose final rational number notations that scaled n/p by subtracting LCM 1/m, i.e.  (n/p – 1/m) = (mn-p).  Earlier Greek rational numbers were also not cited by Sigler that scaled n/p by multiplying LCM m/m, i.e. (n/p)(m/m) = (mn/mp). Greeks also used scaled rational numbers in its final notation, as Egyptians did before them, by finding the best divisors of mp summed to mn. Arabs had replaced Ionian and Dorian alphabet encoding of numerals that mapped 1/2  = beta’, and so forth.

Sigler's important work footnoted most arithmetic, algebra and higher math topics, recorded in final rational number notations and methods. Sigler made sense of Leonardo’s rambling narratives by overlaying modern versions of medieval unit fraction arithmetic.  Sigler's footnotes are important, yet sparse based on his untimely death and confusing modern points of view.

The Sigler translation directly and indirectly includes the fundamental theorem of arithmetic as understood by Arabs. The first seven chapters indirectly discuss theoretical and practical aspects of the 3,200-year old system of Egyptian fractions. At various places Leonardo cites three different arithmetic notations to record rational numbers in unit fraction series. The first arithmetic notation was dominant in the 500 page book. The second and third notations were used for special factoring purposes not intended to be parsed by this paper.

LEONARDO'S SEVEN EGYPTIAN FRACTION METHODS (Distinctions)
This paper analyzes seven rational number conversion methods, or distinctions, as written in “Liber Abaci”  remainder arithmetic. Leonardo used the first notation to scale vulgar fractions to elegant and not-so-elegant Egyptian fractions series. Fibonacci indirectly discusses older Egyptian fraction system as this paper will hi-light.

The seven conversion methods used by Leonardo were built by selecting a subtraction 1/m, a LCM, in a subtraction a step, per “ab initio”, from the beginning details:

(n/p - 1/m) = (mn - p)/mp

 Fibonacci set  (mn –p) = 1 as often as possible, in the first six distinctions, further discussed by:

1. Leonardo’s first distinction

The first method contains three aspects—the simple, the  composite,  and a reversed composite. Two remainder arithmetic notations were used.

a. Simple factoring, 1/2 of 1/9 meant 1/18. In addition 1/2 x1/9 was converted to an Egyptian fraction series 1/2 = 1/3 + 1/6. Note that  1/18 = 1/27 + 1/54, was an older  Egyptian fraction series that Fibonacci did not list.

b. The second composite used a Greek or Arab notation that reported 1/18 = 1/2 0/9, which equals 5/10 0/9, as listed by Leonardo. Aspects of this rule may date to the time of Ahmes, when (64/64), a hekat unity was partitioned into 1/64 quotients 1/320 (ro) remainders. In other examples Ahmes mentally scaled 1/2 to 5/10, a common practice employed by Ahmes (who scaled 1/64 by 5/5 to 5/320, recorded as 5 ro) 

c. The third reversed composite used a Greek or Arab notation that allowed the denominators, 10 and 9, to be switched, stating that:

1/18 = 5/10 0/9 = 5/9 0/10.

Note that 5/10 0/9 = 5/90 = 1/18 and 5/9 0/10 = 5/90 = 1/18.

Sigler summarized Leonardo's first method (distinction) rule, and its three aspects, primarily through the assistance of Dunton and Grimm's "Fibonacci on Egyptian fraction" paper. Dunton and Grimm stressed the algebraic statement k/kl = 1/l, an algebraic identity, as a fair method (which it is not) to capture Leonardo’s three part rule. Note that the third aspect of Leonardo's method (distinction), titled "third reversed composite," is one of three methods that Dunton and Grimm did not parse, a logical omission that Sigler adopted.

2. Second method (distinction)

When greater numbers are not divisible by the lesser, a phrase offered by Leonardo, was clarified by these examples:

a. 5/6 = (3 + 2)/6

= (1/2 1/3), a quotient name for a numerator

b. 7/8 = (4 + 2+ 1)/8 = (1/2 1/4 1/8)

c. A reverse composite is used to solve for

3/40 = (3/4 0/10) meaning that

3/4 = (3/10 0/4)

was used to solve example problems by applying tables of separations, as Leonardo listed lists parts of 6, 8, 12, 20, 24, 60 and 100. This class of table was reported in the Coptic era, by David Fowler and others. In a broad sense, the RMP 2/nth table was an example of this class of table.

(Again, Sigler stressed Dunton and Grimm modern view per the statement (k + 1)/klm = 1/lm + 1/km), an analysis that insufficiently captured the medieval methodology.)

3. Third method (distinction)

a. 2/11 = (1/6 0/11), parts of 2/11

b. 3/11 = (1/4 0/11) = (1/11 1/4)

c. 6/11 = 1/22 1/2

d. 8/11 = 2/11 + 6/11

meant  that a table of values defined the scope of  distinction two.

(Sigler again cited Dunton and Grimm by a proposed modern identity k/(kl -1) = 1/l + 1/(kl -1) rather than the actual ( n/p – 1/m) = (mn – p)/mp = 1 encoding context).

4. Fourth method (distinction)

This distinction  indirectly applied  aspects of Ahmes' 2/p construction methodologies, where highly composite denominators were selected to solve several examples. The vulgar fraction examples selected by Leonardo were, 19/53, 5/11, 7/11, 6/19 and 7/29. The older methodology began to be re-discovered in 1895 by F. Hultsch, and 1944 by E.M. Bruins by the Hultsch-Bruins method.

a. 19/53 - 1/3 = (3 + 1))/(3 x 53) = 1/159 1/53, meant that

19/53 = 1/159 1/53 1/3, a statement found in Egyptian texts.

b. 5/11 - 1/3 = (3 + 1)/(3x 11) = 1/11 1/33, meant that

5/11 = 1/33 1/11 1/3, again a very old form of style and contents

c. 7/11 - 1/2 = (2 + 1)/(2 x 11) = 1/22 1/11 1/2

d. 6/19 - 1/4 = (4 + 1)/(4x 19) = 1/19 1/76, meant

6/19 = 1/76 1/19 1/4

e. 7/29 - 1/5 = (5 + 1)/(5x29) = 1/29 + 1/145, or

7/29 = 1/145 1/29 1/5 in Leonardo’s notation.

(Sigler did not comment on this distinction. This approach was expanded by Ahmes, 2850 years earlier to solve 30/53 = 28/53 + 2/53 as a second basis of the 2/n table.) 

5. Fifth method (distinction)

a. 9/26 - 1/3 = (1/3 0/26 1/3) = 1/78 1/3

b. 11/26

c. 11/29 = (1/78 1/3 1/3) since

11/29 - 1/3 = (3 + 1)/(93 x 29) = 1/79 1/3

d. 11/62 = (0/62 1/31 1/7), since

11/62 - 1/7 = (14 + 1)/(7x 62), or

11/62 = (0/62 1/7 1/31 1/7), an alternate Leonardo notation.

(Again, Sigler cited few historical or pertinent info to summarize this distinction's important set of examples, at least in the eyes of Leonardo.)

6. Sixth method (distinction)

a. 17/27 - 3/27 = 14/27 - 1/2 = 1/54, meant

17/27 = 1/54 1/9 1/2, since 3/27 was found to reduce the vulgar fraction being converted.

b. (20/53 - 18/48) = (960 - 954)/(18x53), meant

20/53 = 18/48 + 6/(18 x 53) = 18/48 1/8 0/53


"7. Seventh method (distinction)

a. (4/9 - 1/13) = 3/(13 x 49)

= (1/319 0/637 1/617 1/319 1/13), not elegant

b. 4/49 - 1/14 = 7/(14 x 49)

= (1/2 0/49 1/14), elegant

c. 4/49 = 1/7 x (4/7) = 1/7 x (4/7 - 1/2 = 1/14)

= (1/2 0/49 1/14), elegant alternative

One of the first rational numbers that could not be scaled to a 2-term series is 4/13. Leonardo would have selected (4/13 – 1/4) =( 3/53 – 1/18) = 1/468 meant 4/13 = (1/4 + 1/18 + 1/468). 

An older parallel is found in Ahmes' shorthand. In RMP 31 and RMP 36, two rational numbers 28/97 and 30/53 could not be converted. Ahmes applied the first 2/n table approach to achieve a conversion to unit fraction series. Ahmes' arithmetic considered the aliquot parts of 56, and 4 for 28/97 and 30 and 2 for 30/53 per:

28/97 = 2/97 + 26/97 = 2/97*(56/56) + 26/97*(4/4) = (97 + 8 + 7 ) + (97 + 4 + 2 + 1)/388

30/53 = 2/53 + 28/53 = 2/53 *(30/30) + 28/53*(2/2) = (53 + 5 + 2)/1590 + (53 + 2 + 1)/106

A version of the older scaling method was condensed by Arabs and Fibonacci in a subtraction context. Leonardo's seventh distinction likely was extended to include 2/53, that converted 30/53 by:
(30/53 - 1/30) = (53 +5 + 2)/1590 = 1/30 + 1/318 + 1/795 
And 3-term series, i.e. (4/13- 1/4) = (3/53 – 1/18) = 1/468 = 4/13 = 1/4 + 1/18 + 1/468 

Leonardo’s square root data was demonstrated in Chapter 14 “The Finding Square and Cubic Roots , and on the  Multiplication, Division and Subtraction of Them, and on the Treatment of Binomials and Apotomes and their Roots”. Only the details of the square root method, also used by Galileo, Archimedes and most likely Egyptians, will be parsed. 

The inverse proportion method estimated the square root of prime numbers, and any number, in 3-steps to 8-decimals, translated into modern arithmetic:

\PMlinkexternal{Archimedes and Leonardo Square Root}{https://www.academia.edu/10778234/Archimedes_Square_Root_of_3_5_6_7_and_29} outlined the square root of 10 by (3 + 1/6), 1/36, 1/228 and by narrative (3 + 1/6 – 1/228)^2

Sigler should have footnoted:

1.	(3 + 1/6)^2 = 10 + 1/36
2.	1/36 x (6/38) = 1/228
3.	(3 + 1/6 – 1/228)^2 

Steps that applied the binomial theorem (a + b)^2  to a^2 + 2ab + b^2,  10 = a^2 + 2ab, and 1/36 = b^2, in step 1; an inverse proportion 1/36: 2ab, in step 2; and reduced error=  1/36 to (1/228)^2, in step 3.

Sigler’s following Leonardo narrative : “Truly according to geometry and not arithmetic, the measure of any root can be of any number can be found, and it is found in this manner”, an untrue and true statement.  In the medieval math era, like today, the binomial theorem was an algebraic idea and not geometry, hence untrue to that extent. “Double false position”, another geometric method named by 20th century scholars makes the statement true, to that extent.

Taken together the seven Arab and medieval distinctions included older arithmetic methods that Leonardo borrowed to mentally compute, and record vulgar fractions in shorthand notations. Leonardo often converted vulgar fractions to either a  2-term or  a 3-term unit fraction series in final notations, as needed. 

CONCLUSiONS:

1.	The “Liber Abaci” demonstrated uses of vulgar fractions in square root intermediate steps, and in other problems, including the scaling of final rational number answers usually with no hint of an algorithm. Use of vulgar fractions in mental and written arithmetic was common.
2.	The formal medieval rational number system was rigorously demonstrated in the first 1/3 of the “Liber Abaci”, but unequally applied in the remaining 2/3 of the book.  For example, the square root of 10 mentioned as (3 + 1/6 – 1/228)^2 was not recorded in a final rational number notation.
3.	Distinction one contained three older arithmetic methods that may date to an Egyptian scribal style of writing parts of a fraction.  Egyptian scribal long hand decoded from the EMLR and RMP  scaled rational numbers by LCM m/n in a multiplication context. Leonardo scaled 1/18 by factoring 1/2 x 1/9, with 1/2 = 1/3 + 1/6 such that 1/18 = 1/27 + 1/64. The EMLR used this method four times, and the RMP used this method to convert 2/101.
4.	Leonardo’s square root method applied the binomial theorem, taken from algebra, and not by geometry as Leonardo’s narrative seems to report. Geometry was also applied in the” Liber Abaci”  to find roots by double false position method,  one of many medieval topics not analyzed in this introductory paper.
References
•	1 Samuel Borofsky, Elementary Theory of Equations, The MacMillian Company, New York, 1963 (fifth printing). 
•	2. Heinz Leueneburg, Leonardi Pisani Liber Abbaci oder Lesevergngen eines Mathematikers, Mannheim: B. I. Wissenschaftsverlag, 1993. 
•	3. Oystein Ore, Number Theory and its History, McGraw-Hill, 1948. 
•	4.  L. E. Sigler, Fibonacci’s Liber Abaci, Leonardo Pisano’s Book of Calculations, Springer, 2002. 
 


\end{document}
