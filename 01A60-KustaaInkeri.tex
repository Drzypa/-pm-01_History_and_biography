\documentclass[12pt]{article}
\usepackage{pmmeta}
\pmcanonicalname{KustaaInkeri}
\pmcreated{2014-12-16 15:56:49}
\pmmodified{2014-12-16 15:56:49}
\pmowner{pahio}{2872}
\pmmodifier{pahio}{2872}
\pmtitle{Kustaa Inkeri}
\pmrecord{9}{40199}
\pmprivacy{1}
\pmauthor{pahio}{2872}
\pmtype{Biography}
\pmcomment{trigger rebuild}
\pmclassification{msc}{01A60}
\pmsynonym{K. Inkeri}{KustaaInkeri}
\pmsynonym{Kustaa Adolf Inkeri}{KustaaInkeri}

\endmetadata

% this is the default PlanetMath preamble.  as your knowledge
% of TeX increases, you will probably want to edit this, but
% it should be fine as is for beginners.

% almost certainly you want these
\usepackage{amssymb}
\usepackage{amsmath}
\usepackage{amsfonts}

% used for TeXing text within eps files
%\usepackage{psfrag}
% need this for including graphics (\includegraphics)
%\usepackage{graphicx}
% for neatly defining theorems and propositions
 \usepackage{amsthm}
% making logically defined graphics
%%%\usepackage{xypic}

% there are many more packages, add them here as you need them

% define commands here

\theoremstyle{definition}
\newtheorem*{thmplain}{Theorem}

\begin{document}
Kustaa Adolf Inkeri (1908-11-12 \`a 1997-3-16), a Finnish number-theorist, was born in Laitila, Finland.\, He has acted in astronomical observatorium of university and as high school teacher.\,  In the Finnish-Russian wars (\PMlinkexternal{1939--1940}{http://en.wikipedia.org/wiki/Winter_War}, \PMlinkexternal{1941--1944}{http://en.wikipedia.org/wiki/Continuation_War}) he was a company commander and later a staff officer; during a comparatively peaceful \PMlinkescapetext{period of stationary warfare, Inkeri held for the soldiers certain courses in mathematics corresponding to the first stage of university degree} studies.\,After the war, he was first docent and then professor of mathematics in the University of Turku.

Inkeri was the founder of the Finnish school of algebraic number 
theory.  He has published studies concerning Fermat's last theorem, 
\PMlinkname{Catalan's conjecture}{catalansconjecture}, the class number of cyclotomic number fields, 
primality testing, Diophantine approximation.\, In 1951, Inkeri was 
appointed to the Academy of Finland.
%%%%%
%%%%%
\end{document}
