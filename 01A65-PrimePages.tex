\documentclass[12pt]{article}
\usepackage{pmmeta}
\pmcanonicalname{PrimePages}
\pmcreated{2013-03-22 16:29:27}
\pmmodified{2013-03-22 16:29:27}
\pmowner{PrimeFan}{13766}
\pmmodifier{PrimeFan}{13766}
\pmtitle{Prime Pages}
\pmrecord{5}{38664}
\pmprivacy{1}
\pmauthor{PrimeFan}{13766}
\pmtype{Definition}
\pmcomment{trigger rebuild}
\pmclassification{msc}{01A65}
\pmclassification{msc}{01A61}
\pmclassification{msc}{01A60}

\endmetadata

% this is the default PlanetMath preamble.  as your knowledge
% of TeX increases, you will probably want to edit this, but
% it should be fine as is for beginners.

% almost certainly you want these
\usepackage{amssymb}
\usepackage{amsmath}
\usepackage{amsfonts}

% used for TeXing text within eps files
%\usepackage{psfrag}
% need this for including graphics (\includegraphics)
%\usepackage{graphicx}
% for neatly defining theorems and propositions
%\usepackage{amsthm}
% making logically defined graphics
%%%\usepackage{xypic}

% there are many more packages, add them here as you need them

% define commands here

\begin{document}
{\em The Prime Pages} are described by their maintainer, Chris Caldwell, as ``The `Guinness book' of prime number records.'' They contain information on the largest known primes with subclassifications by kinds of primes (e.g., Sophie Germain primes, multifactorial primes, etc.)

The primes that the Prime Pages concern themselves are too large to write out in base 10. Even as a binary integer, the largest known Mersenne prime would require a 32 megabyte file to contain it. Thus the primes are written ``indirectly'' (e.g., $2^{32582657} - 1$ for the largest known Mersenne prime).

The Prime Pages also serve as a clearinghouse of information on the latest primality testing algorithms and programs.

Chris Caldwell teaches mathematics at the University of Tennessee at Martin. The prime pages are at \PMlinkexternal{primes.utm.edu}{http://primes.utm.edu}.
%%%%%
%%%%%
\end{document}
