\documentclass[12pt]{article}
\usepackage{pmmeta}
\pmcanonicalname{LeonardoDaPisa}
\pmcreated{2014-03-22 7:22:34}
\pmmodified{2014-03-22 7:22:34}
\pmowner{Mravinci}{12996}
\pmmodifier{milogardner}{13112}
\pmtitle{Leonardo da Pisa}
\pmrecord{41}{40937}
\pmprivacy{1}
\pmauthor{Mravinci}{13112}
\pmtype{Biography}
\pmcomment{trigger rebuild}
\pmclassification{msc}{01A35}
\pmsynonym{Fibonacci}{LeonardoDaPisa}
\pmsynonym{Leonardo Pisano}{LeonardoDaPisa}

\endmetadata

% this is the default PlanetMath preamble.  as your knowledge
% of TeX increases, you will probably want to edit this, but
% it should be fine as is for beginners.

% almost certainly you want these
\usepackage{amssymb}
\usepackage{amsmath}
\usepackage{amsfonts}

% used for TeXing text within eps files
%\usepackage{psfrag}
% need this for including graphics (\includegraphics)
%\usepackage{graphicx}
% for neatly defining theorems and propositions
%\usepackage{amsthm}
% making logically defined graphics
%%%\usepackage{xypic}

% there are many more packages, add them here as you need them

% define commands here

\begin{document}
\emph{Leonardo da Pisa} (1171 - 1249) Italian mathematician, nicknamed \emph{Fibonacci} (``figlio di Bonacci'') is best remembered for the 1202 AD book Liber Abaci. The book reports practical and theoretical arithmetic, algebra, geometry, and weights and measures mathematics. Leonardo's Arab sources had been gathered from Hellene and other sources, like Babylonian square root, and re-written into base 10 numerals after 800 AD. In the modern era the Fibonacci sequence recalls a possible aspect of Leonardo's math life.

Leonardo's father Gugliemo, better known as ``Bonaccio,'' was a merchant in Bugia, a port now part of Algeria. Bonaccio regularly used Arabic numerals and Egyptian fraction arithymetic in the course of his work. Leonardo was born in Pisa (then its own sovereign republic, now part of Italy) but spent his formative years in Bugia, helping his father. Leonardo described the theoretical side of Arabic-Hindu numeral arithmetic. The base 10 numeral aspect of the {\it Liber Abaci} had followed suggestions of Pope Sylvester. Pope Sylvester in 999 AD required Arabic numerals to be used in Latin mathematical documents. 

Fibonacci's 500 page book was fully translated from Latin to English by L.E. Sigler in 2002. Previously the book was translated by chapters, or by math topic. The first 1/4 of the book demonstrates practical, and theoretical examples written in three subtraction arithmetic styles. The subtraction styles were implemented by Arbs in 800 AD. Arabs had modified the Greek multication styles. Leonardo's theoretical arithmetic otherwise followed Greek and Egyptian math methodologies. For example, Leonardo's arithmetic reports 1500 BCE to 2000 BCE $\frac{2}{n}$ arithmetic that had been translated into Hindu-Arabic numerals by Arabs. Arabs and Leonardo reported three distinct arithmetic notations. The first notation allowed vulgar fractions, and aliquot parts, to be linearly summed to larger vulgar fractions. Egyptian notations had not allowed vulgar fractions in final answers. The second, and third, notations condensed aliquot part information into circle patterns, showing Greeks, like Heron's, factoring style. Sigler named the third notation after Euclid. All three notations reveal early uses of the fundamental theorem of arithmetic. All three notations were condensed, shortening Greek and Egyptian unit fraction answers.

It was in the {\it Liber Abaci} that Fibonacci indirectly mentioned a sequence for which he is famous to modern mathematicians. A form of the sequence may have been known to recreational mathematicians prior to Fibonacci. James Joseph Sylvester, in 1891, suggested that the sequence was used by Fibonacci. Sylvester connected a proposed greedy algorithm to rational number conversions within $n$-steps. Leonardo had written a two-step process based in a non-algorithmic method.

Fibonacci's seventh method reported a sequence of unit fractions written in the first of three medieval notations, citing three alternatives, each using two-steps:

a. $\frac{4}{9} - \frac{1}{13} = \frac{3}{13 \times 49}$

$ = (\frac{1}{319} + \frac{0}{637} + \frac{1}{617} + \frac{1}{319} + \frac{1}{13}$), not elegant

b. $\frac{4}{49} - \frac{1}{14} = \frac{7}{14 \times 49}$

$ = \frac{1}{2} + \frac{0}{49} + \frac{1}{14}$, elegant

c. $\frac{4}{49} = \frac{1}{7} \times \frac{4}{7} = \frac{1}{7} \times (\frac{4}{7} - \frac{1}{2} = \frac{1}{14})$

$ = \frac{1}{2} + \frac{0}{49} + \frac{0}{49} + \frac{1}{14})$ alternate elegant

A 2-step conversion method, using multiples 26 and 6, was used by an EMLR student scribe to convert $\frac{1}{8}$. Ahmes also used a related two-step method to convert $\frac{28}{97}$, solving $\frac{2}{97}$, and $\frac{26}{97}$, by combining the unit fraction series. Hence, Fibonacci's 2-step method was a restatement of earlier traditions.

Leonardo also wrote books on geometry and Diophantine equations, discussing these and other topics in the remaining 374 pages of Liber Abaci.3

Galileo was born before Stevin's two books were approved by the Paris Academy, and thus learned square root from the tradition of Fibonacci: 

Note that scribal shorthand notes suggested by academics prior to Dec. 2012 suggested incomplete square root steps even though major operational aspects of the same class of arithmetic and algebraic pesu steps have been found in the medieval era. In May 2013 the shorthand notes of Galileo reveal the same method was also used by Fibonacci and Archimedes.

In the square root of five step 3 was not needed.

B. square root of six (6) ,

step 1: estimated Q = 2, R = (6 -4)/4 = 1/2 such that

(2 + 1/2)^2 = 6 + (1/2)^2, error1 = 1/4

step 2: reduced 1/4 error that divided by (2 + 1/2)

such that

1/4 x (2/10) = 1/20.

hence (2 + 1/4 - 1/400) =((2 + 99/400)*2, error2 = 1/400

Ahmes, Archimedes and Fibonacci may have stopped at this point and recorded

(2 + 99/400) as a unit fraction series

[2 + (80 + 10 + 8 + 1)/400] =

[2 + 1/5 + 1/40 + 1/50 + 1/400 ]

step 3 (as included Archimedes square root of three method) seemed optional

divided 1/400 by (400/1798) = 1/1798,

hence (2 + 99/400 - (1/1798)^2 = accurate (1/1798)^2

Archimedes’ actual square root method would have recorded

[2+ 1/5 + 1/40 + 1/50 + 1/400]

with a note that a longer series, with an error of (1/1798)^2 was easily found.

Looking forward from the life of Fibonacci, Simon Stevin formalized modern base 10 positional decimals, by using an algorithm, about 300 years after Leonardo's death. John Napier improved the utility of Stevin's decimal notation in ways that modern students appreciate, retaining base 10 numerals reported by Leonardo.


\begin{thebibliography}{3}
\bibitem{1}L.E. Sigler, \emph{Fibonacci's Liber Abaci, Leonardo Pisano's Book of Calculations}, Springer, 2002.
\bibitem{2} Heinz Lüneburg, \emph{Leonardi Pisani Liber Abbaci oder Lesevergnügen eines Mathematikers}, Mannheim: B. I. Wissenschaftsverlag , 1993.
\bibitem{3} Oystein Ore, \emph{Number Theory and its History}, McGraw-Hill, 1948.
\end{thebibliography}

\subsection{External links}
\begin{itemize}
\item \PMlinkexternal{Sigler's blog about his book}{http://liberabaci.blogspot.com}
\item \PMlinkexternal{Heron and Fibonacci addition}{http://arxiv.org/PS_cache/math/pdf/0701/0701624v1.pdf}
\item \PMlinkexternal{2/n tables}{http://rmprectotable.blogspot.com/}
\item \PMlinkexternal{The blog of an EMLR student}{http://emlr.blogspot.com}
\end{itemize}

%%%%%
%%%%%
\end{document}
