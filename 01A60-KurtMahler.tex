\documentclass[12pt]{article}
\usepackage{pmmeta}
\pmcanonicalname{KurtMahler}
\pmcreated{2013-03-22 16:22:48}
\pmmodified{2013-03-22 16:22:48}
\pmowner{PrimeFan}{13766}
\pmmodifier{PrimeFan}{13766}
\pmtitle{Kurt Mahler}
\pmrecord{5}{38523}
\pmprivacy{1}
\pmauthor{PrimeFan}{13766}
\pmtype{Biography}
\pmcomment{trigger rebuild}
\pmclassification{msc}{01A60}

% this is the default PlanetMath preamble.  as your knowledge
% of TeX increases, you will probably want to edit this, but
% it should be fine as is for beginners.

% almost certainly you want these
\usepackage{amssymb}
\usepackage{amsmath}
\usepackage{amsfonts}

% used for TeXing text within eps files
%\usepackage{psfrag}
% need this for including graphics (\includegraphics)
%\usepackage{graphicx}
% for neatly defining theorems and propositions
%\usepackage{amsthm}
% making logically defined graphics
%%%\usepackage{xypic}

% there are many more packages, add them here as you need them

% define commands here

\begin{document}
{\em Kurt Mahler} (26 July 1903 - 25 February 1988) was a British mathematician who proved that $$\sum_{i = 1}^\infty \frac{i}{10^{\sum_{j = 1}^i k}}$$ (where $k$ is the number of digits of $j$ in base 10) is a transcendental number (that is, approximately 0.123456789101112131415161718192021...) He also helped prove that certain cases of Waring's problem do not occur infinitely often.

Born in Germany, Mahler left for Manchester to escape Hitler's genocide. After World War II, Mahler taught in America and Australia.
%%%%%
%%%%%
\end{document}
