\documentclass[12pt]{article}
\usepackage{pmmeta}
\pmcanonicalname{ArabicNumerals}
\pmcreated{2014-12-05 15:36:30}
\pmmodified{2014-12-05 15:36:30}
\pmowner{milogardner}{13112}
\pmmodifier{milogardner}{13112}
\pmtitle{Arabic numerals}
\pmrecord{167}{40287}
\pmprivacy{1}
\pmauthor{milogardner}{13112}
\pmtype{Definition}
\pmcomment{trigger rebuild}
\pmclassification{msc}{01A35}
\pmsynonym{Hindu-Islamic numerals}{ArabicNumerals}
%\pmkeywords{1 to 9}
\pmrelated{base3}

% this is the default PlanetMath preamble.  as your knowledge
% of TeX increases, you will probably want to edit this, but
% it should be fine as is for beginners.

% almost certainly you want these
\usepackage{amssymb}
\usepackage{amsmath}
\usepackage{amsfonts}

% used for TeXing text within eps files
%\usepackage{psfrag}
% need this for including graphics (\includegraphics)
%\usepackage{graphicx}
% for neatly defining theorems and propositions
%\usepackage{amsthm}
% making logically defined graphics
%%%\usepackage{xypic}

% there are many more packages, add them here as you need them

% define commands here

\begin{document}
INTRODUCTION: After 800 CE an un-ciphered Arab notation scaled rational numbers by LCM m, written in a subtraction context by 1/m, recoded sometimes awkward unit fraction series. The revised numeration system replaced the older ciphered Egyptian Greek and LCM m written as m/m in a  multiplication context. The Greek and Egyptian system had scaled rational numbers by LCM m/m from 2050 BCE to 800 AD in the business worlds of both cultures. An informal zero had been used in Egyptian double accounting and Greek math statements, one reason that Arabs had not added zero to their numeration system.

The replacement Arab  notation brought Hindu 1-9 numerals from India and a hint of an algorithm from Babylon/Chaldean math traditions that recorded rational numbers to awkward looking unit fraction series scaled (ciphered) by the same Greek LCM m in a subtraction context. A formal zero in a positional notation was missing until 1585 CE. 

After 800 CE  Arabs and medieval scribes scaled (ciphered) rational numbers n/p by LCM m that considered 

(n/p - 1/m) = [(mn -p)/mp]

considered 0 less than n/p less than 1

n/p was  rational numbers recorded in 2-term unit fraction series that set (mn-p) = 1; and 3-term unit fraction series selected a second LCM m.

The new Arab arithmetic scaled rational numbers n/p by LCM m in a subtraction context by seven conversion rules reported by Fibonacci in 1202 CE. The seven rules scaled rational numbers n/p by LCM m to (mn-p)/mp continued operational properties of the older Egyptian fraction arithmetic rules for addition, subtraction, multiplication and division. Intermediate calculations tried to set remainder numerators (mn -p) equal to 1 (unity) statements by an algorithm. Unit fraction answers recorded 2-term and 3-term unit fraction series. 

The new numeration and subtraction based arithmetic system demonstrated:

(n/p - 1/m) equaled  (mn -p)/mp by considering obvious limitations:

Example, (7/27 - 1/4) = 1/104, meant 7/27 = 1/4 + 1/104

More complex examples included 4/13 that could not be solved by 2-term series.
The more complex 4/13 case was solved by: 

(4/13 - 1/4) = (16 -13)/52 = 3/52 - 1/18 = (54 - 52)/936

In difficult cases like 4/13, 3-term unit fraction series is read in modern arithmetic by:

4/13 = 1/4 + 1/18 + 1/468  

The medieval 2-term and 3-term conversions of rational numbers to concise unit fraction system was attempted to  popularized by Pope Sylvester after 999 CE. Pope Sylvester suggested that Latin schools  use Hindu-Arabic numerals and the simplified arithmetic systems. 

Only with the formal end of the Crusades did the generalized use of  Hindu-Arabic symbols and simplified arithmetic, algebra, geometry and economic trading unit systems become popularized by the "Liber Abaci", written in 1202 CE, Europe's arithmetic book for 250 years.

The Liber Abaci was written by Fibonacci (Leonardo de Pisa), a son of a Pisa merchant that had traveled with his father studying the math and trading unit systems. The Liber Abaci book was used as a Latin school textbook until the unit fraction system began to fall out of use in Europe after Byzantium (Constantinople) was over run by the Ottoman Empire in 1454, an event that closed much of the Silk Road.

MAIN POINTS: Arab mathematicians adopted Hindu 1-9 numerals around 800 CE. The Arab innovation replaced Greek and Hellene ciphered numerals by adding East Indian base 10 numerals in a modified base 10 unit fraction system. The older Greek and Hellene ciphered numeration system had followed a 1,500 year older hieratic Egyptian system that mapped numerals on a one-to-one basis to (sound) symbols. Egyptian scribes employed a formal zero, though not positional, an awkwardness that continued with Hindu-Arabic numerals after 800 CE. 

The medieval 1-9 numeration system commenced in 800 CE motivated Europeans to adopt Arab numerals, Arab algorithms, and non-positional zero elements by a Pope Sylvester 999 CE edict. Europe adopted Hindu-Arabic numerals and Arab and Greek mathematics by studying Fibonacci's \PMlinkexternal{Liber Abaci}{http://liberabaci.blogspot.com/} after 1202 CE. 

After the closure of much of  the Silk Road in 1454 CE, Fibonacci's 250 year old arithmetic book dropped out of use. About the same time Liber Abaci math was translated by Nicolas Chuquet in 1453. Chuquet used a lattice multiplication, double false position, Diophantine indeterminate equations, Babylonian square root, algorithms, and aspects of the Chinese Remainder Theorem methods that arrived years earlier via the Silk Road (as documented by Needham). Zero as a positional number was added 130 years later within a new numeration system and algorithm that encoded rational numbers by the binomial theorem. The well known definition of $$n^0 = 1$$ was an element. Construction details of the modern base 10 decimal system was recorded in 1585 CE by Simon Stevin. Stevin rigorously applied zero in two books, one for science, and one for business, as an exponential positional place-holder. Both books were approved by the Paris Academy. Several scholars have given credit to Hindu-Arabic numerals 800 AD introduction spread to Europe and was popularized for 250 years in Fibonacci's "Liber Abaci".

Elements of Napier's Bones, and Arab- Hellene lattice multiplication method were noted by Fibonacci. Napier popularized the new base 10 decimal system by adding logarithms. Napier's numeration publications facilitated several science activities, including \PMlinkexternal{Galileo}{http://www.ams.org/samplings/feature-column/fc-2013-05}'s 1609 astronomical work and logarithms were used to facilitate Huygens' telescope, and Greek square root as late as Galileo.

The last surviving \PMlinkexternal{Ghobar unit fraction document was written in 1637}{http://mathforum.org/kb/message.jspa?messageID=6945365&tstart=45
}. The text pointed the way to Mecca from Morocco. \PMlinkexternal{Modern Arabic script replaced Ghobar script in the 17th century}{http://mathforum.org/kb/message.jspa?messageID=6945477&tstart=0}, a set of actions that formally ended 3,700 of continuous unit fraction arithmetic and mathematics use in the Egyptian, Greek, \PMlinkexternal{Arab}{http://pds.lib.harvard.edu/pds/view/13518906}, and medieval math worlds. 

BACKGROUND: Prior to 800 CE Greek and Egyptian scribes scaled rational numbers to unit fraction arithmetic in concise ways. The 2,800 year old system scaled and ciphered rational number n/p by LCM m in a multiplication context. Rational numbers n/p were scaled by m/m to mn/mp in a multiplication context by four rules before selected concise unit fraction series representations. Scribes selected the best divisors of denominator mp that summed to numerator mn by \PMlinkexternal{four implicit conversion rules}{http://rmprectotable.blogspot.com/}. The older scribal arithmetic demonstrated concise 2-term, 3-term, 4-term and 5-term unit fraction series representations of rational numbers. 

Scholars in the 20th century did not decode the older Greek and Egyptian proto-number theory aspects of the four scribal rules. By the 21st century scholars began pointing out scribal p and q as prime numbers that scaled n/p by LCM m/m to mn/mp before concise unit fraction series were recorded. 

Egyptian and Greek scribes ciphered numbers onto sound symbols (letters) on a one-to-one basis. Zero was used but was not a positional idea. Babylonians also used a practical zero (Neugebauer) near 2050 BCE when Egyptians used the word sfr for zero in double entry accounting and other applications. 

Greeks mapped numerals and rational numbers onto Ionian and Doric letter symbols by adding ('), ie. 1/2 = to beta', 1/3 = gamma', and so forth, recorded as unit fraction series. The Greek zero symbol recorded an oval topped with two dots, a set of notations that began to be replaced with the arrival of Arabic 1-9 numerals. The best unit fraction series were often not intuitive, and thus difficult for modern scholars to report as originally computed.

CONCLUDING COMMENTS: After 800 CE a subtraction context ciphered rational numbers (n/p - 1/m) to (mn -p)/mp to (mn -p)  recorded 2-term and 3-term series.  Fibonacci reported three Arab unit fraction notations. The most popular Liber Abaci notation was translated by Nicolas Chuquet in 1453 CE. Chuquet used a lattice multiplication, double false position, Diophantine indeterminate equations, Babylonian square root, algorithms, and aspects of the Chinese Remainder Theorem methods arrived 1,500 years earlier via the Silk Road. 

In 1585 CE medieval unit fraction arithmetic was written in Hindu-Arabic 1-9 numerals ended in Europe, though unit fraction aspects continued as late a \PMlinkexternal{Galileo}{http://www.ams.org/samplings/feature-column/fc-2013-05}. The medieval Ghobar script version of the unit fraction system contained arithmetic words and math operations also ended when modern Arabic script replaced Ghobar script in the 17th century. The Greek numeration system was last used by  \PMlinkexternal{Galileo}{http://planetmath.org/squarerootof3567and29} to solve square root problems. 

\begin{thebibliography}{10}
\bibitem{1}  A.B. Chace, Bull, L, Manning, H.P., Archibald, R.C., \emph{The Rhind Mathematical Papyrus}, Mathematical Association of Amnerica, Vol I, 1927. NCTM reprints available. 
\bibitem{2} Milo Gardner, \emph{An Ancient Egyptian Problem and its Innovative Solution, Ganita Bharati}, MD Publications Pvt Ltd, 2006.
\bibitem{3}Richard Gillings, \emph{Mathematics in the Time of the Pharaohs}, Dover Books, 1992.
\bibitem{4} Otto Neugebauer, \emph{Exact Sciences in Antiquity}
\bibitem{5} Oystein Ore, \emph{Number Theory and its History}, McGraw-Hill Books, 1948, Dover reprints available.
\bibitem{6} T.E. Peet, \emph{Arithmetic in the Middle Kingdom}, Journal Egyptian Archeology, 1923.
\bibitem{7} Tanja Pommerening, \emph{"Altagyptische Holmasse Metrologish neu Interpretiert" and relevant phramaceutical and medical knowledge, an abstract,  Phillips-Universtat, Marburg, 8-11-2004, taken from "Die Altagyptschen Hohlmass}, Buske-Verlag, 2005.
\bibitem{8} Gay Robins, and Charles Shute \emph{Rhind Mathematical Papyrus}, British Museum Press, Dover reprint, 1987.
\bibitem{9} L.E. Sigler, \emph{Fibonacci's Liber Abaci: Leonardo Pisano's Book of Calculation}, Springer, 2002.
\bibitem{10} Hana Vymazalova, \emph{The Wooden Tablets from Cairo:The Use of the Grain Unit HK3T in Ancient Egypt, Archiv Orientalai}, Charles U Prague, 2002.
\end{thebibliography}




%%%%%
%%%%%
\end{document}
