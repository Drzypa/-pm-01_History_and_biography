\documentclass[12pt]{article}
\usepackage{pmmeta}
\pmcanonicalname{ElieCartan}
\pmcreated{2013-03-22 18:24:25}
\pmmodified{2013-03-22 18:24:25}
\pmowner{bci1}{20947}
\pmmodifier{bci1}{20947}
\pmtitle{\'{E}lie Cartan}
\pmrecord{20}{41055}
\pmprivacy{1}
\pmauthor{bci1}{20947}
\pmtype{Definition}
\pmcomment{trigger rebuild}
\pmclassification{msc}{01A99}
\pmclassification{msc}{01A60}
\pmclassification{msc}{01A70}
\pmsynonym{\'{E}lie Joseph Cartan}{ElieCartan}
%\pmkeywords{\'{E}lie Joseph Cartan}
%\pmkeywords{The Theory of Spinors (1966)}
\pmrelated{Spinor}
\pmrelated{SpinGroup}

\endmetadata

% this is the default PlanetMath preamble.  as your knowledge
% of TeX increases, you will probably want to edit this, but
% it should be fine as is for beginners.

% almost certainly you want these
\usepackage{amssymb}
\usepackage{amsmath}
\usepackage{amsfonts}

% used for TeXing text within eps files
%\usepackage{psfrag}
% need this for including graphics (\includegraphics)
%\usepackage{graphicx}
% for neatly defining theorems and propositions
%\usepackage{amsthm}
% making logically defined graphics
%%%\usepackage{xypic}

% there are many more packages, add them here as you need them

% define commands here
\usepackage{amsmath, amssymb, amsfonts, amsthm, amscd, latexsym}
%%\usepackage{xypic}
\usepackage[mathscr]{eucal}

\setlength{\textwidth}{6.5in}
%\setlength{\textwidth}{16cm}
\setlength{\textheight}{9.0in}
%\setlength{\textheight}{24cm}

\hoffset=-.75in     %%ps format
%\hoffset=-1.0in     %%hp format
\voffset=-.4in

\theoremstyle{plain}
\newtheorem{lemma}{Lemma}[section]
\newtheorem{proposition}{Proposition}[section]
\newtheorem{theorem}{Theorem}[section]
\newtheorem{corollary}{Corollary}[section]

\theoremstyle{definition}
\newtheorem{definition}{Definition}[section]
\newtheorem{example}{Example}[section]
%\theoremstyle{remark}
\newtheorem{remark}{Remark}[section]
\newtheorem*{notation}{Notation}
\newtheorem*{claim}{Claim}

\renewcommand{\thefootnote}{\ensuremath{\fnsymbol{footnote%%@
}}}
\numberwithin{equation}{section}

\newcommand{\Ad}{{\rm Ad}}
\newcommand{\Aut}{{\rm Aut}}
\newcommand{\Cl}{{\rm Cl}}
\newcommand{\Co}{{\rm Co}}
\newcommand{\DES}{{\rm DES}}
\newcommand{\Diff}{{\rm Diff}}
\newcommand{\Dom}{{\rm Dom}}
\newcommand{\Hol}{{\rm Hol}}
\newcommand{\Mon}{{\rm Mon}}
\newcommand{\Hom}{{\rm Hom}}
\newcommand{\Ker}{{\rm Ker}}
\newcommand{\Ind}{{\rm Ind}}
\newcommand{\IM}{{\rm Im}}
\newcommand{\Is}{{\rm Is}}
\newcommand{\ID}{{\rm id}}
\newcommand{\GL}{{\rm GL}}
\newcommand{\Iso}{{\rm Iso}}
\newcommand{\Sem}{{\rm Sem}}
\newcommand{\St}{{\rm St}}
\newcommand{\Sym}{{\rm Sym}}
\newcommand{\SU}{{\rm SU}}
\newcommand{\Tor}{{\rm Tor}}
\newcommand{\U}{{\rm U}}

\newcommand{\A}{\mathcal A}
\newcommand{\Ce}{\mathcal C}
\newcommand{\D}{\mathcal D}
\newcommand{\E}{\mathcal E}
\newcommand{\F}{\mathcal F}
\newcommand{\G}{\mathcal G}
\newcommand{\Q}{\mathcal Q}
\newcommand{\R}{\mathcal R}
\newcommand{\cS}{\mathcal S}
\newcommand{\cU}{\mathcal U}
\newcommand{\W}{\mathcal W}

\newcommand{\bA}{\mathbb{A}}
\newcommand{\bB}{\mathbb{B}}
\newcommand{\bC}{\mathbb{C}}
\newcommand{\bD}{\mathbb{D}}
\newcommand{\bE}{\mathbb{E}}
\newcommand{\bF}{\mathbb{F}}
\newcommand{\bG}{\mathbb{G}}
\newcommand{\bK}{\mathbb{K}}
\newcommand{\bM}{\mathbb{M}}
\newcommand{\bN}{\mathbb{N}}
\newcommand{\bO}{\mathbb{O}}
\newcommand{\bP}{\mathbb{P}}
\newcommand{\bR}{\mathbb{R}}
\newcommand{\bV}{\mathbb{V}}
\newcommand{\bZ}{\mathbb{Z}}

\newcommand{\bfE}{\mathbf{E}}
\newcommand{\bfX}{\mathbf{X}}
\newcommand{\bfY}{\mathbf{Y}}
\newcommand{\bfZ}{\mathbf{Z}}

\renewcommand{\O}{\Omega}
\renewcommand{\o}{\omega}
\newcommand{\vp}{\varphi}
\newcommand{\vep}{\varepsilon}

\newcommand{\diag}{{\rm diag}}
\newcommand{\grp}{{\mathbb G}}
\newcommand{\dgrp}{{\mathbb D}}
\newcommand{\desp}{{\mathbb D^{\rm{es}}}}
\newcommand{\Geod}{{\rm Geod}}
\newcommand{\geod}{{\rm geod}}
\newcommand{\hgr}{{\mathbb H}}
\newcommand{\mgr}{{\mathbb M}}
\newcommand{\ob}{{\rm Ob}}
\newcommand{\obg}{{\rm Ob(\mathbb G)}}
\newcommand{\obgp}{{\rm Ob(\mathbb G')}}
\newcommand{\obh}{{\rm Ob(\mathbb H)}}
\newcommand{\Osmooth}{{\Omega^{\infty}(X,*)}}
\newcommand{\ghomotop}{{\rho_2^{\square}}}
\newcommand{\gcalp}{{\mathbb G(\mathcal P)}}

\newcommand{\rf}{{R_{\mathcal F}}}
\newcommand{\glob}{{\rm glob}}
\newcommand{\loc}{{\rm loc}}
\newcommand{\TOP}{{\rm TOP}}

\newcommand{\wti}{\widetilde}
\newcommand{\what}{\widehat}

\renewcommand{\a}{\alpha}
\newcommand{\be}{\beta}
\newcommand{\ga}{\gamma}
\newcommand{\Ga}{\Gamma}
\newcommand{\de}{\delta}
\newcommand{\del}{\partial}
\newcommand{\ka}{\kappa}
\newcommand{\si}{\sigma}
\newcommand{\ta}{\tau}
\newcommand{\med}{\medbreak}
\newcommand{\medn}{\medbreak \noindent}
\newcommand{\bign}{\bigbreak \noindent}
\newcommand{\lra}{{\longrightarrow}}
\newcommand{\ra}{{\rightarrow}}
\newcommand{\rat}{{\rightarrowtail}}
\newcommand{\oset}[1]{\overset {#1}{\ra}}
\newcommand{\osetl}[1]{\overset {#1}{\lra}}
\newcommand{\hr}{{\hookrightarrow}}
\begin{document}
\section{\'{E}lie Joseph Cartan}
French mathematician and mathematical physicist;\\
Born: April 9th, 1869 in Dolomieu (near Chamb\'ery), Savoie, Rh\^{o}ne-Alpes, France.\\
Died: May 6th, 1951 in Paris, France.\\
His brother, Louis--a member of the `underground' French Resistance-- was beheaded by the Nazis in December 1943.

\subsection{Formal studies:}
\bigbreak
School inspector Dubost was impressed by \'{E}lie Joseph Cartan's abilities and obtained state funds that paid for 
\'{E}lie to attend the Lyc\'ee in Lyons, completed with `distinction in Mathematics'. 
The state stipend was then extended in order to allow him to study at the \'Ecole Normale Sup\'erieure in Paris. 

\begin{enumerate}
\item Doctoral student in Paris at the \'Ecole Normale Sup\'erieure in 1888;
\item Doctorate in 1894
\end{enumerate}

\subsection{Academic Appointments:} 

\begin{enumerate}
\item 1894--1896 Faculty appointment at the University at Montpellier
\item 1896--1903 Lecturer appointment at the University of Lyon
\item 1903--1909 Professor at the University of Nancy
\item 1909--1912 Lecturer at the Sorbonne in Paris
\item 1912--1920 Chair of Differential and Integral Calculus in Paris
\item 1920--1923 Professor of Rational Mechanics at the Sorbonne in Paris
\item 1924--1940 Professor of Higher Geometry at the Sorbonne in Paris
\item 1940:      Emeritus Professor at 71. 
\end{enumerate}

His son, Henri Cartan, later produced brilliant work in Mathematics; he was a close mentor
and early coworker of the German-born, (perhaps greatest) French mathematician Alexander Grothendieck.
Henri Cartan wrote about 
\PMlinkexternal{his father's work and his own}{http://www-gap.dcs.st-and.ac.uk/~history/Biographies/Cartan.html}:
``{\em [My father] knew more than I did about Lie groups, and it was necessary to use this knowledge for the determination of all bounded circled domains which admit a transitive group. So we wrote an article on the subject together [Les transformations des domaines cercl\'es born\'es, \textbf{C. R. Acad. Sci. Paris}: 192 (1931), 709-712]. But in general my father worked in his corner, and I worked in mine.''}

\subsection{Research Results:}

\'{E}lie J. Cartan achieved a mathematical synthesis of continuous groups, Lie algebras and differential equations; he also produced a complete Theory of Spinors of fundamental interest both in Mathematics and Mathematical Physics. 
He also produced results on the representations of semisimple Lie groups; he then developed applications of 
\PMlinkname{Grassmann algebra}{} to the theory of exterior differential forms. Between 1894 and 1904 he applied his theory of exterior differential forms to a wide variety of problems in differential geometry, classical dynamics, special and general relativity (for example, v. Spinor Theory invented by him in 1913; \'E. Cartan published the two volume work Lecons sur la th\'eorie des spineurs in 1938).\\

Another great French mathematician Jean Dieudonn\'e wrote about \'Elie J. Cartan : \\ 

``{\em He discussed a large number of examples, treating them in an extremely elliptic style that was made possible only by his uncanny algebraic and geometric insight and that has baffled two generations of mathematicians. ''}

In 1945 E.J. Cartan published the book {\em ``Les syst\'emes diff\'erentiels ext\'erieurs et leurs applications geom\'etriques''}. 

\subsection{Honors and Awards:}

 Late in his life, \'Elie Cartan received many awards and honors; only the most prominent ones are listed here.
Honorary degrees from the University of Liege in 1934, and from Harvard University in 1936. \\ 
He was elected to the French Academy of Sciences on March 9th, 1931 and he was Vice-President of the Academy in 1945 and President in 1946. He was awarded three honorary degrees in 1947 from the Free University of Berlin, the {\em University of Bucharest, Romania}, and the Catholic University of Louvain, in Belgium. In 1948, he was awarded an Honorary Doctorate by the University of Pisa, Italy. He was elected a Fellow of the Royal Society of London on May 1st, 1947, the Accademia dei Lincei and the Norwegian Academy.


%%%%%
%%%%%
\end{document}
