\documentclass[12pt]{article}
\usepackage{pmmeta}
\pmcanonicalname{JohnForbesNash}
\pmcreated{2013-03-22 16:35:20}
\pmmodified{2013-03-22 16:35:20}
\pmowner{PrimeFan}{13766}
\pmmodifier{PrimeFan}{13766}
\pmtitle{John Forbes Nash}
\pmrecord{4}{38783}
\pmprivacy{1}
\pmauthor{PrimeFan}{13766}
\pmtype{Biography}
\pmcomment{trigger rebuild}
\pmclassification{msc}{01A60}
\pmclassification{msc}{01A61}
\pmclassification{msc}{01A65}

% this is the default PlanetMath preamble.  as your knowledge
% of TeX increases, you will probably want to edit this, but
% it should be fine as is for beginners.

% almost certainly you want these
\usepackage{amssymb}
\usepackage{amsmath}
\usepackage{amsfonts}

% used for TeXing text within eps files
%\usepackage{psfrag}
% need this for including graphics (\includegraphics)
%\usepackage{graphicx}
% for neatly defining theorems and propositions
%\usepackage{amsthm}
% making logically defined graphics
%%%\usepackage{xypic}

% there are many more packages, add them here as you need them

% define commands here

\begin{document}
\emph{John Forbes Nash} (1928 - ) American mathematician whose life story inspired the film {\it A Beautiful Mind}, with Russell Crowe portraying him.

In his early twenties, Nash was already published in mathematics and economics journals. Two years after marrying Alicia Lopez-Harrison de Lard\'e, Nash was admitted to McLean Hospital to treat his schizophrenia. His favorite prime number being 23, he thought that a {\it Time} cover story on Pope John XXIII was some sort of secret message meant for him. He continued to teach at Princeton while receiving various treatments.

Nash won the Nobel Prize in economics in 1994. Sylvia Nassar's biography of Nash won a Pulitzer Prize in 1998 and was adapted into an Academy Award-winning film in 2001.
%%%%%
%%%%%
\end{document}
