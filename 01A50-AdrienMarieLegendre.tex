\documentclass[12pt]{article}
\usepackage{pmmeta}
\pmcanonicalname{AdrienMarieLegendre}
\pmcreated{2013-03-22 16:39:40}
\pmmodified{2013-03-22 16:39:40}
\pmowner{Mravinci}{12996}
\pmmodifier{Mravinci}{12996}
\pmtitle{Adrien-Marie Legendre}
\pmrecord{5}{38866}
\pmprivacy{1}
\pmauthor{Mravinci}{12996}
\pmtype{Biography}
\pmcomment{trigger rebuild}
\pmclassification{msc}{01A50}
\pmclassification{msc}{01A45}

% this is the default PlanetMath preamble.  as your knowledge
% of TeX increases, you will probably want to edit this, but
% it should be fine as is for beginners.

% almost certainly you want these
\usepackage{amssymb}
\usepackage{amsmath}
\usepackage{amsfonts}

% used for TeXing text within eps files
%\usepackage{psfrag}
% need this for including graphics (\includegraphics)
%\usepackage{graphicx}
% for neatly defining theorems and propositions
%\usepackage{amsthm}
% making logically defined graphics
%%%\usepackage{xypic}

% there are many more packages, add them here as you need them

% define commands here

\begin{document}
\emph{Adrien-Marie Legendre} (1752 - 1833) French mathematician, perhaps best known for the Legendre symbol. Born into wealth, he studied physics in Paris and later taught at a military academy because he liked to teach and not because he needed the money. His earliest work in physics concerned the trajectories of cannonballs. As he moved more towards mathematics, he made major contributions to the study of elliptic integrals.

In the French Revolution, Legendre lost his money. His {\it \'El\'ements de G\'eom\'etrie} was a lucrative book and was much reprinted and translated, but it was his various teaching positions and pensions that kept him at an acceptable standard of living. A mistake in office politics in 1824 led to the loss of his pension and he lived the rest of his years in poverty.

Two streets in Paris are named after Legendre as well as a lunar crater.
%%%%%
%%%%%
\end{document}
