\documentclass[12pt]{article}
\usepackage{pmmeta}
\pmcanonicalname{SquareRootOf3567And29}
\pmcreated{2015-02-22 17:09:56}
\pmmodified{2015-02-22 17:09:56}
\pmowner{milogardner}{13112}
\pmmodifier{milogardner}{13112}
\pmtitle{square root of 3, 5, 6, 7 and 29}
\pmrecord{119}{42615}
\pmprivacy{1}
\pmauthor{milogardner}{13112}
\pmtype{Proof}
\pmcomment{trigger rebuild}
\pmclassification{msc}{01A35}
\pmclassification{msc}{01A20}
\pmclassification{msc}{01A16}

\endmetadata

% this is the default PlanetMath preamble.  as your knowledge
% of TeX increases, you will probably want to edit this, but
% it should be fine as is for beginners.

% almost certainly you want these
\usepackage{amssymb}
\usepackage{amsmath}
\usepackage{amsfonts}

% need this for including graphics (\includegraphics)
\usepackage{graphicx}
% for neatly defining theorems and propositions
\usepackage{amsthm}

% making logically defined graphics
%%\usepackage{xypic}
% used for TeXing text within eps files
%\usepackage{psfrag}

% there are many more packages, add them here as you need them

% define commands here

\begin{document}
I INTRODUCTION: An ancient square root method was decoded in 2012. Scholars for 100 years failed to fully decode Archimedes' three steps that estimated unit fraction series answers to four to nine decimal places (modern standards), a method used by Fibonacci and Galileo, applied three properties of the binomial theorem:

N = (a + b)^2 = a^2 + 2ab + b^2  

found N^2 = a^2 + 2ab  (property 1) as often as possible

with an Error = b^2  (property 2), as often as possible

The first step estimated (a + b) 

with a = largest square number less than N^2

and b = (N^2 - a^2)/2a), 

for example the square root of 10 estimated (3 + 1/6), meant a = 3, b = 1/6

A.  when (a + b)^2 greater than  N^2, b^2 was the error

so that the second step considered the inverse proportion of 2ab: (1/2c)

(property 3)

by multiplied (a + 1/2a)^2 = (2a^2 + 1)/2a x (2c/1)

B.  when (a + b) < N^2, c = [(N^2 - (a +b)^2]

also multiplied (2a^2 + 1) x 1/(2(N^2 - (a + b)^2)

for example, using the square root of 10 case, (3 + 1/6)^2 = (9 + 1/2 + 1/2 + 1/36)

found  property 1: 10 = 9 + 1 

and property 2 Error 1 = 1/36

so that 

property 3 wrote (3 + 1/6) as a vulgar fraction 19/6 was doubled 38/6 

error 1/36 divided by 38/6 found

1/36 x 6/38 = 1/6 x 1/38 = 1/228 = error2 was subtracted from 1/6 meant

(3 + 1/6 - 1/228)^2  = (3 + 18/228)^2, second step estimate

A  third step followed the pattern defined by step two such that (1351/780)^2  was  greater than 3  is  greater than (265/153)^2  per facts connected to Archimedes (explained in detail in this article).

II Background: Previously unresolved aspects of  Achimedes square root reported ancient estimations of upper and lower limits of  3^1/2  problem were reported by \PMlinkexternal{Kevin Brown}{http://mathpages.com/home/kmath038/kmath038.htm} and \PMlinkexternal{E.B. Davis}{http://www.mth.kcl.ac.uk/staff/eb_davies/PDFfiles/209.pdf} with (1351/780)^2  greater than 3  is greater than (265/153)^2, an inverse prportion  methodthat quickly estimated the square root of any prime number accurate to 4 to 9 decimal places,
such that \PMlinkexternal{Archimedes' actual lower limit of 3 + 10/71 of  pi is less than the well known upper limit 3 + 1/7}{http://milan.milanovic.org/math/english/sqart3/sqart3.html} . 

A. Archimedes' actual square root of 3^1/2 and N^1/2  method, decoded in Dec 2012 and Jan 2013, calculated the higher limit to 3  limit(1351/780)^2 by:

1. step 1. guess (1 + 2/3)^2 = 1 + 2/3 + 2/3 +  4/9 = 2 + 1/3 + 4/9, meant  2/9 =   error1, considered the first binomial theorem property.  

2. step 2 reduced error1 2/9 

by dividing 2/9 by 2(1 + 2/3)

note that:  2(1 + 2/3)  applied the second  binomial theorem property.

the two steps that meant

2/9 x (3/10) = 1/15

such that

(1 + 2/3 + 1/15)^2, error2 (1/15)^2 = 1/225 = error2

knowing (1 + 11/15) = 26/15

3. step 3 reduced error2 = 1/225 by dividing by 2 x (26/15) = 52/15

1/225 x (15/52) = 1/15 x (1/52) = (1/780)^2 = error 3

reached

(26/15 - 1/780)^2 = (1351/780)^2 in modern fractions

recorded a unit fraction series that began with step 2 data and subtracted 1/780

(1 + 2/3 + 1/15 - 1/780)^2

as Archimedes would have written

(780 + 520 + 39 + 6 + 5 + 1)/780)^2 = (1 + 2/3 + 1/50 + 1/130 + 1/156 + 1/780)^2

B . The lower limit 265/153 modified step 2, used
  
1/17 rather than 1/15, (1+ 2/3 + 1/17) = (1 + 37/51)

such that (1 + 111/153)changed to (1 + 112/153) = 265/153

II. PROOF: Modern translations of scribal square roots of five (5), six (6), seven (7), (29) and any rational number are demonstrated below (A, B, C, D).

Greek and Egyptian algebraic steps were finite. Decoding algorithmic looking finite arithmetic steps 2, 3 and 4 have been demystified.

A. Computed the square root of five(5) that estimated

(Q + R)^2, R= 1/(1/2Q)

step 1: estimated

Q = 2. R,= (1/4) such that

(2 + 1/4)^2 = 5 + (1/4)^2; error1 = 1/16

step 2, reduced error1 that divided

1/16 by 2(2 + 1/4)= 18/4 such that

1/16 x (4/18) = 1/72 = error2 = (1/72)^2

2 + 1/4 - (1/72)^2)^2 = (2 + 1285/5184)^2;

as Archimedes and Ahmes would have re-written

(2 + 1285/5184) as a unit fraction series:

a} [2 + (864 + 398 + 51 + 1/17 + 6)/5184) =

[2 + 1/6 + 1/13 + 1/39 + 1/864]^2

b} [2 + (864 + 370 + 64 + 6 +1 )/5184] =

[2 + 1/6 + 1/14 + 1/81 + 1/864 + 1/5184]^2

Note that scribal shorthand notes suggested by academics prior to Dec. 2012 suggested incomplete square root steps even though major operational aspects of the same class of arithmetic and algebraic pesu steps have been found  in the medieval era.  In May 2013 the shorthand notes of  \PMlinkexternal{Galileo}{http://www.ams.org/samplings/feature-column/fc-2013-05} reveal the same method was also used by Fibonacci and Archimedes.

In the square root of five step 3 was not needed.

B. square root of six (6) ,

step 1: estimated Q = 2, R = (6 -4)/4 = 1/2 such that

(2 + 1/2)^2 = 6 + (1/2)^2, error1 = 1/4

step 2: reduced 1/4 error that divided by (2 + 1/2)

such that

1/4 x (2/10) = 1/20.

hence (2 + 1/4 - 1/400) =((2 + 99/400)*2, error2 = 1/400

Ahmes,  Archimes and Fibonacci may have  stopped at this point and recorded

(2 + 99/400) as a unit fraction series

[2 + (80 + 10 + 8 + 1)/400] =

[2 + 1/5 + 1/40 + 1/50 + 1/400 ]

step 3 (as included Archimedes square root of three method) was optionasl

divided 1/400 by (400/1798) = 1/1798,

hence (2 + 99/400 - (1/1798)^2 = accurate (1/1798)^2

Archimedes' actual square root method  would have recorded

[2+ 1/5 + 1/40 + 1/50 + 1/400]

with a note that a longer series, with an error of (1/1798)^2 was easily found.

C. square root of seven (7)

step 1: estimates Q = 2, R = 3/4 and (2 + 3/4)^2 =

7 + (3/4)^2 , error1 = 9/16

step 2: divides 9/16 by twice (2 + 3/4) =

(9/16)(4/22) = 9/88 = (1/11 + 1/88)

(2 + 3/4 -9/88) = [2 + 1/2 + 13/88]^2 =

[2 + 1/2 + (8 + 4 + 1)/88]^2 =

[2 + 1/2 + 1/11 + 1/22 + 1/88]^2

step 3 may have been required

divide 9/88 by twice (2 + 1/2 + 13/88) =

(9/88)(88/466) =

9/466 = (1/155 + 1/155 + 1/466)

hence [2 + 1/2 + 1/11 + 1/22 + 1/88 - (2/155 +1/466)]^2 would have been recorded as a unit fraction series

Note that Archimedes and Ahmes paired

(1/22 - 2/155) =

and

(1/88 - 1/466) =

readers may choose the most likely final unit fraction series,

D. ESTIMATE the square root of 29.

step 1 found R = (29-25)/2Q = 4/10 = 2/5

such that 

(5 + 2/5)^2 = 29 + 4/25 = error1

STEP 2

reduce error1 4/25 by dividing by 2(5 + 2/5)

4/25 x 5/54 = 2/27

hence a final unit fraction series converted

(5 + 2/5 - 2/27)^2 by considering

(5 + 1/5 + (27-10)/135)^2 = (5 + 1/5 + 1/9 + 2/135)^2

was accurate to (2/27)^2

NOTE: the conversion of 2/135 to a unit fraction series followed Ahmes 2/n table rules

(2/5)(1/27) = (1/3 + 1/15)(1/27) = 1/51 + 1/405

MEANT THE SQUARE ROOT OF 29 WAS ESTIMATED IN TWO STEPS BY

(5 + 1/5 + 1/9 + 1/51 + 1/405)^2

footnote: Fibonacci's square root of 17 method was appropriately cited as used by \PMlinkexternal{Galileo}{http://www.ams.org/samplings/feature-column/fc-2013-05} though not properly analyzed in every detail. Fibonacci guessed (4 + 1/8)^2 = (17 + 1/64) , and Fibonacci  reduced the  estimated 1/64 error foumd an inverse proportion:1/64 x  8/66 = 1/528 which meant (4 + 1/8 - 1/528)^2 = (2177/528)^2 = 17.000003 is accurate to (1/528)^2  per\PMlinkexternal{Archimedes}{http://planetmath.org/squarerootof3567and29} and not by Newton, as suggested by scholars prior to 2012.

III CONCLUSION

Unit fraction square root was formalized by 2050 BCE and used  by Egyptians, Greeks, Arabs, medieval scribes and  as late as \PMlinkexternal{Galileo}{http://www.ams.org/samplings/feature-column/fc-2013-05}. The method estimated irrational square roots of N by 1-step, 2-step, 3-step and 4-steps methods. Step 1 guessed quotients (Q)  and remainders (R) = n/(2Q) with  n = (N - Q^2). Step 2, 3, and 4 reduced error 1, 2 and 3 associated with the previous step by dividing by 2(Q + R). Two may have been the first square root per: guess (1 + 2/5)^2 = 1 + 24/25, error 1 = 1/25 divided by 2(1 + 2/5) = 1/25 x 5/14 = 1/70 meant second estimate became (1 + 2/5 + 1/70)^2 = 2 + 1/4900. A third step divided 1/4900 by 198/70 = 1/4900 x 70/198 = 1/70 x 1/198 = 1/13860 meant 
(99/70 - 1/13860)^2 = 2 + (1/13860)^2.

\begin{thebibliography}{5}
\bibitem{1}A.B. Chace, Bull, L, Manning, H.P., Archibald, R.C., \emph{The Rhind Mathematical Papyrus}, Mathematical 
\bibitem{2}Marshall Clagett \emph{Ancient Egyptian Science, Volume III}, American Philosophical Society, Philadelphia, 1999.
\bibitem{3}Richard Gillings, \emph{Mathematics in the Time of the Pharaohs}, Dover Books, 1992, PAGE 214-217.
\bibitem{4} H. Schack-Schackenburg, \emph{"Der Berliner Papyreys 6619", Zeitscrift fur Agypyische Sprache} , Vol 38 (1900), pp. 135-140 and Vol. 40 (1902), p. 65f.
\bibitem{5} L. E. Sigler, \emph{Fibonacci's Liber Abaci, Leonardo's Book of Calculation} ,Springer, NY, 2002, page 491. 

\end{thebibliography}

%%%%%
\end{document}
