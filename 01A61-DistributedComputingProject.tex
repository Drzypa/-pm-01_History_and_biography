\documentclass[12pt]{article}
\usepackage{pmmeta}
\pmcanonicalname{DistributedComputingProject}
\pmcreated{2013-03-22 17:20:29}
\pmmodified{2013-03-22 17:20:29}
\pmowner{PrimeFan}{13766}
\pmmodifier{PrimeFan}{13766}
\pmtitle{distributed computing project}
\pmrecord{6}{39696}
\pmprivacy{1}
\pmauthor{PrimeFan}{13766}
\pmtype{Topic}
\pmcomment{trigger rebuild}
\pmclassification{msc}{01A61}
\pmclassification{msc}{01A65}
\pmsynonym{distributed computing}{DistributedComputingProject}

% this is the default PlanetMath preamble.  as your knowledge
% of TeX increases, you will probably want to edit this, but
% it should be fine as is for beginners.

% almost certainly you want these
\usepackage{amssymb}
\usepackage{amsmath}
\usepackage{amsfonts}

% used for TeXing text within eps files
%\usepackage{psfrag}
% need this for including graphics (\includegraphics)
%\usepackage{graphicx}
% for neatly defining theorems and propositions
%\usepackage{amsthm}
% making logically defined graphics
%%%\usepackage{xypic}

% there are many more packages, add them here as you need them

% define commands here

\begin{document}
\PMlinkescapeword{power}
\PMlinkescapeword{project}
\PMlinkescapeword{telescope}
A {\em distributed computing project} is an application of computer networking in which the computing power of several different computers across a network (usually the Internet) is harnessed to tackle one monumental problem. Perhaps the most famous distributed computing project is SETI@home, which lets the Search for Extraterrestrial Intelligence use individual computers to analyze radio telescope data. The popularity of this SETI@home has inspired the use of distributed computing in mathematics, with the most famous of these being the Great Internet Mersenne Prime Search. Seventeen or Bust is another famous mathematical distributed computing project.

Most distributed computing projects involve one server supercomputer to co\"ordinate the rest. The main server parcels out tasks to the other computers and receives results from them. For mathematical distributed computing projects, the verification of results is typically done by a computer at the project headquarters.
%%%%%
%%%%%
\end{document}
