\documentclass[12pt]{article}
\usepackage{pmmeta}
\pmcanonicalname{CarlStormer}
\pmcreated{2013-03-22 17:52:36}
\pmmodified{2013-03-22 17:52:36}
\pmowner{Mravinci}{12996}
\pmmodifier{Mravinci}{12996}
\pmtitle{Carl St{\o}rmer}
\pmrecord{4}{40357}
\pmprivacy{1}
\pmauthor{Mravinci}{12996}
\pmtype{Biography}
\pmcomment{trigger rebuild}
\pmclassification{msc}{01A55}
\pmclassification{msc}{01A60}
\pmsynonym{Carl Stormer}{CarlStormer}
\pmsynonym{Carl St\"ormer}{CarlStormer}

\endmetadata

% this is the default PlanetMath preamble.  as your knowledge
% of TeX increases, you will probably want to edit this, but
% it should be fine as is for beginners.

% almost certainly you want these
\usepackage{amssymb}
\usepackage{amsmath}
\usepackage{amsfonts}

% used for TeXing text within eps files
%\usepackage{psfrag}
% need this for including graphics (\includegraphics)
%\usepackage{graphicx}
% for neatly defining theorems and propositions
%\usepackage{amsthm}
% making logically defined graphics
%%%\usepackage{xypic}

% there are many more packages, add them here as you need them

% define commands here

\begin{document}
\emph{Carl St{\o}rmer} (1874 - 1957) Norwegian physicist, mathematician and historian, best known for his atmospherical research, particularly his work on the aurora borealis. He collaborated with Ludvig Sylow and Elling Holst on the published correspondence of Niels Henrik Abel.

In mathematics, St{\o}rmer is known for St{\o}rmer's theorem and the \PMlinkname{St{\o}rmer numbers}{StormerNumber}.

\begin{thebibliography}{1}
\bibitem{jd} Joseph Warren Dauben \& Christoph J. Scriba, {\it Writing the History of Mathematics: Its Historical Development}. Basel: Birkh\"auser (2002): 153
\end{thebibliography}
%%%%%
%%%%%
\end{document}
