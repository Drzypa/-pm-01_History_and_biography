\documentclass[12pt]{article}
\usepackage{pmmeta}
\pmcanonicalname{RemainderArithmetic}
\pmcreated{2013-03-22 15:49:04}
\pmmodified{2013-03-22 15:49:04}
\pmowner{milogardner}{13112}
\pmmodifier{milogardner}{13112}
\pmtitle{remainder arithmetic}
\pmrecord{165}{37783}
\pmprivacy{1}
\pmauthor{milogardner}{13112}
\pmtype{Topic}
\pmcomment{trigger rebuild}
\pmclassification{msc}{01A16}
\pmsynonym{proto-number theory}{RemainderArithmetic}
\pmsynonym{egyptian math}{RemainderArithmetic}
\pmsynonym{egyptian fractions}{RemainderArithmetic}

% this is the default PlanetMath preamble.  as your knowledge
% of TeX increases, you will probably want to edit this, but
% it should be fine as is for beginners.

% almost certainly you want these
\usepackage{amssymb}
\usepackage{amsmath}
\usepackage{amsfonts}

% used for TeXing text within eps files
%\usepackage{psfrag}
% need this for including graphics (\includegraphics)
%\usepackage{graphicx}
% for neatly defining theorems and propositions
%\usepackage{amsthm}
% making logically defined graphics
%%%\usepackage{xypic}

% there are many more packages, add them here as you need them

% define commands here
\begin{document}
\PMlinkescapeword{variation}
\PMlinkescapeword{partition}
\PMlinkescapeword{scaling}
\PMlinkescapeword{independent}
\PMlinkescapeword{translation}
\PMlinkescapeword{jump}
\PMlinkescapeword{term}
\PMlinkescapeword{word}
\PMlinkescapeword{depth}
\PMlinkescapeword{degree}
\PMlinkescapeword{side}
\PMlinkescapeword{terms}
\PMlinkescapeword{connected}
\PMlinkescapeword{link}
\PMlinkescapeword{parent}
\PMlinkescapeword{structure}
\PMlinkescapeword{links}
\PMlinkescapeword{line}
\PMlinkescapeword{AD}
\PMlinkescapeword{basis}

Remainder arithmetic was formalized in ancient Egypt 4,050 years ago. The Akhmim Wooden Tablet (\PMlinkexternal{AWT}{http://en.wikipedia.org/wiki/Akhmim_Wooden_Tablet}), circa 1950 BCE, defined a weights and measures application within a new method for division. The Moscow Mathematical Papyrus and the Rhind Mathematical Papyrus (RMP) divided a hekat unity by 1/3 to (1/4 + 1/16 + 1/4) hejat + 5 ro. The scribe partitioned the unity (64/64) into binary hekat quotients +  1/320  ro remainders. 

Several classe of hekat quotients + remainders were created, three being hin (1/10), dja (1/64) and ro (1/320). 

The new division method defined a quotient and remainder that was inverse to multiplication, as explained in RMP 38. The inverse feature of division defined the multiplication operation, a two-phase definition that anticipated our modern arithmetic operations. 

One aspect of the hekat division method exactly parsed (64/64) by rational numbers n, limited to the range 1/2 less n less 1/64 such that:

(64/64)/n = Q/64 + (R/n)*(1/64)

with Q/64, a binary quotient, and  

(R/n)(1/64), an Egyptian fraction remainder, 

Ahmes scaled the remainder (R/n)(1/64) by (5/5) obtaining (5R/n)*(1/320)

such that:

(64/64)/n = Q/64 + (5R/n)*ro

replacing the number 1/320 with the word ro.

The quotient and scaled remainder method was used hundreds of times in Middle Kingdom texts. 

Scholars did not fully decode the quotient and remainder method until 2005. 

B. A second aspect of the method was reported in ten RMP 47 division problems per:

100-hekat being divided by 10, 20, 30, 40, 50, 60, 70, 80, 90 and 100 in the form

(6400/64)/n = Q/64 + (5R/n)*ro

For example, the n = 70 case reported Q = 91 and R = 30 (since 6400/70 = 91 + remainder 30) such that:

(6400/64)/70 = (64 + 16 + 8 + 2 + 1)/64 + (150/70)ro =

(1 + 1/4 + 1/8 + 1/32 + 1/64)hejat + (2 + 1/7)ro, reported by Robins-Shute in 1987, an arithmetically correct answer.

However, Ahmes converted (2 + 1/7)ro = (2 + [6/42 = (3 + 2 + 1)/42] = (2 + 1/14 + 1/21 + 1/42)ro 

adopting the 2/n table method that converted 2/101 = [12/606 = (6 + 3 + 2 + 1)/606 = 1/101 + 1/202 + 1/303 + 1/606  

followed an earlier EMLR method that wrote 1/101 = 6/606 = (3 + 2 + 1)/606 = 1/202 + 1/303 + 1/606

C. A third aspect of the division method created sub-units m divided by n, Three m/n 'units' included hin (10/n) dja (64/n) and ro (320/n). The divisor n was no longer limited to 1/64 < n < 64. Divisors n could be any rational number. The Rhind Mathematical Papyrus (RMP), circa 1650 BCE, used the method 29 times in problem 81 scaling m to 10, and 320, such that:

10/n hin 

and,

320/n ro

Unit fraction remainders were written in one column, and an equivalent (64/64)/n binary quotient and scaled remainder was written in an adjoining column. The two column data set was reported by Gillings, yet the exact remainder arithmetic aspect of the (64/64) column was not decoded until 2005.

The Akhmim Wooden Table detailed five examples, divisors n: 3, 7, 10, 11 and 13. The Rhind Mathematical Papyrus defined 29 examples in the range 1/64 < n < 64 in one problem, and several other examples in other problems.

The m/n sub-units allowed m to be 10, 64, 320, and other scaled valu8es. For example, the Ebers Medical Papyrus exclusively used the m/n sub-units, several of which have not be scaled. The volume unit used one-part integer quotients, and non-scaled Egyptian fraction remainders units hin, dja, ro, and so forth. For example, a 1/10 hin unit was written 10/n hin, a dja unit was written 64/n dja, and a ro unit was written 320/n ro.

C. The Akhmim Wooden Tablet text was published in 1901, and analyzed by Daressy in 1906. Daressy analyzed the data line by line, correctly reporting three hekat divisions: divisors 3, 7, and 10. Daressy improperly reported divisors 11 and 13.

Hana Vymazalova, a Charles U., Prague grad student published AWT hekat problems by correcting Daressy's two 1906 errors in 2002. Vymazalova proved that all five two-part answers were returned by the AWT scribe to a hekat unity $\frac{64}{64}$ by a reverse process. Vymazalova showed that the AWT scribe had recorded binary quotients, and scaled Egyptian fraction remainders in an innovative exact method, improperly citing Peet's discussion of Daressy's analysis.

D. The second class of remainder arithmetic units are reported in 2,000 Middle and New Kingdom (NK) medical prescriptions recipes. Several of the Middle Kingdom and New Kingdom sub-units had been garbled by translators. In 2002, a German graduate student, Tanja Pommerening, corrected a dja hekat unit, scaled to 1/64.The unit is needed by physicians reconstituting ancient Egyptian medicines, following ancient recipes.

Dr. Tanja Pommerening's 2005 PhD also discussed the dja as a healing unit in a mathematical sense. The dja healing unit corrected rounded-off errors associated with traditional (Horus-Eye) binary series partitions. The healed definition is recorded in hieratic and hieroglyphic symbols. The dja was written in a one-part remainder arithmetic system, 64/n dja. The healed 1/64 unit had corrected an Old Kingdom rounded off error by returning a missing 1/64 unit.

E. The remainder arithmetic was also used by Ahmes to divide rational numbers a/b Today the modern definition of quotient and remainder division is only based on the a/b fraction side of number theory.

F. Summary: Additional details of the 4,000 year old remainder method are found in the \PMlinkexternal{Akhmim Wooden Tablet}{http://akhmimwoodentablet.blogspot.com}. The second remainder arithmetic method was introduced by red auxiliary numbers creating the optimized, but not optimal, \PMlinkexternal{RMP 2/n table}{http://en.wikipedia.org/wiki/RMP_2/n_table}, and expanded to remainder m/n Egyptian fraction sub-units used in several RMP problems and 2,000 medical prescriptions in the medical texts.


\begin{thebibliography}{6}

\bibitem{1} Georges Daressy, \emph{"Calculs Egyptiens du Moyan EmpireÃÃÃÃâ?, Recueil de Travaux Relatifs  De La  Phioogie et al Archaelogie Egyptiennes Et Assyriennes XXVIII, 1906, 62ÃÃÃÃâ72}, Paris, 1906.
\bibitem{2} Milo Gardner, \emph{An Ancient Egyptian Problem and its Innovative Solution, Ganita Bharati}, MD Publications Pvt Ltd, 2006.
\bibitem{3}Richard Gillings, \emph{Mathematics in the Time of the Pharaohs}, Dover Books, 1992.
\bibitem{4} T.E. Peet, \emph{Arithmetic in the Middle Kingdom}, Journal Egyptian Archeology, 1923.
\bibitem{5} Tanja Pommerening, \emph{"Altagyptische Holmasse Metrologish neu Interpretiert" and relevant phramaceutical and medical knowledge, an abstract,  Phillips-Universtat, Marburg, 8-11-2004, taken from "Die Altagyptschen Hohlmass}, Buske-Verlag, 2005.
\bibitem{6} Hana Vymazalova, \emph{The Wooden Tablets from Cairo:The Use of the Grain Unit HK3T in Ancient Egypt, Archiv Orientalai}, Charles U Prague, 2002.
\end{thebibliography}


%%%%%
%%%%%
\end{document}
