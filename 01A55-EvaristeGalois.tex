\documentclass[12pt]{article}
\usepackage{pmmeta}
\pmcanonicalname{EvaristeGalois}
\pmcreated{2013-03-22 18:08:42}
\pmmodified{2013-03-22 18:08:42}
\pmowner{MathNerd}{17818}
\pmmodifier{MathNerd}{17818}
\pmtitle{\'Evariste Galois}
\pmrecord{9}{40700}
\pmprivacy{1}
\pmauthor{MathNerd}{17818}
\pmtype{Biography}
\pmcomment{trigger rebuild}
\pmclassification{msc}{01A55}
\pmsynonym{Evariste Galois}{EvaristeGalois}

\endmetadata

% this is the default PlanetMath preamble.  as your knowledge
% of TeX increases, you will probably want to edit this, but
% it should be fine as is for beginners.

% almost certainly you want these
\usepackage{amssymb}
\usepackage{amsmath}
\usepackage{amsfonts}

% used for TeXing text within eps files
%\usepackage{psfrag}
% need this for including graphics (\includegraphics)
%\usepackage{graphicx}
% for neatly defining theorems and propositions
%\usepackage{amsthm}
% making logically defined graphics
%%%\usepackage{xypic}

% there are many more packages, add them here as you need them

% define commands here



\begin{document}
{\em \'Evariste Galois} (October 25, 1811 in Bourg-la-Reine - May 31, 1832 in Paris)

His father, Nicholas Gabriel Galois, was the director of a boarding school in Bourg-la-Reine, and his mother, Ad\'{e}la\"{i}de-Marie Demante Galois, "was well-educated, clever and intellectually sophisticated." (Muir, 1961) He went to the same boarding school which Voltaire and Victor Hugo went to.

Unfortunately, the school excessively emphasized the teaching of Latin an Greek, something Galois did not care about. The teachers were indifferent to his mathematical skills, such as mental \PMlinkescapetext{arithmetic}, and critical of how much time he devoted to the study of math.

\PMlinkescapetext{Even} the mathematics teacher was unsupportive of Galois and did not offer any help or advice to get into the \'Ecole Polytechnique, then considered "the best school of mathematics in all of France." (Muir, 1961) Though the admissions exam was easy for him, his lack of manners cost him admittance and he spent another year at the boarding school.

Undeterred, he wrote a paper on the solvability of algebraic equations and submitted it to the French Academy of Sciences, but the paper was misplaced before it could be reviewed. At this time his father killed himself. The young Galois became convinced that France needed Napoleon back. Expelled from the \'Ecole Normale, Galois joined the National Guard and served in an artillery \PMlinkescapetext{unit} that planned to overthrow the King. But lacking public \PMlinkescapetext{support}, the coup d'\'{e}tat was aborted and Galois decided to teach \PMlinkescapetext{algebra}. His first lecture was heard by many students but gradually fewer and fewer students came.

Galois remained politically active and was on several occasions threatened with incarceration. On Bastille Day, Galois showed up to a protest wearing his old National Guard uniform and was arrested for and convicted of illegally wearing a uniform. The French Academy of Sciences sent him a letter while he was in jail, saying that they found his previously lost manuscript but did not understand it.

Soon after being released from jail, Galois was challenged to a duel he did not refuse, despite being certain that he would die regardless of the \PMlinkescapetext{outcome}. The night before the duel he spent trying to finish a book on his \PMlinkescapetext{theories}, but he needed a lot more time to properly write and edit the book. He managed to leave sixty pages.

The next day he showed up to the duel and was shot in the stomach. On his deathbed, Galois identified a policeman as the other duelist. The novelist Alexandre Dumas claimed the other duelist was Pescheux d'Herbinville, but d'Herbinville's description was \PMlinkescapetext{inconsistent} with the description given by local newspapers.

It wasn't until long after his death that his genius began to be recognized. He had solved an important open problem pertaining to the condition by which polynomials can be solved by \PMlinkescapetext{radicals}, and he had laid the groundwork for group \PMlinkescapetext{theory}. Important \PMlinkescapetext{algebraic} concepts now bear his name, such as \PMlinkescapetext{Galois theory} and Galois connection. Owing to his extremely short life, it became \PMlinkescapetext{easy to see} him as a Mozart of math, scorned by his generation and the next and only recently beginning to receive due appreciation. Biographers begun to characterize him as a ``misunderstood genius and victim of the Establishment," but more recently others blame the tragedies of Galois' life on Galois himself. (Stillwell, 2002)

Jane Muir, {\it Of Men and Numbers} (1996) Dover 203 - 216

John Stillwell, {\it Mathematics and its History} (2002) Springer 377
%%%%%
%%%%%
\end{document}
