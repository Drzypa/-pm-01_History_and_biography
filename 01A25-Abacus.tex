\documentclass[12pt]{article}
\usepackage{pmmeta}
\pmcanonicalname{Abacus}
\pmcreated{2013-03-22 17:06:25}
\pmmodified{2013-03-22 17:06:25}
\pmowner{PrimeFan}{13766}
\pmmodifier{PrimeFan}{13766}
\pmtitle{abacus}
\pmrecord{5}{39405}
\pmprivacy{1}
\pmauthor{PrimeFan}{13766}
\pmtype{Definition}
\pmcomment{trigger rebuild}
\pmclassification{msc}{01A25}
\pmclassification{msc}{00A07}

\endmetadata

% this is the default PlanetMath preamble.  as your knowledge
% of TeX increases, you will probably want to edit this, but
% it should be fine as is for beginners.

% almost certainly you want these
\usepackage{amssymb}
\usepackage{amsmath}
\usepackage{amsfonts}

% used for TeXing text within eps files
%\usepackage{psfrag}
% need this for including graphics (\includegraphics)
%\usepackage{graphicx}
% for neatly defining theorems and propositions
%\usepackage{amsthm}
% making logically defined graphics
%%%\usepackage{xypic}

% there are many more packages, add them here as you need them

% define commands here

\begin{document}
\PMlinkescapeword{frame}
\PMlinkescapeword{represent}
\PMlinkescapeword{near}
\PMlinkescapeword{place}
\PMlinkescapeword{side}

The {\em abacus} is a computational tool of ancient history. In its simplest form, it is a frame with a number of rods, and on each rod there are beads to represent numbers. Usually, the beads on the leftmost rod represent units, those on the second leftmost rod represent tens, then hundreds, etc. The Chinese added a separator near the top, leaving two beads in the ``heaven'' part of the device to represent five of a given place value, e.g., a ``heaven'' bead on the third rod from the left represents 500 while a bead on the ``earth'' side of the same rod stands for just 100. For calculations involving money, where the unit of currency is usually divided into a hundred cents, the two leftmost rods can be used to represent the cent values while the third rod for the left is then the unit rod.

\begin{thebibliography}{1}
\bibitem{jd} J. Dilson, {\it The Abacus: A Pocket Computer}. New York: St. Martin's Press (1994)
\end{thebibliography}

%%%%%
%%%%%
\end{document}
