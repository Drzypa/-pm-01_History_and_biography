\documentclass[12pt]{article}
\usepackage{pmmeta}
\pmcanonicalname{BenoitMandelbrot}
\pmcreated{2013-03-22 16:58:05}
\pmmodified{2013-03-22 16:58:05}
\pmowner{PrimeFan}{13766}
\pmmodifier{PrimeFan}{13766}
\pmtitle{Beno\^it Mandelbrot}
\pmrecord{5}{39242}
\pmprivacy{1}
\pmauthor{PrimeFan}{13766}
\pmtype{Biography}
\pmcomment{trigger rebuild}
\pmclassification{msc}{01A65}
\pmclassification{msc}{01A61}
\pmclassification{msc}{01A60}
\pmsynonym{Benoit Mandelbrot}{BenoitMandelbrot}

\endmetadata

% this is the default PlanetMath preamble.  as your knowledge
% of TeX increases, you will probably want to edit this, but
% it should be fine as is for beginners.

% almost certainly you want these
\usepackage{amssymb}
\usepackage{amsmath}
\usepackage{amsfonts}

% used for TeXing text within eps files
%\usepackage{psfrag}
% need this for including graphics (\includegraphics)
%\usepackage{graphicx}
% for neatly defining theorems and propositions
%\usepackage{amsthm}
% making logically defined graphics
%%%\usepackage{xypic}

% there are many more packages, add them here as you need them

% define commands here

\begin{document}
\PMlinkescapeword{term}

\emph{Beno\^it Mandelbrot} (1924 - ) French American mathematician best known for the Mandelbrot set, coined the term ``fractal.'' He applied chaos \PMlinkescapetext{theory} to the study of astronomy, economics, etc.

Born in Poland to Jews from Lithuania, he was 11-years-old when his family moved to France before the Nazi invasion. \PMlinkescapetext{Even} as World War II was in full swing, Mandelbrot continued his college studies with Gaston Julia, who introduced him to fractals. Unlike most Jewish mathematicians, Mandelbrot didn't make his way to the United States until after the war, working with John von Neumann in New Jersey, where he met his wife, Aliette Kagan. The couple moved to Switzerland, but came back to New York. Mandelbrot worked for IBM for more than thirty years. Then he taught at Yale for more than twenty years.
%%%%%
%%%%%
\end{document}
