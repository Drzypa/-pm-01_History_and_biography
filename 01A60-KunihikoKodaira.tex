\documentclass[12pt]{article}
\usepackage{pmmeta}
\pmcanonicalname{KunihikoKodaira}
\pmcreated{2013-03-22 16:12:46}
\pmmodified{2013-03-22 16:12:46}
\pmowner{PrimeFan}{13766}
\pmmodifier{PrimeFan}{13766}
\pmtitle{Kunihiko Kodaira}
\pmrecord{7}{38309}
\pmprivacy{1}
\pmauthor{PrimeFan}{13766}
\pmtype{Biography}
\pmcomment{trigger rebuild}
\pmclassification{msc}{01A60}
\pmsynonym{Kodaira Kunihiko}{KunihikoKodaira}

\endmetadata

% this is the default PlanetMath preamble.  as your knowledge
% of TeX increases, you will probably want to edit this, but
% it should be fine as is for beginners.

% almost certainly you want these
\usepackage{amssymb}
\usepackage{amsmath}
\usepackage{amsfonts}

% used for TeXing text within eps files
%\usepackage{psfrag}
% need this for including graphics (\includegraphics)
%\usepackage{graphicx}
% for neatly defining theorems and propositions
%\usepackage{amsthm}
% making logically defined graphics
%%%\usepackage{xypic}

% there are many more packages, add them here as you need them

% define commands here

\begin{document}
\PMlinkescapeword{service}
\PMlinkescapeword{Axis}
\PMlinkescapeword{powers}
\PMlinkescapeword{States}
\PMlinkescapeword{axis}
\PMlinkescapeword{states}

\emph{Kunihiko Kodaira}, the first Japanese mathematician to receive the Fields medal, was best known for his work on the theory of complex manifolds and algebraic geometry.

Born in Nagano on March 16, 1915, Kodaira taught in Tokyo during World War II and was pressed into cryptographic service for the Axis powers. After the war, he came to the United States, where he learned of the latest advances on Hodge theory. He collaborated with D. C. Spencer on building up the deformation theory of complex structures on manifolds.

Kodaira died in Kofu on July 26, 1997. The Kodaira dimension is named in his honor.
%%%%%
%%%%%
\end{document}
