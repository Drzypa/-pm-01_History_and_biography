\documentclass[12pt]{article}
\usepackage{pmmeta}
\pmcanonicalname{HibehPapyrus}
\pmcreated{2014-12-14 7:49:00}
\pmmodified{2014-12-14 7:49:00}
\pmowner{milogardner}{13112}
\pmmodifier{milogardner}{13112}
\pmtitle{Hibeh Papyrus}
\pmrecord{83}{42283}
\pmprivacy{1}
\pmauthor{milogardner}{13112}
\pmtype{Definition}
\pmcomment{trigger rebuild}
\pmclassification{msc}{01A20}
\pmclassification{msc}{01A16}
\pmclassification{msc}{01A15}
\pmsynonym{Egyptian math}{HibehPapyrus}
%\pmkeywords{Egyptian fractions}
\pmdefines{unit fractions}
\pmdefines{Greek}

\endmetadata

% this is the default PlanetMath preamble.  as your knowledge
% of TeX increases, you will probably want to edit this, but
% it should be fine as is for beginners.

% almost certainly you want these
\usepackage{amssymb}
\usepackage{amsmath}
\usepackage{amsfonts}

% used for TeXing text within eps files
%\usepackage{psfrag}
% need this for including graphics (\includegraphics)
%\usepackage{graphicx}
% for neatly defining theorems and propositions
%\usepackage{amsthm}
% making logically defined graphics
%%%\usepackage{xypic}

% there are many more packages, add them here as you need them

% define commands here

\begin{document}
The \PMlinkexternal{Hibeh Papyrus}{http://mathforum.org/kb/plaintext.jspa?messageID=5539288} was written between 300 BCE to 270 BCE by a Greek scribe that lived in Hellenized Egypt. The numerical Greek data was transliterated by David Fowler and Eric Turner in Historia Mathematics, 10 (1983), 344-358. Units of Greek days were recorded in 1/45 parts unit fraction arithmetic. 

An implicit n/45 table was re-constructed by Fowler and other scholars created a 90 percent accurate translation. To decode an additional 5 percent of the text Greek denominators Greek letters  alpha = 1, beta = 2 and so forth denominators must be understood with letters followed by ('). The (') symbol denoted unit fractions, i.e. 1/2 = beta'. That is, unit fraction denominators n recorded 1/n as n' in ciphered Greek letters. 

One of the n/45 table entries 13/45 recorded delta' lambda' rho-pi'

13/45 data that is translated to a concise unit fraction series 

1/4 1/30 (1/(80 +100) 

meant

13/45 = 1/4 + 1/30 + 1/180

The scribal narrative offered alternate series for several n/45 table entries. The scribe  referenced the unit fraction encoding method commonly used in the 300 BCE period. The text mapped numerals onto Greek alpha, beta, gamma method that was adopted from a closely related Egyptian Middle Kingdom system (2050 BCE to 1550 BCE).

Note that  LCM 4 scaled 13/45 to 52/180 such that: 

52/180 = (45 + 6 + 1)/180 = 1/4 + 1/30 + 1/180

In total the 44 n/45 table entries three LCMs 1, 2 and 4 encoded the entire text making the text the simplest unit fraction series reported in any Greek or Egyptian math text.

Additional scribal n/45 scaled facts added back complete arithmetic sentences. In summary, the implicit n/45 table explicitly cited 31 members. There were 13 missing members: 5/45, 6/45, 12/45, 15/45, 18/45, 20/45, 27/45, 30/45, 31/45, 33/45 35/45, 39/45 and 40/45. 

Added back scaled scribal intermediate calculations of each n/45 conversion agrees with Fowler's 1983 analysis. Only least common multiple LCM m goes beyond Fowler's analysis. The complete n/45 tables assists the reader to understand 27 length of day and night of festival date as vivid calendar fact markers by first grasping the n/45 table:

1. 1/45 = 1/45 used LCM 1 (used three times)

2. 2/45 =  1/30 + 1/90  meant 4/90 = (3 + 1)/90 used LCM 2 

3. 3/45 = 1/15 used LCM 1 

4. 4/45 = 1/15 + 1/45  meant 4/45= (3 + 1)/45 used LCM 1 (used four times) an aternate

* 4/45 = 1/12 + 1/180 + meant 16/90 = (15 + 1)/180 used LCM 4 (once)  

5. 5/45 = 1/9 used LCM 1  (missing)

6. 6/45 = 1/10 + 1/30 meant 12/90 = (9 + 3)/90 used LCM 2 (missing)

7. 7/45 = 1/9 + 1/30 + 1/90 meant 14/90 = (10 + 3 + 1)/90 used LCM 2 (used twice) 

8. 8/45 = 1/9 + 1/90 meant 16/90 = (15 + 1)/90 used LCM 2 (used twice)

9. 9/45 = 1/5 used LCM 1

10. 10/45  = 1/5 + 1/45  meant (9 + 1)/45 used LCM 1

11. 11/45 = 1/9 + 1/10 + 1/30 meant 22/90 = (10 + 9 + 3)/90 used LCM 2

12. 12/45 = 1/9 + 1/10 + 1/18  meant  24/90 = (10 + 9 + 5)/90 used LCM 2 (missing)

13. 13/45 = 1/4 + 1/30 + 1/180  meant 52/180 = (45 + 6 + 1)/180 used LCM 4

14. 14/45 = 1/4 +  meant 56/180 = (45 + 9 + 2)/90 used LCM 4  (missing)

15. 15/45 = 1/3 meant the scribe used LCM 1 (missing)

16. 16/45 1/3 + 1/45 meant 16/45 = (15 + 1)/45 used LCM 1 (used twice)

17. 17/45 = 1/3 + 1/30 + 1/90 meant 34/90 =(30 + 3 + 1)/90 used LCM 2 

18. 18/45 = 1/3 + 1/15 meant  18/45 = (15 + 3)/45 used LCM 1 (missing)

19. 19/45 = 1/3 + 1/15 + 1/45 meant 19/45 = (15 + 3 + 1)/45 used LCM 1 (used twice)

20. 20/45 = 1/3 + 1/9 meant 20/45= (30 + 10)/90 = 40/90 used LCM 2 (missing)

21. 21/45 1/3 + 1/10 + 1/30 meant 42/90 = (30 + 9 + 3)/90 = 42/9 used LCM 2 (used twice)

22. 22/45 = 1/3 + 1/10 + 1/18  meant 44/90 = (30 + 9 + 5)/90 used LCM 2? 

23. 23/45 = 1/2 + 1/90 meant  46/90=(45 + 1)/90 = 46/90 used LCM 2

24. 24/45 = 1/2 + 1/30 meant 48/90 = (45 + 3)/90 used LCM 2 (used three times)

25. 25/45 = 1/2 + 1/30 + 1/45 meant 50/90 = (45 + 3 + 2)/90 used LCM 2

26. 26/45 = 1/2 + 15 + 1/90 meant 52/90 = (45 + 6 + 1)/90 used LCM 2 (used twice)

27. 27/44 = 1/2 + 1/15 + 1/30 meant 54/90 = (45 + 6 + 3)/90 used LCM 2 (missing)

28. 28/45  = 1/2 + 1/10 + 1/45 meant 56/90 = (45 + 9 + 2)/90 used LCM 2

29. 29/45 = 1/2 + 1/9 + 1/30 meant 58/90 = (45 + 10 + 3)/90 used LCM 2 (used twice)

*29/58 = 1/2 + 1/10 + 1/30 + 1/90 meant 58/90 = (45 + 9 + 3 + 1)/90

30. 30/45 = 2/3 meant 60/90 used LCM 2  (missing)

31. 31/45 = 2/3 + 1/45 meant 62/90 = (60 + 2)/90 used LCM 2 (missing)

32. 32/45 = 2/3 + 1/30 + 1/90 meant 64/90 = (60 + 3 + 1)/90 used LCM 2

33. 33/45 = 2/3 + 1/15  meant 66/90 = (60 + 6)/90 used LCM 2 (missing)

34. 34/45 = 2/3 + 1/15 + 1/45 meant 68/70 = (60 + 6 + 3)/90 used LCM 2 (used twice) 

35. 35/45 = 2/3 + 1/10 + 1/90 meant 70/90 = (60 + 9 + 1)/90 used LCM 2 (missing)

36. 36/45 = 2/3 + 1/10 + 1/30 meant 72/90 = (60 + 9 + 3)/90 used LCM 2 (used twice)

37. 37/45 = 2/3 + 1/10 + 1/30 + 1/45 meant 74/90 = (60 + 9 + 3 + 2) used LCM 2 (used twice)

38. 38/45 = 2/3 + 1/6 + 1/90 meant 76/90 = (60 + 15 + 1) used LCM 2 (used twice)

39. 39/45 = 2/3 + 1/6 + 1/30 meant 78/90 = (60 + 15 + 3) used LCM 2 (missing)

40. 40/45 = 2/3 + 1/6 + 1/18 meant 80/90 = (60 + 15 + 5)/90 used LCM 2 (missing)

41. 41/45 = 2/3 + 1/5 + 1/30 + 1/90 meant 82/90 = (60 + 18 + 3 + 1)/90 used LCM 2 (used four times) 

42. 42/45 = 2/3 + 1/4 + 1/60 meant 168/180 = (120 + 45 + 3)/180 used LCM 4 

43. 43/45 = 2/3 + 1/4 + 1/30 + 1/180 meant 172/180 = (120 + 45 + 6 + 1)/180 used LCM 4

44. 44/45 = 2/3 + 1/4 + 1/20 + 1/90 meant 176/180 = (120 + 45 + 9 + 2)/180 used LCM 4 (used twice).

LCM 1 meant the divisors of 45: 15, 9, 5, 3, and 1 were summed to eight numerators of 1/45, 3/45, 4/45, 5/45, 9/45, 10/45, 15/45, and 18/45 that allowed ciphered unit fraction series to be recorded.

LCM 2 meant the divisors of 90: 45, 30, 18, 15, 10, 9, 6, 5, 4, 3, 2, 1 were summed to 31 n/45 numerator as intermediate steps, and other n/45 conversions as unit fraction series alternatives.

LCM 4 meant the divisors of 180: 90, 60, 30, 20, 18, 15, 12, 10, 9, 6, 5, 4, 3, 2, 1 appeared in five numerators 13/45, 14/45, 42/45, 43/45, and 44/45, and other n/45 conversions as unit fraction series alternatives.

One purpose of the papyrus listed 27 Greek festivals recorded within an Egyptian civil calendar verified by length of day and night recorded in exact unit fraction series: \PMlinkexternal{local month names}{http://www.mlahanas.de/Greeks/Measurements2.htm} and \PMlinkexternal{classical Greek local calendars}{http://www.mlahanas.de/Egypt/EgyptianCalendar.html}. Ciphered Greek unit fraction series considered an analysis provided by David Fowler and Eric Turner: Hibeh Papyrus Historia Mathematics, 10 (1983), 344-359 and other analysis reported by Bruce Friedman. The LCM method was consistent with Egyptian MK texts, two of which are linked \PMlinkexternal{references}{http://rmprectotable.blogspot.com/}. \PMlinkexternal{The entire Hibeh Papyrus included a wide range of subjects}{http://www.archive.org/stream/hibehpapyri01egypuoft/hibehpapyri01egypuoft_djvu.txt}.

Legend: NT = night time, DT = day time

1. Phaophi; Hathyr; then CHOIAK.

a. 146,55a Iota(10) Gamma(3) iota(10) beta(2) mu(40) epsilon(5) [E/D]

b. mu epsilon= 45, NT 13 + 4/45

c. 146,55b Iota (10) beta(2) epsilon(5) lambda(30) qoppa(90) [E/D], DY 10+41/45 

2. [CHOIAK 16], Arcturus Rises

a. 146,56 Iota(10) Sigma(6) night 16th

b. 146,57a Iota(10) Beta(2) beta(2) iota(10) epsilon(5) mu(40) epsilon(5) [E/D], NT 12+34/45

d. 146,57b Iota(10) Alpha(1) theta(9) iota(10) lambda(30), DY 11+ 11/45

3. [CHOIAK 26], Corona rises

a. 146,58 Kappa(20) Sigma(6), six! 26th

b. 146,60a Iota(10) Beta(2) <(?) lambda(30) [E/D],  NT 12+ 8/15

c. 146,60b Iota(10) Alpha(1) gamma(3) iota(10) lambda(30),  DY 11+ 7/15

4. OSIRIS [TUBI 5] 146,62a Epsilon(5) Tubi 5th

5. [TUBI 20] AN EQUINOX (spring), Feast of Phitorois

a. 146,62b Kappa(20) 20th

b. 146,63a Iota(10) Beta(2), NT 12 (hrs)

c. 146,63b Iota(10) Beta(2), DY 12 (hrs)

6. [TUBI 27], Pleides set in the evening

a. 146,64 Kappa(20) Zeta(7) 27th

b. 146,65 Iota(10) Alpha(1) beta(2) 5(6) qoppa(90),  NT 11+ 38/45

c. 146,66 Iota(10) Beta(2) iota(10) lambda(30) mu(40) epsilon(5), DY 12 + 7/45

7. [MECHEIR 6] Sun enters Taurus, Hyades sets in the evening, 146, Mecheir 6th

a. 146,68 Iota(10) Alpha(1) <(?) iota(10) lambda(30) epsilon(5), NT 11 + 29/45

b. 146,69 Iota(10) Beta(2) gamma(3) mu(40) epsilon(5),  DY 12 + 16/45

8. [MECHEIR 19], Lyra rises in the evening, Assembly at Sais

a. 146,73c Iota(10) Theta(9) Lyra 19th or 16th?

b. 146,75a Iota(10) Alpha(1) gamma(3) iota(10) epsilon(5) mu(40) epsilon(5) [E/D], NT 11 + 19/45

d. 146,75b Iota(10) Beta(2) <(?) iota(10) epsilon(5) omicron(?70), DY 12 + 26/45

9. [MECHEIR 20+ ?] Orion sets or rises?

a. 146,79 Kappa(20)+? [MISSING] 20+

b. 146,80 Iota(10) Alpha(1) +? [MISSING] NT 11+

c. 146,81 Iota(10) Beta(2) Beta(2) [MISSING], DY 12 + 2?.

10. [MECHEIR 27], Lyra sets in the evening, Feast of Prometheus

a. 147,83 Kappa(20) Zeta(7) 27th

b. 147,84a Iota(10) Alpha(1) 5(6) qoppa(90) NT 11+8/45

c. 147,84b Iota(10) Beta(2) Beta(2) iota(10) lambda(30) mu(40) epsilon(5), DY 12+ 37/45

11. [PHAMENOTH 4], Sun enters Gemini, Capella rises in the morning

a. 147,88 Delta(4) 4th

b. 147.89 Iota(10) Alpha(1) epsilon(5) mu(40) NT 11+1/45

c. 147,90 Iota(10) Beta(2) Beta(2)delta(4) kappa(20) qoppa(90) DY 12+44/45

12. [PHAMENOTH 5], Scorpio begins to set in the morning

a. 147,90 Epsilon(5) 5th

b. 147,91 Iota(10) Gamma(3?) NT 11?

c. 147,92 Iota(10) Gamma(3) DY 13

13. [PHAMENOTH 9],Feast of Edu among the Egyptians

147,92 Theta(9) 9th

14. [PHAMENOTH 12], Scorpio sets completely in the morning

a. 147,93 Iota(10) Beta(2) 12th

b. 147,94 Iota(10) Beta(2) 5(6) qoppa(90) NT 10+38/45

c. 147,95 Iota(10)Gamma(3) iota(10) lambda(30) lambda 30) mu(40) epsilon(5), DY 13+ 7/45

15. [PHAMENOTH 13], Pleiades rise in the morning

147,95 Iota(10) Gamma (3) 13th

16. [PARMOUTHI 3] Sun enters Cancer, Aquila rises in the morning

a. 147,107 Gamma(3) 3rd

b. 147,109a Iota(10) gamma(3) lambda(30) qoppa(90) [E/D] NT 10+ 17/45

c. 147,109b Iota(10) Gamma(3)<(?) qoppa(90) mu(40) epsilon(5) DY 13+ 28/45

17. [PARMOUTHI 11] Delphinus rises in the evening

a. 148,110 Iota(10)Alpha(1) 11th

b. 148,111a Iota(10)epsilon(5) [E/D] NT 10+ 1/5

c. 148,111b Iota(10)Gamma(3) beta(2) iota(10) alpha(1) lambda(30).  DY 13+ 4/5

18. [PARMOUTHI 17] Orion rises in the morning

a. 148,113 Iota(10)Zeta(7) 17th

b. 148,114a Iota(10) iota(10) epsilon(5) [E/D] NT 10+ 1/15

c. 148,114b Iota(10) Gamma(3) beta(2) delta(4) xi(60) DY 13+ 14/15

19. [PARMOUTHI 20] The Sun rises in the same place for 3 days.

a. 148,115 Kappa(20) 20th

b. 148,115a Iota(10) NT 10

c. 148,115b Iota(10) Delta(4) DY 14

d. 148,117 Gamma(3) (nuepas) 3 Days

20. [PARMOUTHI 21]

a. 148,117 Kappa(20) Alpha(1) 21st

b. 148,117b Iota(10) NT 10

c. 148,118 Iota(10)Delta(4) DY 14

21. [PARMOUTHI 22]

a. 148,118 Kappa(20) Beta(2) 22nd

b. 148,118b Iota(10) NT 10

c. 148, Iota(10)Delta(4) DY 14

22. [PARMOUTHI 23]

a. 148,119 Kappa(20) Gamma(3) 23rd

b. 148,119b Iota(10) NT 10

c. 148,120 Iota(10)Delta(4) DY 14

24. [PARMOUTHI 24] Summer Solstice, the night gains upon the day by 1/45 of an hour

a. 148,120b Kappa(20)Delta(4) 24th

b. 148,122 mu(40) epsilon(5) 1/45

c. 148,123 Iota(10) epsilon(5) mu(40) NT 10+ 1/45

d. 48,124 Iota(10)Gamma(3) beta(2) delta(4) kappa(20) qoppa(90), DY 13+ 44/45

25. [PARMOUTHI 25],Etesian winds begin to blow, river begins to rise

a. 148,124 Kappa(20) Epsilon(5) 25th

b. 148,127 Iota(10) lambda(30) qoppa(90), NT 10+ 2/45

c. 148,128 Iota(10)Gamma(3)beta(2) delta(4) lambda(30) rho(100) pi(80), DY 13+ 43/45

26. [PACHON 6], Sun enters Leo, Vindemitor rises?

a. 148,129 5(6) 6th

b. 148,131 Iota(10) delta(4) lambda(30) rho(100) pi(80) NT 10+ 13/45

c. 148,132 Iota(10)Gamma(3) [Beta(2) lambda(30)] qoppa(90) DY 13+ 32/45

27. Orion rises completely in the morning:

a. 148,132 Theta(9) 9th: 148,134a Iota(10) gamma(3) mu(40) [ epsilon](5) [E/D] NT 10+ 16/45

b. 148,134b Iota(10)Gamma(3) <(?) iota(10) lambda(30) qoppa(90) DY 13+ 29/45

David Fowler mentioned the 1202 AD Liber Abaci (written by Fibonacci) used unit fraction arithmetic without mentioning that a major Arab algorithm change took place after 800 AD. Prior to 800 AD the Akhmim Papyrus, Hibeh Papyrus, and the RMP used non-algorithm LCM m methods. Scribes encoded 4/13 by LCM 4 to 16/52 =(13 + 2+ 1)/52 = 1/4 + 1/26 + 1/52 (in the RMP and Akhmim Papyrus). Fibonacci applied an algorithm and LCM 4 to 4/13. Fibonacci subtracted 1/4 from 4/13, obtained 3/52, subtracted 1/18 from 3/52, and obtained 4/13 = 1/4 + 1/18 + 1/468. 

Summary: LCM 1, 2 and 4 scaled 27 festival dates and day and night lengths to trivial and non-trivial unit fraction series. The HP LCM scaling method followed a 1,500 year older Egyptian Middle Kingdom tradition. The Egyptian-Greek LCM unit fraction conversion method was continuously used for 2,800 years until \PMlinkexternal{800 AD}{http://mathforum.org/kb/message.jspa?messageID=7485114&tstart=0}, when an Arab algorithm formally changed rational number conversions to a subtraction context (n/p - 1/m = (mn - p)/mp).

REFERENCES:

1. \PMlinkexternal{Hibeh Papyrys(1906), Grenfell, Bernard P.; Hunt, Arthur S.}{http://openlibrary.org/books/OL6992589M/The_Hibeh_papyri}.

2. David Fowler personally introduced the Hibeh Papyrus to me about 20 years ago by email and by a formal HM 10 reference, co-authored with Sir Eric Turner: Hibeh Papyrus i 27: An early example of Greek arithmetical notation, Historia Mathematics 10 (1983), 344-359

The HM 10 article was included in \PMlinkexternal{Greek era texts}{http://www.maths.warwick.ac.uk/maths/papers/dhf.html}
that were published in Fowler's productive math history life-time.

3. Egyptian Mathematical Leather Roll (1950 BCE - 1650 BCE)

4. \PMlinkexternal{Rhind Mathematical Papyrus (1650 BCE)}{http://ahmespapyrus.blogspot.com/} \PMlinkexternal{2/n table}{http://rmprectotable.blogspot.com/}

5. \PMlinkexternal{Akhmim Papyrus (500 AD - 800 AD)}{http://mathforum.org/kb/message.jspa?messageID=7485114&tstart=0}

%%%%%
%%%%%
\end{document}
