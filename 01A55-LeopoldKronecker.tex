\documentclass[12pt]{article}
\usepackage{pmmeta}
\pmcanonicalname{LeopoldKronecker}
\pmcreated{2013-03-22 18:15:23}
\pmmodified{2013-03-22 18:15:23}
\pmowner{Mravinci}{12996}
\pmmodifier{Mravinci}{12996}
\pmtitle{Leopold Kronecker}
\pmrecord{4}{40852}
\pmprivacy{1}
\pmauthor{Mravinci}{12996}
\pmtype{Biography}
\pmcomment{trigger rebuild}
\pmclassification{msc}{01A55}

% this is the default PlanetMath preamble.  as your knowledge
% of TeX increases, you will probably want to edit this, but
% it should be fine as is for beginners.

% almost certainly you want these
\usepackage{amssymb}
\usepackage{amsmath}
\usepackage{amsfonts}

% used for TeXing text within eps files
%\usepackage{psfrag}
% need this for including graphics (\includegraphics)
%\usepackage{graphicx}
% for neatly defining theorems and propositions
%\usepackage{amsthm}
% making logically defined graphics
%%%\usepackage{xypic}

% there are many more packages, add them here as you need them

% define commands here

\begin{document}
\emph{Leopold Kronecker} (1823 - 1891) German mathematician, famous for saying ``God created the integers, all else is the work of man.'' Concepts named after him include the Kronecker symbol and the Kronecker product.

Born to Jewish parents in what is now Legnica, Poland, Kronecker studied with Ernst Eduard Kummer there, and later at the university in Berlin with Peter Gustav Dirichlet. After earning a doctorate, rather than teach at the university, Kronecker returned to his birthplace and put his mathematical knowledge to practical use in the family business, becoming quite wealthy in addition to his family inheritance. In the 1850s he published a lot of papers which were quite influential and gained him acceptance to various societies, starting with the Berlin Academy.

Being a member of the Berlin Academy, Kummer persuaded Kronecker to take advantage of his privilege of being able to lecture at the university. Not many students showed up to his lectures, so when he was offered a position at G\"ottingen, he declined. It wasn't until 1883 that Kronecker began teaching as a professor, at his old alma mater. There he taught Franz Mertens and Kurt Hensel, to name just a few. In 1880, Kronecker became editor of {\it Crelle's Journal}. A year before his death, Kronecker converted to Lutheranism.

\begin{thebibliography}{1}
\bibitem{jt} James Tanton, {\it Encyclopedia of Mathematics}. New York: Facts on File Library (2005):296 - 297
\end{thebibliography}
%%%%%
%%%%%
\end{document}
