\documentclass[12pt]{article}
\usepackage{pmmeta}
\pmcanonicalname{AcanoALunarCalendarMethod}
\pmcreated{2016-03-25 7:14:12}
\pmmodified{2016-03-25 7:14:12}
\pmowner{milogardner}{13112}
\pmmodifier{milogardner}{13112}
\pmtitle{Acano a lunar calendar method}
\pmrecord{127}{40690}
\pmprivacy{1}
\pmauthor{milogardner}{13112}
\pmtype{Definition}
\pmcomment{trigger rebuild}
\pmclassification{msc}{01A15}
\pmclassification{msc}{01A13}
\pmdefines{lunar calendar}

% this is the default PlanetMath preamble.  as your knowledge
% of TeX increases, you will probably want to edit this, but
% it should be fine as is for beginners.

% almost certainly you want these
\usepackage{amssymb}
\usepackage{amsmath}
\usepackage{amsfonts}

% used for TeXing text within eps files
%\usepackage{psfrag}
% need this for including graphics (\includegraphics)
%\usepackage{graphicx}
% for neatly defining theorems and propositions
%\usepackage{amsthm}
% making logically defined graphics
%%%\usepackage{xypic}

% there are many more packages, add them here as you need them

% define commands here

\begin{document}
INTRODUCTION 

Diffusion of ancient near East lunar and solar calendars to the Canary Islands likely have contained a \PMlinkexternal{3 x 4 matrix calendar}{http://docs.google.com/viewer?a=v&q=cache:PPTp1ozN-60J:webpages.ull.es/users/jbarrios/pdf/Acano1996.pdf+Jose+Barrios+Garica&hl=en&gl=us&pid=bl&srcid=ADGEESjjIisIz44I64cov5HSo4gOnWkP1A0w5vEovKeJgNApuUMLm8oysGQSNuAG4RQyg0mMsOhESGv_FK6oRPcHMDeta5W8Fx_C-JbTpgkn5D9GibwuVf9IzI8ujRc8Q6PAtS68gnEM&sig=AHIEtbQR9ZIgZTzAlaoQ5o1npe5A_VdjAw}. 


The raw data was parsed and formally reported by Jose Barrios Garcia. An older \PMlinkexternal{Babylonian, Egyptian 

an/or Phoenician}{http://www.crystalinks.com/bailysbeads.html} 135-moon  cycle may have combined with a lunar solar calendar marks the movement of solstices (SS) across 12-moons, adding 11  days each year and alternatively subtracting 29 or 30 days every time. The acano count gives the location of the SS and locations of the other seasonal moons. Making the acano count as possibly reported in \PMlinkexternal{Babylonian  astronomy}{http://en.wikipedia.org/wiki/Babylonian_calendar} in an 135-moon context, used until 499 BCE, and replaced  by Metonic 235-moon method, an equivalent count is 29+30+29 days = 8*11 days. Thus, the SS stays 8 solar years in the  first three moons, the first column. The SS then jumps to the second column. This give us the basic statement:

8 solar year = 8 lunar year + 3 moons = 99 moons, which forms the \PMlinkexternal{Octaeteris}{http://en.wikipedia.org

/wiki/Octaeteris}, another luni-solar calendar from antiquity.

The basics of the Canary Island 9-moon, 12-moon and 270-moon lunar eclipse calendars may have been used during the 

period 900  BCE to 400 AD, and passed down in oral traditions after that time. It is unknown if these calendars were 

based on an Babylonian Egyptian or Phoencian (Libyan) 135-moon lunar calendar. 

The Canary Island calendar is used today and is named \PMlinkexternal{acano}{

http://webpages.ull.es/users/jbarrios/pdf/Acano1996.pdf}. The acano calendar used three colors and is parsed

by a 3 x  4 matrix revealing multiple mathematical topics. 

Matrix math subjects may include aspects of a 260 day Mayan calendar, 520 eclipse  calendar,  135-moon, 270-moon,

and Mayab 405-mpon lunar eclipse calendar. 

In addition there is solstice number system that is easy to explain face-to-face, with drawings, reports 

Jose Barrios Garcia, Professor of Mathematics at the University of Laguna, Canary Islands (Spain). Without 

drawings, and a lengthy discussion the specifics of the ancient acano calendar are difficult to grasp. An 

attempt to explain details outside of its three-colors follows.

LUNAR CALENDARS
Imagine a chessboard of 3 (horizontal) by 4 (vertical) files, equaling 12 squares alternatively painted in 
\PMlinkexternal{black and red}{http://webpages.ull.es/users/jbarrios/pdf/Acano1996.pdf}. Each square represented one synodic moon, let us say, from new moon to new moon. So the acano represents 12 consecutive moons, alternatively colored red and black. The moons are ordered vertical, let us say, from left to right.

 

1     4     7    10

2     5     8    11

3     6     9    12

 

From this point of view, the acano was an (eventually eternal) temporal framework measuring the pass of time from the endless repetition of this exact group of 12 synodic moons. There are no intercalary moons in this pattern. Time is measured by the eternal sequence of synodic moons packed in groups of exactly 12 moons. Red and black chessboards have been preserved in ancient caves on the Canary Islands.

Attending to a 19th century indirect source (La luna es la madre de los tiempos) the alternative coloring helps to track the duration of an actual moon. If we assume an average synodic month of 29.5 days, let red moons to be 29 days and black moons to be 30 days (or vice versa). On this way, two consecutive moons average 29.5 days. The acano accounts for a total of 6*29+6*30=354 days.

Let the Sun move across the acano, assume a solar year of 365 days, and suppose that we begin to count just from the coincidence of the New moon with the Summer Solstice in just the first day of the first month of the first acano.

First acano:

Since equinoxes and solstices are never less than three moon spaced, if Summer solstice (SS) occurs in the first day of the first moon of the first acano, the Autumn equinox (AE) occurs in the 4th moon, the Winter solstice (WS) in the 7th moon and the Spring Equinox (SE) in the 10th moon. So the equinoctial and solstice moons (from now on the seasonal moons) are aligned in the first file of the acano.

A Canarian priest working on this pattern would have readily located on the acano the very important (from astronomical, calendrical, social and agricultural perspective) the four seasonal moons. So they would prepare the social and economical activities for the year knowing exactly and anticipated in which moon everything occurs.

Second acano:

Now 12 moons have passed and we are in the first moon of the second acano. Since there is a 11 days difference between the acano and the Solar year, SS would occur in the 11th day of the first moon, so the seasonal moons continue to be aligned in the first file of the acano.

Third acano:

Another 12 moons have passed so the SS would occur in the 22th day of the first month and  the seasonal moons continue to be aligned in the first file of the acano.

Fourth acano:

Another 12 moons have passed so the SS would occur in the 33th day of the acano, that is to say, in the 4th day of the second moon. Since the SS has jumped to the second file of the acano, the other three seasonal moons also jump, so they remain aligned!, now in the second file.

And so on for the next followings years.  A report from the Canary Islands considers 520 days as one and one-half eclipse cycle, as discussed by Barrios.  The Canary Island 520 day cycle was a minor cycle within the dominate 270-moon calendar. The 270-moon calendar calculated 135-moon and 405-moon calendars within the 260, 520, ... 11960 day cycles. 

Acano Number System

Quoting Barrios, "As a matter of fact, to record a date on the acano you only need to write a number from 1 to 30 on one of its squares. The selected square fixes the moon while the number fixes the day of the moon counted, let us say, from new to new. Accordingly, it is possible to record unambiguously on a single acano the 33 successive dates fixing a whole round of the summer solstice through the lunar year. What is of the utmost importance is that this can be accomplished either through the years by actual observation, either at any desired moment by performing an easy arithmetical exercise on the acano. 

Indeed, once recorded on the acano the date of a particular summer solstice, we obtain the dates of the next summer solstices simply adding 11 days by year to the previous number. Each time the accumulated shift is greater than 29 or 30 days, we jump to the next square, reduce the shift by 29 or 30 days, write the new date on the square and continue the count. Actually, this exercise can be done even mentally for a number of years." 

DISCUSSIONS

Aaboe argues for several classes of lunar eclipse calendars in ancient cultures.  One was the acano. Several scholars argue the Egyptian and Babylonian 135-moon case, including red to denote life, and black to denote death, following an acano-like cycle. Jose Barrios Garcia argues for Canarian 520 day eclipse and 270-moon calendars in alternating red and black day colors. F. Lounsbury and \PMlinkexternal{A. Lebeuf}{http://www.le.ac.uk/archaeology/rug/aa/progs/pohualli.html} argue the Mayan 260 day calendar, a Canarian 520 day calendar and a 405-moon calendar, that include red to denote east (good), and black to denote west(almost bad), followed acano cycles. Aztecs and Mesoamericans used black to denote north, and death. Acano luni-solar eclipse calendars were well known to ancient cultures, such as 9-moon (260 day) and 99-moon calendars, cycles that may relate to ancients predicting eclipses.  

Two recent acano-type uses aligned civil and scientific calendars connected to eclipses tables, and validated longitudes in land (map making) and \PMlinkexternal{ocean navigation}{
http://phoenicia.org/proutes.html}. The far ranging Phoenicians circumnavigated \PMlinkexternal{Africa}{http://www.livius.org/he-hg/herodotus/hist01.htm} in 600 BCE, made \PMlinkexternal{later trips}{http://en.wikipedia.org/wiki/Hanno_the_Navigator}, as did Columbus in 1492 assisted by lunar eclipse tables and the \PMlinkexternal{"Armed Guards of Polaris"}{http://mathforum.org/kb/message.jspa?messageID=1180034&tstart=0}, calibrated daily by ' 1/2 hour glasses' for \PMlinkexternal{longitude}{http://www.navworld.com/navcerebrations/columbus.htm}, and Polaris for latitude.

Acano-type lunar eclipse prediction tables, referenced to Spain port city longitudes, were carried by Columbus on four New World trips. Columbus, like the \PMlinkexternal{Phoenicians/Libyans}{
http://www.sciencedirect.com/science?_ob=ArticleURL&_udi=B7586-4D2MWSD-2&_user=10&_rdoc=1&_fmt=&_orig=search&_sort=d&view=c&_acct=C000050221&_version=1&_urlVersion=0&_userid=10&md5=f6f4727667f3a345724b8b46b505410d}
 had used the 135-moon Egyptian eclipse tables.  \PMlinkexternal{Phoenicians/Libyans}{http://www.burlingtonnews.net/redhairedmummiescanaryislands.html} reached the Canary Islands in 900 BCE as deposed Egyptian pharaohs (denoted by \PMlinkexternal{mummification}{
http://books.google.com/books?id=njALAAAAIAAJ&pg=PA67&lpg=PA67&dq=canary+islands,+mummies&source=web&ots=pmiExR214p&sig=IgJRJOoej7-rjJpwqaB0TPBeP6s&hl=en&sa=X&oi=book_result&resnum=10&ct=result}
 and calendars), and stayed there until 400 AD, as a stop over on long voyages. Columbus debarked from the Canary Islands and observed \PMlinkexternal{two eclipses}{http://starryskies.com/The_sky/events/lunar-2003/columbus.eclipse.html} on his voyages, determining \PMlinkexternal{longitudes}{http://sunsite.utk.edu/math_archives/.http/hypermail/historia/aug99/0131.html} more accurate than historians have credited him (by only reading Bishop de Landa's edited data). Columbus had based his longitude calculations on Ptolemy's Almagest definition of the earth's circumference. Bishop de Landa edited Columbus' 'armed guards of Polaris' data as political protection from the \PMlinkexternal{treaty of 1494}{
http://www.zum.de/whkmla/period/disc/tordesillas.html} sanctions - in which Spain may have lost New World lands had Columbus' longitude hand written data been given to the \PMlinkexternal{Pope or to Portugal}{http://en.wikipedia.org/wiki/Portugal_in_the_Age_of_Discovery}.

Mayan black and red numbers cite seasonal almanacs in the Dresden Codex (pages 61-69) and Madrid Codex almanacs. A  

405-moon lunar calendar maybe implied one day by ring number 7.2.14.9 (51419 days) or other information in the seasonal

almanac. The \PMlinkexternal{Brickers in 

1988}{http://articles.adsabs.harvard.edu/cgi-bin/nph-iarticle_query?1988JHAS...19....1B&defaultprint=YES&filetype=.pdf}.

Pages 65-68 of the Dresden Code report eclipse, solstice and equinoxes almanac about 800 years (per 15 reference dates mentioned on pages 61-64) and shorter seasonal 1910, 1924 and 1911 day cycles on pages 65-68 analyzed by Bruce Friedman in 2012. Bruce Friedman argues for matrix table within three lines of raw data across pages 65-68:

11 1 13 1 11 12 1 13 8 8 6 13 4 5 2 7 13 7 6 13 6 6 8 1

11 11 13 11 11 9 1 10 8 5 6 11 4 2 2 4 13 4 6 10 6 3 8 11

9 9 5 1 1 2 10 12 6 5 2 7 11 5 7 12 3 2 12 1 8 9 4 13
 
By following the Brickers' paired "framed" instructions within crosswise sums: 

274 432 355 163 432 177 167 178 432 163 355 432 274 

Note that the symmetry is not perfect because 178 "appears" where 177 is "expected." This is due to the part of line 

two that referred to previously  as an anomaly (namely line 2 on p.65 where 13,11 appears to break a consistent pattern
 
of 10 day mod 13 additions to alternate entries in all of lines one and two. 13,10 is "expected" by the pattern and

would have made the 178 "appears" as 177.  The Bricker's considered this anomaly as a scribal error (it MAY NOT). 

All these numbers above are processed from the first two lines only but yield 163 (twice) and 167 (once at middle). 

163 and 167 are still obscure but these are an anchor (of some kind) to the third line where the same values "appear" 

as a delta from the first and last entries on line 3 p.65 (167 as the delta between 9,9 and 1,2) and the first and last

entries of line 3 p.67 (163 as the delta between 11,5 and 3,2). Shelving analysis of these two quantities [163 and 167]

in favor of the more peculiar four-time appearance of 432 seems important. To date 4 x 432 is 1728 and trivially is

1000 + (2 x 364), data that would be in error if the Brickers are correct (a strong possibility as their 2011 book

is digested).

However: Line three's 13 quantities total 1911 which is 21 x 91 and if 1728 id deducted 183 as a result. Also look at

432 less one cycle of 260 and find result 172. 172+183 is 355. 355 "appears" above (as may have been intended) to be

the 177+178 spacing of lunar eclipses.  The 432 day number appears four times in a manner that needs to be precisely

understood. The other matrix numbers are  associated with well-known lunar eclipse cycles. 

Mayans added red and black numbers as other keys, possibly in the acano manner, another issue that calls out to be

studied with the summer solstice acting as a possible beginning point.

Mayan astronomers worked from the mid-year season, outwards, denoted by feet icons going forward and back,

and year bearers, aspects ofthe Brickers' 2011 book. The matrix seasonal almanac data double checked itself by several 

techniques. Transliteration errors will likely be eliminated by double checking the almanac's  matrix structure within 

a study being conducted by Bruce Friedman (with minor assistance by myself).
  

SUMMARY

The best known \PMlinkexternal{acano}{
http://webpages.ull.es/users/jbarrios/pdf/Acano1996.pdf} cycle is marked by 33 solar years and 34 lunar years. The 

acano cycle uses 270 = 2 x 135 moons in which 46 lunar eclipses, and a complete number system on the Canary Islands. 

Lounsbury reports Mayan eclipse data, beginning with a 260 day calendar in the Dresden Codex as 405 moons = 3x 135 

moons, as possibly extended to a 405-month cycle. Related lunisolar calendars have been discovered around the world.
 
Several ancient 135-moon, 270-moon and 405-moon lunar calendars show common features. One  regional lunar calendar 

used a red and black acano 3 x 4 matrix placed in \PMlinkexternal{Canary Island caves next to 

mummies}{http://www.mummytombs.com/mummylocator/group/guanche.htm}. The red and black 3 x 4 type matrix/abacus 

innovation may pop up elsewhere in paintings, suggesting an academic connection. As individual cultural achievements 

lunar eclipse tables were prepared by Egyptians, Libyans, Phoenicians, Mesoamericans, and other advanced cultures. 

Several ancient lunar eclipse, solstice and equinox tables (almanacs) assisted in aligning civil and religious 

calendars, map making, and ocean navigation are under investigation.


\begin{thebibliography}{2}

\bibitem{1} Asger Aaboe, \emph{"Remarks on the theoretical treatment of eclipses in antiquity"},Journal for the History of Astronomy (Cambridge), 1972.
\bibitem{2} Jose Barrios Garica, \emph{TARA: A Study on the Canarian Astronomical Pictures, Part II: The acano chessboard}, Universidad de Laguna (Spain), 1996.

\end{document}
