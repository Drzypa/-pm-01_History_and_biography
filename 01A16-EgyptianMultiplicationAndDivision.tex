\documentclass[12pt]{article}
\usepackage{pmmeta}
\pmcanonicalname{EgyptianMultiplicationAndDivision}
\pmcreated{2013-03-22 18:28:42}
\pmmodified{2013-03-22 18:28:42}
\pmowner{milogardner}{13112}
\pmmodifier{milogardner}{13112}
\pmtitle{Egyptian multiplication and division}
\pmrecord{67}{41152}
\pmprivacy{1}
\pmauthor{milogardner}{13112}
\pmtype{Definition}
\pmcomment{trigger rebuild}
\pmclassification{msc}{01A16}
%\pmkeywords{aliquot parts}

\endmetadata

% this is the default PlanetMath preamble.  as your knowledge
% of TeX increases, you will probably want to edit this, but
% it should be fine as is for beginners.

% almost certainly you want these
\usepackage{amssymb}
\usepackage{amsmath}
\usepackage{amsfonts}

% used for TeXing text within eps files
%\usepackage{psfrag}
% need this for including graphics (\includegraphics)
%\usepackage{graphicx}
% for neatly defining theorems and propositions
%\usepackage{amsthm}
% making logically defined graphics
%%%\usepackage{xypic}

% there are many more packages, add them here as you need them

% define commands here

\begin{document}
Since the 1880s an incomplete and muddled view defined ancient Egyptian multiplication and division operations. \PMlinkexternal{Springer's}{http://eom.springer.de/A/a013260.htm} on-line encyclopedia summarizes a 1920s view on \PMlinkexternal{Wikpedia}{http://en.wikipedia.org/wiki/Egyptian_multiplication_and_division}:

The art of computation arose and developed before the time of the oldest written records. The oldest advanced mathematical records include the Kahun (Cahoon) papyri and the famous Rhind papyrus, which, among other papyri, date to 2050 BCE. Earlier additive hieroglyphic arithmetic methods represented numbers in ways that Old Kingdom Egyptians performed addition and subtraction operations in relatively simple ways. For example, multiplication was carried out by doubling, i.e. the factors were decomposed into sums of powers of two, the individual summands were multiplied, and the components added. Operations on fractions were reduced in Ancient Egypt to operations on aliquot fractions. More complicated fraction operations were decomposed with the aid of tables as the sums of aliquot fractions. 

In 1862 the first Egyptian text, the Berlin Papyrus, was published. Formal scholarly solutions appeared in 1900 with the work of Schack-Shackenberg. European scholars did not consider the Berlin Papyrus when 1920s transliterations of the RMP were published in incomplete additive versions of Egyptian multiplication. The 1920s historians had also not followed up a 1895 report (by F. Hultsch) that suggested a second form of multiplication method was present in Ahmes' 2/n table, and other RMP problems (i.e. RMP 38). The second method included aliquot parts, as Springer suggests above. 

Aliquot part were reported by \PMlinkexternal{F. Hultsch}{http://planetmath.org/encyclopedia/HultschBruinsMethodEgyptianFractions2.html} in 1895. Hultsch parsed Ahmes' 2/n table revealing scribal patterns of divisors of scaled rational number denominators that parsed 2/n table data. Springer's Egyptian multiplication encyclopedia entry did not specify scribal aliquot part details a topic that was not resolved until the 21th century.

Ahmes' aliquot part division steps, sensed in the 19th century, were not decoded during the 20th century for several reasons. Two reasons misdirected 1920s math historians by considering transliterations rather than translations. The first class of reason prematurely closed the subject of Egyptian fraction arithmetic operations by concluding Egyptian multiplication contained only additive steps. The Second class of reason scribal division was suggested have followed an non-inverse process called 'single false position', or 'trial and error' reported by scholars related to RMP 24-35, Ahmes' algebra problems. 

The aliquot part story line remained unsolved until 2008. Shortly after 2002 the \PMlinkexternal{Kahun Papyrus}{http://planetmath.org/encyclopedia/KahunPapyrusAndArithmeticProgressions.html} and the \PMlinkexternal{Rhind Papyrus 2/n table}{http://rmprectotable.blogspot.com/} were decoded revealing two aliquot part operational methods: (1) new inverse multiplication and division methods, and (2) a LCM number method written in red. The multiplication and division methods had been hidden in same set of aliquot part operational steps, including \PMlinkexternal{red auxiliary numbers}{http://planetmath.org/encyclopedia/FirstLCMMethodRedAuxiliaryNumbers.html} steps. In 2006, the 1895 Hultsch-Bruins method was confirmed from a second direction, detailing a common aliquot method used in the RMP and Egyptian Mathematical Leather Roll, and published on-line in 2008.

Springer's summary of Egyptian multiplication and division followed the misleading 1920s definition of Egyptian division suggesting: "Division was carried out by subtracting from the number to be divided the numbers obtained by successive doubling of the divisor." Math historians from 1920 to 1990 (i.e.; Howard Eves) called the proposed Egyptian division method 'single false position', and 1990 scholars (i.e. Spalinger) called the scribal information 'trial and error'. Ironically, 'single false position' was first documented in 800 AD, and at no earlier data. Later Arab texts improved up its root finding 'double \PMlinkexternal{false position}{http://en.wikipedia.org/wiki/False_position} ', method. 'Trial and error' in Ahmes' algebra problems selected LCM m to scale rational numbers n/p to mn/pm, and therefore can not be a scribal division fragment.

Springer's definition of Egyptian division was historically incomplete on several levels. A complete definition of Egyptian scribal division should include discussion of the first six RMP problems. RMP 1-6 reported division by 10 labor rates defined in the \PMlinkexternal{Reisner Papyrus}{http://planetmath.org/encyclopedia/ReisnerPapyrus.html}. In addition, RMP algebra problems and methods are consulted. For example, Ahmes divided 28 by 97, in \PMlinkexternal{RMP 31}{http://mathforum.org/kb/thread.jspa?threadID=1768698&tstart=0}(confirmed in RMP 34) by solving:  x + (2/3 + 1/2 + 1/7)x = 33 and x + (2/3 + 1/2 + 1/7)x = 37 as other vulgar fraction problems were solved in the Kahun Papyrus and Rhind Papyrus 2/n tables. Aliquot part steps were hidden in theoretical multiplication and division operations for over 100 years. 

Ahmes, therefore, did not use 'single false position' in any arithmetic operation, a point made by Robins-Shute in 1987. The 1920s 'false position' and 1990 'trial and error' scholarly ideas were false suppositions. For example, 28/97, in RMP 31, and data from RMP 21- 23 expose Ahmes' LCM method. In RMP 23 where 45 was introduced solving most of the problem, but 360 was needed to Ahmes to complete the problem as all other algebra problems were solved.

In the 21st century, Ahmes is being reported converting vulgar fractions into optimized unit fractions series within a LCM m multiplication method. The LCM method replaced the aliquot parts of the denominator in the numerator. To convert 2/97 in RMP 31, and the 2/n table. Ahmes converted 28/97 into two problems, 2/97 and 26/97, such that:

1. To convert 2 by 97: As Ahmes' 2/n table wrote for all 2/n conversions less than 2/101, he first selected a highly divisible number m as an optimizing multiplier m/m. In the 2/97 case 56 was selected, creating a multiplier 56/56 such that the aliquot parts of 56 (28, 14, 8, 7, 4, 2, 1) were introduced into the solution by writing: 

2/97*(56/56) = 112/(56*97) = (97 + 8 + 7)/56*97)

and,

2/97 = 1/56 + 1/679 + 1/776 

2. To convert 26/97 Ahmes looked for a multiplier m/m that would increase the numerator to greater than 97. Ahmes found 4/4. By considering the aliquot parts of 4 (4 , 2, 1) Ahmes wrote out:

26/97*(4/4) = 104/(4*97)= (97 + 4 + 2 + 1)/(4*97)

such that:

26/97 = 1/4 +  1/97 + 1/194 + 1/388

and,

3. Ahmes combined steps 2/97 and 26/97 into one Egyptian fraction series by writing:

28/97 = 1/4 + 1/56 + 1/97 + 1/194 + 1/388 + 1/679 + 1/77

'Single false position' was false 20th century supposition that failed to parse Ahmes' actual division operation. Ahmes division operation is correctly parsed as inverse to Egyptian multiplication. Egyptian scribes applied theoretical ideas in \PMlinkexternal{Ahmes math tool box}{http://planetmath.org/encyclopedia/AnOverViewOfAhmesPapyrus.html} to convert rational numbers to Egyptian fractions.

In, \PMlinkexternal{RMP 35-RMP 38 and RMP 66}{http://ahmespapyrus.blogspot.com/2009/01/ahmes-papyrus-new-and-old.html} a hekat was replaced by its 1/320 unit equivalent, 320 ro. In RMP. 10 hekat, 3200 ro, was divided by 365, the number of civil days in the year obtaining: 

8 + 280/365

the expected daily use of fat.

The quotient 8 was proven by the binary steps 1 - 365, 2 - 730, 4 - 1460, and 8 - 2920. The  remainder 280 (3200 - 2920 = 280). Translated to modern notation  Ahmes' proof multiplied the initial divisor 365 by (8 + 2/3 + 1/10 + 2190) summing (2920 + 243 1/3 + 36 1/2 + 1/6) obtaining 3200 ro, the exact initial value of 10 hekat of fat.     

RMP 38 reported Ahmes multiplying 320 ro, one hekat, by 7/22, obtaining 101 9/11. The 101 9/11 answer was proven by multiplying 101 9/11 by 22/7, with the initial divisor 7/22 parsed by binary steps 34/11 times 1/10 = 7/22 showing that inverse multipliers define division, one of several modern arithmetic ideas that were anticipated by Ahmes.

Egyptian division was an inverse of Egyptian multiplication, reported in the RMP and the 1900 BCE \PMlinkexternal{Akhmim Wooden Tablet}{http://en.wikipedia.org/wiki/Akhmim_Wooden_Tablet} (AWT) and other Middle Kingdom texts. 

Conclusion: To parse ancient Egyptian multiplication and division written with a weights and measures context , Ahmes' 2/n table and other arithmetic operations must be stripped away to reveal rational numbers. Ahmes multiplication and division of rational numbers were inverse to each other.

Egyptian multiplication contained two aspects, a theoretical side, and a practical side. Egyptian division was an inverse of Egyptian multiplication, and visa verse. Prior to the 21st century AD Egyptian math scholars had not considered theoretical aspects of the RMP and other Egyptian texts. Theoretical definitions hid an aliquot part and other arithmetic  definitions, two being a hekat unity stated as (64/64), and a hekat stated as 320 ro. Egyptian division was quotient and exact remainder based, aspects that scholars are increasingly studying, linked to aliquot parts, 2/n tables, and other ancient scribal applications, such as weights and meaures, after 2005.
 


%%%%%
%%%%%
\end{document}
