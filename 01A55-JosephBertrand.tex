\documentclass[12pt]{article}
\usepackage{pmmeta}
\pmcanonicalname{JosephBertrand}
\pmcreated{2013-03-22 16:24:21}
\pmmodified{2013-03-22 16:24:21}
\pmowner{Mravinci}{12996}
\pmmodifier{Mravinci}{12996}
\pmtitle{Joseph Bertrand}
\pmrecord{4}{38554}
\pmprivacy{1}
\pmauthor{Mravinci}{12996}
\pmtype{Biography}
\pmcomment{trigger rebuild}
\pmclassification{msc}{01A55}
\pmsynonym{Joseph Louis-Fran\c{c}ois Bertrand}{JosephBertrand}
\pmsynonym{Joseph Louis-Francois Bertrand}{JosephBertrand}

\endmetadata

% this is the default PlanetMath preamble.  as your knowledge
% of TeX increases, you will probably want to edit this, but
% it should be fine as is for beginners.

% almost certainly you want these
\usepackage{amssymb}
\usepackage{amsmath}
\usepackage{amsfonts}

% used for TeXing text within eps files
%\usepackage{psfrag}
% need this for including graphics (\includegraphics)
%\usepackage{graphicx}
% for neatly defining theorems and propositions
%\usepackage{amsthm}
% making logically defined graphics
%%%\usepackage{xypic}

% there are many more packages, add them here as you need them

% define commands here

\begin{document}
\emph{Joseph Louis-Fran\c{c}ois Bertrand} (1822 - 1900) French mathematician, economist and translator who translated Carl Friedrich Gauss's books to French. Bertrand is perhaps best known for Bertrand's conjecture regarding the distribution of prime numbers between a given integer and its double.

In 1884 he took over seat 40 of the Academie Fran\c{c}aise for Jean-Baptiste Dumas. Upon his death, Bertrand's seat was given to Marcellin Berthelot.
%%%%%
%%%%%
\end{document}
