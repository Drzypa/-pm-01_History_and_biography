\documentclass[12pt]{article}
\usepackage{pmmeta}
\pmcanonicalname{EulerLeonhard}
\pmcreated{2013-03-22 14:10:25}
\pmmodified{2013-03-22 14:10:25}
\pmowner{mathwizard}{128}
\pmmodifier{mathwizard}{128}
\pmtitle{Euler, Leonhard}
\pmrecord{15}{35599}
\pmprivacy{1}
\pmauthor{mathwizard}{128}
\pmtype{Biography}
\pmcomment{trigger rebuild}
\pmclassification{msc}{01A50}
\pmclassification{msc}{01A70}
\pmsynonym{Euler}{EulerLeonhard}

% this is the default PlanetMath preamble.  as your knowledge
% of TeX increases, you will probably want to edit this, but
% it should be fine as is for beginners.

% almost certainly you want these
\usepackage{amssymb}
\usepackage{amsmath}
\usepackage{amsfonts}

% used for TeXing text within eps files
%\usepackage{psfrag}
% need this for including graphics (\includegraphics)
%\usepackage{graphicx}
% for neatly defining theorems and propositions
%\usepackage{amsthm}
% making logically defined graphics
%%%\usepackage{xypic}

% there are many more packages, add them here as you need them

% define commands here
\begin{document}
Leonhard Euler was born on April the 15th 1707 as the son of a Protestant minister in Basel (Switzerland). Already in his childhood he exhibited great mathematical talents, but his father wanted him to study theology and become a minister. In 1720 Euler began his studies at the University of Basel. There Euler met Daniel and Nikolaus Bernoulli, who noticed Euler's skills in mathematics. Paul Euler, Leonhard's father, had attended Jakob Bernoulli's mathematical lectures and respected his family.
When Daniel and Nikolaus Bernoulli asked him to allow his son to study mathematics he finally agreed and Euler began to study mathematics.

In 1727 Euler was called to St. Petersburg by Catherine I. and became professor of physics in 1730. Finally in 1733 he became professor of mathematics. His work was both in physics and mathematics. Euler was the first to publish a systematic introduction to mechanics in 1736: ``Mechanica sive motus scientia analytice exposita'' (Mechanics or motion explained with analytical science (that is, calculus)). 1735 he lost much of his vision in the right eye because he had looked into the sun for too long.

In 1733 he married Katharina Gsell, the daughter of the director of the academy of arts. They had thirteen children, of whom only three sons and two daughters survived. The descendants of these children, however, were in high positions in Russia in the 19th century.

In the year 1741 Euler went to the Prussian Academy of Sciences in Berlin and became director of the mathematical class. His time in Berlin was very productive; however, he did not have an easy position because he was not much liked by the king. Therefore he returned to St. Petersburg in 1766, now ruled by Catherine II., where he would remain for the rest of his life.

Also in that time Euler was very productive, though he very soon lost his vision completely. This was possible because he had an extraordinary memory and could calculate very well. It is reported that once he let his assistant calculate a series to 17 summands and noticed that his own result and the assistant's result differed in the 50th digit. A recalculation showed that Euler was right! 

It has been calculated that it would take 50 years eight-hour work per day to copy all his works by hand. It was not till the year 1910 that a collection of his complete works was published and it took about 70 volumes. It is reported by Legendre that often he would write down a complete mathematical proof between the first and the second call for supper.

In contrast to most intellectuals of his time he was conservative and a convinced Christian. There is a story, which is often told in books and on the web, saying that once at the court of Catherine the Great he met the French philosopher Denis Diderot, who was a convinced atheist and tried to convince the Russians of atheism, much to the annoyance of Catherine. Therefore she asked Euler to stop him. Euler thought about it and when Catherine invited Diderot to have a theological discussion with Euler, Euler said: ``$\frac{a+b^n}{n}=x$, therefore God exists, answer!'' Diderot, who knew almost nothing about algebra knew not what to answer and therefore returned to Paris. This story however is almost certainly an urban myth and Diderot knew enough algebra to answer Euler. However it is said that Euler published some other (not really serious) proofs of the existence of God, which may well be, since at that time people were wondering about the possibility to give an algebraic proof of the existence of God.

When Euler died on 18th of September 1783 the mathematician and philosopher Marquis de Condorcet said ``...et il cessa de calculer et de vivre'' (and he stopped calculating and living).
\subsubsection*{References}
\begin{itemize}
\item Hermann Heimpell, Theodor Heuss, Benno Reifenberg (editors): Die gro\ss en Deutschen volume 2, Ullstein Verlag Berlin, 1956
\item Simon Singh: Fermats letzter Satz, Deutscher Taschenbuch Verlag Munich, 2000
\item Lexikon der Naturwissenschaftler, Spektrum Akademischer Verlag Heidelberg, 2000
\item \PMlinkexternal{An online biography}{http://www-gap.dcs.st-and.ac.uk/\%7Ehistory/Mathematicians/Euler.html}\\
\end{itemize}

\PMlinkescapeword{children}
\PMlinkescapeword{descendants}
\PMlinkescapeword{class}
\PMlinkescapeword{calculate}
\PMlinkescapeword{collection}
\PMlinkescapeword{complete}
\PMlinkescapeword{sources}
%%%%%
%%%%%
\end{document}
