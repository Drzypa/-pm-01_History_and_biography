\documentclass[12pt]{article}
\usepackage{pmmeta}
\pmcanonicalname{IvanPervushin}
\pmcreated{2013-03-22 18:04:53}
\pmmodified{2013-03-22 18:04:53}
\pmowner{Mravinci}{12996}
\pmmodifier{Mravinci}{12996}
\pmtitle{Ivan Pervushin}
\pmrecord{5}{40618}
\pmprivacy{1}
\pmauthor{Mravinci}{12996}
\pmtype{Biography}
\pmcomment{trigger rebuild}
\pmclassification{msc}{01A55}
\pmsynonym{Ivan Mikheevich Pervushin}{IvanPervushin}

\endmetadata

% this is the default PlanetMath preamble.  as your knowledge
% of TeX increases, you will probably want to edit this, but
% it should be fine as is for beginners.

% almost certainly you want these
\usepackage{amssymb}
\usepackage{amsmath}
\usepackage{amsfonts}

% used for TeXing text within eps files
%\usepackage{psfrag}
% need this for including graphics (\includegraphics)
%\usepackage{graphicx}
% for neatly defining theorems and propositions
%\usepackage{amsthm}
% making logically defined graphics
%%%\usepackage{xypic}

% there are many more packages, add them here as you need them

% define commands here

\begin{document}
\emph{Ivan Mikheevich Pervushin} (1827 - 1900) Russian priest and mathematician, best known for finding factors of the Fermat numbers $2^{2^{12}} + 1$ and $2^{2^{23}} + 1$ and finding the ninth Mersenne prime, $2^{61} - 1$.

Born in Perm, he studied to be a priest in Kazan and later moved to practice in Zamaraevo. When Pervushin was not saying Mass, he devoted himself to studying number theory. It was in Zamaraevo that Pervushin pondered Marin Mersenne's list of prime numbers of the form $2^p - 1$ and discovered that Mersenne was mistaken to exclude 61, for 2305843009213693951 is in fact prime. Today computers can verify this practically instantaneously, but in Pervushin's day, it took Hudelot 54 hours in 1887 to confirm Pervushin's result, performing the Lucas-Lehmer primality test by hand.

The same year that Pervushin discovered the ninth Mersenne prime, he visited Shadrinsk and wrote an article critical of the government there. The government exiled him to Mehonskoe, where he spent the rest of his life. Still, the ninth Mersenne prime, the largest prime number known at the time, was for a while known as ``Pervushin's number.''

\begin{thebibliography}{1}
\bibitem{eb} Eric Bach \& Jeffrey Shallit, {\it Algorithmic Number Theory: Volume 1: Efficient Algorithms}. Cambridge: MIT Press (1996): 7
\end{thebibliography}
%%%%%
%%%%%
\end{document}
