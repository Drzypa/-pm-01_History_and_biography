\documentclass[12pt]{article}
\usepackage{pmmeta}
\pmcanonicalname{FirstPrimitivePythagoreanTriplets}
\pmcreated{2013-10-30 13:12:58}
\pmmodified{2013-10-30 13:12:58}
\pmowner{pahio}{2872}
\pmmodifier{pahio}{2872}
\pmtitle{first primitive Pythagorean triplets}
\pmrecord{12}{37289}
\pmprivacy{1}
\pmauthor{pahio}{2872}
\pmtype{Example}
\pmcomment{trigger rebuild}
\pmclassification{msc}{01A16}
\pmclassification{msc}{11-00}
\pmsynonym{least coprime Pythagorean triplets}{FirstPrimitivePythagoreanTriplets}
%\pmkeywords{right triangle}
%\pmkeywords{integers}
%\pmkeywords{Egyptian numbers}
\pmrelated{PythagorasTheorem}
\pmrelated{IncircleRadiusDeterminedByPythagoreanTriple}
\pmrelated{ContraharmonicMeansAndPythagoreanHypotenuses}
\pmrelated{PythagoreanHypotenusesAsContraharmonicMeans}
\pmdefines{Egyptian triangle}

% this is the default PlanetMath preamble.  as your knowledge
% of TeX increases, you will probably want to edit this, but
% it should be fine as is for beginners.

% almost certainly you want these
\usepackage{amssymb}
\usepackage{amsmath}
\usepackage{amsfonts}

% used for TeXing text within eps files
%\usepackage{psfrag}
% need this for including graphics (\includegraphics)
%\usepackage{graphicx}
% for neatly defining theorems and propositions
 \usepackage{amsthm}
% making logically defined graphics
%%%\usepackage{xypic}

% there are many more packages, add them here as you need them

% define commands here

\theoremstyle{definition}
\newtheorem*{thmplain}{Theorem}
\begin{document}
 
$(\mbox{odd cathetus})^2+(\mbox{even cathetus})^2 = (\mbox{hypotenuse})^2$\\
\\
 $3^2+4^2 = 5^2 $\quad ($\leftarrow$ these form the so-called {\em Egyptian triangle}, known by the pyramid builders)\\$
 5^2+12^2 = 13^2 $\\$
 15^2+8^2 = 17^2 $\\$
 7^2+24^2 = 25^2 $\\$
 21^2+20^2 = 29^2 $\\$
 9^2+40^2 = 41^2 $\\$
 35^2+12^2 = 37^2 $\\$
 11^2+60^2 = 61^2 $\\$
 45^2+28^2 = 53^2 $\\$
 33^2+56^2 = 65^2 $\\$
 13^2+84^2 = 85^2 $\\$
 63^2+16^2 = 65^2 $\\$
 55^2+48^2 = 73^2 $\\$
 39^2+80^2 = 89^2 $\\$
 15^2+112^2 = 113^2 $\\$
 77^2+36^2 = 85^2 $\\$
 65^2+72^2 = 97^2 $\\$
 17^2+144^2 = 145^2 $\\$
 99^2+20^2 = 101^2 $\\$
 91^2+60^2 = 109^2 $\\$
 51^2+140^2 = 149^2 $\\$
 19^2+180^2 = 181^2 $\\$
 117^2+44^2 = 125^2 $\\$
 105^2+88^2 = 137^2 $\\$
 85^2+132^2 = 157^2 $\\$
 57^2+176^2 = 185^2 $\\$
 21^2+220^2 = 221^2 $\\$
 143^2+24^2 = 145^2 $\\$
 119^2+120^2 = 169^2 $\\$
 95^2+168^2 = 193^2 $\\$
 23^2+264^2 = 265^2 $\\$
 165^2+52^2 = 173^2 $\\$
 153^2+104^2 = 185^2 $\\$
 133^2+156^2 = 205^2 $\\$
 105^2+208^2 = 233^2 $\\$
 69^2+260^2 = 269^2 $\\$
 25^2+312^2 = 313^2 $\\$
 195^2+28^2 = 197^2 $\\$
 187^2+84^2 = 205^2 $\\$
 171^2+140^2 = 221^2 $\\$
 115^2+252^2 = 277^2 $\\$
 75^2+308^2 = 317^2 $\\$
 27^2+364^2 = 365^2 $\\$
 221^2+60^2 = 229^2 $\\$
 209^2+120^2 = 241^2 $\\$
 161^2+240^2 = 289^2 $\\$
 29^2+420^2 = 421^2 $\\$
 255^2+32^2 = 257^2 $\\$
 247^2+96^2 = 265^2 $\\$
 231^2+160^2 = 281^2 $\\$
 207^2+224^2 = 305^2 $\\$
 175^2+288^2 = 337^2 $\\$
 135^2+352^2 = 377^2 $\\$
 87^2+416^2 = 425^2 $\\$
 31^2+480^2 = 481^2 $\\$
 285^2+68^2 = 293^2 $\\$
 273^2+136^2 = 305^2 $\\$
 253^2+204^2 = 325^2 $\\$
 225^2+272^2 = 353^2 $\\$
 189^2+340^2 = 389^2 $\\$
 145^2+408^2 = 433^2 $\\$
 93^2+476^2 = 485^2 $\\$
 33^2+544^2 = 545^2 $\\$
 323^2+36^2 = 325^2 $\\$
 299^2+180^2 = 349^2 $\\$
 275^2+252^2 = 373^2 $\\$
 203^2+396^2 = 445^2 $\\$
 155^2+468^2 = 493^2 $\\$
 35^2+612^2 = 613^2 $\\$
 357^2+76^2 = 365^2 $\\$
 345^2+152^2 = 377^2 $\\$
 325^2+228^2 = 397^2 $\\$
 297^2+304^2 = 425^2 $\\$
 261^2+380^2 = 461^2 $\\$
 217^2+456^2 = 505^2 $\\$
 165^2+532^2 = 557^2 $\\$
 105^2+608^2 = 617^2 $\\$
 37^2+684^2 = 685^2 $\\$
 399^2+40^2 = 401^2 $\\$
 391^2+120^2 = 409^2 $\\$
 351^2+280^2 = 449^2 $\\$
 319^2+360^2 = 481^2 $\\$
 279^2+440^2 = 521^2 $\\$
 231^2+520^2 = 569^2 $\\$
 111^2+680^2 = 689^2 $\\$
 39^2+760^2 = 761^2 $\\$
 437^2+84^2 = 445^2 $\\$
 425^2+168^2 = 457^2 $\\$
 377^2+336^2 = 505^2 $\\$
 341^2+420^2 = 541^2 $\\$
 185^2+672^2 = 697^2 $\\$
 41^2+840^2 = 841^2 $\\$
 483^2+44^2 = 485^2 $\\$
 475^2+132^2 = 493^2 $\\$
 459^2+220^2 = 509^2 $\\$
 435^2+308^2 = 533^2 $\\$
 403^2+396^2 = 565^2 $\\$
 315^2+572^2 = 653^2 $\\$
 259^2+660^2 = 709^2 $\\$
 195^2+748^2 = 773^2 $\\$
 123^2+836^2 = 845^2 $\\$
 43^2+924^2 = 925^2 $\\$
 525^2+92^2 = 533^2 $\\$
 513^2+184^2 = 545^2 $\\$
 493^2+276^2 = 565^2 $\\$
 465^2+368^2 = 593^2 $\\$
 429^2+460^2 = 629^2 $\\$
 385^2+552^2 = 673^2 $\\$
 333^2+644^2 = 725^2 $\\$
 273^2+736^2 = 785^2 $\\$
 205^2+828^2 = 853^2 $\\$
 129^2+920^2 = 929^2 $\\$
 45^2+1012^2 = 1013^2 $\\$
 575^2+48^2 = 577^2 $\\$
 551^2+240^2 = 601^2 $\\$
 527^2+336^2 = 625^2 $\\$
 455^2+528^2 = 697^2 $\\$
 407^2+624^2 = 745^2 $\\$
 287^2+816^2 = 865^2 $\\$
 215^2+912^2 = 937^2 $\\$
 47^2+1104^2 = 1105^2 $\\$
 621^2+100^2 = 629^2 $\\$
 609^2+200^2 = 641^2 $\\$
 589^2+300^2 = 661^2 $\\$
 561^2+400^2 = 689^2 $\\$
 481^2+600^2 = 769^2 $\\$
 429^2+700^2 = 821^2 $\\$
 369^2+800^2 = 881^2 $\\$
 301^2+900^2 = 949^2 $\\$
 141^2+1100^2 = 1109^2 $\\$
 49^2+1200^2 = 1201^2 $\\$
 675^2+52^2 = 677^2 $\\$
 667^2+156^2 = 685^2 $\\$
 651^2+260^2 = 701^2 $\\$
 627^2+364^2 = 725^2 $\\$ 
 595^2+468^2 = 757^2 $\\$
 555^2+572^2 = 797^2 $\\$
 451^2+780^2 = 901^2 $\\$
 387^2+884^2 = 965^2 $\\$
 315^2+988^2 = 1037^2 $\\$
 235^2+1092^2 = 1117^2 $\\$
 147^2+1196^2 = 1205^2 $\\$
 51^2+1300^2 = 1301^2 $\\$
 725^2+108^2 = 733^2 $\\$
 713^2+216^2 = 745^2 $\\$
 665^2+432^2 = 793^2 $\\$
 629^2+540^2 = 829^2 $\\$
 533^2+756^2 = 925^2 $\\$
 473^2+864^2 = 985^2 $\\$
 329^2+1080^2 = 1129^2 $\\$
 245^2+1188^2 = 1213^2 $\\$
 53^2+1404^2 = 1405^2 $

N.B. that the lengths of the even cathetus and the 
hypotenuse are consecutive integers (as 1404 and 1405) 
always when the corresponding seed numbers $m$ and $n$ 
(see the \PMlinkname{parent}{PythagoreanTriplet} entry) are successive integers.
%%%%%
%%%%%
\end{document}
