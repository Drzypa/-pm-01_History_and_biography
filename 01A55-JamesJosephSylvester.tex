\documentclass[12pt]{article}
\usepackage{pmmeta}
\pmcanonicalname{JamesJosephSylvester}
\pmcreated{2013-03-22 18:20:12}
\pmmodified{2013-03-22 18:20:12}
\pmowner{Mravinci}{12996}
\pmmodifier{Mravinci}{12996}
\pmtitle{James Joseph Sylvester}
\pmrecord{6}{40969}
\pmprivacy{1}
\pmauthor{Mravinci}{12996}
\pmtype{Biography}
\pmcomment{trigger rebuild}
\pmclassification{msc}{01A55}

\endmetadata

% this is the default PlanetMath preamble.  as your knowledge
% of TeX increases, you will probably want to edit this, but
% it should be fine as is for beginners.

% almost certainly you want these
\usepackage{amssymb}
\usepackage{amsmath}
\usepackage{amsfonts}

% used for TeXing text within eps files
%\usepackage{psfrag}
% need this for including graphics (\includegraphics)
%\usepackage{graphicx}
% for neatly defining theorems and propositions
%\usepackage{amsthm}
% making logically defined graphics
%%%\usepackage{xypic}

% there are many more packages, add them here as you need them

% define commands here

\begin{document}
\emph{James Joseph Sylvester} (1814 - 1897) British mathematician, poet and author, best known for Sylvester's sequence and for coining mathematical terms, some of which are still in use today (such as ``totient'' and ``totative'').
Together with Arthur Cayley, he invented the theory of subject of invariant theory, which became one of the
most actively studied branches of algebra during the nineteenth century. 

Born in England of a Jewish family, Sylvester studied in England and Scotland, and taught in the United States. He also studied law.

\begin{thebibliography}{3}
\bibitem{ra} R. C. Archibald, ``James Joseph Sylvester'', {\it Studies and Essays in the History of Science and Learning Offered in Homage to George Sarton on the Occasion of his Sixtieth Birthday}. New York: H. Schuman (1947): 209 - 217
\bibitem{lf} L. S. Feuer, ``Sylvester in Virginia'' {\it The Mathematical Intelligencer} {\bf 9} (1987): 13 - 20
\bibitem{kp} K. H. Parshall, {\it James Joseph Sylvester: Jewish Mathematician in a Victorian World}. Baltimore: The Johns Hopkins University Press (2006)
\end{thebibliography}
%%%%%
%%%%%
\end{document}
