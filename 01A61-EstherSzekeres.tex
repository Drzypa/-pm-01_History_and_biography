\documentclass[12pt]{article}
\usepackage{pmmeta}
\pmcanonicalname{EstherSzekeres}
\pmcreated{2013-03-22 16:16:11}
\pmmodified{2013-03-22 16:16:11}
\pmowner{Mravinci}{12996}
\pmmodifier{Mravinci}{12996}
\pmtitle{Esther Szekeres}
\pmrecord{5}{38380}
\pmprivacy{1}
\pmauthor{Mravinci}{12996}
\pmtype{Biography}
\pmcomment{trigger rebuild}
\pmclassification{msc}{01A61}
\pmclassification{msc}{01A60}
\pmsynonym{Esther Klein}{EstherSzekeres}
\pmsynonym{Esther Szekeres-Klein}{EstherSzekeres}

% this is the default PlanetMath preamble.  as your knowledge
% of TeX increases, you will probably want to edit this, but
% it should be fine as is for beginners.

% almost certainly you want these
\usepackage{amssymb}
\usepackage{amsmath}
\usepackage{amsfonts}

% used for TeXing text within eps files
%\usepackage{psfrag}
% need this for including graphics (\includegraphics)
%\usepackage{graphicx}
% for neatly defining theorems and propositions
%\usepackage{amsthm}
% making logically defined graphics
%%%\usepackage{xypic}

% there are many more packages, add them here as you need them

% define commands here

\begin{document}
\emph{Esther Szekeres} n\'ee \emph{Klein} (1910 - 2005) Hungarian mathematician, best known for the happy ending problem she pondered with Paul Erd\H{o}s and which led to her marriage to the mathematician George Szekeres.

During World War II, Szekeres and her husband moved to Australia. In Sydney, she lectured at Macquarie University and lobbied to improve mathematics education in high schools in Australia and New Zealand.

As she collaborated with Erd\H{o}s, Janice L. Malouf and John Selfridge on the paper {\it Subsets of an interval whose product is a power}, she has \PMlinkname{Erd\H{o}s number}{ErdHosNumber} 1.
%%%%%
%%%%%
\end{document}
