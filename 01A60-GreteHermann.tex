\documentclass[12pt]{article}
\usepackage{pmmeta}
\pmcanonicalname{GreteHermann}
\pmcreated{2013-03-22 16:57:11}
\pmmodified{2013-03-22 16:57:11}
\pmowner{Mravinci}{12996}
\pmmodifier{Mravinci}{12996}
\pmtitle{Grete Hermann}
\pmrecord{5}{39222}
\pmprivacy{1}
\pmauthor{Mravinci}{12996}
\pmtype{Biography}
\pmcomment{trigger rebuild}
\pmclassification{msc}{01A60}

\endmetadata

% this is the default PlanetMath preamble.  as your knowledge
% of TeX increases, you will probably want to edit this, but
% it should be fine as is for beginners.

% almost certainly you want these
\usepackage{amssymb}
\usepackage{amsmath}
\usepackage{amsfonts}

% used for TeXing text within eps files
%\usepackage{psfrag}
% need this for including graphics (\includegraphics)
%\usepackage{graphicx}
% for neatly defining theorems and propositions
%\usepackage{amsthm}
% making logically defined graphics
%%%\usepackage{xypic}

% there are many more packages, add them here as you need them

% define commands here

\begin{document}
\emph{Grete Hermann} (1901 - 1984) German mathematician perhaps most famous for discovering a flaw in John von Neumann's proof of the impossibility of hidden variable theory in quantum mechanics. As a student of Emmy Noether at G\"ottingen University, Hermann showed in her thesis that her teacher's proof of the Lasker-Noether theorem could be turned into an efficient algorithm for computing primary decompositions of polynomial ideals in Noetherian rings.

As Adolf Hitler came to \PMlinkescapetext{power} in Germany, Hermann participated in the underground movement against the Nazis, but by 1936 she left Germany for Denmark and later England. She returned when World War II was over. In her later years she was more interested in politics and philosophy than in physics and mathematics.
%%%%%
%%%%%
\end{document}
