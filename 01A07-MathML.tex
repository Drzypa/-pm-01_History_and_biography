\documentclass[12pt]{article}
\usepackage{pmmeta}
\pmcanonicalname{MathML}
\pmcreated{2013-03-22 16:54:19}
\pmmodified{2013-03-22 16:54:19}
\pmowner{PrimeFan}{13766}
\pmmodifier{PrimeFan}{13766}
\pmtitle{MathML}
\pmrecord{4}{39164}
\pmprivacy{1}
\pmauthor{PrimeFan}{13766}
\pmtype{Definition}
\pmcomment{trigger rebuild}
\pmclassification{msc}{01A07}
\pmsynonym{Mathematical Markup Language}{MathML}

\endmetadata

% this is the default PlanetMath preamble.  as your knowledge
% of TeX increases, you will probably want to edit this, but
% it should be fine as is for beginners.

% almost certainly you want these
\usepackage{amssymb}
\usepackage{amsmath}
\usepackage{amsfonts}

% used for TeXing text within eps files
%\usepackage{psfrag}
% need this for including graphics (\includegraphics)
%\usepackage{graphicx}
% for neatly defining theorems and propositions
%\usepackage{amsthm}
% making logically defined graphics
%%%\usepackage{xypic}

% there are many more packages, add them here as you need them

% define commands here

\begin{document}
{\em MathML} (or {\em Mathematical Markup Language}) is an application of XML customized for the transmission of documents pertaining to mathematics. The official Document Type Definition was created by the World Wide Web Consortium.

MathML is concerned as much with the meaning of mathematical formulas as it is with their presentation. This means, for example, that any instances of the tacit multiplication operator must be stated in the MathML source even though they are not meant to be shown on the display.

To give one example of the famous Euler identity $e^{i\pi} = -1$ in MathML:

% I tried to put an example in here, but instead of the usual errors, I got unexpected errors

There are programs available that convert \TeX{} to MathML.
%%%%%
%%%%%
\end{document}
