\documentclass[12pt]{article}
\usepackage{pmmeta}
\pmcanonicalname{EgyptianGeometryAreasCalculatedInCubitsKhetsAndSetats}
\pmcreated{2013-03-22 18:56:40}
\pmmodified{2013-03-22 18:56:40}
\pmowner{milogardner}{13112}
\pmmodifier{milogardner}{13112}
\pmtitle{Egyptian geometry areas calculated in, cubits, khets and setats}
\pmrecord{70}{41801}
\pmprivacy{1}
\pmauthor{milogardner}{13112}
\pmtype{Definition}
\pmcomment{trigger rebuild}
\pmclassification{msc}{01A16}

% this is the default PlanetMath preamble.  as your knowledge
% of TeX increases, you will probably want to edit this, but
% it should be fine as is for beginners.

% almost certainly you want these
\usepackage{amssymb}
\usepackage{amsmath}
\usepackage{amsfonts}

% used for TeXing text within eps files
%\usepackage{psfrag}
% need this for including graphics (\includegraphics)
%\usepackage{graphicx}
% for neatly defining theorems and propositions
%\usepackage{amsthm}
% making logically defined graphics
%%%\usepackage{xypic}

% there are many more packages, add them here as you need them

% define commands here

\begin{document}
An Egyptian scribe, Ahmes, recorded length, width, area, height, and volume calculations in the \PMlinkexternal{Rhind Mathematical Papyrus}{http://ahmespapyrus.blogspot.com/2009/01/ahmes-papyrus-new-and-old.html}. Three problems \PMlinkexternal{RMP 53-55}{http://rmp50-60.blogspot.com/} outlined basic geometry units as algebraic and finite. Length, width, area, height, and volume formulas stressed arithmetic quotient and remainders statements. 

Volume units were often limited to 1/64 of a hekat quotients statements. Remainders were expressed in 1/320 of a hekat (named ro) remainders such that(1/64)(5/5) = 5/320 of a hekat) = 5 ro.  At other times Ahmes recorded hekat units only in terms of 1/320 quotient and remainder statements.

Ahmes used 400 year old standardized quotient and remainder arithmetic. The unit fraction system had been poorly decoded by 20th century scholars. By 2002 CE, scribal weights and measures revealed (1/30)cubit-cubit-cubit = (1/20)khar = 1 hekat and other ancient finite math facts. Middle Kingdom scribes often replaced one hekat with (64/64) hekat, 10 hin, 64 dja, and 320 ro depending upon the application. The (64/64) hekat identity method may have first appeared in the 1900 BCE Akhmim Wooden Tablet. Hana Vymazalova first published the (64/64) hekat unity fact in 2002 CE.

The (64/64) hekat unity method was used over 60 times in the RMP and five times in the 300 year older Akhmim Wooden Tablet using divisors n in the range 1/64 < n < 64.

Exact quotient and remainder statements were recorded in concise unit fraction series. Units fraction answers were sometimes proven by being multiplied by initial divisors, and returned to unity. Five examples proofs appeared related to (64/64) being returned in the Akhmim Wooden Tablet and (320 ro) was returned in RMP 38. 

The Akhmim Wooden Tablet returned (64/64) validating five 'ab initia' (64/64)/n using divisors n: 3, 7, 10, 11 and 13 calculating binary quotient (Q/64) and (5/5) scaled ro remainders, a method that Ahmes used to convert 7/9 hekat in RMP 43, line 5. The RMP included over 40 examples of the (64/64)/n hekat unity division method. 

RMP 38 used a related method. Ahmes multiplied 320 ro by 7/22 and obtained (101 + 9/11)ro. A proof multiplied (101 + 9/11)ro by 22/7 and returned. 320 ro. The proof documents that scribal division was inverse to scribal multiplication, a well-known modern math fact.  

RMP 53 calculated two areas of 45/8 setat by 63/8 setat, and a third area of an undefined shape discussed by the note, 1/10 of 1 3/8 mh added to 10 cubits of land (COL). 

The exact setat measurement unit was 100 cubit by 100 cubit, or 10,000 square cubits. A cubit of land (COL), or mh, was one cubit wide by 100 cubits long, or 1/100 setat.

The first area, 45/8 setat, a triangle, had an altitude of 5 khet,  and a base of 9/4 khet, was found by triangle formula: 

1/2 the base times the altitude, 5*(9/4)*(1/2)= (45/8) = 5 5/8 setat.

The second area, also a triangle, had an altitude of 7 khet, and a base of 9/4 khet, was also found by a triangle formula: 

1/2 the base times the altitude, Ahmes calculated 7*(9/4)*(1/2) = 63/8 = 7 7/8 setat   

The third calculation found an area of undefined shape discussed by:

11/8 mh = 110/8 mh + 10 mh = 23 3/4 mh = 1/8 setat + 11 1/4 mh

since 12 1/2 mh = 1/8 setat.

Alternative views outline the third shape that could have defined a truncated pyramid (base 6, top 3, height 95/18), or a triangle (base 6, and altitude 95/12).

To assist a validated decoding of the third RMP 53 area RMP 54, and RMP 55 scribal guidelines have been consulted.

RMP 54 partitioned 7/10 setat by 10, 5, 2 1/2 and 1 1/4 segments. Proof was provided by multiplying one setat by 7/10, 14/10, 28/10 and 56/10 within a quotient and remainder context. A quotient setat and a scaled remainder mh were scaled as the 2/n table and a ro unit in hekat (volume unit) were scaled, by writing:  

a. (7/10)*(4/4) = 28/40 = (24 + 3)/40 = 3/8 setat + 300/40 mh = 5/8 setat + 7 1/2 mh

b. (14/10)*(4/4) = 56/10 = (55 + 1)/40 = 11/8 setat + 100/4 mh = 1 3/8 setat + 2 1/2 mh

c. (28/10)*(2/2) = 56/20 = (55 + 1)/20 = 11/4 setat + 100/20 mh = 2 3/4 setat + 5 mh

d. (56/10) = (55 + 1)/10 = 11/2 setat + 100/10 COL = 5 1/2 setat + 10 mh    

Ahmes may have also made calculations thinking in mh unuts. For example, 

Ahmes shorthand partition of 7/10 setat, (1/2 + 1/5) setat, may have focused upon 1/5 setat written as 20 mh. Knowing 12 1/2 mh was 1/8 setat, an answer may have been recorded by:

(1/2 + 1/5)setat = (1/2 + 1/8 + (20 - 12 1/2 mh) = 5/8 setat + 7 1/2 mh. 

RMP 55 took 3/5 of 5 setat to obtain 3 setat by four steps (a to d):

a. 1/2 setat + 10 mh

b. (1 + 1/8) setat + (7 + 1/2) mh

c. (1 + 3/8) setat + (2 + 1/2) setat

d. adding steps a and c, and knowing (12 + 1/2) mh = 1/8 setat

(1/2 setat + 10 mh) + [(1 + 3/8) setat + (2+ 1/2)mh] = (2 + 7/8)setat + (1 + 2 + 1/2) mh = 3 setat

Scribes in RMP 41, 42, 43, MMP 10 and the Kahun Papyrus used of four area (A) and volume (V) formulas that replaced radius (R) by diameter (D/2) and pi by 256/81 in the area of circle formula allowing 

A = (256/81)(D/2)(D/2) = (64/81)(D)(D), and adding height (H) four formula were written as:

a. A = (8/9)(8/9)(D)(D) cubits   (MMP 10, and RMP 41)

b. V = (H)(8/9)(8/9)(D)(D) cubits  (RMP 42)

c. V = (3/2)(H)(8/9)(8/9)(D)(D) khar (RMP 42)

d. V = (2/3)(H)((4/3)(4/3)(D)(D) khar (RMP 43 and the Kahun Papyrus)

e.In RMP 44 1500 khar times (1/20) = (75) 400-hekat, was not (75) '100-quadruple hekat'  The correct 400-hekat point was a point that was true for RMP 41, 42 and 43 by: "... I refer you again to Spalinger's  SAK 17 article and his discussions of RMP Book II (pp. 320-323), and to Griffith's writings on the RMP (as referenced by Spalinger) in PSBA 13-16, and to Griffith again in PSBA 14. As Griffith says in the latter (p. 429) in regard to whether single, double or quadruple hekats are intended: "the meaning is implied by context", and he adds in regard to hundreds of quadruple hekats that: "I conclude, therefore, that the scribe uses a rather inaccurate but perfectly intelligible abbreviation of language, in saying that '75 is the number of quadruple hekt' when (in may not be the truth) that 75 is the number of complete squares (of 100 each) of the quadruple hekt" (p. 430) ...".

RMP 43 4096/9 x 1/20 = (22 + 1/2 + 1/4 + 1/180) 100 hekat was listed on line 5 as:

*f. 1/180 400 hekat = 

4 x [100/180 = (5/9)x (64/64) = 320/576 = (288 + 18 + 9)/576 + 5/576 = 1/2 + 1/32 + 1/64 + (25/9)ro] =

4 x (100/180) = (5/9) x 4 x (64/64) = 4 x 320/576 = 4x (288 + 18 + 9)/576 + 5/576 = 1/2 + 1/32 + 1/64 + (25/9)4-ro] 

with (25/9)4-ro = (2) 4-ro + (7/9)4-ro = (2) 4-ro + (28/36)4-ro = 

(2) 4-ro + (18 + 9 + 1)/36 4-ro = (2) 4- ro + (1/2 + 1/4 + 1/36) 4-ro 

1 cubit-cubit-cubit = (3/2)khar =30 hekat  (reported in RMP 41, 42, and 43),

(2/3)cubit-cubit-cubit = 1 khar = 20 hekat (reported in RMP 41, 42 and 43),

(2/15) cubit-cubit-cubit = (1/5)khar = 4 hekat (reported in RMP 41, 42 and 43)

The hekat was also used to pay wages. An inverse pesu unit monitored exact amounts of grain in a loaf of bread, a glass of beer, and other products, such that wages were paid from two to eight hekats per worker per 30 days. The Reisner Papyri documents terseset badges were worn that authorized 10 day payments. RMP 1 to 6 dealt with labor managements issues on another level. 


\begin{thebibliography}{9}
\bibitem{1}  A.B. Chace, Bull, L, Manning, H.P., Archibald, R.C., \emph{The Rhind Mathematical Papyrus}, Mathematical Association of Amnerica, Vol I, 1927. NCTM reprints available.
\bibitem{2} Milo Gardner, \emph{An Ancient Egyptian Problem and its Innovative Solution, Ganita Bharati}, MD Publications Pvt Ltd, 2006.
\bibitem{3}Richard Gillings, \emph{Mathematics in the Time of the Pharaohs}, Dover Books, 1992.
\bibitem{4} Oystein Ore, \emph{Number Theory and its History}, McGraw-Hill Books, 1948, Dover reprints available.
\bibitem{5} T.E. Peet, \emph{Arithmetic in the Middle Kingdom}, Journal Egyptian Archeology, 1923.
\bibitem{6} Tanja Pommerening, \emph{"Altagyptische Holmasse Metrologish neu Interpretiert" and relevant phramaceutical and medical knowledge, an abstract,  Phillips-Universtat, Marburg, 8-11-2004, taken from "Die Altagyptschen Hohlmass}, Buske-Verlag, 2005.
\bibitem{7} Gay Robins, and Charles Shute \emph{Rhind Mathematical Papyrus}, British Museum Press, Dover reprint, 1987.
\bibitem{8} L.E. Sigler, \emph{Fibonacci's Liber Abaci: Leonardo Pisano's Book of Calculation}, Springer, 2002.
\bibitem{9} Hana Vymazalova, \emph{The Wooden Tablets from Cairo:The Use of the Grain Unit HK3T in Ancient Egypt, Archiv Orientalai}, Charles U Prague, 2002.
\end{thebibliography}



%%%%%
%%%%%
\end{document}
