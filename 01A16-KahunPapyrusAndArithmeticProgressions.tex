\documentclass[12pt]{article}
\usepackage{pmmeta}
\pmcanonicalname{KahunPapyrusAndArithmeticProgressions}
\pmcreated{2014-11-24 5:38:48}
\pmmodified{2014-11-24 5:38:48}
\pmowner{milogardner}{13112}
\pmmodifier{milogardner}{13112}
\pmtitle{Kahun Papyrus and Arithmetic Progressions}
\pmrecord{113}{41008}
\pmprivacy{1}
\pmauthor{milogardner}{13112}
\pmtype{Definition}
\pmcomment{trigger rebuild}
\pmclassification{msc}{01A16}

\endmetadata

% this is the default PlanetMath preamble.  as your knowledge
% of TeX increases, you will probably want to edit this, but
% it should be fine as is for beginners.

% almost certainly you want these
\usepackage{amssymb}
\usepackage{amsmath}
\usepackage{amsfonts}

% used for TeXing text within eps files
%\usepackage{psfrag}
% need this for including graphics (\includegraphics)
%\usepackage{graphicx}
% for neatly defining theorems and propositions
%\usepackage{amsthm}
% making logically defined graphics
%%%\usepackage{xypic}

% there are many more packages, add them here as you need them

% define commands here

\begin{document}
\PMlinkexternal{Kahun Papyrus}{http://en.wikipedia.org/wiki/Kahun_Papyrus} (From Wikipedia)

The Kahun Papyyrus was discovered by Flinders Petrie in 1889. The fragmented hieratic text dates to 1825 BCE Egypt and includes mathematical and medical topics. Most Kahun fragments are kept at the University College London date to the reign of Amenemhat III. One of the fragments referred to as the Kahun Gynaecological Papyrus dealt with gynecological illnesses and conditions.

A second fragment began with a brief 2/n table. Middle Kingdom scribal converted rational numbers 2/n to exact and concise unit fraction series as the RMP scribe followed 200 years later. The KP 2/n table converted 2/3, ..., 2/21 to concise unit fraction series and implied n/p conversions in shorthand proofs as the RMP scribe also demonstrated. 

A longer Rhind Mathematical Papyrus (RMP) 2/n table converted 50 rational numbers, 2/3, ..., 2/101 to concise unit fraction series. A longhand LCM m conversion method waa defined in \PMlinkexternal{RMP 36}{http://planetmath.org/encyclopedia/RMP36AndThe2nTable.html}. The method scaled n/p by LCM m to mn/mp with the best divisors of mp recorded in red auxiliary numbers that summed to numerator mn calculated exact and concise unit fraction series. 

A. Considering KP arithmetic and arithmetic progressions the raw data contained finite arithmetic operating components. Arithmetic progressions and geometric proportions were the highest forms of Egypotian mathematics. The KP scribe defined a 10-term arithmetic progression summed to 100, with a difference (d) of 5/6. The KP arithmetic progression was discussed in two RMP problems.

Scribal arithmetic progressions were also found in the RMP. Ahmes listed two columns of data (published by Gillings in 1972). Ahmes's thinking is shown in Gillings' column 11 by multiplying 5/12 times 9, a fact that was needed to find the largest term of the RMP progression. Ahmes then added 10 and wrote out the correct largest term of the arithmetic progression, and subtracted 5/6, nine times. Gillings found the remaining terms of the progressions by using the KP's method. To understand the KP method, readers must make arithmetic calculations as the Middle Kingdom scribes wrote down in their three problems, double and triple checking your work with several tools.

Gillings' 1972 analysis of both RMP versions of Middle Kingdom arithmetic progression failed to parse the method in a manner that was comparable, in every respect, to the KP method. For example, Gillings noticed similar problems in the RMP (RMP 40, 64). Gillings muddled three pages of his analysis on the topic.

B. In 1987, Egyptologist Gay Robins, and Charles Shute, wrote a book on the Rhind Mathematical Papyrus (RMP). Five years later Egyptologist John Legon wrote on the KP and the same class of arithmetic proportions used in the RMP. The KP and RMP report the same arithmetic proportion method to find the largest term. The method: take 1/2 of the difference, 1/2 of 5/6 (5/12 in the KP) times the number of differences (nine times 5/12 = 15/4 in the KP) plus the sum of the A.P progression (100 in the KP) divided by the number of terms (10 meant 100/10 = 10 in the KP). Finally add column 11's result, 3 3/4, to 10, and the largest term, 13 3/4.

In unit fractions, the context of the text, add column 11: 5/12 times 9 writing 3 3/4 as 3 2/3 1/12 to 10 in column 12 beginning with the largest term 13 2/3 1/12. The scribe subtracted 5/6 nine times created remaining terms of the arithmetic progression.

Robins-Shute confused aspects of the problem by omitting the sum divided by the number of terms, a topic cited in a closely related RMP 40 problem. A scribal algebraic statement matched pairs added to 20 reporting five pairs summed to 100, a set of facts included in RMP 40.

The complete KP method found the largest term facts reported in RMP 64 and RMP 40 by John Legon in 1992. Scholars have parsed Rhind Mathematical Papyrus 40 a problem that asked that 100 loaves of bread to be shared between five men by finding the smallest term of an arithmetic progression.

C. A confirmation of the Kahun Papyrus arithmetic progression method must include discussions of RMP 40 and RMP 64. In RMP 64 Ahmes asked 10 men to share 10 hekats of barley with a differential of 1/8 defining an arithmetical progression. Robins and Shute reported: "the scribe knew the rule that, to find the largest term of the arithmetical progression, he must add half the difference to the average number of terms as many times as there are common differences, that is, one less than the number of terms".

1. number of terms: 10

2. arithmetical progression difference: 1/8

3. arithmetic progression sum: 10

The scribe used the following facts to find the largest term.

1. one-half of differences, 1/16, times number of terms minus one, 9,

    1/16 times 9 = 9/16

2. The computed parameter(1), was found by 10, the sum, divided by 10, the number of terms. It was inserted by Robins-Shute, but had not been high-lighted, citing 1 + 1/2 + 1/16, or 1 9/16, the largest term. The remaining nine terms were found by subtracting 1/8 nine times to obtain the remaining barley shares.

That is, the KP scribe used formula 1.0:1111111

(1/2)d(n-1) + S/n = Xn (formula 1.0)

with,

d = differential, n = number of terms in the series, S = sum of the series, Xn = largest term in the series allowed three(of the four) parameters: d, n, S and Xn, to algebraically find the fourth parameter. 

When n was odd, x (n/2) = S/n, 

and x 1 + xn = x2 + x(n -1) = x3 + x(n -2) = ... = x(n/2) = S/n, 

Note that Robins-Shute omitted the sum divided by the number of terms (S/n):

A modern footnote cites \PMlinkexternal{Carl Friedrich Gauss}{http://planetmath.org/encyclopedia/CarlFriedrichGauss.html} implementing as a grammar school student a solution to the n = even case. Ahmes and Gauss found the sum for 1 to 100 by using d = 1 following the same rule. Ahmes and Gauss reached the sum 5050 based on 50 pairs of 101 (1 + 101 = 2 + 99 = 3 + 98 = ...) by using an identical arithmetic progression rule.

D. A four level review of a Kahun Papyrus problem that reported 1365 1/3 khar as the volume of cylinder with a diameter of 12 cubit and height of 8 cubits summarized by:

1. Level 1 shows that pi was set to 256/81, and knowing one khar equaled 3/2 of a hekat, the scribe computed 1365 1/3 khar began with the area of a circle, $$A = (pi)r^2$$, and input pi = 256/81 and D = 2, considering:

a. A = (256/81)(D/2)(D/2) = (64/81)(D)(D)

b. A = (8/9)(8/9)(D)(D)   (algebraic formula 1.0)

This formula appeared in MMP 10, RMP 41, RMP 42, RMP 43, RMP 44, RMP 45, and RMP 46. In RMP 42 Ahmes adding height (H) and created two volume formulas.

c. V = (H)(8/9)(D)(8/9)(D) cubits squared  (algebraic geometry formula 1.1)

d. V = (3/2)(H)(8/9)(8/9)(D)(D) khar (algebraic geometry formula 1.2) converted cubits to khar unit

In the Kahun Papyrus and RMP 43 algebraic geometry formula 1.2 indirectly scaled by 3/2 considering

e. (3/2)V = (3/2)(H)(3/2)(8/9)(8/9)(D)(D) = (H)(4/3)(4/3)(D)(D)

f. V = (2/3)(H)[(4/3)(D)(4/3)(D)] khar  (algebraic geometry formula 1.3)  

g. or directly considering V = (3/2)(H)(8/9)(8/9)(D)(D) =(32/27)(D)(D)(H) =(2/3)(H)(4/3)(4/3)(D)(D) khar

published by Robins-Shute in "Rhind Mathematical Papyrus" 1987, on page 46, confirming a scribal algebraic relationship.

To numerically verify scribal algebraic geometry was used, input scribal raw data D = 12, H = 8 (from the Kahun Papyrus):

h. (4/3)(12) was reported as (16)(16) = 256 such that

i. V = (2/3)8(256)=(1365 + 1/3)khar

RMP 43 input D = 8 and H = 6 into the same formula

j. V = (2/3)(6)[(4/3)(8)(4/3)(8)] = (4)(32/3)(32/3) = 4096/9 = (455 + 1/9) khar

k. A 4-quadruple hekat (400 hekat) division required RMP 43 and the Kahun Papyrus by the volume (V) formula.

V = (2/3)(H)[(4/3)(4/3)(D)(D)] (khar) 

RMP 41, 42, 44, 45, 46, and 47 also input khar times 1/20 data as 4-hekat units to scale the ancient hekat to a modern 4800 ccm. In RMP 47 400-hekat was multiplied by 1/10 and 1/20 to 10 4-hekat and 5 4-hekat, respectivelu. As important RMP 47 multiplied 100-quadruple hekat (100 1-hekat) written as (6400/64)hekat by 1/30, 1/40, 1/50, 1/60, 1/70, 1/80, 1/90 and 100 to obtain 1-hekat quotients and 1-ro remainders. 

An archaeological study can test scribal feeding rates for quail, dove, duck and geese by considering RMP 83 data. This class of study can test 1/10 a hekat scaled to 480 ccm. A recent feasibility study shows1 hekat may have equaled 250-500 ccm for geese and ducks. 

2. Level 2 reported relative values of 12 fowls in terms of a set-duck unit paid in the following problem by:

a. 3 re-geese unit value 8 set-ducks = 24

b. 3 terp-geese unit value 4 set-ducks = 12

c. 3 Dj. Cranes unit value 2 set-ducks = 6

d. 3 set-duck unit value 1 set-duck = 3

total value 45 set-ducks. 

Not calculated, but included in the valuation of (12 - 1) = 11 with 100 - 45 = 55, cited 55/11 as the total value as 5 times the value of one set-duck, with each water fowl given a value based on the daily, 10 day, 30 day and total hekat of grain consumed at rates reported in \PMlinkexternal{RMP 83}{http://planetmath.org/encyclopedia/AhmesBirdFeedingRateMethod.html}. RMP 83 used the same (64/64) scaling method reported in the Akhmim Wooden Tablet and in \PMlinkexternal{RMP 47}{http://mathforum.org/kb/message.jspa?messageID=7212156&tstart=0}.

e. The RMP included subtle and therefore complex Egyptian \PMlinkexternal{bird feeding}{http://planetmath.org/encyclopedia/AhmesBirdFeedingRateMethod.html} problems that directly discussed MK bird valuations in hekats. The scribal manner implied that Egyptian fraction arithmetic was focused upon \PMlinkexternal{economic issues}{http://planetmath.org/encyclopedia/EconomicContextOfEgyptianFractions.html}, setting prices to pre-determined standards within:

Hekat/n recorded in binary quotients (hekat) and scaled ro remainders
=====================================================================

a. 1/40 (64/64)/40 = 1/64 hekat + [24/40(64) = (120/40(ro)]   

1/64 hekat + 3 ro

Ahmes' set-goose, dove and quail hekat feeding/consumption rate*

b. 1/20: (64/64)/20 = 3/64 + 20/20 ro = (2 + 1)/64 + ro

(1/32 + 1/64)hekat + 1 ro 

Ahmes' valuation of a set-duck based a feeding/consumption rate*

c. 1/16: (64/64)/16 = 4/64 = 1/16 hekat

1/10: (64/64)/10 = 6/64 + 20/20 =

[(4 + 2)/64 = (1/16 + 1/32)hekat + 1 ro

An AWT number ... and a djendjen feeding/consumption rate*

d. 1/9: (64/64)/9 = 7/64 + 5/9 ro = ( 4 + 2 + 1)/64 + [5/9 = (10/18)ro = (9 + 1)/18)]

(1/16 + 1/32 + 1/64)hekat + (1/2 + 1/18)ro

e. 1/8: (64/64)/8 = 8/64 = 1/8 hekat

f. 1/7: (64/64)/7 = 9/64 + 5/7 ro =

(8 + 1)/64 + [(10/14)ro = (7 + 2 + 1) /14 ro] =

(1/8 + 1/64)hekat + (1/2 + 1/7 + 1/14)ro

(another AWT number)

g. 1/6: (64/64)/6 = 10/64 + 20/6 ro =

(8 + 2)/64 + (3 + 1/3)ro=

(1/8 + 1/32) hekat + (3 + 1/3)ro

An Ahmes geese and crane feeding/consumption rate*

h. 1/5: (64/64)/5 = 12/64 + 20/6 ro =

(8 + 4)/64 hekat + (3 + 1/3)ro

(1/8 + 1/16)hekat + (3 + 1/3)ro

(An Ahmes' terp-goose hekat feeding/consumption rate*)

i. 1/4: (64/64)/4 = 16/64 = 1/4 hekat

j. 1/3: (64/64)/3 = 21/64 + 5/3 ro =

(16 + 4 + 1)/64 + (1 + 2/3)ro =

(1/4 + 1/16 + 1/64 hekat + ( 3 + 2/3) ro

(another AWT number)

k. 1/(5/2): (64/64)/2 + (64/64)/8 =

(1/2 + 1/8)hekat

(An Ahme' re-goose hekat feeding/consumption rate*)

l. 1/2: (64/64)/2 = 32/64 = 1/2 hekat

m. 1/1. (64/64)/1 = 64/64 = 1 hekat

Raw  hekat consumption and division of a hekat by 40, 20, 10,  9, 8, 7, 6, 5, 4,3, 5/2, 2, and 1 data was  taken and extended from Clagett's data  to a table of fowl valuations based on fowl hekat consumption rates* ...a basic economic fact that Ahmes understood very well.

3. Level 3 showed that Ahmes in \PMlinkexternal{RMP 38}{http://planetmath.org/encyclopedia/RMP35To38PlusRMP66.html} may have corrected the KP scribal over-estimate of grain volume in the cylinder. Ahmes may have down-sized the volume of a 12 cubit diameter and 8 cubit high cylinder to 1356 3/14 khar by approximating pi to 22/7, improving the volume estimate by 9 5/42 khar, though is no direct evidence of that suggestion.

4. Level four considers \PMlinkexternal{Greek}{http://planetmath.org/encyclopedia/PlatosMathematics.html} arithmetic
when Plato spoke of mathematics by:

"How do you mean?

I mean, as I was saying, that arithmetic has a very great and elevating effect, compelling the soul to reason about abstract number, and rebelling against the introduction of visible or tangible objects into the argument. You know how steadily the masters of the art repel and ridicule any one who attempts to divide absolute unity when he is calculating, and if you divide, they multiply, taking care that one shall continue one and not become lost in fractions.

That is very true.

Now, suppose a person were to say to them: O my friends, what are these wonderful numbers about which you are reasoning, in which, as you say, there is a unity such as you demand, and each unit is equal, invariable, indivisible, -what would they answer? "

from Chapter 7. "The Republic" (Jowell translation)."

and the work of \PMlinkexternal{Archimedes}{http://planetmath.org/encyclopedia/ArchimedesCalculus.html}.

that includes Archimedes showing that pi was an irrational number limited to well-defined rational number limits smaller than Egyptians recorded by 256/81 and 22/7 approximations.

E. The Kahun Papyrus calculation of the 1365 1/3 khar volume would likely have been corrected by Ahmes with pi set at 22/7, 175 years later, and improved by Archimedes including calculus, with Egyptian and Greek businessmen freed from the abstract number of Egyptians.

F. The Kahun Papyrus contains other numerical information. One data set, eight lines of large quotient and remainder rational numbers was preceded by 14 lines of missing data. The fragmented data may relate to a calculation ending with 1/12. The historical context of the data, cited below is unclear. 

15. 925157 + 1/3 

16. 708453 + 1/3 

17. 709533 + 1/3 

18. 508098 + 2/3 + 1/8 + 1/16 

19. 407042 + 2/3 

20. 440003 + 1/6 

21 209200 

22. 1/12

As a wild guess, a prime number analysis may offer a few hints to decoding aspects of the ancient data:

15. 2775460 divided by 3, factors (2, 2, 5, 73, 1901) divided by 3

16. 21283600 divided by 3, factors 2, 2, 2 ,5 ,13, 4093) divided by 3

17. 2128600 divided by 3, factors  (2, 2, 2, 5, 5, 29, 367) divided by 3

18. 508098 + 41/48 = 243887050 divided by 48, or

  121943525 divided by 24, factors (5, 5, 11, 443431) divided by 24

19. 1221128 divided by 3, factors (2, 2, 2, 152641) divided by 3

20. 2640019 divided by 6, factors (61, 113, 383) divided by 6

21. factors (2, 2, 2, 2, 5, 5, 523)

22. 209200 times 12 = 25700900

The data was converted to rational numbers and prime factors to consider astronomical cycles as a possible ancient context. More on this data when, or ever, reliable data becomes available.   

G. In summary, the Kahun Papyrus (KP) was notable for a 2/n table, an arithmetic progression problem, a calculation of the volume of a cylinder (in khar and hekats), a valuation of four classes of birds by the lowest valued bird ) a set-duck), an economic trading unit system used regionally in the Ancient Near East, that was passed down to the Greeks, and a large number problem that discussed an unclear aspect of scribal numerical abilities. 

Ancient Egyptian fraction arithmetic was finite. The finite system represented positive rational numbers that scaled 2/n to 2m/mn, and n/p to mn/mp. Selected divisors of mn and mp recorded red numbers that assisted scribes to record concise unit fraction series. Unit fraction seroes were typically 5-terms 1/a + 1/b + 1/c + 1/d + 1/e, or less, scaled in a manner that confused 19th and 20th century scholars. The Egyptian fraction Middle Kingdom notation was continuously used for 3,600 years. Scholars in the 21st century AD have finally parsed the 4,000 year old unit fraction notation that fell into disuse after 1454 AD. With the dominance of algorithmic base 10 decimal arithmetic, approved by the Paris Academy in 1585 AD, and updated by Napier and others, the earliest Egyptian fraction notation, and all its  descendants were replaced by 1600 AD and forgotten by 1900 AD.

The KP and RMP scribes used identical methods that calculated largest terms in arithmetic progressions. Formula 1.0 defined four variables (d, n, S and xn). Algebraically identical methods found any one of the four variables, knowing three variables.
 
Formula 1.0 did not rely on rational number differences(d) being converted to Egyptian fraction series. Agreement on the larger questions, i.e., what were the beginning, and intermediate arithmetic steps of the arithmetic progression formula, have been algebraically resolved, though not fully published in the academic journals. 

Other common calculations were used in the KP and the RMP. The KP scribe converted rational numbers 2/n and n/p to concise and exact Egyptian fractions based on finite Egyptian fraction arithmetic, and red auxiliary numbers in identical ways.

Research continues with respect to the scribal economic valuation problems, scaled by binary quotients and 1/320 (ro) remainders, recorded in the Akhmim Wooden Tablet, the RMP, and the KP. Other meta mathematics issues cited in the KP, an important mathematical text, are also being investigated.


\begin{thebibliography}{4}

\bibitem{1} Richard Gillings, \emph{"Mathematics in the Time of the Pharaohs"}, pages 176-180, MIT Press, Cambridge, 1972
\bibitem{2} John Legon, \emph{"A Kahun Papyrus Fragment"}, pages 21-24, Discussions in Egyptology 24, 1992.
\bibitem{3} Luca Miatello, \emph{"The difference 5 1/2 in a problem of rations from the Rhind mathematical papyrus"}, Historia Mathematica, vol 34, issue 4, pages 277-284, Nov. 2008.
\bibitem{4} Gay Robins and Charles Shute, \emph{"The Rhind Mathematical Papyrus"}, pages 41-43, British Museum Press, Dover Reprint, 1987.

\end{thebibliography}


\subsection{External links}
\begin{itemize}
\item \PMlinkexternal{John Legon's 1992 paper}{http://www.legon.demon.co.uk/kahun.htm}
\item \PMlinkexternal{Math-history-list discussion}{http://mathforum.org/kb/thread.jspa?threadID=1543145&tstart=0}
\item \PMlinkexternal{Luca Miatello's 2008 paper}{
http://www.sciencedirect.com/science?_ob=ArticleURL&_udi=B6WG9-4TB0V84-1&_user=10&_rdoc=1&_fmt=&_orig=search&_sort=d&view=c&_acct=C000050221&_version=1&_urlVersion=0&_userid=10&md5=09294f1424847a8cddb37965be55e3c6}

\end{itemize}

%%%%%
%%%%%
\end{document}
