\documentclass[12pt]{article}
\usepackage{pmmeta}
\pmcanonicalname{HenriLebesgue}
\pmcreated{2013-03-22 16:40:23}
\pmmodified{2013-03-22 16:40:23}
\pmowner{PrimeFan}{13766}
\pmmodifier{PrimeFan}{13766}
\pmtitle{Henri Lebesgue}
\pmrecord{5}{38880}
\pmprivacy{1}
\pmauthor{PrimeFan}{13766}
\pmtype{Biography}
\pmcomment{trigger rebuild}
\pmclassification{msc}{01A60}
\pmclassification{msc}{01A55}
\pmsynonym{Henri L\'eon Lebesgue}{HenriLebesgue}
\pmsynonym{Henri Leon Lebesgue}{HenriLebesgue}

\endmetadata

% this is the default PlanetMath preamble.  as your knowledge
% of TeX increases, you will probably want to edit this, but
% it should be fine as is for beginners.

% almost certainly you want these
\usepackage{amssymb}
\usepackage{amsmath}
\usepackage{amsfonts}

% used for TeXing text within eps files
%\usepackage{psfrag}
% need this for including graphics (\includegraphics)
%\usepackage{graphicx}
% for neatly defining theorems and propositions
%\usepackage{amsthm}
% making logically defined graphics
%%%\usepackage{xypic}

% there are many more packages, add them here as you need them

% define commands here

\begin{document}
\emph{Henri L\'eon Lebesgue} (1875 - 1941) French mathematician and author best known for the Lebesgue integral, the last professional mathematician to publically call 1 a prime number.

The son of a typesetter who died of tuberculosis, young Henri continued his studies thanks to the efforts of his mother and attended the Ecole Normale Sup\'erieure. By the time he turned 30, Lebesgue was a published book author with a book on primitive functions and another one on the trigonometric series. But to this day Lebesgue is remembered more for the Lebesgue integral and the Lebesgue measure. Lebesgue's saying that ``Mathematics reduced to general theories would be a beautiful form without content'' has found its way into many quotation dictionaries.
%%%%%
%%%%%
\end{document}
