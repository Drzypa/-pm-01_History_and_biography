\documentclass[12pt]{article}
\usepackage{pmmeta}
\pmcanonicalname{TheTop10MostBeautifulTheorems}
\pmcreated{2013-03-22 18:53:52}
\pmmodified{2013-03-22 18:53:52}
\pmowner{PrimeFan}{13766}
\pmmodifier{PrimeFan}{13766}
\pmtitle{the top 10 most beautiful theorems}
\pmrecord{5}{41745}
\pmprivacy{1}
\pmauthor{PrimeFan}{13766}
\pmtype{Feature}
\pmcomment{trigger rebuild}
\pmclassification{msc}{01A60}
\pmclassification{msc}{00A99}

\endmetadata

% this is the default PlanetMath preamble.  as your knowledge
% of TeX increases, you will probably want to edit this, but
% it should be fine as is for beginners.

% almost certainly you want these
\usepackage{amssymb}
\usepackage{amsmath}
\usepackage{amsfonts}

% used for TeXing text within eps files
%\usepackage{psfrag}
% need this for including graphics (\includegraphics)
%\usepackage{graphicx}
% for neatly defining theorems and propositions
%\usepackage{amsthm}
% making logically defined graphics
%%%\usepackage{xypic}

% there are many more packages, add them here as you need them

% define commands here

\begin{document}
A 1988 poll of readers of the {\it Mathematical Intelligencer} ranked some of the most well-known theorems in mathematics thus:

\begin{enumerate}
\item Euler's identity, $e^{i \pi} = -1$
\item Euler's formula for a polyhedron, $V + F = E + 2$
\item There are infinitely many prime numbers. See Euclid's proof that there are infinitely many primes.
\item There are only 5 regular polyhedra
\item The sum of the reciprocals of the squares of the positive integers is $\frac{\pi^2}{6}$. See the Basel problem.
\item A continuous mapping of a closed unit disk into itself has a fixed point
\item The square root of 2 is irrational
\item $\pi$ is a transcendental number
\item Every plane map can be colored with just 4 colors
\item Every prime number of the form $4n + 1$ is the sum of two square integers in only one way
\end{enumerate}

\begin{thebibliography}{1}
\bibitem{dw} David Wells, {\it The Penguin Book of Curious and Interesting Mathematics}. London: Penguin Books (1997): 126 - 127
\end{thebibliography}
%%%%%
%%%%%
\end{document}
