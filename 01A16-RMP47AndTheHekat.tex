\documentclass[12pt]{article}
\usepackage{pmmeta}
\pmcanonicalname{RMP47AndTheHekat}
\pmcreated{2015-02-02 15:22:26}
\pmmodified{2015-02-02 15:22:26}
\pmowner{milogardner}{13112}
\pmmodifier{milogardner}{13112}
\pmtitle{RMP 47 and the hekat}
\pmrecord{45}{42244}
\pmprivacy{1}
\pmauthor{milogardner}{13112}
\pmtype{Definition}
\pmcomment{trigger rebuild}
\pmclassification{msc}{01A16}

% this is the default PlanetMath preamble.  as your knowledge
% of TeX increases, you will probably want to edit this, but
% it should be fine as is for beginners.

% almost certainly you want these
\usepackage{amssymb}
\usepackage{amsmath}
\usepackage{amsfonts}

% used for TeXing text within eps files
%\usepackage{psfrag}
% need this for including graphics (\includegraphics)
%\usepackage{graphicx}
% for neatly defining theorems and propositions
%\usepackage{amsthm}
% making logically defined graphics
%%%\usepackage{xypic}

% there are many more packages, add them here as you need them

% define commands here

\begin{document}
Scaling aspects of hekat sub-divisions were reported in Rhind Mathematical Papyrus (RMP) 41, 42, 43, 44, 45, 46, 47, the Kahun Papyrus, and the Northumberland Papyri 1, 2 and 3.  To correct confusing scribal partitions of the hekat, muddled by 1920s scholars accepted until 1999 by Clagett and others by reporting transliterations as translations, please consider:

!. In  RMP 43 and the Kahun Papyrus a scribal volume formula was well defined by:

V = (2/3)(H)[(4/3)(4/3)(D)(D)] (khar) 

2. Ahmes mentioned a 400-hekat unit and/or a conflicting (and possibly equivalent) 100-hekat unit by multiplying a khar unit by 1/20. In RMP 41, 42, 43, 44, 45, 46, and 47 Ahmes mixed hieratic symbols related to 400-hekat and 100-hekat units. Which one was the correct unit?

3. In RMP 47 a mixed  table calculated 100 (1-hekat) quotients, matched with(1-ro) remainders, and 100 (4-hekat) quotients matched with (4-ro) remainders \PMlinkexternal{Clagett's 1999 transliteration}{http://books.google.com/books?id=8c10QYoGa4UC&pg=PA469&dq=Ancient+Egyptian+Science} of Ahmes' shorthand notes reported muddled quotients in binary fractions and remainders scaled to ro, 1/320 of (1- hekat) and (4-hekat). Clagett's data paired (4-hekat) quotients with (1-ro) remainders is corrected by:

1/10 = 10 (1-hekat)

1/20 = 5 (1-hekat) 

1/30 = (3 + 1/4 + 1/16 + 1/64) (1-hekat) + (1 + 2/3)(1-ro)

1/40 = 2 1/2 (4-hekat)

1/50 = 2 (4-hekat)

1/60 = 1 1/2 1/8 1/32 (4-hekat) +  (3 1/3) (4-ro )

1/70 = 1 1/4 1/8 1/32 1/64 (4-hekat) +  (2 1/14 1/21) (4-ro) 

1/80 = 1 1/4 (4-hekat)
  
1/90 = 1 1/16 1/32 1/64 (4-hekat) +  1/2 1/18 (4-ro) 

1/100 = 1 (4-hekat) 

\PMlinkexternal{Marianne Michel}{https://www.academia.edu/8899140/Les_mathématiques_de_lÉgypte_ancienne._Numération_métrologie_arithmétique_géométrie_et_autres_problèmes_Safran_2014_  } showed in 2014 that the MMP partitioned 5 (4-hekat), 10 (4-hekat), 20 (4-hekat) and 40 (4-hekat) by 2/3. For example 20 (4-hekat) mulltiplied by 2/3 = (16 1/4 + 1/16 + 1/64)(4-hekat) (1 + 2/3) (4-ro) meant that the (1-hekat) partitions by 1/n required (1-ro )remainders, and (4-hekat) partitions by 1/n required (4-ro) remainders, with al other fractional facts remained unchanged.

4. Correcting transliterated data, by adding back scribal omissions, must consider one hekat written as (64/64), as reported by Hana Vymazalova as a hekat unity, per:
 to obtain translations to modern arithemtic:
 
a. (64/64)/n = Q/64 + (5R/n)ro

a formula proven in the 1925 BCE Akhmim Wooden Tablet. The formula was applied in the RMP over 40 times.

b. Proceed to RMP 47 in line 7, the 1/70 case, Clagett's transliteration error was numerically corrected by Robins-Shute, 1987, without citing the AWT formula. Corrected raw data states: 

(2 + 1/14 + 1/21+ 1/42)hekat 

c. translated into modern arithmetic by adding back scribal calculations such that

(150/70)ro  = (2 + 1/7)ro 

reported a 1/7 sentence:

1/7 = 1/7(6/6) = 6/42 = (3 + 2 + 1)/42 = 1/14 + 1/21 + 1/42  

The complete scribal calculations started at the beginning says:

(6400/64) x 1/70 =  91/64 + 30/4480 in modern arithmetic and

(64 + 16 + 8 + 2 + 1)/64 hekat + (150/70)ro in scribal arithmetic, citing

(1 + 1/4 + 1/8 + 1/32 + 1/32)hekat + (2 + 1/14 + 1/21 + 1/42)ro 

a corrected modern and ancient arithmetic answer.   

d. Ahmes mixed 400-hekat and 100-hekat quotient and remainder notations in RMP 47. The situation caused Peet, Chace, Robins-Shute and others to report muddled transliterated data. Ahmes encoded 400-hekat multiplications by 1/10 and 1/20 and 100-hekat multiplications by 1/30, 1/40, 1/50, 1/60, 1/70, 1/80, 1/90 and 1/100 in ways that require corrections when writing equivalent modern arithmetic statements. 

Adding back initial scribal data was mentioned by Tanja Pemmerening in 2002 and 2006, a rigorous point of view that stresses modern metric scalings is also corrected to ancient scaling formulas by applying a modified AWT formula

4 x (6400/64) x 1/n = (Q/64)4-hekat + (5R/n)4-ro

RMP 47's ancient formulas (by myself)  and metric units (by Tanja Pemmerening) are fairly decoded by: 

1/10 = (10)4-hekat  = (4800 x 4 x 10) =  192000 cc   

1/20 = (5)4-hekat = (4800 x 4 x 5) = 96000 cc 

1/30 = (3 + 1/4 + 1/16 + 1/64)4-hekat + (1 + 2/3)4-ro = (57600 + 4800 + 1200 + 300 + 60 + 40) = 64000 cc
           

1/40 = (2 + 1/2)4-hekat = (38400 + 9600) = 48000 cc     

1/50 = (2) 4-hekat  = (4800 x 4 x 2) = 38400 cc    

1/60 = (1 + 1/2 + 1/8+ 1/32)4-hekat + (3 + 1/3)4-ro = (19200 + 9600 + 2400 + 600 + 180 + 20) = 32000 cc

1/70 = (1 + 1/4 + 1/8 + 1/32 + 1/64)4-hekat + (2 + 1/14 + 1/21 + 1/42)4-ro =(19200 + 4800 + 2400 + 600 + 300 + 120 + 8 + 4/7) = 27428 + 4/7 cc
   
1/80 = (1 + 1/4)4-hekat =(19200 + 4800) = 24000 cc        

1/90 = (1 + 1/16 + 1/32 + 1/64)4-hekat + (1/2 + 1/18)4-ro = (19200 + 1200 + 600 + 300 + 30 + 3 + 1/3) = 21333 + 1/3 cc

1/100 = (1)4-hekat = (4800 x 4) = 19200 cc

5. Additional scribal light can be shed on the 4-hekat issue by considering economic 4-sack shipping units recorded in Northumberland Papyri 1, 2 and 3 published by Barns and Gunn, 1948. Quadruple-sack and quadruple-hekat units were monitored into 4-hekat units of grain bread units and did not mix 4-hekat quotients with 1-ro remainders. Balanced (4-hekat + 4-ro) quotient or remainders or (1-hekat + 1-ro) quotients and remainders were recorded by Ahmes. RMP 47 seemed to mix these two mutually exclusive methods. Ahmes correctly reported khar divided by 20 into 400-hekat with 4-ro remainders, and did not intended 1-ro remainders to be understood. Clagett was one of many scholars that have improperly recorded 400-hekat quotients and 1-ro remainders within the same statements. Scribes recorded 400-hekat and 100-hekat divisions by rational numbers (that were inverted and multiplied). Answers can be correctly translated as one hekat = 4800 ccm;  1/10 of a hekat (hin)= 480 ccm;  4-ro = 60 ccm; and 1-ro = 15 ccm. Scholars that stress transliterations rather than translations often associated 4-hekat quotients with 1-ro remainders, rather than the correct 4-hekat quotient with the correct 4-ro remainders.

6. G. Reference (2) Marshall Clagett, 1999, Egyptian Science and Mathematics (Volume III) includes Chace’s 1927 views of the RMP, as well as transliterations of the Kahun Papyrus, the Moscow Mathematical Papyrus. The valid transliterations should not be considered complete translations. Missing initial and intermediate arithmetic steps were not parsed, and inserted, as well as the other mathematics, i.e. the attested arithmetic operations used by all Middle Kingdom scribes, info required to prepare complete translations.

\begin{thebibliography}{10}

\bibitem{1}A.B. Chace, Bull, L., Manning, H.P. and Archibald, R.C. \emph{The Rhind Mathematical Papyrus}, Mathematical Association of America, Vol 1, 1927, vol 2, 1929, and reprint 1979 (NCTM).
\bibitem{2} Georges Daressy, \emph{"Calculs Egyptiens du Moyan Empire, Recueil de Travaux Relatifs  De La  Philologie et al Archaelogie Egyptiennes Et Assyriennes XXVIII, 1906}, Paris, 1906.
\bibitem{3} Milo Gardner, \emph{An Ancient Egyptian Problem and its Innovative Solution, Ganita Bharati}, MD Publications Pvt Ltd, 2006.
\bibitem{4} Milo Gardner, \emph{The Egyptian Mathematical Leather Roll, Attested Short Term and Long Term}, History of the Mathematical Sciences, Editors: Ivor Gratton-Guinness, and B.S. Yadav, Hindustan Book Agency, 119-134, 2002.
\bibitem{5}Richard Gillings, \emph{Mathematics in the Time of the Pharaohs}, Dover Books, 1992.
\bibitem{6} T.E. Peet, \emph{Arithmetic in the Middle Kingdom}, Journal Egyptian Archeology, 1923.
\bibitem{7} Tanja Pommerening, \emph{"Altagyptische Holmasse Metrologish neu Interpretiert" and relevant phramaceutical and medical knowledge, an abstract,  Phillips-Universtat, Marburg, 8-11-2004, taken from "Die Altagyptschen Hohlmass}, Buske-Verlag, 2005.
\bibitem{8} Gay Robins, Charles Shute \emph{"Rhind Mathematical Papyrus", London, British Museum}, British Museum Press, 1987.
\bibitem{9} Anthony Spalinger \emph{"Rhind Mathematical Papyrus", SAK-17}, 1990.
\bibitem{10} Hana Vymazalova, \emph{The Wooden Tablets from Cairo:The Use of the Grain Unit HK3T in Ancient Egypt, Archiv Orientalai}, Charles U Prague, 2002.
\end{thebibliography}


%%%%%
%%%%%
\end{document}
