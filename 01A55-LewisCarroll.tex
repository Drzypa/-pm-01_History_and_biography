\documentclass[12pt]{article}
\usepackage{pmmeta}
\pmcanonicalname{LewisCarroll}
\pmcreated{2013-03-22 17:05:42}
\pmmodified{2013-03-22 17:05:42}
\pmowner{PrimeFan}{13766}
\pmmodifier{PrimeFan}{13766}
\pmtitle{Lewis Carroll}
\pmrecord{4}{39390}
\pmprivacy{1}
\pmauthor{PrimeFan}{13766}
\pmtype{Biography}
\pmcomment{trigger rebuild}
\pmclassification{msc}{01A55}
\pmsynonym{Charles Lutwidge Dodgson}{LewisCarroll}

\endmetadata

% this is the default PlanetMath preamble.  as your knowledge
% of TeX increases, you will probably want to edit this, but
% it should be fine as is for beginners.

% almost certainly you want these
\usepackage{amssymb}
\usepackage{amsmath}
\usepackage{amsfonts}

% used for TeXing text within eps files
%\usepackage{psfrag}
% need this for including graphics (\includegraphics)
%\usepackage{graphicx}
% for neatly defining theorems and propositions
%\usepackage{amsthm}
% making logically defined graphics
%%%\usepackage{xypic}

% there are many more packages, add them here as you need them

% define commands here

\begin{document}
{\em Charles Lutwidge Dodgson} (January 27, 1832 - January 14, 1898), better known by the pen name {\em Lewis Carroll}, was an English author, mathematician, logician, Anglican clergyman, and photographer.

His most famous writings are {\it Alice's Adventures in Wonderland} and its sequel {\it Through the Looking-Glass} as well as the poems ``The Hunting of the Snark'' and ``Jabberwocky'', all usually assessed to be within the genre of literary nonsense. However, amid the nonsense there are hints of an interest in mathematics, which prompted mathematician Martin Gardner to publish {\it The Annotated Alice}.

There is an urban legend that Queen Victoria, having enjoyed one of Carroll's children's books, wrote to him graciously suggesting that he dedicate his next book to her. Carroll, according to the story, obligingly did so dedicate it, but the work happened to be a mathematical opus (which did not amuse her) entitled {\it An Elementary Treatise on Determinants}. Although he did write a work with that title, he did not dedicate it to anyone in particular.

His early academic career veered between high-octane promise and irresistible distraction. He may not always have worked hard, but he was exceptionally gifted and achievement came easily to him. In 1852 he received a first in Honour Moderations, and shortly after he was nominated to a Studentship, by his father's old friend Canon Edward Pusey. However, a little later he failed an important scholarship through his self-confessed inability to apply himself to study. Even so, his talent as a mathematician won him the Christ Church Mathematical Lectureship, which he continued to hold for the next twenty-six years. The income was good, but the work bored him. Many of his pupils were older and richer than he was, and almost all of them were uninterested. However, despite early unhappiness, Dodgson was to remain at Christ Church, in various capacities, until his death.

There are societies dedicated to the enjoyment and promotion of his works and the investigation of his life in many parts of the world including North America, Japan, the United Kingdom, and New Zealand.

{\it This entry was adapted from the Wikipedia article \PMlinkexternal{Lewis Carroll}{http://en.wikipedia.org/wiki/Lewis_Carroll} as of May 16, 2007.}
%%%%%
%%%%%
\end{document}
