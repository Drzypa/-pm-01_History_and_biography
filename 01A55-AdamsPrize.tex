\documentclass[12pt]{article}
\usepackage{pmmeta}
\pmcanonicalname{AdamsPrize}
\pmcreated{2013-03-22 16:59:25}
\pmmodified{2013-03-22 16:59:25}
\pmowner{PrimeFan}{13766}
\pmmodifier{PrimeFan}{13766}
\pmtitle{Adams Prize}
\pmrecord{4}{39270}
\pmprivacy{1}
\pmauthor{PrimeFan}{13766}
\pmtype{Definition}
\pmcomment{trigger rebuild}
\pmclassification{msc}{01A55}
\pmclassification{msc}{01A60}
\pmclassification{msc}{01A61}
\pmclassification{msc}{01A65}
\pmsynonym{John Couch Adams Prize}{AdamsPrize}

% this is the default PlanetMath preamble.  as your knowledge
% of TeX increases, you will probably want to edit this, but
% it should be fine as is for beginners.

% almost certainly you want these
\usepackage{amssymb}
\usepackage{amsmath}
\usepackage{amsfonts}

% used for TeXing text within eps files
%\usepackage{psfrag}
% need this for including graphics (\includegraphics)
%\usepackage{graphicx}
% for neatly defining theorems and propositions
%\usepackage{amsthm}
% making logically defined graphics
%%%\usepackage{xypic}

% there are many more packages, add them here as you need them

% define commands here

\begin{document}
The {\em Adams Prize} is a prize awarded by the University of Cambridge to a young mathematician working in the United Kingdom. Like the Fields Medal, age 40 is the specific cutoff for this prize.

These prizes were instituted in honor of the British mathematician John Couch Adams in 1850, soon after he discovered Neptune. Each year the prize committee chooses a specific field of mathematics and awards the prize to a mathematician working in that field.
%%%%%
%%%%%
\end{document}
