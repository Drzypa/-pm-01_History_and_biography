\documentclass[12pt]{article}
\usepackage{pmmeta}
\pmcanonicalname{GoogleCalculator}
\pmcreated{2013-03-22 16:29:30}
\pmmodified{2013-03-22 16:29:30}
\pmowner{CompositeFan}{12809}
\pmmodifier{CompositeFan}{12809}
\pmtitle{Google calculator}
\pmrecord{8}{38665}
\pmprivacy{1}
\pmauthor{CompositeFan}{12809}
\pmtype{Definition}
\pmcomment{trigger rebuild}
\pmclassification{msc}{01A65}
\pmclassification{msc}{01A61}

\endmetadata

% this is the default PlanetMath preamble.  as your knowledge
% of TeX increases, you will probably want to edit this, but
% it should be fine as is for beginners.

% almost certainly you want these
\usepackage{amssymb}
\usepackage{amsmath}
\usepackage{amsfonts}

% used for TeXing text within eps files
%\usepackage{psfrag}
% need this for including graphics (\includegraphics)
%\usepackage{graphicx}
% for neatly defining theorems and propositions
%\usepackage{amsthm}
% making logically defined graphics
%%%\usepackage{xypic}

% there are many more packages, add them here as you need them

% define commands here

\begin{document}
The {\em Google calculator} is a software calculator available to anyone using the Google search engine by typing commands in the search box. The Google calculator can handle basic arithmetic, namely addition, subtraction, multiplication, division and square root (including those commands containing parentheses); unit conversions; percentages; calculations relating to certain very well-known constants.

The first example provided by Google is \verb=5+2*2=. The Google calculator obeys the rules of operator precedence and understands parentheses: \verb=3*(18-2)-1= correctly evaluates to 47. Sometimes the calculator understands the tacit multiplication operator, but it is best to always state it explicitly. The calculator will sometimes insert parenthesis into the expression to show it has understood correctly, e.g., in response to ``\$125 + 6\% of \$125'' it answers with ``(US\$ 125) + (6\% of (US\$ 125)) = 132.5 U.S. dollars.'' For sufficiently large numbers (for example, powers of two greater than $2^{42}$) it switches to scientific notation. Like most hardware calculators, the Google calculator also responds to \verb=0^0= with 1.

The calculator does not understand reverse Polish notation, nor is it capable of giving the prime factorization of small composite numbers.

Commands can also be given in word form, for example, ``two times two'' or ``thrice pi'', which evaluate to ``four'' and ``9.42477796'' respectively. Some of the more archaic number names can also be used, such as ``gross'' (for 144) and ``score'' (for 20).
%%%%%
%%%%%
\end{document}
