\documentclass[12pt]{article}
\usepackage{pmmeta}
\pmcanonicalname{WilliamFeller}
\pmcreated{2013-03-22 16:24:10}
\pmmodified{2013-03-22 16:24:10}
\pmowner{PrimeFan}{13766}
\pmmodifier{PrimeFan}{13766}
\pmtitle{William Feller}
\pmrecord{6}{38551}
\pmprivacy{1}
\pmauthor{PrimeFan}{13766}
\pmtype{Biography}
\pmcomment{trigger rebuild}
\pmclassification{msc}{01A60}
\pmsynonym{Willibrord Feller}{WilliamFeller}
\pmsynonym{Vilim Feller}{WilliamFeller}

\endmetadata

% this is the default PlanetMath preamble.  as your knowledge
% of TeX increases, you will probably want to edit this, but
% it should be fine as is for beginners.

% almost certainly you want these
\usepackage{amssymb}
\usepackage{amsmath}
\usepackage{amsfonts}

% used for TeXing text within eps files
%\usepackage{psfrag}
% need this for including graphics (\includegraphics)
%\usepackage{graphicx}
% for neatly defining theorems and propositions
%\usepackage{amsthm}
% making logically defined graphics
%%%\usepackage{xypic}

% there are many more packages, add them here as you need them

% define commands here

\begin{document}
\emph{William Feller} (1906 - 1970) American mathematician who specialized in probability \PMlinkescapetext{theory}, best known for the Feller process, and his classic textbook ``An Introduction to Probability Theory and Its Applications'', \PMlinkescapetext{Volume} 1.

Feller was born in Zagreb (now the capital of Croatia) and christened after Saint Willibrord. He moved to Germany to study at G\"{o}ttingen University, where he received his doctoral \PMlinkescapetext{degree}. Because of Nazism, he fled to Denmark, later Sweden and finally America, becoming a citizen just prior to World War II. As a member of the American Mathematical Society, Feller began the publication of the journal {\it Mathematical Reviews}.
%%%%%
%%%%%
\end{document}
