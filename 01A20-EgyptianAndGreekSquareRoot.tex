\documentclass[12pt]{article}
\usepackage{pmmeta}
\pmcanonicalname{EgyptianAndGreekSquareRoot}
\pmcreated{2015-02-06 6:04:24}
\pmmodified{2015-02-06 6:04:24}
\pmowner{milogardner}{13112}
\pmmodifier{milogardner}{13112}
\pmtitle{Egyptian and Greek square root}
\pmrecord{206}{42610}
\pmprivacy{1}
\pmauthor{milogardner}{13112}
\pmtype{Definition}
\pmcomment{trigger rebuild}
\pmclassification{msc}{01A20}
\pmclassification{msc}{01A16}
\pmsynonym{square root}{EgyptianAndGreekSquareRoot}

% this is the default PlanetMath preamble.  as your knowledge
% of TeX increases, you will probably want to edit this, but
% it should be fine as is for beginners.

% almost certainly you want these
\usepackage{amssymb}
\usepackage{amsmath}
\usepackage{amsfonts}

% used for TeXing text within eps files
%\usepackage{psfrag}
% need this for including graphics (\includegraphics)
%\usepackage{graphicx}
% for neatly defining theorems and propositions
%\usepackage{amsthm}
% making logically defined graphics
%%%\usepackage{xypic}

% there are many more packages, add them here as you need them

% define commands here

\begin{document}
A. Premise: Egyptian numeration scaled rational numbers to concise unit fraction series in a wide range of applications.  An unexpected application reported an inverse proportion encoded an algebraic arithmetic \PMlinkexternal{Egyptian square root}{http://mathforum.org/kb/thread.jspa?threadID=2421477} method in one to three steps. 

The once difficult to decode problem was one of the oldest in the history of mathematics. The problem was documented by \PMlinkexternal{Kevin Brown}{http://www.mathpages.com/home/kmath038/kmath038.htm} and \PMlinkexternal{E.B. Davis}{http://www.mth.kcl.ac.uk/staff/eb_davies/PDFfiles/209.pdf}. Scribal estimations were often accurate to four to nine decimal places.    

1. Middle Kingdom, Greek and medieval square root methods were used by scribes from 2050 BCE to 1454 AD. Modern code breaking methods tracked raw facts in 2012 that decoded both the upper and lower limits reported by Archimedes in algebraic arithmetic.
 
a. The square root of three problem was decoded and solved per: 

(1) step 1. guess (1 + 2/3)^2 = 1 + 4/3 + 4/9 = (2 + 3/9 + 4/9) = 2 + 7/9 = error 2/9

(2) step 2 reduce error 2/9 (3/10) = 1/15 

  divided 2/9 by [2(1 + 2/3) = 10/3]  add = 1/15

  added  1/15 to ( 1 + 2/3) such that

(3) (1 + 2/3 + 1/15)^2, error (1/15)^2 = 1/225, meant   (1 + 11/15) = 26/15

step 3 reduced error 1/225 (15/52) = 1/15(52) = 1/780 (same as step 2) 

reached 

(26/15 + 1/780)^2 = (1353/780)^2 in modern fractions

rather than algebraic geometry steps implied by Heron and E.B. Davis.

(4) Archimedes recorded a unit fraction series that began with step 2 data. 
step 3 1/780 was subtracted

(1 + 2/3 + 1/15 - 1/780)^2

such that (likely) reported

(1 + 2/3 + 1/30 + (13 + 6 + 4 + 2)/780)^2 = (1 + 2/3 + 1/30 + 1/60 + 1/130 + 1/195 + 1/390)^2

(5) E.J. Dijkerhuis "Archimedes" (1987) page 435 discussed

(27^1/2)/3 in terms of a continuing fraction series

cited 3^1/2, 26/15, 265/153, 1351/780

followed by

"... what would a more satisfactory solution of the question look like?"

this method solved (27^1/2)/3 in two steps

1. [(5 + 1/5)^]2/3 = (27 + 1/25)/3

2. (1/25) x 5/25)]/3 = (1/260}/3

such that

(5 + 1/5 + 1/25 - 1/260)/3 = 1351/780

(6) The basic method was decoded in reverse chronological order by taking off the algebraic geometry blinders worn by 20th century researchers, specifically: Heron's "Metrica", and Babylonian base 60 methods.

b. the lower limit 265/153 replaced step 2 info 1/15 by 1/17 by

1/17 rather than 1/15, (1+ 2/3 + 1/17) = (1 + 37/51)

such that (1 + 111/153) increased to (1 + 112/153) = 265/153 

proof:  

Archimedes estimated pi by

22/7 > pi > 223/71

(1) the method scaled 22/7 by 10/10 = 220/70

(2) obtained the lower limit by 22/7(10/10) = 220/70

such that:

220/71 was modified to 223/71

followed the  'best guess' method that

265/153 was calculated. 

2. A second application was used for the \PMlinkexternal{pesu}{http://planetmath.org/EconomicContextOfEgyptianFractions.html}. The pesu scaled Middle Kingdom Egyptian commodities that measured exact portions of grain used to produce one loaf of bread, one glass of beer and one of other products.

B. To demonstrate how and why Egyptian and square root worked LE Sigler translated Leonardo de Pisa (Fibonacci) "Liber Abaci" a 1202  CE Latin text (translated to English in 2002 CE). Sigler transliterated a square root of 10 problem that exposed steps two and three. Archimedes modified step three that estimated the square root of three. Commonly the three step square root method was used from 2050 BCE Egypt, the Classical Greek era (Archimedes), and in the medieval era (Fibonacci) until 1453 AD, when the "Liber Abaci" fell out of favor in Latin schools.

1. guess (3 + 1/6)^2, since 1/(3 + 3) offers an estimate 10 + (1/6)^2 = 10 + 1/36

2. divide 1/36 by 1/2 of 2(10 + 1/6) = 1/36 (6/38) = 1/228 .

Sigler stopped at this point, and continued with a misleading geometric root supposition that included graphs.

The raw data easily extends to:

(3 + 1/6 - 1/228)^2 =(3 + 37/228)= (3 + [10 + (1/228)^2],

seems to be an accurate translation, based on the following two data sets.

C. Parker reported a Demotic era square root of 200 as (14 + 1/7)^2. Note that only the first step of Fibonacci's method approximated 200^1/2 by

1. guess (14 + 1/7)^2 since 1/(14 + 14) estimate

200 + (1/7)^2 = 200 + 1/49

Had a more accurate estimate been needed, the Demotic scribe could have

2. divided 1/49 by 1/2 of (14 + 1/7) = 1/49(7/198) = 1/1372 estimated = 200 + (1/686)*2 = 200 + 1/470596 =

and given (1/7 - 1/1372)= 195/1372
  
(14 + 195/1372)^2

is accurate to (1/1372)^2

D. Gillings in 1972 in "Mathematics in the Time of the Pharaohs", page 217, cited a 200 BC Greek square root 164 by Archibald:

(12 + 2/3 + 1/15 + 1/26 + 1/32)^2 or ((12 + 2/3 + 1/15 + 1/24 + 1/32)^2,

quoted Archibald, "If 1/26 was an error for 1/24 the approximation would indeed be remarkable".

Let us find the remarkable approximation, if there was one, by applying a scribal three step method.

1. guess (12 + 5/6)^2 since 20/(12 + 1/12) estimated 200 + (5/6)^2

Rather than divide 25/36 by 2(12 + 5/6) an awkward data set a new guess

2. (12 + 4/5)^2 estimated (144 + 48/5 + 48/5 + 16/25) = 163 + 21/25, hence an error of 4/25.

3. divide 4/25 by 1/2 of (12 + 4/5) = 4/25(5/128) = 1/160 estimated 164 + (1/160)*2 = 164 + 1/25600 = (12 + 4/5 + 1/160)^2

For those skilled in Egyptian numeration the scribe scaled 4/5 in 24 parts by

4/5(24/24) = 96/120 = (2/3 + 1/15 + 1/24 + 1/40), add 1/160

since (1/32 -1/40) = 1/160

meant 164^1/2 approx as

(12 + 2/3 + 1/15 + 1/24 + 1/32)^2

was the historical unit fraction series.

4. The method also solved the square root of 114 per:

Step 1. guess#1 (10 + 14/20)^2 = (10 + 7/10)^2 = 100 + 7 + 7 + 49/100, 49/100 is large error

Step 2. guess#2 (10 + 2/3)^2 = 100 + 6 2/3 + 6 2/3 + 4/9 = 113 +( 3/9 +4/9) = 113 + 7/9, an acceptable 2/9 error

Step 3. reduce 2/9,  divide 2/9 by 1/2 of the inverse of (10 + 2/3) = 32/3 meant 1/2 of  3/32 = 3/64 such that 

(2/9)(3/64) = 1/96

best estimate: square root of 114: (10 + 2/3 + 1/96)^2 contained a small(1/96)^2 error

E. CONCLUSION: Algebraic arithmetic was written in vulgar and unit fractions from 2050 BCE to 1454 AD . A two-three step square root method was 'encoded'. The method lasted 3,500 year and accurately estimated irrational roots. The second and third steps operated on prime and composite numerators and denominators as inverse proportion
The methology employed three properties of the binomial theorem:
 
N = (a + b)^2 = a^2 + 2ab + b^2

found N^2 = a^2 + 2ab (property 1) as often as possible

with an Error = b^2 (property 2), as often as possible

The first step estimated (a + b)

with a = largest square number less than N^2

and b = (N^2 - a^2)/2a),

for example the square root of 10 estimated (3 + 1/6), meant a = 3, b = 1/6

A. when (a + b)^2 greater than N^2, b^2 was the error

so that the second step considered the inverse proportion of 2ab: (1/2c)

(property 3)

by multiplying (a + 1/2a)^2 = (2a^2 + 1)/2a x (2c/1)

B. when (a + b) ¡ N^2, c = [(N^2 - (a +b)^2]

also multiplied (2a^2 + 1) x 1/(2(N^2 - (a + b)^2)

for example, using the square root of 10 case, (3 + 1/6)^2 = (9 + 1/2 + 1/2 + 1/36)

found property 1: 10 = 9 + 1 and property 2 Error 1 = 1/36 so that property 3 wrote (3 + 1/6) as a vulgar fraction 19/6 so that

1/36 x 6/38 = 1/6 x 1/38 = 1/228 found an improved estimate

(3 + 1/6 - 1/228)^2 = (3 + 18/228)^2

with with a third step following the second step, if necessary, such that (1351/780)^2 was greater than PI was greater than (265/153)^2

 
references: 

1. \PMlinkexternal{Kevin Brown}{http://www.mathpages.com/home/kmath038/kmath038.htm}

2. \PMlinkexternal{E.B. Davis}{http://www.mth.kcl.ac.uk/staff/eb_davies/PDFfiles/209.pdf}

3. \PMlinkexternal{Planetmath}{http://planetmath.org/encyclopedia/5ArchimdesAndAhmesSquareRootOf3.html}

\begin{thebibliography}{5}
\bibitem{1}A.B. Chace, Bull, L, Manning, H.P., Archibald, R.C., \emph{The Rhind Mathematical Papyrus}, Mathematical 
\bibitem{2}Marshall Clagett \emph{Ancient Egyptian Science, Volume III}, American Philosophical Society, Philadelphia, 1999.
\bibitem{3}Richard Gillings, \emph{Mathematics in the Time of the Pharaohs}, Dover Books, 1992, PAGE 214-217.
\bibitem{4} H. Schack-Schackenburg, \emph{"Der Berliner Papyreys 6619", Zeitscrift fur Agypyische Sprache} , Vol 38 (1900), pp. 135-140 and Vol. 40 (1902), p. 65f.
\bibitem{5) L. E. Sigler, \emph{Fibonacci's Liber Abaci, Leonardo's Book of Calculation},Springer, NY, 2002, page 491  
\end{thebibliography}


\PMlinkexternal{Math Forum}{http://mathforum.org/kb/message.jspa?messageID=7930283} discussions; 
\PMlinkexternal{Why Study Egyptian Fractions}{http://mathforum.org/kb/thread.jspa?threadID=2404732}

\PMlinkexternal{Planetmath Remainder Arithmetic}{http://planetmath.org/encyclopedia /RemainderArithmeticVsEgyptianFractions.html} entry;

\PMlinkexternal{Planetmath Kahun Papyrus}{http://planetmath.org/encyclopedia/KahunPapyrusAndArithmeticProgressions.html} entry.


A. BACKGROUND CONSIDERATIONS

To directly decode Egyptian Middle Kingdom, Classical Greece and Hellene square root recorded in unit fraction systems raw data from 2050 BCE to 800 CE have been parsed. Raw data analyses show that square roots of 2, 6, 164 and other irrational roots were first approximated by scribes by 'quick and dirty' estimates. If the approximation was unacceptable to the scribe, a second estimate considered the family of integer quotients (Q) and n/5-type remainders(R) scaled by 24/24 or other unities. Final estimates were recorded to the scribal higher standard in (Q + R)^2  unit fraction series. These historical facts were freshly decoded and written up in late 2012.

B. PREDICATES  

1. Middle Kingdom Egyptians, Greeks and Hellenes until 800 CE scaled rational numbers n/p by highly divisible unities m/m to mn/mp in hard-to-read shorthand notations before easy-to-read unit fraction series were recorded. After 800 AD Arabs scaled rational numbers by subtraction in ways that continues aspects of square root operations. 

Egyptians Final unit fraction series scaled from rational numbers mn/mp, a multiplication context, that were created by inspecting the best divisors of mp that best summed to numerator mp. The Egyptian numeration system, that lasted until 800 AD was first reported in the \PMlinkexternal{EMLR}{http://emlr.blogspot.com/} 1/n tables, Kahun and Ahmes Papyri \PMlinkexternal{2/n}{ http://rmprectotable.blogspot.com/} tables. The unit fraction notation generally converted rational numbers n/p to concise unit fraction series that solved over 200 Middle Kingdom weights and measures, and a decentralized economy problems that controlled inventories and paid workers wages in scaled commodity units.

a. Middle Kingdom Egyptians stressed\PMlinkexternal{1/p}{http://emlr.blogspot.com/} and \PMlinkexternal{2/p}{ http://rmprectotable.blogspot.com/} conversions that taught scribal students to generally convert \PMlinkexternal{n/p}{http://www.academia.edu/617613/Egyptian_Fractions_Unit_Fractions_Hekats_and_Wages_-_an_Update} to concise unit fractions series. 

There are 10 quality ancient texts that described Egyptian finite rational number system written in 1/n tables, 2/n tables, volume units. wage payments, solutions to \PMlinkexternal{second degree equations}{http://planetmath.org/encyclopedia/BerlinPapyrusAndSecondDegreeEquations.html} and \PMlinkexternal{higher order math problems}{http://planetmath.org/encyclopedia/AnOverViewOfAhmesPapyrus.html}.

b. Available Greek era texts used the same class of unit fraction n/p tables that included 1/p, 2/p ... (n-1)/p conversions to concise unit fraction series. For example, an implicit \PMlinkexternal{Hibeh Papyrus}{http://planetmath.org/encyclopedia/HibehPapyrus.html} n/45 table written in 300 BC may used trivial LCM 1, 2 and 4 scaling factors to record time units. 

2. Beginning with Parker's Coptic approximate square root of 200 as (14 + 1/7)^2, it is clear that a 'quick and dirty' method was available to construction workers and student scribes. The easy method inspected the quotient 14, doubled it, and multiplied the inverse 1/28 by 4 (200 - 196) = 4/28 = 1/7. The easy method considered (200 - Q^2), with Q = 14,  hence, the 'quick and dirty' method estimated the square root of 200 as:

(14 + 1/7)^2 = 196 + 14/7 + 14/7 + 1/49 = 200 + 1/49  

accurate to 1/49.

Modern vulgar fraction proof: 

(14 + 1/7)^2 = (98/7 + 1/7)^2 = (99/7)^2 = 9801/49 = (9800 + 1)/49 = 200 + 1/49

C. GREEK ERA SQUARE ROOT: The Hibeh Papayrus  data offers examples of Greek numeration based on Egyptian numeration as Gillings, following the work of R.C. Archibald reported unified rational number subjects connected to a 200 BCE text discussed in MATH IN THE TIME OF THE PHARAOHS on PAGES 214-217

Square root of 164 denoted as [164]^1/2 estimated {12 + 2/3 + 1/15 + 1/26 + 1/32]

without citing alternative precise historical methods, was there one best square root method that Egyptians and Greeks used?

1. An advanced square root method follows \PMlinkexternal{Occam's razor}{http://math.ucr.edu/home/baez/physics/General/occam.html} says:


a. "Quick and dirty" Method

To quickly solve the square root of 164, begin with the estimate

(12 + R)^2 

R estimated by double quotient 12 + 12 = 24. Second the inverse 1/24. Third multiply 1/24 by (164 - 144) = 20/24 = 

(5/6)^2 = E(Q.D.)

given that 25/36 was unacceptable, the 200 BCE scribe used an improved method

b. 164^1/2 was near (12 + 4/5)^2, stated in terms of (Q + R)^2, quotient (Q) and remainder (R) meant:

(1) 164^ 1/2 = (144 + 48/5 + 48/5 + 16/25) = (163 + 21/25), an error = 4/25  

(2) to reduce the E1 below 4/25 

(a) divide (4/25) by 2(12 + 4/5) = 4/5 x 5/128 =  1/160  (Sigler, 2002)

and convert (12 + 4/5 + 1/160) to a unit fraction

(b) the scribe converted 4/5 to a 24/24 scaled unit fraction series:

(4/5)(24/24) = 96/120 = (80 + 8 + 5 + 3)/120 = [2/3 + 1/15 + 1/24 + 1/40 +/160]

based on

(1/32 - 1/40) = 1/160, allowed

[2/3 + 1/15 + 1/24 + 1/32]^2 to be recorded  

c. Finally, note that the scribe could not have considered 1/26 as the 3rd term for several reasons, 

(a) the first reason notices that (1/15 - 1/26) = (21/410) was not an easy to use unit fraction. 

(b) the second reason notes that one scholar had poorly transliterated 1/26 (from Doric or Ionian cipher to modern fractions). The actual base 10 rational number, 1/24, fits the scribal raw data pattern based on \PMlinkexternal{ Ahmes}{http://ahmespapyrus.blogspot.com/2009/01/ahmes-papyrus-new-and-old.html} and Greek scaled rational numbers.

(c) a third reason notes that the "quick and dirty" method had not been clearly identified by scholars in a 'first try' role. 

(d) a fourth reason notes that scholars had not been aware of the wide array choices that scribes had available to scale rational numbers to concise unit fraction series. One recently decoded scaling method, decoded in 2003 by Hana Vymazalova, involved a volume unit (hekat) scaled by (64/64). Applying this scaling choice quotients were binary hekat units, and remainders were scaled by 5/5 to 1/320 (ro) unit fraction series. 

3. The 200 BCE Greek scribe then reduced the rational number error E1 (164 - 163 21/25 = 4/25)associated with 1/40. 

A replacement 1/32  for the last term 1/40 reduced rational error E1 (4/25) to an acceptable E2 = (1/160)^2.

4. Modern vulgar fraction proof: 

(12 + 2/3 + 1/15 + 1/24 + 1/32)^2 = [12 + 320/480 + 32/480 + 20/480 + 15/480] = [12 + 129/160]^2=

 [144 + 19 + 7/20 + (129/160)^2 = 164 + (1/160)^2 

(readers may wish to double check the arithmetic)

D. MIDDLE KINGDOM MIDDLE ERA SQUARE ROOT 
Gillings called his view of the Egyptian method an Aha problem, not disclosing precise scribal link to the Berlin Papyrus and other texts. 
 
The Middle Kingdom pesu method, the basis of Gillings Aha method, was recorded in \PMlinkexternal{RMP 69}{http://planetmath.org/encyclopedia/RMP69AndTheBerlinPaprusProportionMethod.html}, 70, 71, 72, 73, 74, 75, 76, 77, 78,  and the Kahun Papyrus. A parallel connected the BP, RMP 69-78, and the KP was recognized by Schack-Schackenberg in 1900. The pesu fact footnoted by Clagett in 1999, but misunderstood by Clagett in the narrative. Clagett misreported the inverse proportion's division operation and basic calculations as 'single false position' rather than the two-sided single variable method that Schack-Schackenberg reported. 

1. Gillings confused the BP scribal solution by reporting the Egyptian scribal inform as simultaneous equations are solved today:

   x2 + y2 = 100,   4x - 3y = 0,  what are x and y?

a. Clagett's single false position suggestion was borrowed from 1920s attempts to read closely related Rhind Mathematical Papyrus problems and methods, an approach that not involved in the Berlin Papyrus either. 

b. The 1900 BCE Berlin Papyrus solution was reported by Schack-Schackenberg in 1900 AD as Ahmes reported his 1650 BCE solution in RMP 69. 

Assume the square of the first side (y) to be 1 cubit. 

Then the other side (x) will be 1/2 + 1/4. 

Then y2 = 1, and using Egyptian multiplication determined 

x2 = (1/2 + 1/4 + 1/2)* (1/4 + 1/8 1/4)* (1/8 + 1/16 1/2 + 1/4 1/4 + 1/8 + 1/8 + 1/16) 

= 1/2 + 1/16
 
Thus, x2 + y2 = 1 + 1/2 + 1/16. 

meant (1 + 1/2 + 1/16)^1/2 = (1 + 1/4) and (100)^1/2 = 10. 

Divide 10 by 1 + 1/4 and you get 8.

2. The Middle Kingdom Egyptian era's actual square root method was found by discussing the square root of 6

that scaled (6 x 6/6)^1/2 = (36/6)61/2 = 6(1/6)^1/2, an incorrect guess.

Scribes had scaled inverse aspect equivalent to the Berlin Papyrus method in another scaled manner


A Nov. 2012 decoded method approximated the square root of 6 in terms of s

(Q+ R)^2 + E2  and specifically (2 + 2/5 + 1/20)^2

given that:

[6(2/5]'^2 = [12/5]^2 = [2 + 2/5]^2 = (4 + 4/5 + 4/5 + 4/25) = (5 + 15/25 + 4/25 = 5 + 19/25)

searched for a E1 = 4/25 correction that estimated the square root value as

(2 + 2/5 + 1/20)*2  = {5 + 19/25  + 4/5 +  81/400 = 6 + 1/400}  

with E2 = 1/400

3. The Egyptian rational number method was copied by Greek in the 2nd century BCE. The Greek method solved the square root of 164 and other problems.  

For example the square root of 2, connected to the earlier square root of 6 and 1/64 may be interesting to review:

a. 2^1/2 approx. by (1 + 2/5)^2 = ( 1 + 24/25)  = understated error E1 = 1/25 

the scribe considered

1/5(24/24) = 48/120 = (40 + 5 + 3))/120 = 1/3 + 1/24 + 1/40) in the context

2^1/2 approx ( 1 + 2/5)^2 with an error of 1/5 = [2 - (1 + 4/5 + 4/25] 

a reduced error was obtained by dividing 1/5 by 2 times (1 + 2/5) = 1/5 x 5/14 = 1/14
and adding (1 + 2/5) = (1 + 29/70)^2 = 98001/4900 = 2 + 1/4900

E. CONCLUSION: Egyptian and Greek square root exactly solved squared rational numbers 4, 9, 16, 25, .... by
a 'quick and dirty' method. The same 'quick and dirty' method estimated the square roots 2, 3, 5, 6, ...   in the form (Q + R)^1/2, with R being a unit fraction (or vulgar fraction). If the error was too high a second rational number error amount (E1) calculated a smaller error(E2). The second advanced method's final step squared (Q + R) and determined rational unit fraction roots, plus  irrational errors(E2)in proofs.

That is scribes focused upon were remainders(R). Initial remainders were often estimated in  n/5 parts such that n/5 was scaled by 24/24 = 24n/120 was discussed by:


1/5(24/24) = 24/120 = (20 + 3 + 1)/120

2/5(24/24).= 48/120 = (40 + 5 + 3)/120

3/5(24/24) = 72/120 = (60 + 8 + 3 + 1)/120

4/4(24/24) = 96/120 = (80 + 8 + 5 + 3)/120

and other partitions.

as Ahmes and earlier Middle Kingdom scribes scaled a volume unit hekat to (64/64) in terms of binary quotients and remainders scaled by (5/5) to 1/320 (ro) of a hekat.

Initial estimated square root of N statements used quotient (Q) in the form

(Q + n/5)^2

The raw square root data was processed in shorthand formats that considered rational errors E1 that scribes reduced to lower acceptable errors E2.

A broadly defined historical method, found by Occam's Razor, has been posted. Data bases will be attached to this discussion to validate unit fraction series that estimated square root for over 3,000 years. The long history of square root as a three step method was shared by Egyptians and Greeks. After 800 AD Arabs and medieval scribes like Fibonacci condensed the three steps to a less accurate two step method. Unit fraction square root ended with the 1585 AD arrival of the modern base 10 decimal system. 

At other times, like the square root of 114 other estimations took place:

For now, consider the square of 114 as a modern example of a finite arithmetic method that works today:

1. guess#1 (10 + 14/20)^2 = (10 + 7/10)^2 = 100 + 7 + 7 + 49/100 = 100 plus a large error

2. guess#2 (10 + 2/3)^2 = 100 + 6 2/3 + 6 2/3 + 4/9 = 113 +( 3/9 +4/9) = 113 + 7/9, a smaller 2/9 error

3. reduce 2/9,  divide 2/9 by 2(10 + 2/3) = 2/9(3/64) = 1/96

hence, (10 + 2/3 + 1/96)^2 is correct with (1/96)^2

proof (10 + 2/3 + 1/96) = 10 + 65/96 = (1025/96)^2 = 1050625/9216 = 114 + 1/9216

with a few notes 

The last step, step 3,  is the key to the method ...

it takes the delta of the estimate (N - (Q + R)^2) = delta1, delta2

or [(Q + R)^2 - N] = delta1, delta2

near to the pesu inverse proportion, as Bruce and I know very well ..

in modern math let N = the number that a square root estimate is computed .

if delta1 is near or exceeds 1/2; guess a new delta2

example:  N = 114

step 1: guess (10 + 14/20)^ = (10+ 7/10)^ =  49/100 =  delta1, too large

step 2: guess (10 + 2/3)^2 = (100 + 20/3 + 20/3 + 4/9) =  2/9 = delta2

usually  R^2 and reduced delta1 or delta2 to a new and acceptable  delta'

(delta1 or delta2)/(Q + R) = delta times 1/[2((Q + R)]

example:

step 3: 2/9 divided by (10 + 2/3) in the context

2/9[2(3/32) = 2/9(3/64) = 1/96

final estimate error (1/96)^2

a much better mathematical proof of the square root method will be created. In additon about a dozen examples spread across Egyptian, Greek and medieval eras when Phase II drafts a formal paper for publication. 

F. PHASE II 

1. The Greek era square root data was grossly under valued by authors like T.L Heath "\PMlinkexternal{Archimedes}{http://archive.org/details/worksofarchimede029517mbp}", 1897. Health imposed an odd modern geometric construction method, unknown to Pythagoras, Archimedes and other Greeks, by his own admission, rather than searching for a historical method,  

a. A Babylonian base 60 standard of 6-place accuracy for the square root of 3 cited Ptolemy 

(103/60 + 55/60^2 + 23/60^3)^2  

Surds seemed to be a closed subject to Heath after this misleading conclusion.  

b. Misleading meant that Heath had not known of the two-sided Egyptian and Greek 

3-step square root accurate to 5-places 

1. guess (1 + 1/2)^, or (1 + 1/3)^2

2. both yielded (1 + 5/12)^2, with an error of 1/144

3. apply step 2 a second time 

    1/144 x 12/34 = 1/408, hence

   (1 + 1/3 + 1/12 - 1/408)^2 = (579/408)^2

    was accurate to (1/408)^2 =  5-places 

4. Archimedes found (1 + 571/780)*2 = (1351/780)^2

   What was Archimedes' first guess? Inspecting 1/2, 7/12,  2/3 and 3/4, 
  
   it is clear that 2/3 was selected. Readers are advised to work 1/2
  
   and 3/4, two easy calculations to see how and why  the  max 1/780 

   denominator term was involved. 
    

Summary: Heath's misleading low opinion of Greek square root and surds should have been amended upward, made equal to, and likely better than, an awkward Babylonian method that had not been written in easy to use Greek unit fractions.

%%%%%
%%%%%
\end{document}
