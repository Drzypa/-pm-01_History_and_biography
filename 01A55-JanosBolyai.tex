\documentclass[12pt]{article}
\usepackage{pmmeta}
\pmcanonicalname{JanosBolyai}
\pmcreated{2013-03-22 16:32:45}
\pmmodified{2013-03-22 16:32:45}
\pmowner{Mravinci}{12996}
\pmmodifier{Mravinci}{12996}
\pmtitle{J\'anos Bolyai}
\pmrecord{6}{38728}
\pmprivacy{1}
\pmauthor{Mravinci}{12996}
\pmtype{Biography}
\pmcomment{trigger rebuild}
\pmclassification{msc}{01A55}
\pmsynonym{Janos Bolyai}{JanosBolyai}
\pmsynonym{Bolyai J\'anos}{JanosBolyai}
\pmsynonym{Bolyai Janos}{JanosBolyai}

% this is the default PlanetMath preamble.  as your knowledge
% of TeX increases, you will probably want to edit this, but
% it should be fine as is for beginners.

% almost certainly you want these
\usepackage{amssymb}
\usepackage{amsmath}
\usepackage{amsfonts}

% used for TeXing text within eps files
%\usepackage{psfrag}
% need this for including graphics (\includegraphics)
%\usepackage{graphicx}
% for neatly defining theorems and propositions
%\usepackage{amsthm}
% making logically defined graphics
%%%\usepackage{xypic}

% there are many more packages, add them here as you need them

% define commands here

\begin{document}
\emph{J\'anos Bolyai} (1802 - 1860) Hungarian mathematician, son of Farkas Bolyai. In the 1820s, J\'anos studied in Vienna, where he developed a keen interest in Euclid's parallel postulate, to the dismay of his father. By 1832, however, Farkas acknowledged his son's innovations in the field and published them as an appendix to his textbook. He accomplished a lot in the advancement of the study of non-Euclidean geometry. A 135 km crater on the moon is named after J\'anos Bolyai.
%%%%%
%%%%%
\end{document}
