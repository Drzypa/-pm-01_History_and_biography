\documentclass[12pt]{article}
\usepackage{pmmeta}
\pmcanonicalname{BernhardRiemann}
\pmcreated{2013-03-22 18:34:21}
\pmmodified{2013-03-22 18:34:21}
\pmowner{pahio}{2872}
\pmmodifier{pahio}{2872}
\pmtitle{Bernhard Riemann}
\pmrecord{5}{41296}
\pmprivacy{1}
\pmauthor{pahio}{2872}
\pmtype{Biography}
\pmcomment{trigger rebuild}
\pmclassification{msc}{01A70}
\pmclassification{msc}{01A55}
\pmsynonym{Riemann}{BernhardRiemann}

% this is the default PlanetMath preamble.  as your knowledge
% of TeX increases, you will probably want to edit this, but
% it should be fine as is for beginners.

% almost certainly you want these
\usepackage{amssymb}
\usepackage{amsmath}
\usepackage{amsfonts}

% used for TeXing text within eps files
%\usepackage{psfrag}
% need this for including graphics (\includegraphics)
%\usepackage{graphicx}
% for neatly defining theorems and propositions
 \usepackage{amsthm}
% making logically defined graphics
%%%\usepackage{xypic}

% there are many more packages, add them here as you need them

% define commands here

\theoremstyle{definition}
\newtheorem*{thmplain}{Theorem}

\begin{document}
The German mathematician Georg Friedrich Bernhard Riemann (17 September 1826 -- 20 July 1866) was born in Breselenz, Hanover, and died in Selasca, Italy.\, In spite of his short life, he made important pioneering works in mathematical analysis, differential geometry, analytic number theory and mathematical physics.\, He is one of the most notable mathematicians.

In his PhD thesis in 1851, supervised by Gauss, Riemann developed the theory of functions of one complex variable, the function theory, introducing among other things the surfaces which carry his name, the Riemann surfaces, and the Riemann sphere.\, In his habilitation work {\em \"Uber die Hypothesen welche der Geometrie zu Grunde liegen} he created the \PMlinkescapetext{basis} of the differential geometry, which has profoundly changed the notion of geometry, especially opening the \PMlinkescapetext{way} for the non-Euclidean geometries and the \PMlinkexternal{theory of relativity}{http://en.wikipedia.org/wiki/Theory_of_relativity}.

The only work of Riemann on number theory, {\em \"Uber die Anzahl der Primzahlen unter einer gegebenen Gr\"o\ss e}, forms together with some works of Chebyshev and Dirichlet the \PMlinkescapetext{foundation} of analytic number theory.\, It was a question of proving and sharpening the prime number theorem, assumed by Gauss.\, Riemann applies function theory making far-reaching consequences on the distribution of primes.\, In this context is born the Riemann hypothesis on the \PMlinkname{zeroes}{ZeroOfAFunction} of Riemann zeta function.
%%%%%
%%%%%
\end{document}
