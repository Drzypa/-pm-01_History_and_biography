\documentclass[12pt]{article}
\usepackage{pmmeta}
\pmcanonicalname{NinaBari}
\pmcreated{2013-03-22 16:38:41}
\pmmodified{2013-03-22 16:38:41}
\pmowner{Mravinci}{12996}
\pmmodifier{Mravinci}{12996}
\pmtitle{Nina Bari}
\pmrecord{4}{38848}
\pmprivacy{1}
\pmauthor{Mravinci}{12996}
\pmtype{Biography}
\pmcomment{trigger rebuild}
\pmclassification{msc}{01A60}
\pmsynonym{Nina Karlovna Bari}{NinaBari}

\endmetadata

% this is the default PlanetMath preamble.  as your knowledge
% of TeX increases, you will probably want to edit this, but
% it should be fine as is for beginners.

% almost certainly you want these
\usepackage{amssymb}
\usepackage{amsmath}
\usepackage{amsfonts}

% used for TeXing text within eps files
%\usepackage{psfrag}
% need this for including graphics (\includegraphics)
%\usepackage{graphicx}
% for neatly defining theorems and propositions
%\usepackage{amsthm}
% making logically defined graphics
%%%\usepackage{xypic}

% there are many more packages, add them here as you need them

% define commands here

\begin{document}
\emph{Nina Karlovna Bari} (1901 - 1961) Soviet mathematician renowned for her work in the trigonometric series. In her lifetime, her interest in descriptive mathematics was derided as being ``mathematics for ladies.'' This didn't stop her from being appointed professor of mathematics at the university in Moscow. Not directly related to Ruth Bari.

\begin{thebibliography}{1}
\bibitem{js} J. Spetich \& Douglas E. Cameron ``Nina Karlovna Bari'' in {\it Women of Mathematics: A Bibliographic Sourcebook} L. Grinstein, P. Cambpell, ed.s New York: Greenwood Press (1987): 6 - 12
\end{thebibliography}
%%%%%
%%%%%
\end{document}
