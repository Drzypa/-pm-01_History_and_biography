\documentclass[12pt]{article}
\usepackage{pmmeta}
\pmcanonicalname{UnitedStatesOfAmericaMathematicalOlympiad}
\pmcreated{2013-03-22 16:52:51}
\pmmodified{2013-03-22 16:52:51}
\pmowner{PrimeFan}{13766}
\pmmodifier{PrimeFan}{13766}
\pmtitle{United States of America Mathematical Olympiad}
\pmrecord{5}{39134}
\pmprivacy{1}
\pmauthor{PrimeFan}{13766}
\pmtype{Definition}
\pmcomment{trigger rebuild}
\pmclassification{msc}{01A65}
\pmclassification{msc}{01A61}
\pmclassification{msc}{01A60}

% this is the default PlanetMath preamble.  as your knowledge
% of TeX increases, you will probably want to edit this, but
% it should be fine as is for beginners.

% almost certainly you want these
\usepackage{amssymb}
\usepackage{amsmath}
\usepackage{amsfonts}

% used for TeXing text within eps files
%\usepackage{psfrag}
% need this for including graphics (\includegraphics)
%\usepackage{graphicx}
% for neatly defining theorems and propositions
%\usepackage{amsthm}
% making logically defined graphics
%%%\usepackage{xypic}

% there are many more packages, add them here as you need them

% define commands here

\begin{document}
The {\em United States of America Mathematical Olympiad} (USAMO) is an invitation-only mathematical competition sponsored by the Mathematical Association of America and hosted by the University of Nebraska-Lincoln. It is a 2-day event in which the contestants try to give proofs to six mathematical problems. The judges award 0 to 7 points to each solution, thus a perfect score is \PMlinkname{42}{FortyTwo} points. The winners go on to represent the United States in the International Mathematical Olympiad. Currently, the Internet is used to present the problems the first day of the competition; on the second day the contestants fax back their answers. The official website of the competition is \PMlinkexternal{http://www.unl.edu/amc/e-exams/e8-usamo/usamo.shtml}{http://www.unl.edu/amc/e-exams/e8-usamo/usamo.shtml}.
%%%%%
%%%%%
\end{document}
