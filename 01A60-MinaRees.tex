\documentclass[12pt]{article}
\usepackage{pmmeta}
\pmcanonicalname{MinaRees}
\pmcreated{2013-03-22 17:25:09}
\pmmodified{2013-03-22 17:25:09}
\pmowner{Mravinci}{12996}
\pmmodifier{Mravinci}{12996}
\pmtitle{Mina Rees}
\pmrecord{4}{39792}
\pmprivacy{1}
\pmauthor{Mravinci}{12996}
\pmtype{Biography}
\pmcomment{trigger rebuild}
\pmclassification{msc}{01A60}

% this is the default PlanetMath preamble.  as your knowledge
% of TeX increases, you will probably want to edit this, but
% it should be fine as is for beginners.

% almost certainly you want these
\usepackage{amssymb}
\usepackage{amsmath}
\usepackage{amsfonts}

% used for TeXing text within eps files
%\usepackage{psfrag}
% need this for including graphics (\includegraphics)
%\usepackage{graphicx}
% for neatly defining theorems and propositions
%\usepackage{amsthm}
% making logically defined graphics
%%%\usepackage{xypic}

% there are many more packages, add them here as you need them

% define commands here

\begin{document}
\PMlinkescapeword{straight}
\PMlinkescapeword{degree}
\PMlinkescapeword{development}
\PMlinkescapeword{way}

\emph{Mina Rees} (1905 - 1977) American mathematician, the first woman to be president of the American Association for the Advancement of Science.

Born in Cleveland of an insurance salesman and a housewife, young Mina began her education in the public schools of New York, earning straight A's on all her report cards. Later on, when barely a freshman at Hunter College, she was offered a teaching position there. She was a member of Phi Beta Kappa. After earning a degree from Columbia, Rees resumed teaching at Hunter College. In 1932 she earned a doctorate from the University of Chicago. After the United States entered World War II, Rees heeded the call of duty and began working for the Applied Mathematics Panel of the Office of Scientific Research \& Development. After the war, Rees worked for the Office of Naval Research and by 1952 she was appointed deputy science director. In 1953 she returned to Hunter College as a dean, and later was a dean at the City University of New York. In 1985, the CUNY honored Rees by naming one of its libraries after her.

Rees has \PMlinkname{Erd\H{o}s number}{ErdHosNumber} 4 by way of Richard Courant, Kurt Otto Friedrich and Harold Shapiro.

\begin{thebibliography}{1}
\bibitem{pf} P. Fox ``Mina Rees'' in {\it Women of Mathematics: A Bibliographic Sourcebook} L. Grinstein, P. Cambpell, ed.s New York: Greenwood Press (1987): 175 - 181
\end{thebibliography}

%%%%%
%%%%%
\end{document}
