\documentclass[12pt]{article}
\usepackage{pmmeta}
\pmcanonicalname{EgyptianMathematicalLeatherRoll}
\pmcreated{2014-11-15 7:38:18}
\pmmodified{2014-11-15 7:38:18}
\pmowner{milogardner}{13112}
\pmmodifier{milogardner}{13112}
\pmtitle{Egyptian Mathematical Leather Roll}
\pmrecord{128}{40108}
\pmprivacy{1}
\pmauthor{milogardner}{13112}
\pmtype{Definition}
\pmcomment{trigger rebuild}
\pmclassification{msc}{01A16}
%\pmkeywords{rational numbers}
\pmdefines{Egyptian fractions}

% this is the default PlanetMath preamble.  as your knowledge
% of TeX increases, you will probably want to edit this, but
% it should be fine as is for beginners.

% almost certainly you want these
\usepackage{amssymb}
\usepackage{amsmath}
\usepackage{amsfonts}

% used for TeXing text within eps files
%\usepackage{psfrag}
% need this for including graphics (\includegraphics)
%\usepackage{graphicx}
% for neatly defining theorems and propositions
%\usepackage{amsthm}
% making logically defined graphics
%%%\usepackage{xypic}

% there are many more packages, add them here as you need them

% define commands here

\begin{document}
ABSTRACT: The Egyptian Mathematical Leather ROLL (EMLR) was a 1900 BCE text that may have been written as late as 1650 BCE. The 26 line text offered conversions of 21 1/p and 1/pq unit fraction series. The text demonstrated six classes of conversion rules as predicates to a single and double least common multiple (LCM) rule. The single LCM rule demonstrated that 1/p was scaled by LCM m to m/mp that obtained concise and non-concise unit fraction series. The double LCM aspect of the rule introduced the student to a hekat double LCM scaling method. 
The EMLR scribe did not use red ink numbers as the Ahmes, the Rhind Mathematical Papyrs (RMP) scribe used in RMP 36 and RMP 37 (two problems that rigorously defined the Middle Kingdom LCM scaling rule).

INTRODUCTION: The \PMlinkexternal{EMLR}{http://en.wikipedia.org/wiki/Egyptian_Mathematical_Leather_Roll} has been housed at the British Museum since 1864. The EMLR and the Rhind Mathetical Papyrus(RMP) were placed at the BM by the family of Henry Rhind. The leather roll was softened and unrolled in 1927, almost 50 years after its sibling document, the RMP, was pirated from the British Museum and published in Germany in 1879. 

The first chapter included two Berlin Papyrus second degree equations was opened to scholarly discussions/debate in 1860, was resolved in 1900 by considering the pesu as an inverse proportion. The RMP 2/n table opened a second chapter that decoded Egyptian mathematics was resolved in 2006 by considering LCM m. The 50 member Rhind Mathematical Papyrus (RMP) 2/n table demonstrated that optimized LCMs scaled 2/p to 2m/mp such that concise unit fraction series were obtained. The RMP conversion rule, demonstrated in RMP 36 and RMP 37, denominator mp was factored into prime divisors. The scribe then selected the best divisors in red ink that summed to numerator mp and thereby obtained concise and not-so concise unit fraction series. 

Scholars reported the EMLR's arithmetic relationships that unintentionally overlooked theoretical aspects of unit fraction series. Other theoretical aspects of the \PMlinkexternal{RMP 2/n tables}{http://rmprectotable.blogspot.com} and the RMP's 87 problems continue to be overlooked by staff of the BM in 2010  \PMlinkexternal{British Museum publications and BBC programs}{http://www.bbc.co.uk/programmes/b00qg5mc} clearly documents.

UPDATES TO A 2004 EMLR PAPER: The EMLR text was an introduction to Egyptian fraction mathematics. Six scribal methods non-optimally converted 1/p and 1/pq to 26 awkward-looking unit fraction series by applying LCM m, a scaling factor. LCM m scaled rational numbers 1/2n, 1/p, and 1/pq to m/2mn, m/mp and m/mpq by six rules:

a. (1/2n)(m/m)to m/2mn

with divisors of denominator 2mn selected that summed to numerator m and a unit fractions series.                       

b. (1/p)(m/m) to m/mp 

with divisors of denominator mp that summed to numerator m and a unit fraction series.

c. (1/pq)(m/m) to mp/mpq 

with divisors of denominator mpq that summed to numerator mp and a unit fraction series.
                                                          ,  
d. (1/8)(25/8) to 25/200 to (8 + 17)/200 to 1/25 + 17/200 demonstrates the rule.
                         
e. (17/200)(6/6) to 102/1200 to(80 + 16 + 6)/1200 = 1/15 + 1/75 + 1/200 demonstrates the rule.                  

f. 1/8 = 1/25 + 1/15 + 1/75 + 1/2009 demonstrates a  rule used to scaled the hekat.             
                        
Rules d, e, and f show that LCM 25 scaled 1/18 and1/16 before 1/25 was subtracted calculating remainders 17/200 and 17/400. The remainders were scaled by LCM 6 thereby calculating three of the four unit fractions in the series, as the hekat remainders were scaled by 5/5 to ro 1/320th of a hekat units. 

Scholars before 2002 considered the EMLR as a possible decoding door to RMP 2/n table construction methods.  But no door was opened until the above six EMLR rules converted 22 rational numbers, 1/8 three times, 1/7 two times, and 1/16 two times, reported 26 non-optimized patterns in 2002.

From 2004 to 2006 the EMLR was suggested to include 1/p = 1/A x A/25, with 1/8 = 1/25 x 25/8.  

By 2006 an "Occam's Razor" update showed that 20 rational numbers were multiplied by one of seven LCM m: 2, 3, 5, 6, 7, 10, and 25, written as 2/2, 3/3, 5/5, 6/6, 7/7, 10/10 and 25/25. Each conversion obtained an awkward-looking unit fraction series showing that scribal students easily converted unit fractions to other unit fraction series. 

A deeper problem of selecting the best series was resolved by creating 2/n tables. One recent \PMlinkexternal{EMLR update}{http://www.york.cuny.edu/~malk/mycourses/math479-spring/egyptian.html} suggests: 

1/28 = [1/(a + b)](1/a + 1/b) = [1/(4 + 7)](1/4 + 1/7) = 1/44 + 1/77

a relationship that the EMLR student or Ahmes may have used to convert 2/35 and 2/91 discussed by:

2/35 = [2/(5 + 7)](1/5 + 1/7) = (1/6)(1/5 + 1/7) = 1/30 + 1/42

2/91 = [2/(7 + 13)](1/7 + 1/13) = (1/10)(1/7 + 1/13) = 1/70 + 1/130  

However, Ahmes shorthand notes cited 6/210 related to the 2/35 conversion

2/35(6/6)= 12/210 = (7 + 5)/210 = 1/30 + 1/42

a LCM m relationship that hold for

2/91(10/10) = 20/910 = (7 + 13)/910 = 1/70 + 1/130 

Hence, the reader must decide which approach was the historical method. For myself, Occam's Razor points in the LCM m scaling direction.

Wrapping up, the EMLR student used one LCM 23 times, and two LCMs twice, that computed 26 non-optimal Egyptian fraction series. For example, 1/8 was converted by LCM 3, 5, 6 and 25.  LCM 3 and 5 obtained non-optimal Egyptian fraction series by:

(1/8)(3/3) = 3/24 = (2 + 1)/24 = 1/12 + 1/24, 

(1/8)(5/5) = 5/40 = (4 + 1))/40 = 1/10 + 1/40, and

Two multiples, 25 and 6, created an out-of-order series: 

multiple one: 1/8(25/25) = 25/200 = (17 + 8)/200 =  17/200 + 1/25

multiple two: 17/200(6/6) = 102/1200 = (80 + 16 + 6)/200 = (1/15 + 1/75 + 1/200 

final answer: 1/25 + 1/15 + 1/75 + 1/200

Footnote one: a modified method converted 28/97 in RMP 31 and 30/53 in RMP 36 by converting 2/97 + 26/97 in RMP 31 and 2/53 + 28/97 in RMP 36 by two LCMs.  In RMP 31 2/97 was scaled by 56/56 and 26/97 was scaled by 4/4; and in RMP 36 2/53 was scaled by 30/30 and 28/53 was scaled by 2/2.

At one time it was suggested that another two step method had decreased the denominator by:

multiple one: 1/8(25/25) = 25/200 = (24 + 1)/200 = 24/200 + 1/200;

factor by 1/5: 24/200 = 1/5*(3/5);

multiple 3:1/5[3/5*(3/3)]

final answer: 1/15 + 1/25 + 1/75 + 1/200, 

with the out-of-order unit fractions denoting an unknown two-phase method.

LCM 6 converted rational numbers 1/7, 1/9, 1/11 and 1/15 by:

1/7(6/6)= 6/42 =(3 + 2+ 1))/42 = 1/14 + 1/21 + 1/42,

1/9(6/6)= 6/54 = (3 +  2 + 1)/54 = 1/18 + 1/27 + 1/54,

1/11(6/6)= 6/66 =(3 + 2 + 1)/66 = 1/22 + 1/33 + 1/66,

1/15(6/6)= 6/90 = (3 + 2 + 1)/90 = 1/30 + 1/45 + 1/90.

Interestingly, the EMLR student wrote out an error 

1/13 = 1/28 + 1/49 + 1/96 = 3/49.

rather than the implied 3/39, an impossible vulgar fraction to convert to a unit fraction series. 

Any one of three othwe LCMs 6, 12 and 14 scale 1/13 to solvable vulgar fraction series 6/78, 12/156 and 14/182

The student could have written 

1/13(6/6) = 6/78 = (3 + 2 + 1)/78 = 1/26+ 1/52 + 1/78

1/13(12/12) = 12/156 = (6 + 3 + 2 + 1)/156 = 1/26 + 1/52 + 1/78 + 1/156 

and LCM 14 by an unlikely (n + 1) pattern that coincides with four \PMlinkexternal{RMP 2/n Table}{http://rmprectotable.blogspot.com} patterns. The (n + 1) pattern was generalized in the RMP by 14 LCMs.

The EMLR student therefore could have written:

1/13(14/14) = 14/182 = (13 + 1)/182 = 1/14 + 1/182

Footnote two: Looking forward to the RMP the LCM m pattern was modified 20 times in the \PMlinkexternal{RMP 2/n table}{http://rmprectotable.blogspot.com} that easily converted 2/p and n/p rational numbers to concise unit fraction series.  The modified RMP  LCM m pattern was reported 3,000 years later as one of Fibonacci's seven \PMlinkexternal{Liber Abaci}{http://liberabaci.blogspot.com} methods.

Overall, the EMLR reported 22 unit fractions, prime, and composite denominators, converted to Egyptian fractions by using eight least common multiples: 2, 3, 4, 5, 6, 7, 10, and 25, including a two phase multiples 25 and 6 method (that converted 1/8 and 1/16). 

A broader narrative describing the EMLR is the found on \PMlinkexternal{Egyptian Mathematical Leather Roll}{http://en.wikipedia.org/wiki/Egyptian_Mathematical_Leather_Roll}.

An EMLR error incorrectly scaled rational number 1/13. The error showed that the student was not ready to graduate, or, as equally likely, an introduction to an advanced conversion method that was not understood. 

An updates in 2011 shows that six EMLR conversion methods connect to three advanced RMP conversion methods. For example the two out-of-order EMLR series 

1/8 = 1/25 + 1/15 + 1/75 + 1/200 

and 

1/16 = 1/50 + 1/30 + 1/150 + 1/400

were each scaled by LCM 25 and LCM 6 as an introduction to the second RMP rule

n/p = (n -2)/p + 2/p

a method that used two LCMs (examples 30/53 (RMP 36) and 28/97(RMP 31).

Footnote three: Advanced Egyptian scribes converted 2/n, n/p, and n/pq to short and concise unit fraction series. Scribes were expected to convert small and large rational numbers to concise Egyptian fraction series as needed. Egyptian fraction scribes were introduced to non-optimal conversions in the EMLR and advanced conversions in the \PMlinkexternal{RMP 2/n table}{http://rmprectotable.blogspot.com}, and a Greek Hibeh Papyrus n/45 table. 

CONCULSION: The EMLR is a 4,000 year of leather roll that demonstrated that 26 rational numbers were converted to non-optimal unit fraction series by a single LCM m such that 1/p x (m/m) = m/mp before inspecting the divisors of mp that summed to numerator m. Six introductory LCM rules considered 8 LCMs and more than 8 sets of red auxiliary numbers that summed to numerator m. LCMs and red auxiliary numbers were not explicitly included in the EMLR. The EMLR scribe converted 26 rational numbers Egyptian fraction series by applying six closely related conversion rules. The six rules validated that unit fractions 1/p and 1/pq were scaled to solvable exact unit fraction series as needed. Three of six EMLR rules scaled 1/8 and 1/16 by a pair of multiples, 25 and 6. The two LCM rule anticipated the second of three advanced RMP conversion rules: n/p = 2/p + (n -2)/p was also solved by two LCMs showed that scribes could select between alternatives. The EMLR was written by a student that may have gone on and learned Kahun Papyrus and RMP 2/n and n/p conversion rules and became a professional scribe.


\begin{thebibliography}{6}
\bibitem{1} Milo Gardner, \emph{The Egyptian Mathematical Leather Roll Attested Short Term and Long Term, History of Mathematical Sciences}, Hindustan Book Company, 2004.
\bibitem{2}Milo Gardner, \emph{"Mathematical Roll of Egypt", Encyclopaedia of the History of Science, Technology, and Medicine in Non-Western Cultures}, Springer, 2005
\bibitem{3} Richard J Gillings, \emph{The Egyptian Mathematical Leather Roll}, Australian Journal of Science 24 pgs 339-344, 1962.
\bibitem{4} Richard J Gillings, \emph{The Egyptian Mathematical Leather Roll}, Archive for History of Exact Sciences pgs 291-298, 1974
\bibitem{5} Richard J Gillings, \emph{The Egyptian Mathematical Leather Roll Line 8, how did the Scribe do it?}, Historia Mathematica pgs 456-457, 1981.
\bibitem{6}S.R.K Glanville, \emph{"Mathematical Leather Roll in the British Museum"}, Journal of Egyptian Archaeology pgs 232-8, 1927
\end{thebibliography}

%%%%%
%%%%%
\end{document}
