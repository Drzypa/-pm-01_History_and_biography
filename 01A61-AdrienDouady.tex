\documentclass[12pt]{article}
\usepackage{pmmeta}
\pmcanonicalname{AdrienDouady}
\pmcreated{2013-03-22 17:16:08}
\pmmodified{2013-03-22 17:16:08}
\pmowner{Mravinci}{12996}
\pmmodifier{Mravinci}{12996}
\pmtitle{Adrien Douady}
\pmrecord{5}{39608}
\pmprivacy{1}
\pmauthor{Mravinci}{12996}
\pmtype{Biography}
\pmcomment{trigger rebuild}
\pmclassification{msc}{01A61}
\pmclassification{msc}{01A60}

\endmetadata

% this is the default PlanetMath preamble.  as your knowledge
% of TeX increases, you will probably want to edit this, but
% it should be fine as is for beginners.

% almost certainly you want these
\usepackage{amssymb}
\usepackage{amsmath}
\usepackage{amsfonts}

% used for TeXing text within eps files
%\usepackage{psfrag}
% need this for including graphics (\includegraphics)
%\usepackage{graphicx}
% for neatly defining theorems and propositions
%\usepackage{amsthm}
% making logically defined graphics
%%%\usepackage{xypic}

% there are many more packages, add them here as you need them

% define commands here

\begin{document}
\PMlinkescapeword{level}

\emph{Adrien Douady} (1935 - 2006) French mathematician, best known for his work on dynamical systems.

From the outset, Douady intended to teach mathematics at the university level and trained specifically for that occupation, then taught at a university in Paris. His research focused on homological algebra at first, but gradually he became more interested in the work of Pierre Fatou and Gaston Julia on dynamical systems. In 1997, Douady was appointed to the Acad\'emie des Sciences. A quadratic Julia set is named after him, the Douady rabbit.

Douady has \PMlinkname{Erd\H{o}s number}{ErdHosNumber} 2.  He wrote a paper with Jacques Dixmier on fiber spaces in a French journal, and Dixmier wrote a paper with Erd\H{o}s on the number of fundamental invariants of binary forms in another French journal.
%%%%%
%%%%%
\end{document}
