\documentclass[12pt]{article}
\usepackage{pmmeta}
\pmcanonicalname{MultiplyPerfectNumbersPage}
\pmcreated{2013-03-22 17:50:49}
\pmmodified{2013-03-22 17:50:49}
\pmowner{PrimeFan}{13766}
\pmmodifier{PrimeFan}{13766}
\pmtitle{Multiply Perfect Numbers Page}
\pmrecord{4}{40318}
\pmprivacy{1}
\pmauthor{PrimeFan}{13766}
\pmtype{Definition}
\pmcomment{trigger rebuild}
\pmclassification{msc}{01A60}
\pmclassification{msc}{01A61}
\pmclassification{msc}{01A65}

\endmetadata

% this is the default PlanetMath preamble.  as your knowledge
% of TeX increases, you will probably want to edit this, but
% it should be fine as is for beginners.

% almost certainly you want these
\usepackage{amssymb}
\usepackage{amsmath}
\usepackage{amsfonts}

% used for TeXing text within eps files
%\usepackage{psfrag}
% need this for including graphics (\includegraphics)
%\usepackage{graphicx}
% for neatly defining theorems and propositions
%\usepackage{amsthm}
% making logically defined graphics
%%%\usepackage{xypic}

% there are many more packages, add them here as you need them

% define commands here

\begin{document}
{\em The~Multiply Perfect Numbers Page}, hosted by Bielefeld University, is currently the most up-to-date Webpage for information on multiply perfect numbers. Maintained by Achim Flammenkamp, this archive of more than five thousand multiply perfect numbers ``grew out'' of a similar database of a little more than two thousand which Richard Schroeppel compiled in 1995 but did not update any further.

Because multiply perfect numbers get very large very fast, it is actually not at all practical to archive a text representation of the numbers written in a particular base. Instead, the archive gives the logarithm of the logarithm, the number of distinct prime factors, the number of consecutive small factors, the exponents of the primes, etc. (in short, enough information to construct the base representation of a desired multiply perfect number) plus historical information such as the name of the discoverer and the date of discovery.

The Multiply Perfect Numbers Page is at \PMlinkexternal{http://wwwhomes.uni-bielefeld.de/achim/mpn.html}{http://wwwhomes.uni-bielefeld.de/achim/mpn.html}.
%%%%%
%%%%%
\end{document}
