\documentclass[12pt]{article}
\usepackage{pmmeta}
\pmcanonicalname{LarsAhlfors}
\pmcreated{2013-03-22 17:39:46}
\pmmodified{2013-03-22 17:39:46}
\pmowner{jirka}{4157}
\pmmodifier{jirka}{4157}
\pmtitle{Lars Ahlfors}
\pmrecord{6}{40097}
\pmprivacy{1}
\pmauthor{jirka}{4157}
\pmtype{Bibliography}
\pmcomment{trigger rebuild}
\pmclassification{msc}{01A65}
\pmsynonym{Lars Valerian Ahlfors}{LarsAhlfors}
\pmrelated{ErnstLindelof}

\endmetadata

% this is the default PlanetMath preamble.  as your knowledge
% of TeX increases, you will probably want to edit this, but
% it should be fine as is for beginners.

% almost certainly you want these
\usepackage{amssymb}
\usepackage{amsmath}
\usepackage{amsfonts}

% used for TeXing text within eps files
%\usepackage{psfrag}
% need this for including graphics (\includegraphics)
%\usepackage{graphicx}
% for neatly defining theorems and propositions
\usepackage{amsthm}
% making logically defined graphics
%%%\usepackage{xypic}

% there are many more packages, add them here as you need them

% define commands here
\theoremstyle{theorem}
\newtheorem*{thm}{Theorem}
\newtheorem*{lemma}{Lemma}
\newtheorem*{conj}{Conjecture}
\newtheorem*{cor}{Corollary}
\newtheorem*{example}{Example}
\newtheorem*{prop}{Proposition}
\theoremstyle{definition}
\newtheorem*{defn}{Definition}
\theoremstyle{remark}
\newtheorem*{rmk}{Remark}

\begin{document}
Lars Valerian Ahlfors (April 18, 1907 -- October 11, 1996) was born in Helsinki, Finland.  He studied and later worked at the Helsinki University.  He visited Harvard in the late 1930's.  For a short time at the end of World War II he worked at Swiss Federal Institute of Technology at Z\"{u}rich, but apparently did not enjoy his stay there.  Soon he left for Harvard University where he worked until he retired at 1977.

In 1936 he was among the first two recipients of the Fields medal.  He also won
the \PMlinkexternal{Wihuri Prize}{http://www.wihurinrahasto.fi/prizes.html} in 1968 and the Wolf Prize in Mathematics in 1981.  At Harvard he was
a William Caspar Graustein Professor of Mathematics from 1964 onward.

Lars Ahlfors is well known for his work on Riemann surfaces and complex analysis in general.  He is the author of one of the most used text books on complex analysis in one variable.
%%%%%
%%%%%
\end{document}
