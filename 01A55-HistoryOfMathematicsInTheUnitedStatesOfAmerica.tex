\documentclass[12pt]{article}
\usepackage{pmmeta}
\pmcanonicalname{HistoryOfMathematicsInTheUnitedStatesOfAmerica}
\pmcreated{2013-03-22 16:53:11}
\pmmodified{2013-03-22 16:53:11}
\pmowner{PrimeFan}{13766}
\pmmodifier{PrimeFan}{13766}
\pmtitle{history of mathematics in the United States of America}
\pmrecord{11}{39141}
\pmprivacy{1}
\pmauthor{PrimeFan}{13766}
\pmtype{Topic}
\pmcomment{trigger rebuild}
\pmclassification{msc}{01A55}
\pmclassification{msc}{01A60}
\pmclassification{msc}{01A61}
\pmclassification{msc}{01A65}
\pmclassification{msc}{01A50}
\pmclassification{msc}{01A45}
\pmsynonym{United States}{HistoryOfMathematicsInTheUnitedStatesOfAmerica}
\pmsynonym{United States of America}{HistoryOfMathematicsInTheUnitedStatesOfAmerica}

\endmetadata

% this is the default PlanetMath preamble.  as your knowledge
% of TeX increases, you will probably want to edit this, but
% it should be fine as is for beginners.

% almost certainly you want these
\usepackage{amssymb}
\usepackage{amsmath}
\usepackage{amsfonts}

% used for TeXing text within eps files
%\usepackage{psfrag}
% need this for including graphics (\includegraphics)
%\usepackage{graphicx}
% for neatly defining theorems and propositions
%\usepackage{amsthm}
% making logically defined graphics
%%%\usepackage{xypic}

% there are many more packages, add them here as you need them

% define commands here

\begin{document}
The {\em United States of America} \PMlinkescapetext{is a democratic republic founded in 1776 from the 13 British colonies in the American continent. After years of fighting, the British military was unable to defeat the American insurgents. The fact that the new nation was separated from Europe and Asia by the Atlantic and Pacific Oceans, and that communication was much slower in those days, meant that news of the latest scientific and mathematical advances reached America months or years after the fact.}

By necessity, the leading intellectuals of the insurgency era were Renaissance men who dabbled in various arts and sciences, a far cry from today's mathematician who specializes in one \PMlinkescapetext{field of mathematics almost to the point of ignorance of all other fields}. So, for example, Benjamin Franklin was well-versed in various astronomical computations, but he also studied magic squares, such as the Franklin magic square, and he conducted some of the earliest experiments in electrodynamics and had a hand in the founding of many important American government agencies.

\PMlinkescapetext{Lesser known but equally important to the development of the young nation were applied mathematicians like Nathaniel Bowditch (who studied the mathematics of marine navigation) and Joseph Caldwell (an important mathematics educator). Even as late as the Civil War, America lacked prominent specialized mathematicians: Benjamin Alvord was a brigadier general of volunteers during the Civil War; after the war he divided his free time between botany and the study of the} geometry of circles and spheres.

In 1888, Thomas Fiske founded the New York Mathematical Society, later renamed the American Mathematical Society.

\PMlinkescapetext{As Adolf Hitler rose through the ranks in Germany in the 1930s, a massive exodus of Jewish artists and scientists made their way to the United States after passing through western Europe (a few took even more indirect routes). Among the most important scientists and mathematicians to come to America were} Albert Einstein, John von Neumann, Enrico Fermi, Hans Albrecht Bethe and Edward Teller (Ede Teller). Many of them helped in the war effort, with cryptographers forming a signficant portion of the American mathematicians working during the war.

In 1950, by \PMlinkescapetext{order} of Congress, the National Science Foundation was founded.

For the second half of the 20th Century, the United States had significantly more native-born prominent specializing mathematicians than in the past: Carl Pomerance, William Fulton and John Cocke to name just a few.

\PMlinkescapetext{With the Internet becoming more important in the 21st Century, American mathematicians are increasingly studying the Web from a mathematical point of view, as well as of course using it to stay up to date on the latest mathematical developments around the world.}
%%%%%
%%%%%
\end{document}
