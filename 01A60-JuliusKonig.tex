\documentclass[12pt]{article}
\usepackage{pmmeta}
\pmcanonicalname{JuliusKonig}
\pmcreated{2013-03-22 17:02:43}
\pmmodified{2013-03-22 17:02:43}
\pmowner{WM}{16977}
\pmmodifier{WM}{16977}
\pmtitle{Julius K\"onig}
\pmrecord{19}{39334}
\pmprivacy{1}
\pmauthor{WM}{16977}
\pmtype{Biography}
\pmcomment{trigger rebuild}
\pmclassification{msc}{01A60}
\pmclassification{msc}{01A55}
\pmsynonym{Gyula K\"onig}{JuliusKonig}
\pmsynonym{K\"onig Gyula}{JuliusKonig}
\pmrelated{BijectionBetweenUnitIntervalAndUnitSquare}

\endmetadata

% this is the default PlanetMath preamble.  as your knowledge
% of TeX increases, you will probably want to edit this, but
% it should be fine as is for beginners.

% almost certainly you want these
\usepackage{amssymb}
\usepackage{amsmath}
\usepackage{amsfonts}

% used for TeXing text within eps files
%\usepackage{psfrag}
% need this for including graphics (\includegraphics)
%\usepackage{graphicx}
% for neatly defining theorems and propositions
%\usepackage{amsthm}
% making logically defined graphics
%%%\usepackage{xypic}

% there are many more packages, add them here as you need them

% define commands here

\begin{document}
\textbf{Julius K\"onig}, born 16 December 1849 in Gy\"or, Hungary,
died 8 April 1913 in Budapest, was a Hungarian mathematician.

\subsection{Biography} 

The Hungarian translation of the Latin name ``Julius'' is
``Gyula''. But when K\"onig contributed to German mathematical
journals, he called himself ``Julius''.

Julius (Gyula) K\"onig was literary and mathematically highly
gifted. He studied medicine in Vienna and, from 1868 on, in
Heidelberg. After having worked, instructed by Hermann von Helmholtz,
about electrical stimulation of nerves, he switched to mathematics and
obtained his doctorate under the supervision of Leo K\"onigsberger, a
very famous mathematician at those times. His thesis \emph{Zur Theorie
der Modulargleichungen der elliptischen Functionen} covers 24
pages. As a post-doc he completed his mathematical studies in Berlin
attending lessons by Leopold Kronecker and Karl Weierstrass. He then
returned to Budapest where he was appointed as a dozent at the
University in 1871. He became a professor at the Teacher's College in
Budapest in 1873 and, in the following year, was appointed professor
at the Technical University of Budapest. He remained with the
university for the rest of his life. He was on three occasions Dean of
the Engineering Faculty and also on three occasions he was Rector of
the University.  In 1889 he was elected a member of the Hungarian
Academy of Sciences. In 1905 he retired but continued to give lessons
on topics of his interest. His son D\'enes also became a distinguished
mathematician.

\subsection{Works}

K\"onig worked in many mathematical fields. His work on polynomial
ideals, discriminants and elimination theory can be considered as a
link between Leopold Kronecker and David Hilbert as well as Emmy
Noether. Later on his ideas were grossly simplified. So they are only
of historical interest today.

K\"onig already considered material influences on scientific thinking
and the mechanisms which stand behind thinking.

``The foundations of set theory are a formalization and legalization
of facts which are taken from the internal view of our consciousness,
such that our 'scientific thinking' itself is an object of scientific
thinking.''

But mainly he is remembered for his contributions to and his
opposition against set theory.

\subsection{K\"onig and Set Theory}

One of the greatest achievements of Georg Cantor was the construction
of a one-to-one correspondence between the points of a square and the
points of one of its edges by means of continued fractions. K\"onig
found a simple method involving decimal numbers which had escaped
Cantor.

1904, on the III. international mathematical congress at Heidelberg
K\"onig gave a talk to disprove Cantor's continuum hypothesis. The
announcement was a sensation and was widely reported by the press. All
section meetings were cancelled so that everyone could hear his
contribution.

K\"onig applied a theorem proved in the thesis of Felix Bernstein,
alas this theorem was not as generally valid as Bernstein had
claimed. Ernst Zermelo, the later editor of Cantor's collected works,
found the error already the next day. In 1905 there appeared short
notes by Bernstein, correcting his theorem, and K\"onig, withdrawing
his claim.

Nevertheless K\"onig continued his efforts to disprove parts of set
theory. In 1905 he published a paper proving that not all sets could
be well-ordered. It is easy to show that the finitely defined elements
of the continuum form a subset of the continuum of cardinality
$\aleph$$_0$. The reason is that such a definition must be given
completely by a finite number of letters and punctuation marks, only a
finite number of which is available.

This statement was doubted by Cantor in a letter to Hilbert in 1906: ``\emph{Infinite definitions} (which are not possible in finite time) are absurdities. If K\"onig's claim concerning the cardinality $\aleph$$_0$ of all 'finitely definable' real numbers was correct, it would imply that the whole \PMlinkescapetext{continuum} of real numbers was countable; this is most certainly wrong. Therefore K\"onig's assumption must be in error. Am I wrong or am I right?''

Cantor was wrong. Today K\"onig's assumption is generally
accepted. Contrary to Cantor, presently the majority of mathematicians
considers undefinable numbers not as absurdities. This assumption
leads, according to K\"onig, ``in a strangely simple way to the result
that the continuum cannot get well-ordered. If we imagine the elements
of the continuum as a well-ordered set, those elements which cannot be
finitely defined form a subset of that well-ordered set which
certainly contains elements of the continuum so.  Hence in this
well-order there should be a first not finitely definable element,
following after all finitely definable numbers. This is
impossible. This number has just been finitely defined by the last
sentence. The assumption that the continuum could be well-ordered has
led to a contradiction.''

K\"onigs conclusion is not stringent. His argument rests upon a change
of language.

The last part of his life K\"onig spent working on his own approach to
set theory, logic and arithmetic, which was published in 1914, one
year after his death. When he died he had been working on the final
chapter of the book.

\subsection{About K\"onig}

At first Georg Cantor highly esteemed K\"onig. In a letter to Philip
Jourdain in 1905 he wrote: ``You certainly heard that Mr. Julius
\emph{K\"onig} of \emph{Budapest} was lead astray, by a theorem of
Mr. \emph{Bernstein} which in \emph{general is wrong}, to give a talk
at Heidelberg, on the international congress of mathematicians,
opposing my theorem according to which every set, i.e., every
consistent multitude can be assigned an aleph. Anyway, the positive
contributions from K\"onig himself are well done.'' Later on Cantor
changed his attitude: ``What \emph{Kronecker} and his pupils as well
as \emph{Gordan} have said against set theory, what \emph{K\"onig},
\emph{Poincar\'e}, and \emph{Borel} have written against it, soon will
be recognized by \emph{all} as a \emph{rubbish}'' (Letter to Hilbert,
1912). ``Then it will show up that \emph{Poincar\'e's} and
\emph{K\"onig's} attacks against set theory are nonsense'' (Letter to
Schwarz, 1913).

\subsection{Some Papers and Books by K\"onig} 
\emph{Zur Theorie der Modulargleichungen der elliptischen Functionen},
Thesis, Heidelberg 1870.

\emph{\"Uber eine reale Abbildung der s.g. Nicht-Euclidischen
Geometrie}, Nachrichten von der K\"onig. Gesellschaft der
Wissenschaften und der Georg-August-Universit\"at zu G\"ottingen, No. 9
(1872) 157-164.

\emph{Einleitung in die allgemeine Theorie der Algebraischen
Groessen}, Leipzig 1903.

\emph{Zum Kontinuum-Problem'}, Mathematische Annalen 60 (1905)
177-180.

\emph{\"Uber die Grundlagen der Mengenlehre und das Kontinuumproblem},
Mathematische Annalen 61 (1905) 156-160.

\emph{\"Uber die Grundlagen der Mengenlehre und das Kontinuumproblem}
(Zweite Mitteilung), Mathematische Annalen 63 (1907) 217-221.

\emph{Neue Grundlagen der Logik, Arithmetik und Mengenlehre}, Leipzig
1914.

\subsection{Literature and Websources}
Brockhaus: Die Enzyklop\"adie, 20th ed. vol. 12, Leipzig 1996, p. 148.

W. Burau: Dictionary of Scientific Biography vol. 7, New York 1973, p. 444.

J. J. O'Connor, E. F. Robertson: The MacTutor History of Mathematics archive.

H. Meschkowski, W. Nilson (eds.): Georg Cantor Briefe, Berlin 1991.

W. M\"uckenheim: Die Mathematik des Unendlichen, Aachen 2006.

B. Sz\'en\'assy, History of Mathematics in Hungary until the 20th Century, Berlin 1992.

English Wikipedia: Article Julius K\"onig.


\PMlinkescapeword{argument}
\PMlinkescapeword{covers}
\PMlinkescapeword{fields}
\PMlinkescapeword{link}
\PMlinkescapeword{nerves}
\PMlinkescapeword{right}
\PMlinkescapeword{section}
\PMlinkescapeword{simple}
\PMlinkescapeword{theory}
\PMlinkescapeword{translation}
%%%%%
%%%%%
\end{document}
