\documentclass[12pt]{article}
\usepackage{pmmeta}
\pmcanonicalname{MayanMath}
\pmcreated{2015-04-29 18:58:00}
\pmmodified{2015-04-29 18:58:00}
\pmowner{milogardner}{13112}
\pmmodifier{milogardner}{13112}
\pmtitle{Mayan math}
\pmrecord{210}{40649}
\pmprivacy{1}
\pmauthor{milogardner}{13112}
\pmtype{Definition}
\pmcomment{trigger rebuild}
\pmclassification{msc}{01A25}
\pmclassification{msc}{01A12}
\pmdefines{congruences}

\endmetadata

% this is the default PlanetMath preamble.  as your knowledge
% of TeX increases, you will probably want to edit this, but
% it should be fine as is for beginners.

% almost certainly you want these
\usepackage{amssymb}
\usepackage{amsmath}
\usepackage{amsfonts}

% used for TeXing text within eps files
%\usepackage{psfrag}
% need this for including graphics (\includegraphics)
%\usepackage{graphicx}
% for neatly defining theorems and propositions
%\usepackage{amsthm}
% making logically defined graphics
%%%\usepackage{xypic}

% there are many more packages, add them here as you need them

% define commands here

\begin{document}
INTRODUCTION Two mathematical systems, one linear and the second modular, encoded Mayan astronomical texts. The linear system was positional with respect to the \PMlinkexternal{Olmec long count}{https://en.wikipedia.org/wiki/Mesoamerican_Long_Count_calendar} bases 18 and 20. The long count recorded day numbers in 1-18 and 1-20, respectively.  

Two examples 1.1.1.1 and 1.1.0.0.0.0 disclose distance numbers that are nearly equal to 19 sidereal cycles of Jupiter and 400 sideral cycles of Saturn, respectively, thereby extending visual retrogrades discussions of the planets (Brickers).  Mayans therefore nominally selected 399 for Jupiter and 378 for Saturn within a methodology that commenserated actual and nominal cycles of Mercury, Venus, earth, and Mars in paired planetary almanacs, measured against the paired Saturn and Jupiter long count.

The modular system also recorded base 13 remainder arithmetic in a manner that defined  least common multiple (LCM) almanacs. Almanac remainders were written in red and black, and often double and triple checked, recorded multiples of calendar rounds 18,980 days. Several LCMs 117, 260, 360, 584, 585, 780 and higher scaled planetary cycles in base  13 remainders. Calendars were recorded 1-13 remainders and distance numbers were recorded in 0-12 remainders.  

The long count can be seen as a numeration system predated the Maya positional system that allowed Mayans to encode 260 day, 360 day, 364 dates followed by four-part lunar months and days and solar months and days. The linear system followed the sun across the window of the annual migration of solstices equinoxes. The mid-year summer solstice (SS) marked the beginning of the primary Mayan solar time-keeping method.

Mayan astronomy linked lunar and solar modular calendars that followed aspects of the Chinese 'string of pearls' visual approach in families of almanacs (i.e Dresden Codex) that discussed nominal and actual planetary cycles.    

The linear aspect, written from right to left by Mayans, is written here left to right to conform with modern conventions , was positional, with exponent n = 0, 1, 2, ...:  

The long count calculated four distance numbers on a 419 AD wall (near Tikal. Guatemala) with four super-numbers. The first is divisible by 117, 260, 360, 365, 584, 585 and 780: Mercury, lunar, earth, Venus, Mars and 18 periods of the Mayan 52 year calendar round period of 18980 days; the second divisible by 117, 260, 360, 364, 365, 780 and 63 (18980); the third by  117, 260, 365, 780 and 91(18980), and the fourth by 117, 260, 365, 780 and 129(18980) as parsed by:

1. 341640 = (8)(365)(117) = (2)(3)(3)(73)(260) = (13)(73)(360) = (6)(156)(365)= (5)(9)(13)(584) = (6)(73)(780)= (3)(6)(18980) = 18-CR. An independent analysis of 120 LCM combinations of Mayan nominal planetary cycles 260, 360, 364, 365, 584, 585 and 780 yields 18-CR 28 times. The larger LCM CR data base is ”implied” and thus not on the wall.

In 2012 Aveni, et al, data reported the remaining super-numbers:

2. 1195740 = (4)(7)(365)(117) = (3)(3)(7)(73)(260) = (3)(3)(5)(73)(364)= (21)(156)(365)= (4)(7)(73)(585)= (21)(73)(780)= (3)(21)(18940) = 21(56940) meant LCM (260, 364, 365, 585) = LCM (364, 365, 585,780)= LCM (260, 364, 365, 585, 780) = LCM(364, 365, 584, 585).

3. 1765140 = (31)(219)(260) = (31)(156)(365) = (31)(73)(780)= (3)(31)(18980) = 31(56940)= LCM (260, 365, 780, 2263)

4. 2448420 = (43)(219)(260) = (43)(156)(365) = (43)(73)(780) = (3)(43)(18980)= 43(56940) = LCM (260, 365, 780, 3139)

Quotients of 260, 365, 780 and 56940 validates a Mars focus per 73(160) = 11680 rather than 11679 cited by Aveni and the Brickers. Note 20 Venus synodic cycles approximates 11680, off by one day. Powell ”New View of Mayan Astronomy” detail Mayan LCM methods that expose Mars, Jupiter and Saturn nominal cycles and methods that coincide with the LCM premise of this paper.


The  planets Mercury, Venus, Mars, Saturn, and Jupiter, plus the moon lined up in "string of pearls" combinations that aligned Chinese calendars to Feb. 2, 1951 BCE (facts seen on Stellarum and other astronomical programs), related evnets that Mayans placed at the center of their mythic and scientific worlds that double checked Mayan calendars.   

Mayan rational numbers scaled super-number distance numbers in exacting ways. Planetary almanacs divided calendar round periods that defined rational number quotients  and day remainders.

Mayan calendars were recorded in day quotient and day remainders. Solar calendars used in China, India, and the Hellene world only used remainder arithmetic. Ancient Near East lunar calendars and weights and measures were also written in quotient and exact remainder arithmetic. 

\PMlinkexternal{Joseph Needham}{http://en.wikipedia.org/wiki/Joseph_Needham} that wrote of the CRT indeterminate equation solution method reached the Hellene and medieval worlds via the Silk road no later than 100 AD. Fibonacci used the CRT in the Liber Abaci that solve several non-astronomical indeterminate problems.

A 405-moon lunar calendar decodes Mayan arithmetic texts within a practical 3 x 4  (base 4 x base 5) \PMlinkexternal{abacus}{http://share.shutterfly.com/action/picture dating method/ The lunar eclipse calendar dating method has been used by several ancient cultures. Given that Mayans and Mesoamericans began with a nine (9) lunar month equivalent, 260 days, the smallest lunar eclipse calendar, and a 360 day calendar, the second 3 x4 lunar calendar was likely unknown to Mesoamericans.

Sanchez' 1961 book "Arithmetic in Mayan" offers an abacus and other pertinent facts.

PEDAGOGY Astronomy and lunar calendars were the birth-parents of mathematics. The observable cycles of our u Writing, as a method of secondary thought, was built upon 6,000 years of 'star gazer' number systems. Unobservable lunisolar calendars emerged late, 499 BCE in the Babylonian Metonic 19 year cycle. Hence unobservable lunisolar calendars will not play a central role in this discussion.

LUNAR CALENDARS Mesoamerican calendars may be directly related to the Canary Island acano number system. Both used black and red colors to denote monthly lunar information. Jose Barrios Garcia reported the relationship this way, As Aaboe (1972) has shown, this ancient 135-moon eclipse count, most likely known in Babylon and Egypt and clearly known in China and Mesoamerica were derived from a simple arithmetical scheme that estimated eclipse years within eclipse year limits.

Dates were created in  to say, "As a matter of fact, to record a date on the acano you only need to write a number from 1 to 30 on one of its squares. The selected square fixes the moon while the number fixes the day of the moon counted, let us say, from new to new. Accordingly, it is possible to record unambiguously on a single acano the 33 successive dates fixing a whole round of the summer solstice through the lunar year. What is of the utmost importance is that this can be accomplished either through the years by actual observation, eithbtain the dates of the next summer solstices simply adding 11 days by year to the previous number. Each time the accumulated shift is greater than 29 or 30 days, we jump to the next square, reduce the shift by 29 or 30 days, write the new date on the square and continue the count. Actually, this exercise can be done even mentally for a number of years."  

MAYAN LUNAR CALENDAR DATING SYSTEM and nearby Mesoamericans altered lunar eclipse calendar dating method within two abaci. George I. Sanchez's 1961 book \PMlinkexternal{"Arithmetic in Maya"}{http://share.shutterfly.com/share/received/welcome.sfly?fid=7b22ffc939909759&sid=8AYsmzhi0ZNGPm} reports one of two Mesoamerican abacus. The first abacus was base five on one hand, with digits 1, 2, 3, and 4, and base 4 on the second hand, with digits 1, 2 and 3. The self-published Austin, Texas book is available at the CSU-Sacramento Library, and elsewhere.

\PMlinkexternal{Wikipedia}{http://en.wikipedia.org/wiki/Abacus} reports a second Mesoamerican abacus. David Esparsa Hidalgo, 1977: "Some sources mention the use of an abacus called a nepohualtzintzin in ancient Mayan culture. This Mesoamerican abacus used a 5-digit base-20 system.[23] The word Nepohualtzintzin comes from the Nahuatl and it is formed by the roots; Ne - personal -; pohual or pohualli - the account -; and tzintzin - small similar elements. And its complete meaning was taken as: counting with small similar elements by somebody. Its use was taught in the "Kalmekak" to the "temalpouhkeh", who were students dedicated to take the accounts of skies, from childhood. Unfortunately the Nepohualtzintzin and its teaching were among the victims of the conquering destruction, when a diabolic origin was attributed to them after observing the tremendous properties of representation, precision and speed of calculations.[citation needed].

The arithmetic tool was based on the vigesimal system (base 20)... The bases 4, 5, 13, 20 meant cycles as noted by the first abacus. The Nepohualtzintzin was divided in two main parts separated by a bar or intermediate cord. In the left part there were four beads, which in the first row have unitary values (1, 2, 3, and 4), and in the right side there are three beads with values of 5, 10, and 15 respectively. In order to know the value of the respective beads of the upper rows, it is enough to multiply by 20 (by each row), the value of the corresponding account in the first row.

Altogether, there were 13 rows with 7 beads in each one, which made up 91 beads in each Nepohualtzintzin. Factoring 91 reveals 7 times 13, a natural phenomena (7), the underworld (13) and the cycles of the heavens (91). One Nepohualtzintzin (91) represented the number of days of a season, two Nepohualtzitzin (182) the corn's cycle, from sowing to its harvest, three Nepohualtzintzin (273) is the number of days of a baby's gestation, and four Nepohualtzintzin (364) completed a cycle and near a year (1 1/4 days short). It is worth mentioning that the Nepohualtzintzin amounted to the rank from 10 to the 18 in floating point, which calculated stellar as well as infinitesimal amounts with absolute precision, meant that no round off was allowed, when translated into modern computer arithmetic."

The two base 5, base 4 abaci described \PMlinkexternal{"Arithmetic in Maya", 1961,}{http://jaie.asu.edu/v1/V1S2book.htm} (used by Lowland Maya) and Hidalgo, 1977, (used by nearby  Mesoamericans )exposes Mesoamerica as one mathematical region with several astronomical and abacus sub-regions.

AZTEC ARITHMETIC A second entry point to Mesoamerican number system story line is provided by an Aztec text reported in "Science" journal article. The article asks a Mesoamerican arithmetic question phrased within a regional use of prime number divisors and Mayan mathematical astronomy methods. The 4/4/08 issue of (\PMlinkexternal{"Science"}{http://www.sciencemag.org/cgi/content/abstract/320/5872/72?ck=nck}) proposes that several algorithms may have calculated area in an Aztec manner. Two manuscripts - one found in a library in France and the other in Mexico - were written on European paper by Aztecs a couple of decades after the conquest, using the Aztec system. The article offers an optimist view that nearby Mayan cosmology and prime number arithmetic was involved. The paper did not include modern or ancient views of the fundamental theorem of arithmetic, as reported from the history of number theory, and a lunar calendar door to the past. The paper's oversight may be corrected by overlaying a Mayan abacus consistent with the Chinese Remainder theorem to connect lunisolar 260 day calendars, lunisolar 405 moon calendars, great cycle calendars of 18980 days, and larger great years calendars without remainders.

Arithmetic examples are needed to shed appropriate light on related Aztec and Mayan topics. The first defines a \PMlinkexternal{modular congruence}{http://planetmath.org/encyclopedia/ResidueClass.html}. Floyd Lounsbury, writing on the number 1.5.5.0 of the Mayan Venus Table shows that a good history of number theory textbook allows observers to view the Mayan data outside our modern 10 decimal context, an academic mandate. Using only the history of number theory's use of prime numbers large chunks of Mesoamerican mathematical astronomy and arithmetic 'jump out of' otherwise unreadable codices. Lounsbury chose Burton's number theory text. Sanchez selected Ore's history of number theory text.

Floyd Lounsbury and other scholars have pointed out great years within Mayan lunar cycles connected to the 584 solar day Venus calendar context. The ancient Mesoamerican calendar info was likely processed by a Mesoamerican abacus, detailed by Sanchez in 1961. That is, an error in the 4/4/08 paper's use of a Western 'arrow of time' geometric proportion, reading the paper within modern base 10 decimals, is correctable by overlaying number theory congruences, and a Mayan abacus arithmetic detailed by Sanchez. The interesting 4/4/08 "Science" paper inappropriately named a geometric congruence, rather than a number theory congruence as a pre-1521 method that assisted Aztecs in determining taxes/tributes paid on the areas of properties owned/operated.

Modern number theory congruences also solve other planetary indeterminate equation problems and lunar eclipse counts. Mayans reduced eclipse findings to 260 day, 405 moon, 11960 day, and 11960 day calendars. A 408 solar moon cycle (33 solar years and 34 lunar years) aligned Islamic calenders and an acano method, close to a 405 moon cycle that assisted alignment of Mayan \PMlinkexternal{calendars}{http://www.jqjacobs.net/mesoamerica/meso_astro.html}. Alignment connected a 260 day lunar calendar, a 365 day calendar, and a LCM baed 18980 day calendar. Anthony Aveni summarized the connection this way, "Students of the codices need not be reminded that the eclipse table immediately follows and is attached to the Venus table. Moreover, these juxtaposed tables are related numerically: one great cycle (GC) = 37960 days - 3 x 11960 + 8 x 260 days... My attempts to seek solutions to this problem were stimulated by the work of Lounsbury (1983) who offered a plausible scheme for the placement of the Venus table in real time."

The Mayan great cycle calendar was not new. An ancient Chinese \PMlinkexternal{oral tradition}{http://www.halexandria.org/dward851.htm} reported the Chinese calendar as aligned by the visible planets, moon, and sun lining up as a 'string of pearls', within a sidereal lunar calendar. NASA 'proved' the Chinese tradition by modeling the 'string of pearls' planetary cycles to Feb 26, 1953 BCE through March 5, 1953 BCE. It may be important to note that a solar lunar calendar of 135 was commonly used in China, Babylon, Egypt, and Mesoamerica (Aaboe). 

In 2012 four super-numbers were dated to 419 AD. The long count dates were painted on a wall near Tikal, Guatemala reports the LCM (584, 585):

1. 341640 = (8)(365)(117) = (2)(3)(3)(73)(260) = (13)(73)(360) = (6)(156)(365)= (5)(9)(13)(584) = (6)(73)(780)= (3)(6)(18980) = 18-CR. An independent analysis of 120 LCM combinations of Mayan nominal planetary cycles 260, 360, 364, 365, 584, 585 and 780 yields 18-CR 28 times. The larger LCM CR data base is ”implied” and thus not on the wall.

In 2012 Aveni, et al, data reported the remaining super-numbers:

2. 1195740 = (4)(7)(365)(117) = (3)(3)(7)(73)(260) = (3)(3)(5)(73)(364)= (21)(156)(365)= (4)(7)(73)(585)= (21)(73)(780)= (3)(21)(18940) = 21(56940) meant LCM (260, 364, 365, 585) = LCM (364, 365, 585,780)= LCM (260, 364, 365, 585, 780) = LCM(364, 365, 584, 585).

3. 1765140 = (31)(219)(260) = (31)(156)(365) = (31)(73)(780)= (3)(31)(18980) = 31(56940)= LCM (260, 365, 780, 2263)

4. 2448420 = (43)(219)(260) = (43)(156)(365) = (43)(73)(780) = (3)(43)(18980)= 43(56940) = LCM (260, 365, 780, 3139)

Quotients of 260, 365, 780 and 56940 validates a Mars focus per 73(160) = 11680 rather than 11679 cited by Aveni and the Brickers. Note 20 Venus synodic cycles approximates 11680, off by one day. Powell ”New View of Mayan Astronomy” detail Mayan LCM methods that expose Mars, Jupiter and Saturn nominal cycles and methods that coincide with the LCM premise of this paper that scaled five cycles of 584 to 2920 and 2-Calendar Rounds, and longer relate to Floyd Lounsbury's decoding of long count 1.5.5.0 = 9100 4(18980) - 61(2340)._

\begin{REFERENCES}{7}

\bibitem{1}Anthony F. Aveni, \emph{"The Moon and the Venus Table", THE SKY IN MAYAN LITERATURE},Oxford Press, 1992.The Moon and the Venus Table", THE SKY IN MAYAN LITERATURE},Oxford Press, 1992.
\bibitem{2} Harvey M. Bricker and Victoria Bricker, \emph{"The Question of Jupiter and Saturn, Astronomy in the Maya Codices"}, American Philosophical Society, 2011.
\bibitem{3}David Burton, \emph{"Elementary Number Theory}, Allyn and Bacon, 1976.
\bibitem{4}David Espersa Hidalgo, \emph{"Nepohualtzintzin. Computador Prehispanico en Vigencia" [The Nepohualtzintzin: a pre-Hispanic computer in use]}, Mexico City, Mexico: Editorial Diana, 1977.
\bibitem{5} Floyd Lounsbury, \emph{"A Solution for the Number 1.5.5.0 of the Mayan Venus Table", THE SKY IN MAYAN LITERATURE, ed. A. Aveni}, Oxford Press, 1992.
\bibitem{6} Oystein Ore, \emph{"Number Theory and its History"}, McGraw-Hill, 1948.
\bibitem{7}George I. Sanchez, \emph{"Arithmetic in Maya"}, Austin-Texas, 1961.
\end{thebibliography}

\end{document}
