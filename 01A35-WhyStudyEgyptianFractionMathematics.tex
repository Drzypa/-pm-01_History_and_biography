\documentclass[12pt]{article}
\usepackage{pmmeta}
\pmcanonicalname{WhyStudyEgyptianFractionMathematics}
\pmcreated{2015-03-27 21:30:52}
\pmmodified{2015-03-27 21:30:52}
\pmowner{milogardner}{13112}
\pmmodifier{milogardner}{13112}
\pmtitle{Why Study Egyptian Fraction Mathematics}
\pmrecord{277}{41416}
\pmprivacy{1}
\pmauthor{milogardner}{13112}
\pmtype{Definition}
\pmcomment{trigger rebuild}
\pmclassification{msc}{01A35}
\pmclassification{msc}{01A30}
\pmclassification{msc}{01A20}
\pmclassification{msc}{01A16}

% this is the default PlanetMath preamble.  as your knowledge
% of TeX increases, you will probably want to edit this, but
% it should be fine as is for beginners.

% almost certainly you want these
\usepackage{amssymb}
\usepackage{amsmath}

% used for TeXing text within eps files
%\usepackage{psfrag}
% need this for including graphics (\includegraphics)
%\usepackage{graphicx}
% for neatly defining theorems and propositions
%\usepackage{amsthm}
% making logically defined graphics
%%%\usepackage{xypic}

% there are many more packages, add them here as you need them

% define commands here

\begin{document}
BACKGROUND: Old Kingdom (OK) hieroglyphic and Middle Kingdom numeration systems were written in base 10. OK Numerals were mapped many-to-one by a binary cursive algorithm onto number symbols. The OK numeration system rounded-off rational numbers by throwing away 1/64 units in arithmetic, algebra, geometry and weights and measures problems. Hieroglyphic math solved spiritual and secular math problems in infinite series statements from about 3,000 BCE to 2050 BCE. 

Hieroglyphic infinite series math was used to solve spiritual \PMlinkexternal{Eye of Horus problems} {http://en.wikipedia.org/wiki/Eye_of_Horus} in the OK, Middle Kingdom (MK) and later periods. Hieroglyphic arithmetic considered the range from zero and one (1) as important limits. For example, when one (1) was reached or stated in a math problem the idea was an ideal unity. 

MIDDLE KINGDOM  HIERATIC MATH: After 2050 BCE Egyptian Middle Kingdom (MK) math system was written in finite 10 arithmetic, algebra, algebraic geometry, and weights and measure systems that stressed an inverse proportion pesu unit. The MK finite math resolved to OK infinite series questions, namely, how to exactly write finite series rational numbers whenever possible.

One purpose of the hieratic system eliminated OK round-off errors whenever possible. Hieratic geometry problems corrected round-off errors by adding-back previously rounded-off OK 1/64 units. MK scribes replaced radius R by semi-diameter D/2, and pi by 256/81 in area and cylinder volume problems and 1/64 and 1/320 hekat units and in RMP 38 used 22/7n for pi to correct for expected hekat losses. In other cases MK arithmetic, algebraic geometry and weights and measures methods exactly scaled hekat units (of grain) and smaller 1/320 units of beer, bread, domesticated fowl, and grain based wage payments such that the pesu inverse proportion unit exactly measured the grain contained in a glass of beer and a loaf of bread so that grain was used as the primary wage payment system (2 hekats per 30 days for the lowest worker and 8 hekat for foreman with a family).

A secondary purpose assisted scribes by decentralizing the Egyptian economy in a http://en.wikipedia.org/wiki/Eye_of_Horus new absentee landlord system}{http://www.reshafim.org.il/ad/egypt/texts/heqanakht.htm}. Exact grain, gold and silver weights units defined the monetary system that paid Pharaoh taxes at the rate of 1/3 of 1/3 percent of net profits. Egyptian finite arithmetic scaled rational numbers n/p by LCM m to a mn/mp in a multiplication context before recording weights and measure units in concise unit fraction series. 

Three hieratic texts, the Moscow Mathematical Papyrus (MMP), the \PMlinkexternal{Kahun Papyrus}{http://planetmath.org/KahunPapyrusAndArithmeticProgressions.html}(KP) and the Rhind Mathematical Papyrus (RMP) encoded areas a circle (A), and volumes (V) of a circle, by naming height (H) of circle granaries by the same meta formulas: 

A = pi(R)(R) = (2567/81)(D/2)(D/2)  

A = (8/9)(D((8/9)(D) cubits (algebraic algebra formula 1.0)

replaced radius (R) with diameter (D/2) and pi by 256/81 (an easy to manipulate number) in

V  = (H)(8/9)(D)(8/9)(D) cubits (algebraic algebra formula 2.0)

V  = (3/2)(H)(8/9)(D)(8/9)(D) Khar (algebraic algebra formula 2.1)

V = (2/3)H(4/3)(D((4/3((D) Khar (algebraic algebra formula 2.2)

derived from scaling algebraic formula 2.1 by 3/2 considering

(3/2)V  =(3/2)(3/2)(H)(8/9)(D)(8/9)(D) = (H)(4/3)(D)(4/3)D) and multiplying both sides by 2/3

in a finite \PMlinkexternal{economic context}{http://planetmath.org/encyclopedia/EconomicContextOfEgyptianFractions.html}.

CLASSICAL GREEK, ARAB AND MEDIEVAL MATH:  Greek texts, like the Hibeh Papyrus, used closely related ciphered numeration and finite arithmetic systems that stressed an inverse proportion square root method estimated irrational numbers.  Arabs and medieval mathematicians modified the Egyptian and Greek numeration and finite arithmetic systems used the inverse proportion square root method. Arabs and Fibonacci (1202 AD) scaled n/p by LCM m in a subtraction 

(n/p - 1/m) = (mn -p)/mp,  

context that applied an algorithm often set numerator (mn -p) unity (1). 

Even when Fibonacci's "Liber Abaci" Europe's arithmetic book fro 250 years fell out of use in 1454 likely related to the Ottoman Empire conquering Constantinople, the inverse proportion square root method continued in use as late as Galileo.

MODERN TRANSLATIONS: Most 20th century transliterations and translations of Middle Kingdom arithmetic texts have been incomplete and therefore misleading in serious ways. Ahmes' actual Middle Kingdom arithmetic was finite. The arithmetic used LCM m, \PMlinkexternal{a number theory concept}{http://www.academia.edu/617613/Egyptian_Fractions_Unit_Fractions_Hekats_and_Wages_-_an_Update}. Egyptian scribes scaled rational number by LCM m within (n/p)(m/m) = mn/mp that recorded concise unit fraction series in a multiplication context. Egyptian scribes selected the best divisors of denominator mp (a GCD) that best summed to numerator mn by following the implicit algebraic context:

a. n/p = n/p(m/m) = mn/mq

example: 4/13 = 4/13(4/4) = 16/52 = (13 + 2 + 1)/52 = 1/4 + 1/26 + 1/52

with the divisors of mp often recorded in red that best summed to numerator mn created concise unit fraction series.

A second algebraic context was recorded in RMP 37. The subtraction context was emulated by Arabs and Fibonacci per:

b. (n/pq - 1/m)= (mn -pq)/mp

example: (4/13 - 1/4)= (16 - 13)/52 = (2 + 1)/52 = 1/26 + 1/52

also meant 4/13 = 1/4 + 1/26 + 1/52

MEDIEVAL UNIT FRACTION TEXTS: Arab and Fibonacci's finite notations applied LCM m in a subtraction context. Rational number n/p was encoded to unit fraction series by seven rules within

(n/p - 1/m)= (mn - p)/mp 

The seventh rule (distinction reported by L.E. Sigler in 2002 AD) demonstrated 

4/13 = 1/4 +  1/18 + 1/468 that considered

(4/13 - 1/4) = ([16 - 13)/52 - 1/18] = (54 - 52)/936 = 1/468

Arab and medieval statements employed the same LCM m that Ahmes employed 2850 years earlier. Ahmes employed two rational number conversion methods. The first method was recorded in RMP 36 by a multiplication use of LCM m recorded as (m/m).

ANALYSIS: Fragmented Greek, \PMlinkexternal{Arab}{http://planetmath.org/encyclopedia/ArabicNumerals.html}, and medieval unit fraction texts have been re-evaluated in the 21st century. Scholars have found several classes of ancient meta rules that replace fragmented hard-to-read scribal shorthand statements with readable scribal long hand statements.

Egyptian fraction eras are being studied within the 3,700 year reign of the Egyptian unit fractions. The first consideration creates transliterations of Egyptian, Greek, Arab and medieval Egyptian fraction texts. The most difficult documents to parse on the transliteration level are the scribal shorthand hieratic texts. Adding back scribal longhand versions of hieratic base 10 decimals was not a focus of 20th century scholars. The well intended 20th century scholars transliterated initial and intermediate Egyptian fraction arithmetic statements within scribal shorthand calculations as if the data was complete. Improved translations of the transliterations add back missing scribal shorthand steps. Emerging scribal longhand steps outline initial, intermediate, and final steps that are very close to the intellectual level understood by MK scribes

Most 4,000 year old discussions of Egyptian fraction math focus upon the 800 year old Liber Abaci at some point. The "Liber Abaci "was Europe's arithmetic book for 250 years. The book offered clear unit fraction methods that indirectly connected to the older Egyptian unit fraction methods. Fibonacci gathered base 10 numeral Arabic texts written from 800 AD to 1200 AD. Prior to 800 AD Greek and other ciphered systems were recorded in Arabic and other regional languages. There are a large number of 800 AD to 1200 AD Arabic numeral texts that amplify the Liber Abaci's math foundations. Scholars gather algebraic texts before and after 800 AD. Scholars also gather arithmetic texts from several eras that amplify the first 124 pages of Fibonacci's 500 page book. 

The first 124 pages of the Liber Abaci, counted by Sigler's 2002 translation, and footnotes, show that Fibonacci converted rational numbers to unit fraction series by seven subtraction rules (distinctions). Five of the medieval distinctions were written as(n/pq - 1/m) = mn/(mpq) connect to the Egyptian multiplication method. \PMlinkexternal{Ahmes'}{http://ahmespapyrus.blogspot.com/2009/01/ahmes-papyrus-new-and-old.html} understood(2/n - 1/m) = (2m -n)/(mn) statements as proofs. Four medieval distinctions look and act like Ahmes red auxiliary method, though translating into a subtraction context. Arabic linguists teaming with Classical scholars offer additional resources that further Greek and immediate Arabic sources from which the Liber Abaci was written into Hindu-Arabic numerals.

Considering the small number of Greek mathematical texts (written in Greek), Egyptian language introductions to Euclid, Archimedes' calculus, Plato's mathematics offer interesting clues. For example, on the numeration level Greeks ciphered numerals 1:1 onto Ionian and Doric alphabets as Egyptian scribes ciphered the counting numerals 1:1 onto hieratic symbols. The Greek mathematical texts show that language hints left by Plato, and others, are of value. 

Generally medieval units of measures fell into disuse when Europeans found new trade routes, after 1492 AD. Portugal, Spain, and other Europeans used regional weights and measures units after the birth of base 10 decimals in 1585 AD. It took several yeas for the the innovative base 10 decimal notation to birth the modern metric system. With two books, one for science and one for business, base 10 decimals were approved by the Paris Academy. Napier added logarithms, improving the base 10 decimal notation. Other improvement have been added since Napier. 

After 1585 AD, and the arrival of base 10 decimals, only a hint of Fibonacci's unit fraction notations were used in Europe. Galileo was the major exception by use of the very old square root method.  Ghobar, and regional Arabic scripts were replaced in the 17th century by modern Arabic script caused additional unit fraction words and math methods to die in the Arab world. Today, Arab and European scholars have few written language clues to decode medieval unit fraction arithmetic texts.  

ADDITIONAL EGYPTIAN FRACTION MATH CONSIDERATIONS: The \PMlinkexternal{Ahmes}{http://ahmespapyrus.blogspot.com/2009/01/ahmes-papyrus-new-and-old.html} includes updates of scribal initial, intermediate, answers, and proof statements. Updated scribal initial, intermediate and final answers followed by confirming duplation proofs that update scribal arithmetic operations. Middle Kingdom 2/n tables converted 2/n to optimized unit fraction series by scaling 2/n by LCMs written as m/m writing 2m/mn such that the additive aliquot parts of mn were selected to write out optimized, but not always optimal, unit fraction series. 

To review garbled transliterated texts \PMlinkexternal{Marshall Clagett, Ancient Egyptian Science, Vol III,1999}{http://books.google.com/books?id=8c10QYoGa4UC&pg=PA469&dq=Ancient+Egyptian+Science}, \PMlinkexternal{Joran Friberg, Unexpected links of Egyptian and Babylonian Mathematics, 2005}{http://books.google.com/books?id=1qQtWFHd8noC&printsec=frontcover&dq=clagett,+Egyptian&source=gbs_similarbooks_s&cad=1#v=onepage&q=&f=false} and Victor Katz, editor "The Mathematics of Egypt, Mesopotamia, China, India an d Islam", 2008. Clagett, Friberg and Katz published 20th century transliterations of the Rhind Mathematical Papyrus, the \PMlinkexternal{Kahun Papyrus}{http://planetmath.org/encyclopedia/KahunPapyrusAndArithmeticProgressions.html}, and other hieratic texts. Clagett, Friberg and Katz(editor for Imhausen,Robson) published recent books that omitted important discussions of the Akhmim Wooden Tablet and misreported important scribal arithmetic and algebraic geometry recorded in the RMP, MMP and Kahun Papyhrus.

The raw transliterated Egyptian fraction, algebra and algebraic geometry made available by Clagett, Friberg and Katz (i.e. Annette Imhausen) omitted vital discussions of once unified scribal methods and other considerations that are being made available by 21st century journal articles.

Twenty-first century journal articles are adding back missing 20th century initial and intermediate calculations to correct scribal shorthand omissions, reported in the 20th century as uncorrectable. 

Increasingly, 21st century journal articles, beginning with Hana Vymazalova and Tanja Pemmerening in 2002 have pointed out under valued Egyptian fraction decoding themes. Theoretical and practical decoding theme were reported in a 2006 journal article. The theoretical and practical aspects parsed the 1900 BCE Akhmim Wooden Tablet (AWT) and 1650 BCE Rhind Mathematical Papyrus (RMP). The AWT began and ended with five divisions of a hekat unity written as (64/64). The (64/64) theme corrected Georges Daressy's 1906 AWT transliteration errors by showing that the AWT scribe exactly divided a hekat unity (64/64) by 3, 7, 10, 11 and 13, writing out binary quotient (Q/64) and a scaled 1/320 of a hekat (5R/n)ro remainder. The RMP used the AWT method over 60 times, 10 times in RMP 47. 

The RMP stressed hard-to-read shorthand unit fraction calculations. At times scribal proofs used a well traveled method that was passed down from Egyptian to the Greek era, as mentioned by Plato, and medieval scribes, as mentioned by Fibonacci. Plato's mathematics, in "The Republic", mentioned shopkeepers applied theoretical unities within Greek era weights and measures units. To study Fibonacci's Egyptian fraction algebra, geometry, finite arithmetic, weights and measures, decoding doors need to be opened. Reading Fibonacci's 1202 AD Liber Abaci, including Sigler's footnotes, it is clear that medieval weights and measures closely followed Platonic, Greeks and 1,500 older Egyptian scribal methods, as did Fibonacci's seven rational number conversion methods, written as (n/pq - 1/m) = (mn -pq)/mpq statements, with m an LCM.

Considering the Egyptian mathematical texts, hieratic was the most common. The longest hieratic text is the Rhind Mathematical Papyrus. The first 1/3 of the text is taken up the 2/n table and 51 optimized unit fraction series that converted 2/3, 2/5, 2/7, ..., 2/101 into a table. The 2/n table red auxiliary numbers assisted in creating the table, a topic that has been controversial since the Hultsch-Bruins method was published in 1895. 

In 1900 and 1906 Egyptian multiplication. division and square root began to be decoded by 
\PMlinkexternal{Schack.Schackenberg}{http://planetmath.org/encyclopedia/BerlinPapyrusAndSecondDegreeEquations.html}, and \PMlinkexternal{Daressy}{http://planetmath.org/encyclopedia/AhmesBirdFeedingRateMethod.html}.  Peet, Chace, 1920s scholars and Gillings muddled the pre-1906 views by only reporting the additive aspects of the hieratic texts. Sadly the pre-1906 views of Hultsch's 2/n table, Schack-Schackenberg's RMP 69-78, Berlin Papyrus, and Daressy's RMP 81-83, Akhmim Wooden Tablet were downgraded and became controversial topics for 100 years. 

Gillings in 1970, "Mathematics in the Time of the Pharaohs", had only considered two additive rules, stated on page 110:

1. "Working only with the same methods, techniques, and notations available to the Egyptian scribe, we now attempt to reproduce some of these unit fraction tables 'ab initio'.

The Latin phrase 'ab initio' introduces a legalistic term 'at the beginning' to imply that scholarship started at the beginning. In retrospect insufficient self-awareness had been applied. The error of not finding Ahmes' beginning point, by pointing out an attested calculation, is parallel to the 1799-1825 barrier that hid the three meanings of Rosetta Stone's hieroglyphic writing for 25 years.

2. "We must of course eschew any modern refinements that could lead us to obvious simplifications. It may be a little irksome, but w have to try to think how the scribe would have thought, to imagine we are writing in hieratic, and to be logical only to the extent that we could expect the scribe to have been logical" 

Concerning rule 1, it was self-serving for anyone to suggest an 'ab initio' posture before attested classifications of the RMP problems, and the methods, were accepted by independent interdisciplinary teams. A  validated 'bi-lingual' context of Ahmes Egyptian fraction has been found. Modern rational number operations were taken from Fibonacci's 800 AD to 1600 AD number theory, methods that been taken Archimedes rational numbers, methods that had been taken from Ahmes rational numbers. Note the unbroken chain of custody, an appropriate legal term. 

Concerning rule 2, one aspect of the statement is true, that Ahmes' text must not introduce modern ' broken feelings'. Gillings attempted to offer a rigorous 'outside of the box' rule to avoid modern guesses. Attested decoding paths, found in the 21st century connect 3,600 years of continuous Egyptian fraction use by a third 'outside the box' rule. Gillings got lost in the modern and ancient forests, amidst the trees. Each ancient tree, each red auxiliary number, and so forth, must be identified and validated, as the third box names and validates.

A 'third outside of the box' rule assisted in the decoding of the Egyptian Mathematical Leather Roll (EMLR) vy returning missing scribal steps.. The EMLR used non-optimal LCMs as the 2/n table, RMP 7-20 and RMP 36 used optimized LCMs. RMP 36 identified the red number aspect of Ahmes' thinking. Gillings missed the subtle details of Ahmes thinking by grouping RMP 7-20, and the EMLR data as identities, rather than several types of LCMs. No attempt was made by Gillings to read RMP 7- 20 in the context of the EMLR's implied use of LCMs and RMP 36's actual use LCMs that exposed red auxiliary numbers added to numerators and optimized unit fraction series. The Kahun Papyrus and its arithmetic progression used a meta arithmetic progression formula that was common to RMP 40 and RMP 64, a fact ignored by many. The Moscow Mathematical Papyrus and arithmetic geometry used several unique formulas. All five texts have been re-parsed after 2005 finding like-problems in the RMP, and other texts, following new decoding paths, and classifications. Concerning the AWT over 40 quotient and exact remainder examples were written in the RMP, 29 times in RM9 81, and seven times in Ahmes' bird-feeding rate method (RMP 83), related facts that Gillings and others had garbled. Removing the garbled statements with the simplest reading of Egyptian texts requires Egyptian fractions to be written in vulgar fraction form - thereby revealing the four Egyptian arithmetic operations. 

The KP arithmetic progression was read in 1992 and slightly improved upon in 2005 by finding the simplest decoding path that connects to the largest number of ancient texts. For example, the RMP 40 and RMP 64 defines a KP decoding path preferred by John Legon. Legon's approach is superior to RMP 24-27, a path preferred by algorithmic researchers, since RMP 40, RMP 64 and other like data was excluded. Legon's approach is also superior to RMP 39, a path that excluded RMP 40, RMP 64, and Reisner Papyrus quotient and remainder data.

The simplest rule resolves certain controversies and thereby decodes the poorly reported texts. To read hieratic numbers, the numbers must be parsed from parent ciphered words. This step takes practice. For example, ro meant 1/320 of a hekat. The ro's relationship to the hekat had multiple uses. First, ro was a scaled remainder. Second 320 ro replaced one hekat for partitioning by large divisor purposes.

Ciphered numerals, such as Ahmes' 2/n table, and the RMP 40's arithmetic progression applications, several classes of ancient problems should be investigated to deeper levels. Excessive modern abstractions, however, reading sensed pyramid and other numerical text images, should be minimized. Mathematicians have fairly avoided this risk by excessively following Gillings 'ab initio' rules. A balanced decoding path is available by replacing the long unit fraction series with one vulgar fraction. Note that the ancient arithmetic operations are read by modern looking and acting statements.

Note that standard Egyptology 'dictionary' meanings that introduce classes of problems should be set aside, at the outset. Group like-problems. Follow the numbers. Allow grouped problems to assist the ancient numbers to speak for themselves, no more, or no less.

That is, prejudging numerical output without meta parsing efforts is short sighted. Disinterested third parties are needed to confirm assumptions, analyses, and conclusions. It should not be rare to find unique 'number' decoding paths in transliterated texts. Note that multiple decoding paths were needed to decode Ahmes' 2/n table members, and the RMP's 87 problems over 150 years. Several decoding paths come to mind: addition, subtraction, multiplication, division, optimized LCMs, quotients and remainder division, arithmetic progressions, inverse proportions. algebra, and geometry formulas. 

Ahmes' 2/n table and 87 problems were written in additive and non-additive operations. Singular additive or multiplication decoding paths, reported in the 1920s, seriously retarded Egyptian fraction research. By the 2000 non-additive subtraction and divisions methods have been parsed that Ahmes and other scribes used to scale 2/n tables, other encode other classes of mathematical methods. 

Today,  a dozen new decoding paths decode over long forgotten scribal mathematical methods.

Several decoding paths parse hieratic mathematical problems and methods. It has been well documented since 1999 that incomplete 1920s translations of the RMP need to be updated. Additive math classifications of Ahmes' mathematics overlooked a great deal. Updated classifications since 2000 have been attested by opening a dozen new decoding doors. The new decoding projects have corrected RMP additive texts by adding back missing scribal shorthand addition, subtraction, multiplication, division and inverse operations. 

It has important to independently confirm, and double check ,every added back missing data element. For example, Ahmes' 51  2/n table members, were scaled by LCM m/m in ways that would have surprised 1920's investigators. Double checking the decoding paths used to parse each problem have been obtained from other problems and often other hieratic texts. Working in interdisciplinary teams requires conflicting decoding paths to be resolved, one issue at a time, often by applying "Occam's Razor". 

Ahmes loved numerical formulas, though few were fully described. Hence decoding door need to fully parse each scribal formula. Ancient texts are revealing new classes of formulas.

A third class of formula, decoded in 2005, shows that the Akhmim Wooden Tablet (AWT) weights and measures units were used in over 40 RMP examples. The 1923 additive translation of the AWT by Peet (1923) only read the 1/320 hekat aspect, thereby missing its formula. The majority of Peet's oversights are easily corrected. For example, the formula:

(64/64)/n = Q/64 + (5R/n)*ro (AWT)

(6400/64) times 1/n = Q/64 + (5R/n)*ro (RMP 47)

recorded quotient (Q), and scaled (5/5) remainder (R) to 1/320 (ro) parts of the hekat. Ahmes used two-part quotient and remainder over 60 times. The division by n and multiplication by 1/n formula calculated two-part answers that Peet, Gillings and 20th century scholars had not identified. Adding back the scaled remainder arithmetic formula to Ahmes' tool kit began with Hana Vymazalova, a graduate student, Charles University, Prague, by publishing an AWT paper in 2001. In 2006 six AWT and 30 RMP data elements were published that reported a generalize multiplication formula was used 10 times in \PMlinkexternal{RMP 47}{http://mathforum.org/kb/message.jspa?messageID=7259234&tstart=0}.

Decoding new formulas, including unscaled and scaled arithmetic methods, 'dictionary' sides of the elements can be parsed by teams of mathematicians and linguists. Teams should discuss scopes and details contained in Middle Kingdom economic issues connected to Egyptian fraction discussions. When Middle Kingdom fragmented Egyptian fraction sentences are reconstructed and double checked by attested Middle Kingdom formulas decoding chapters can be closed. 

Concerning cubits, RMP 53-55 data reports a few interesting facts. These problems discuss cubit and khet units written in setats, 100 cubit by 100 cubit areas and setats divided into 1/100 setat strips, and mh units. Reading RMP 54 includes Ahmes' implicit use of the LCM 2/n table conversion method. Ahmes scaled a setat to (4/4) and (2/2) before multiplying by 7/10, 14/10 and 28/10

1. (7/10)*(4/4) setat = 28/40 setat = (25 + 3)/40 setat

as the 2/n table LCM conversion method would have written out

2. 5/8 setat + 300/40 mh = 5/8 setat + 7 1/2 mh

Ahmes' answer.


CONCLUSIONS:

(1) Old Kingdom (OK) infinite series math rounded-off representations of rational numbers to six-binary fractions. OK numerals were mapped many-to-one to number symbols in secular and spiritual situations. A cursive algorithm dominated eras in which the Eye of Horus was considered.

(2) Middle Kingdom (MK) finite series math added back rounded-off rational number representations to unity statements when ever possible. Finite math ciphered rational numbers one-to-one onto sound symbols. Hieroglyphic infinite series math continued in us for only spiritual situations.

(3) Greek infinite series problems were solved by finding finite series. Classical Greeks and Hellenes generally wrote finite series as scaled rational numbers [n/p by LCM m such that n/p(m/m)= mn/mp, with the best divisors of mp summed to ]recorded concise unit fraction series, included a three-step inverse proportion \PMlinkexternal{square root method}{http://planetmath.org/squarerootof3567and29}.  Greek numeration mapped number symbols one-to-one onto Ionian and Doric alphabets until 800 CE.

(4) Post-800 CE Arab scribes introduced Hindu numerals, and formalized an algorithm in Egyptian fraction arithmetic. Rational numbers were scaled by LCM m in a subtraction context such that (n/p - 1/m) = (mn -p)/mp set (mn -p) = 1 whenever possible. Following Pope Sylvester's 999 AD edict, Fibonacci in 1202 AD wrote up medieval arithmetic, algebra, geometry and weights and measures systems in the Liber Abaci, Latin writing Europe's primary math book for 250 years. 

(5) Post-1600 CE base 10 decimal arithmetic added an algorithm to arithmetic defined zero as an exponential place-holder. Over the last 500 years algorithms have become dominate ideas in many math fields. 

\begin{thebibliography}{10}
\bibitem{1}  A.B. Chace, Bull, L, Manning, H.P., Archibald, R.C., \emph{The Rhind Mathematical Papyrus}, Mathematical Association of Amnerica, Vol I, 1927. NCTM reprints available. 
\bibitem{2} Milo Gardner, \emph{An Ancient Egyptian Problem and its Innovative Solution, Ganita Bharati}, MD Publications Pvt Ltd, 2006.
\bibitem{3}Richard Gillings, \emph{Mathematics in the Time of the Pharaohs}, Dover Books, 1992.
\bibitem{4} Otto Neugebauer, \emph{Exact Sciences in Antiquity}
\bibitem{5} Oystein Ore, \emph{Number Theory and its History}, McGraw-Hill Books, 1948, Dover reprints available.
\bibitem{6} T.E. Peet, \emph{Arithmetic in the Middle Kingdom}, Journal Egyptian Archeology, 1923.
\bibitem{7} Tanja Pommerening, \emph{"Altagyptische Holmasse Metrologish neu Interpretiert" and relevant phramaceutical and medical knowledge, an abstract,  Phillips-Universtat, Marburg, 8-11-2004, taken from "Die Altagyptschen Hohlmass}, Buske-Verlag, 2005.
\bibitem{8} Gay Robins, and Charles Shute \emph{Rhind Mathematical Papyrus}, British Museum Press, Dover reprint, 1987.
\bibitem{9} L.E. Sigler, \emph{Fibonacci's Liber Abaci: Leonardo Pisano's Book of Calculation}, Springer, 2002.
\bibitem{10} Hana Vymazalova, \emph{The Wooden Tablets from Cairo:The Use of the Grain Unit HK3T in Ancient Egypt, Archiv Orientalai}, Charles U Prague, 2002.
\end{thebibliography}



POST-SCRIPT

1. (*) RMP 38 discussed 22/7 in a hekat context as a possible improved pi estimate. Details disclosed that Ahmes applied the modern rule, to divide, invert the divisor and multiple, per

a. 1 hekat = 320 ro 
b. 320 ro x 7/22 = 101 9/11
c. 101 9/11 x 22/7 = 320 ro = 1 hekat

2. Old Kingdom connections to MK Egyptian fractions are being translated into modern arithmetic statements in the 21st century. One of the topics shows that Old Kingdom infinite series system and the finite series Egyptian fraction system developed on separate tracks in the Old Kingdom. After 2,050 BCE Egyptian scribes corrected Old Kingdom cursive round-off  errors by writing finite unit fraction statements. The replacement Egyptian fraction system dominated the ancient Near East, Greece, Arab and medieval and its regional economic systems for 3700 years until base 10 decimals ended in modern Arabic script in the 17th century (a Moorish 1637 AD text being the last known example). 

Elements of modern arithmetic operations, and algebraic geometry, especially multiplication and division, were hidden hieratic scribal notes. Confirming elements of scribal 2/n table methods were discussed in RMP 36. In RMP 36 Middle Kingdom rational number conversion methods detailed four arithmetic operations built upon modern arithmetic operations. Scribes proved unit fraction answers by applying one Old Kingdom multiplication operation, returning answers to beginning numbers, often identities (or unities). Two RMP 38 proofs multiplied 320 by 7/22 (101 9/11) and 101 9/11 by multiplying by 22/7, obtaining 320. RMP 66 did the same thing by Ahmes dividing 10 hekat (3200 ro) by 365 (obtaining 8 + 280/365), and proved the unit fraction answers 8 + 2/3 + 1/10 + 1/2190) times 365 = 3200 ro showing that Middle Kingdom multiplication and division operations were inverse to one another (in the modern sense). In RMP 36 and 37 three discussions of red auxiliary numbers fully expose scribal alignments of a red number numerator to one unit fraction. RMP 41,42 and 43 vividly recorded algebraic geometry with radius (R) replaced by diameter (D/2) and pi buy 256/81, using four formulas. MMP 10 used sqrt(A) = (8/9)D cubits squared, the simplest formula, and the Kahun Papyrus used sqrt (V)= (2/3)H(4/3)D) khar, in RMP 43; multiplied 1500 khar by 1/20 into (75) 400-hekat in RMP 44; 400-hekat and
100-hekat multiplications by 1/n into quotient (Q/64) 4-hekat and 1-hekat quotients and remainder (5R/n) 4-ro and 1 ro remainder in RMP 47.

Concerning the 3,700 year life of Egyptian fraction arithmetic, stripped of Greek, Arabic and medieval cultural differences, a 2,800 year life of Egyptian fraction system was translated by Arabs into Hindu-Arabic numerals in 800 AD. The modified unit fraction system maintained ancient arithmetic definitions. The replacement Hindu-Arabic numeration system dominated Latin speaking and writing Europe after 999 AD (urged by Pope Sylvester) until 1454 AD with the fall of Byzantium. Egyptian fraction mathematics formally died in Europe when a new algorithmic decimal numeration system was published in 1585 AD (approved by the Paris Academy that accepted Simon Stevin's two books, one for business and one for science).

%%%%%
%%%%%
\end{document}
