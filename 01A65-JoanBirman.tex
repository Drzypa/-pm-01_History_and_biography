\documentclass[12pt]{article}
\usepackage{pmmeta}
\pmcanonicalname{JoanBirman}
\pmcreated{2013-03-22 16:41:47}
\pmmodified{2013-03-22 16:41:47}
\pmowner{Mravinci}{12996}
\pmmodifier{Mravinci}{12996}
\pmtitle{Joan Birman}
\pmrecord{6}{38909}
\pmprivacy{1}
\pmauthor{Mravinci}{12996}
\pmtype{Biography}
\pmcomment{trigger rebuild}
\pmclassification{msc}{01A65}
\pmclassification{msc}{01A61}
\pmclassification{msc}{01A60}
\pmsynonym{Joan Lyttle}{JoanBirman}
\pmsynonym{Joan Sylvia Birman}{JoanBirman}
\pmsynonym{Joan Sylvia Lyttle}{JoanBirman}

% this is the default PlanetMath preamble.  as your knowledge
% of TeX increases, you will probably want to edit this, but
% it should be fine as is for beginners.

% almost certainly you want these
\usepackage{amssymb}
\usepackage{amsmath}
\usepackage{amsfonts}

% used for TeXing text within eps files
%\usepackage{psfrag}
% need this for including graphics (\includegraphics)
%\usepackage{graphicx}
% for neatly defining theorems and propositions
%\usepackage{amsthm}
% making logically defined graphics
%%%\usepackage{xypic}

% there are many more packages, add them here as you need them

% define commands here

\begin{document}
\emph{Joan Sylvia Birman} n\'ee \emph{Lyttle} (1927 - ) American mathematician and educator. Considered one of the pioneers of braid theory, she has published extensively on the topic. Now a professor emeritus at Columbia University, she has won several awards since 1974, such as the Chauvenet Prize in 1996. Birman currently has a National Science Foundation grant to further research braids.

Birman has \PMlinkname{Erd\H{o}s number}{ErdHosNumber} 3: she co-authored ``Abelian and solvable subgroups of mapping class groups of surfaces'' in {\it Duke Math. J.} {\bf 50} No. 4 with Alex Lubotzky, who co-authored ``Discrete groups of slow subgroup growth'' in {\it Israel J. Math.} {\bf 96} with L\'aszl\'o Pyber, who co-authored ``Isomorphic subgraphs in a graph'' in {\it Colloquium Math. Soc. J\'anos Bolyai} {\bf 52} with Erd\H{o}s.
%%%%%
%%%%%
\end{document}
