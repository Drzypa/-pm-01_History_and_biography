\documentclass[12pt]{article}
\usepackage{pmmeta}
\pmcanonicalname{KarlWeierstrass}
\pmcreated{2013-03-22 17:06:40}
\pmmodified{2013-03-22 17:06:40}
\pmowner{Mravinci}{12996}
\pmmodifier{Mravinci}{12996}
\pmtitle{Karl Weierstra{\ss}}
\pmrecord{7}{39410}
\pmprivacy{1}
\pmauthor{Mravinci}{12996}
\pmtype{Biography}
\pmcomment{trigger rebuild}
\pmclassification{msc}{01A55}
\pmsynonym{Karl Theodor Wilhelm Weierstra{\ss}}{KarlWeierstrass}
\pmsynonym{Karl Theodor Wilhelm Weierstrass}{KarlWeierstrass}
\pmsynonym{Karl Weierstrass}{KarlWeierstrass}

\endmetadata

% this is the default PlanetMath preamble.  as your knowledge
% of TeX increases, you will probably want to edit this, but
% it should be fine as is for beginners.

% almost certainly you want these
\usepackage{amssymb}
\usepackage{amsmath}
\usepackage{amsfonts}

% used for TeXing text within eps files
%\usepackage{psfrag}
% need this for including graphics (\includegraphics)
%\usepackage{graphicx}
% for neatly defining theorems and propositions
%\usepackage{amsthm}
% making logically defined graphics
%%%\usepackage{xypic}

% there are many more packages, add them here as you need them

% define commands here

\begin{document}
\emph{Karl Theodor Wilhelm Weierstra{\ss}} (often \emph{Karl Theodor Wilhelm Weierstrass} in English texts) (1815 - 1897) German mathematician, the father of modern analysis.

Born in Ostenfelde, of a government official from Paderborn, young Karl showed an aptitude for mathematics but his father intended for him to follow in his footsteps to a career in public office. The son went to the University of Bonn to study law and economics but focused his attention on mathematics to the \PMlinkescapetext{point} of ignoring the requirements of the \PMlinkescapetext{degrees} his father intended him to obtain. Karl left Bonn for M\"unster and studied elliptic functions extensively. After being appointed chairman at an university in Berlin, Karl Weierstra{\ss} proved the fundamental calculus notion of the limit, regaining Bernard Bolzano's forgotten results in the famous \PMlinkname{Bolzano-Weierstra{\ss} theorem}{BolzanoWeierstrassTheorem}. Several other important calculus notions bear Weierstra{\ss}'s name.
%%%%%
%%%%%
\end{document}
