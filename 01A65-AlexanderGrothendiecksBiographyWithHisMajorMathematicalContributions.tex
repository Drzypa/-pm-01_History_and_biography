\documentclass[12pt]{article}
\usepackage{pmmeta}
\pmcanonicalname{AlexanderGrothendiecksBiographyWithHisMajorMathematicalContributions}
\pmcreated{2013-03-22 18:21:29}
\pmmodified{2013-03-22 18:21:29}
\pmowner{bci1}{20947}
\pmmodifier{bci1}{20947}
\pmtitle{Alexander Grothendieck's biography with his major mathematical contributions}
\pmrecord{104}{40996}
\pmprivacy{1}
\pmauthor{bci1}{20947}
\pmtype{Topic}
\pmcomment{trigger rebuild}
\pmclassification{msc}{01A65}
\pmclassification{msc}{01A61}
\pmclassification{msc}{01A60}
%\pmkeywords{extended biography of AlexanderGrothendieck}
%\pmkeywords{with several selected publications}
\pmrelated{AlexanderGrothendieck}
\pmrelated{AlexanderGrothendiecksAvailableSeminarsAndBooks}
\pmrelated{NicolasBourbaki}
\pmrelated{AlexsMathematicalGenealogy}
\pmrelated{AlexSMathematicalHeritageEsquisseDunProgramme}
\pmrelated{GrothendiecksTheorem}

\endmetadata

% this is the default PlanetMath preamble.  as your 
\usepackage{amssymb}
\usepackage{amsmath}
\usepackage{amsfonts}

% used for TeXing text within eps files
%\usepackage{psfrag}
% need this for including graphics (\includegraphics)
%\usepackage{graphicx}
% for neatly defining theorems and propositions
%\usepackage{amsthm}
% making logically defined graphics
%%%\usepackage{xypic}
% define commands here
\usepackage{amsmath, amssymb, amsfonts, amsthm, amscd, latexsym}
%%\usepackage{xypic}
\usepackage[mathscr]{eucal}

\setlength{\textwidth}{6.5in}
%\setlength{\textwidth}{16cm}
\setlength{\textheight}{9.0in}
%\setlength{\textheight}{24cm}

\hoffset=-.75in     %%ps format
%\hoffset=-1.0in     %%hp format
\voffset=-.4in

\theoremstyle{plain}
\newtheorem{lemma}{Lemma}[section]
\newtheorem{proposition}{Proposition}[section]
\newtheorem{theorem}{Theorem}[section]
\newtheorem{corollary}{Corollary}[section]

\theoremstyle{definition}
\newtheorem{definition}{Definition}[section]
\newtheorem{example}{Example}[section]
%\theoremstyle{remark}
\newtheorem{remark}{Remark}[section]
\newtheorem*{notation}{Notation}
\newtheorem*{claim}{Claim}

\renewcommand{\thefootnote}{\ensuremath{\fnsymbol{footnote%%@
}}}
\numberwithin{equation}{section}

\newcommand{\Ad}{{\rm Ad}}
\newcommand{\Aut}{{\rm Aut}}
\newcommand{\Cl}{{\rm Cl}}
\newcommand{\Co}{{\rm Co}}
\newcommand{\DES}{{\rm DES}}
\newcommand{\Diff}{{\rm Diff}}
\newcommand{\Dom}{{\rm Dom}}
\newcommand{\Hol}{{\rm Hol}}
\newcommand{\Mon}{{\rm Mon}}
\newcommand{\Hom}{{\rm Hom}}
\newcommand{\Ker}{{\rm Ker}}
\newcommand{\Ind}{{\rm Ind}}
\newcommand{\IM}{{\rm Im}}
\newcommand{\Is}{{\rm Is}}
\newcommand{\ID}{{\rm id}}
\newcommand{\GL}{{\rm GL}}
\newcommand{\Iso}{{\rm Iso}}
\newcommand{\Sem}{{\rm Sem}}
\newcommand{\St}{{\rm St}}
\newcommand{\Sym}{{\rm Sym}}
\newcommand{\SU}{{\rm SU}}
\newcommand{\Tor}{{\rm Tor}}
\newcommand{\U}{{\rm U}}

\newcommand{\A}{\mathcal A}
\newcommand{\Ce}{\mathcal C}
\newcommand{\D}{\mathcal D}
\newcommand{\E}{\mathcal E}
\newcommand{\F}{\mathcal F}
\newcommand{\G}{\mathcal G}
\newcommand{\Q}{\mathcal Q}
\newcommand{\R}{\mathcal R}
\newcommand{\cS}{\mathcal S}
\newcommand{\cU}{\mathcal U}
\newcommand{\W}{\mathcal W}

\newcommand{\bA}{\mathbb{A}}
\newcommand{\bB}{\mathbb{B}}
\newcommand{\bC}{\mathbb{C}}
\newcommand{\bD}{\mathbb{D}}
\newcommand{\bE}{\mathbb{E}}
\newcommand{\bF}{\mathbb{F}}
\newcommand{\bG}{\mathbb{G}}
\newcommand{\bK}{\mathbb{K}}
\newcommand{\bM}{\mathbb{M}}
\newcommand{\bN}{\mathbb{N}}
\newcommand{\bO}{\mathbb{O}}
\newcommand{\bP}{\mathbb{P}}
\newcommand{\bR}{\mathbb{R}}
\newcommand{\bV}{\mathbb{V}}
\newcommand{\bZ}{\mathbb{Z}}

\newcommand{\bfE}{\mathbf{E}}
\newcommand{\bfX}{\mathbf{X}}
\newcommand{\bfY}{\mathbf{Y}}
\newcommand{\bfZ}{\mathbf{Z}}

\renewcommand{\O}{\Omega}
\renewcommand{\o}{\omega}
\newcommand{\vp}{\varphi}
\newcommand{\vep}{\varepsilon}

\newcommand{\diag}{{\rm diag}}
\newcommand{\grp}{{\mathbb G}}
\newcommand{\dgrp}{{\mathbb D}}
\newcommand{\desp}{{\mathbb D^{\rm{es}}}}
\newcommand{\Geod}{{\rm Geod}}
\newcommand{\geod}{{\rm geod}}
\newcommand{\hgr}{{\mathbb H}}
\newcommand{\mgr}{{\mathbb M}}
\newcommand{\ob}{{\rm Ob}}
\newcommand{\obg}{{\rm Ob(\mathbb G)}}
\newcommand{\obgp}{{\rm Ob(\mathbb G')}}
\newcommand{\obh}{{\rm Ob(\mathbb H)}}
\newcommand{\Osmooth}{{\Omega^{\infty}(X,*)}}
\newcommand{\ghomotop}{{\rho_2^{\square}}}
\newcommand{\gcalp}{{\mathbb G(\mathcal P)}}

\newcommand{\rf}{{R_{\mathcal F}}}
\newcommand{\glob}{{\rm glob}}
\newcommand{\loc}{{\rm loc}}
\newcommand{\TOP}{{\rm TOP}}

\newcommand{\wti}{\widetilde}
\newcommand{\what}{\widehat}

\renewcommand{\a}{\alpha}
\newcommand{\be}{\beta}
\newcommand{\ga}{\gamma}
\newcommand{\Ga}{\Gamma}
\newcommand{\de}{\delta}
\newcommand{\del}{\partial}
\newcommand{\ka}{\kappa}
\newcommand{\si}{\sigma}
\newcommand{\ta}{\tau}
\newcommand{\med}{\medbreak}
\newcommand{\medn}{\medbreak \noindent}
\newcommand{\bign}{\bigbreak \noindent}
\newcommand{\lra}{{\longrightarrow}}
\newcommand{\ra}{{\rightarrow}}
\newcommand{\rat}{{\rightarrowtail}}
\newcommand{\oset}[1]{\overset {#1}{\ra}}
\newcommand{\osetl}[1]{\overset {#1}{\lra}}
\newcommand{\hr}{{\hookrightarrow}}

\begin{document}
\subsection{Alexander Grothendieck's Biography and His Major Mathematical Contributions}
{\bf Born:} March 28th, 1928 in Berlin, Germany  


\subsubsection{The Beginnings}
 A concise quote from an \PMlinkexternal{article by J J O'Connor and E F Robertson is:}{http://www-groups.dcs.st-andrews.ac.uk/~history/Biographies/Grothendieck.html} 

\subsubsection{Quotes}
  ``Alexander Grothendieck's father was Russian and he (Alex's father) was murdered by the Nazis.'' 

  ... (His mother, Hanka Grothendieck, was German); 
  ...``Grothendieck moved to France in 1941 and later entered Montpellier University. After graduating from Montpellier he spent the year 1948-49 at the $\'E$cole Normale Sup$\'e$rieure in Paris.''


{\em ...``people are accustomed to work with fundamental groups and generators and relations for these and stick to it, even in contexts when this is wholly inadequate, namely when you get a clear description by generators and relations only when working simultaneously with a whole bunch of base-points chosen with care - or equivalently working in the algebraic context of groupoids, rather than groups. Choosing paths for connecting the base points natural to the situation to one among them, and reducing the groupoid to a single  group, will then hopelessly destroy the structure and inner symmetries of the situation, and result in a mess of generators and relations no one dares to write down, because everyone feels they won't be of any use whatever, and just confuse the picture rather than clarify it. I have known such perplexity myself a long time ago, namely in Van Kampen type situations, whose only understandable formulation is in terms of (amalgamated sums of) groupoids."}  by Alexander Grothendieck.


\subsubsection{The Functional Analysis Phase}
\begin{itemize}
\item 1949 Alex Grothendieck worked on functional analysis with Jean Dieudonn$\'e$e at the 
University of Nancy in France; he was only for a short time one of the 
\PMlinkname{`Nicolas Bourbaki' group}{NicolasBourbaki} of mathematicians that included at various times: Andr$\'e$ Weil, Henri Cartan, Charles Ehresmann and Dieudonn$\'e$e. A quote from 
\PMlinkexternal{``Who Is Grothendieck ?'':}{http://www.ams.org/notices/200808/tx080800930p.pdf}
``{\em To begin with, (L) Schwartz gave Grothendieck a paper to read that he had just written with Dieudonn$\'e$e, which ended with a list of fourteen unsolved problems. After a few months, Grothendieck had solved all of them. Try to visualize the situation: on one side, Schwartz, who had just received a Fields Medal and was at the top
of his scientific career, and on the other side the unknown student from the provinces, who had
a rather inadequate and unorthodox education. Grothendieck was awarded a Ph.D. for his work
on topological vector spaces and stuck with that field for a while.''}

\item Alexander Grothendieck's doctoral thesis supervised by his advisor Laurent Schwartz, and co-advised 
by Jean Dieudonn$\'e$e was entitled ``Produits tensoriels topologiques et espaces nucl$\'e$aires'';
\end{itemize}

\subsubsection{Academic Appointments}
\begin{itemize}
\item 1953-1955 Visiting at the University of S$\~a$o Paulo, supported by the Centre National de la R$\'e$cherche Scientifique;

\item 1956 Returned to France at the Centre National de la R$\'e$cherche Scientifique;

\item 1960: Visiting at the University of Kansas in the USA working on topology and geometry, supported by the Centre National de la Recherche Scientifique beginning with 1956.
\item 1970-72 Visiting Professor at Coll$\'e$ge de France. 

\item  1972-73 Visiting Professor at Orsay. 

\item 1973 Professor at the University of Montpellier. 
\item 1984-88 On leave-- to direct research at the Centre National de la  Recherche Scientifique.

\end{itemize}

\subsubsection{His Golden Age at IHES}

1959-1970: Chair of the newly formed Institut des Hautes $\'E$tudes Scientifiques (IHES);
the IHES years have been referred to as his `Golden Age', when an entire new school of 
Abstract Mathematics flourished under Grothendieck's extremely creative leadership; thus,
Grothendieck's S$\'e$minaire de G$\'e$om$\'e$trie Alg$\`e$brique  \cite{Alexsem1, Alexsem2} established 
IHES as the \emph{World's Center of Algebraic Geometry} during 1960-1970, with Alex as its 
driving force. He travelled widely across Europe, including the Soviet-occupied Eastern Europe (such as 
the invited visit he made in the Summer of 1968 when he delivered a lecture at the 
School of Mathematics in Bucharest at the invitation of Acad. Prof. Dr. Miron Nicolescu of the Romanian Academy 
({\em supported from 1866 by Prince Charles von Hohenzollern-Sigmaringen--who became in 1881--King Carol I of Romania}), and across the World. Grothendieck is a very strong pacifist with very high ideals and goals, of real honesty and also extreme modesty; Alexander Grothendieck campaigned against the military built-up of the 1960s, which built-up almost ended up in total annihilation of our planet during the Cuban missile crisis. 

 Alexander Grothendieck's work during the `Golden Age' period established unifying themes in:
Algebraic Geometry, Number theory, Topology, Category Theory and Functional/Complex Analysis. 
He introduced his own `theory of schemes' in the 1960's which allowed certain of  A. Weil's number theory conjectures to be solved. He worked on the theory of topoi/toposes that are relevant not only to mathematical logic and category theory, but also to computer software/programming and institutional ontology classification and bioinformatics. 
He provided an algebraic proof of the Riemann-Roch theorem, algebraic definition of the 
fundamental group of a curve, the definition of the fundamental functor for a categorical 
Galois theory, the re-definition of Abelian categories,(as for example in the case 
of $\A b5$ categories that carry his name-the Grothendieck and local Grothendieck categories),
he outlined the {\em `Dessins d' Enfants'} combinatorial topology theory and much, much more. His 
``S$\'em$inaires de G$\'e$ometrie alg$\`e$briques" alone are several thousands of pages in (typewritten) printed length, or close to 500 Mb in electronic format. Later in the '80's in his \emph{`Esquisse d'un Programme'} he outlined
the `anabelian' homology theory, what is called today in different fields by different names:
Non-Abelian Homology Theory (that has not yet been achieved as he planned to do), non-Abelian
Algebraic Topology, Noncommutative geometry, Non-Abelian Quantum Field theories, or ultimately,
non-Abelian Categorical Ontology, fields that are still in need of future developments.



\subsubsection{Honors and Awards}

\begin{itemize}

\item Speaker at the International Congress of Mathematicians in 1958; 

\item Alexander Grothendieck received the Fields Medal in 1966, which he accepted; 

\item Alexander Grothendieck was awarded, but declined, the Crafoord Prize in 1988;
the prize was instead accepted by one of his 
\PMlinkexternal{French former students}{http://www.genealogy.math.ndsu.nodak.edu/id.php?id=61289};

\item Emeritus Professor in 1988 on his 60th birthday. 

\end{itemize}


\subsubsection{Author's direct, First-hand impressions of Alexander Grothendieck:}

 \emph{One was struck immediately upon meeting him by his generosity and the energy with which 
Alex shared his ideas with colleagues and students, as well as the excitement that 
he incited through his brilliantly clear lecturing style, thus inspiring others to share
in his excitement for all of Mathematics, not just some highly specialized subject,
as if they were `to set out to explore a completely new land, or white territory'}.

\subsubsection{A Brief Summary of  Alexander Grothendieck's Best-Known Contributions to Mathematics:}
\begin{itemize}
\item Topological tensor products and nuclear spaces, 
\item Sheaf cohomology as derived functors, schemes, K-theory and Grothendieck-Riemann-Roch,
\item \'Etale Cohomology and the Cohomological interpretation of L-functions,
\item Crystalline cohomology,
\item Defining and constructing geometric objects {\em via} Representable Functors, 
\item Descent, fibred categories and stacks, 
\item Grothendieck topologies (sites) and topoi, 
\item Derived categories, 
\item Formalisms for local and global duality (the 'six operations'), 
\item Motives and the 'yoga of weights', 
\item Tensor Categories and Motivic Galois Groups.
\item Proofs of two generalized Riemann-Roch-Grothendieck theorems conjectured by Andr\'e Weil.
\end{itemize}

{\em Note:}
 Alexander Grothendieck's \PMlinkexternal{mathematical `genealogy'}{http://www.genealogy.math.ndsu.nodak.edu/id.php?id=31245&fChrono=1} is claimed to go back through many successive doctoral advisor generations from Laurent Schwartz
to Borel, Darboux,..., Simeon Poisson, Joseph Lagrange, Leonhard Euler, Bernoulli, Gottfried Leibniz (in 1666, with a 53,763-long sequence of `descendants'), Weigel and Christiaan Huygens, and the record finally stops at Ludolph van Ceulen at the Universiteit Leiden in 1607 AD!


 {\em A most valuable resource in Algebraic Geometry, ``Ho- and Coho- mology''}: 
 \PMlinkexternal{Grothendieck-Serre Correspondence--Bilingual Edn.}{http://books.google.com/books?hl=en&id=FBfygannPSUC&dq=Alexandre+Grothendieck&printsec=frontcover&source=web&ots=Rwmt1x2weX&sig=y330F6qaDddY6_zyIJJA0SU602U&sa=X&oi=book_result&resnum=9&ct=result}

\begin{thebibliography}{99}

\bibitem{WSA2}
Winfried Scharlau: \PMlinkexternal{``Who Is Alexander Grothendieck ?''}{http://www.ams.org/notices/200808/tx080800930p.pdf}

\bibitem{Alex1}
Alexander Grothendieck. 1971, Revetements Etales et Groupe Fondamental (SGA1),
chapter VI: Categories fibrees et descente, \emph{Lecture Notes in Math.}
\textbf{224}, Springer--Verlag: Berlin.

\bibitem{Alex2}
Alexander Grothendieck. 1957, Sur quelque point d-algebre homologique. , \emph{Tohoku Math. J.}, \textbf{9:} 119-121.

\bibitem{Alex3}
Alexander Grothendieck and J. Dieudon\'{e}.: 1960, Elements de geometrie algebrique., \emph{Publ. Inst. des Hautes Etudes de Science}, \textbf{4}.

\bibitem{ALEXsem1}
Alexander Grothendieck et al.,1971. Seminaire de Geometrie Algebrique du Bois-Marie, Vol. 1--7, Berlin: Springer-Verlag.

\bibitem{Alexsem2}
Alexander Grothendieck. 1962. Seminaire de Geometrie Algebrique du Bois-Marie, Vol. 2 - Cohomologie Locale des Faisceaux Coherents et Theoremes de Lefschetz Locaux et Globaux. , pp.287. (with an additional contributed expose by Mme. Michele Raynaud). 
\PMlinkexternal{Typewritten manuscript available in French}{http://modular.fas.harvard.edu/sga/sga/2/index.html};
\PMlinkexternal{see also a brief summary in English}{http://planetmath.org/?op=getobj&from=books&id=78}
References Cited: 
\begin{enumerate}
\item J. P. Serre. 1964. {\em Cohomologie Galoisienne}, Springer-Verlag: Berlin.
\item J. L. Verdier. 1965. {\em Algebre homologiques et Categories derivees}. North Holland Publ. Cie.
\end{enumerate}

\bibitem{ALEX57}
Alexander Grothendieck. 1957, Sur Quelques Points d'algebre homologique, {\em Tohoku Mathematics Journal}, 9, 119--221.
 
\bibitem{Alexsem1}
Alexander Grothendieck et al. \emph{Seminaires en Geometrie Algebrique- 4}, Tome 1, Expose'e 1 
(or the Appendix to Expose'e 1, by `N. Bourbaki' for more detail and a large number of results.
AG4 is \PMlinkexternal{freely available}{http://modular.fas.harvard.edu/sga/sga/pdf/index.html} in French;
also available here is an extensive 
\PMlinkexternal{Abstract in English}{http://planetmath.org/?op=getobj&from=books&id=158}.

\bibitem{Alex84}
Alexander Grothendieck, 1984. ``Esquisse d' un Programme'', (1984 manuscript), 
{\em finally published in ``Geometric Galois Actions''}, L. Schneps, P. Lochak, eds., 
{\em London Math. Soc. Lecture Notes} {\bf 242}, Cambridge University Press, 1997, pp.5-48;
English transl., ibid., pp. 243-283. MR 99c:14034 .

\bibitem{Alex81}
Alexander Grothendieck, ``La longue marche in a travers la theorie de Galois'' 
\emph{= ``The Long March Towards/Across the Theory of Galois''}, 1981 manuscript, University of Montpellier preprint series 1996, edited by J. Malgoire. 

\bibitem{LS94}
Leila Schneps. 1994. 
\PMlinkexternal{The Grothendieck Theory of Dessins d'Enfants}{http://planetmath.org/?op=getobj&from=books&id=163}.
(London Mathematical Society Lecture Note Series), Cambridge University Press, 376 pp.

\bibitem{DHSL2k}
David Harbater and Leila Schneps. 2000.
\PMlinkexternal{Fundamental groups of moduli and the Grothendieck-Teichmuller group}{http://www.ams.org/tran/2000-352-07/S0002-9947-00-02347-3/home.html}, \emph{Trans. Amer. Math. Soc}. 352 (2000), 3117-3148. 
MSC: Primary 11R32, 14E20, 14H10; Secondary 20F29, 20F34, 32G15.

\end{thebibliography}

%%%%%
%%%%%
\end{document}
