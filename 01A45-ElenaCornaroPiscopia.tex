\documentclass[12pt]{article}
\usepackage{pmmeta}
\pmcanonicalname{ElenaCornaroPiscopia}
\pmcreated{2013-03-22 17:26:04}
\pmmodified{2013-03-22 17:26:04}
\pmowner{PrimeFan}{13766}
\pmmodifier{PrimeFan}{13766}
\pmtitle{Elena Cornaro Piscopia}
\pmrecord{4}{39811}
\pmprivacy{1}
\pmauthor{PrimeFan}{13766}
\pmtype{Definition}
\pmcomment{trigger rebuild}
\pmclassification{msc}{01A45}
\pmsynonym{Elena Lucrezia Piscopia Cornaro}{ElenaCornaroPiscopia}
\pmsynonym{Elena Lucrezia Cornaro Piscopia}{ElenaCornaroPiscopia}
\pmsynonym{Elena Cornaro}{ElenaCornaroPiscopia}

\endmetadata

% this is the default PlanetMath preamble.  as your knowledge
% of TeX increases, you will probably want to edit this, but
% it should be fine as is for beginners.

% almost certainly you want these
\usepackage{amssymb}
\usepackage{amsmath}
\usepackage{amsfonts}

% used for TeXing text within eps files
%\usepackage{psfrag}
% need this for including graphics (\includegraphics)
%\usepackage{graphicx}
% for neatly defining theorems and propositions
%\usepackage{amsthm}
% making logically defined graphics
%%%\usepackage{xypic}

% there are many more packages, add them here as you need them

% define commands here

\begin{document}
\PMlinkescapeword{languages}
\PMlinkescapeword{degree}

\emph{Elena Lucrezia Piscopia Cornaro} (1646 - 1684) Italian theologist and mathematician, the first woman to earn a doctorate at the University of Padua (the second did so in the 20th Century).

Born of a church lawyer and a housewife, young Elena, unlike other girls of her time, was taught Greek and Latin, and other languages. She also studied mathematics and theology.
After becoming a deaconess at a Benedictine convent, she went on to study at the University of Padua where she earned a philosophy doctorate, becoming the first woman to earn such a degree. She began teaching mathematics at the university until her death, however, she is more remembered for her charity work.

\begin{thebibliography}{1}
\bibitem{church} ``Elena Lucrezia Piscopia Cornaro'' in {\it Catholic Encyclopedia} New York: Encyclopedia Press (1913)
\end{thebibliography}
%%%%%
%%%%%
\end{document}
