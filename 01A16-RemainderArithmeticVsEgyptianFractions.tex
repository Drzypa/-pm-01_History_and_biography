\documentclass[12pt]{article}
\usepackage{pmmeta}
\pmcanonicalname{RemainderArithmeticVsEgyptianFractions}
\pmcreated{2013-03-22 15:50:28}
\pmmodified{2013-03-22 15:50:28}
\pmowner{milogardner}{13112}
\pmmodifier{milogardner}{13112}
\pmtitle{remainder arithmetic vs Egyptian fractions}
\pmrecord{129}{37819}
\pmprivacy{1}
\pmauthor{milogardner}{13112}
\pmtype{Definition}
\pmcomment{trigger rebuild}
\pmclassification{msc}{01A16}
\pmsynonym{remainder arithmetic}{RemainderArithmeticVsEgyptianFractions}
\pmsynonym{Egyptian fractions}{RemainderArithmeticVsEgyptianFractions}
\pmsynonym{Egyptian math}{RemainderArithmeticVsEgyptianFractions}
%\pmkeywords{Rhind Mathematical Papyrus}
%\pmkeywords{Kahun Papyrus}
\pmdefines{Egyptian mathematics}
\pmdefines{arithmetic progressions}

% this is the default PlanetMath preamble.  as your knowledge
% of TeX increases, you will probably want to edit this, but
% it should be fine as is for beginners.

% almost certainly you want these
\usepackage{amssymb}
\usepackage{amsmath}
\usepackage{amsfonts}

% used for TeXing text within eps files
%\usepackage{psfrag}
% need this for including graphics (\includegraphics)
%\usepackage{graphicx}
% for neatly defining theorems and propositions
%\usepackage{amsthm}
% making logically defined graphics
%%%\usepackage{xypic}

% there are many more packages, add them here as you need them

% define commands here

\begin{document}
INTRODUCTION: Scholarly Egyptian fraction debates date to 1862. These debates decode aspects of the 1800 BCE  \PMlinkexternal{Berlin Papyrus}{http://planetmath.org/?op=getobj&from=objects&id=12108} math and math from other texts. The Berlin Papyrus hinted at a scribal square root method  solved two second degree equations, and  a pesu (inverse proportion) method (decoded in 1900). 

Closely related textual debates include the 1650 BCE Rhind Mathematical Papyrus(RMP)(published by 1879 in Germany), the 1900 BCE Akhmim Wooden Tablet (published in 1906)  and the 1900 BCE Egyptian Mathematical Leather Roll (EMLR) (published by 1927 England) offer subtle Egyptian fraction issues that slowly give up ancient scribal math secrets.  For example, a  \PMlinkexternal{generalized scribal square root method}{http://planetmath.org/archimdesandahmessquarerootof3567and29} was published in Dec. 2012.

The 1879 Egyptian fraction phase of the debate began after a bootleg copy of the RMP was taken from the British Museum and published in Germany. British, German, European, USA and Arab scholars debate the RMP math themes begin with a hard-to-read RMP 2/n table. The 1927 arithmetic debate placed in concrete an additive conclusion with the publication of Peet and Chace's additive view of the RMP 2/n table offered a misleading point of view. 

In 1900 the inverse proportion pesu was decoded as a method that solved two second degree equation. Later analysis of the pesu revealed additional remainder arithmetic properties.

In 1901 and 1906 the Akhmim Wooden Tablet reported a volume unit (a hekat) scaled by 1/3, 1/7, 1/10, 1/11 and 1/13. Georges Daressy found the exact aspect of the 1/3, 1/7 and 1/10 cases. It took until 2001 for Hana Vymazalova to report the exact aspect of the 1/11 and 1/13 cases by pointing out an initial hekat unity written as (64/64).   

In 1927 scholars suggested the EMLR text would shed light on the RMP and other scribal Egyptian fraction methods. The 26 line 1800 BCE EMLR was unrolled and additively read by British Museum scholars. An anticipated deeper understanding of Egyptian arithmetic was not reported. Several early 1930's German scholars suspected that the \PMlinkexternal{EMLR}{http://en.wikipedia.org/wiki/Egyptian_Mathematical_Leather_Roll}, at some point, would provide deeper insights. 

In 1933 the events that led up to World War II virtually stopped Egyptian fraction research. Research started up again in 1945. Research slowly progressed over the next 50 years. Gillings published an excellent summary of the available Egyptian fraction texts in 1972 and indirectly mentioned the Akhmim Wooden Tablet. Gillings accepted the majority of the 1920's additive views of Peet and Chace while adding several minor suggestions as aids to read the most popular texts.

By 2004, scholars began to discuss abstract Egyptian math themes by identifying scaled aspects of 2/n table unit fraction series used in the RMP's 87 problems and the 26 line EMLR. Additional Egyptian fraction texts have been brought into the larger debate also report abstract themes. 

In 2004 three publications jump-started abstract aspects of Egyptian fraction arithmetic. The first was an algebraic version of 22 EMLR conversions of rational numbers created by six multiples. The multiples may have been non-additive multiples. They hinted at a deeper arithmetic likely used by Ahmes. The EMLR multiple method was soon connected to the second publication, the 1202 AD Liber Abaci's and seven rational number conversion methods published by Sigler. The Liber Abaci had been read for years in fragmentary ways. Finally the full text was translated from Latin to English. The third publication was the 1900 BCE Akhmim Wooden Tablet. This text hinted at an abstract form of Egyptian remainder arithmetic. The paper was published by Vymazalova, a Charles U. graduate student.

In 2007 a lease common multiple method connected the first 2002 publication to the 2004 publication. The LCM method allowed EMLR and RMP Egyptian fraction data to be computed by an identical method. By considering the entire scope of Egyptian fraction literature that six non-optimal EMLR multiples had been adapted by Ahmes, the RMP scribe, into a single optimal multiple method. Ahmes seemed to easily convert 51 2/n table rational numbers to optimized Egyptian fraction series by selecting an optimal multiple. Research continues to parse Ahmes' math specifics. All that is known for sure is that Ahmes used 'red auxiliary' numbers, an LCM method. Ahmes' selection of an optimal multiple may have also considered Akhmim Wooden Tablet and RMP remainder arithmetic.

In 2008 the RMP 2/n table was read in terms of RMP 36. Additional scribal 2/n table construction methods were published on-line in \PMlinkexternal{2008}{http://rmprectotable.blogspot.com/} and \PMlinkexternal{2010}{http://rmp36.blogspot.com/}. 

BACKGROUND: Using Webster's new collegiate dictionary, an Egyptian is defined by: 1. a native or inhabitant of Egypt; 2. the Afro-Asiatic  language of the ancient Egyptian from the earliest time to the 3rd century A.D. 

By adding the word fraction to the word Egyptian, creates the phrase: Egyptian fraction. This narrative will show that Egyptians living before the 3 century A.D. wrote Egyptian fractions in ways that took over 115 years of debate to 'break the ancient scribal code' of Egyptian fraction arithmetic. The 115 year narrative's definition of an Egyptian fraction disallows the interjection of post-300 AD non-Egyptian fraction ideas and methods, such as represented by the modern greedy algorithm, and other modern decoding attempts that had hidden the ancient scribal methods from full view. 

For example, by removing the modern idea of algorithm, and its 800 AD birth, as a \PMlinkexternal{RMP}{http://en.wikipedia.org/wiki/Rhind_Mathematical_Papyrus} decoding possibility, the 2/n table and the EMLR methods began to be fairly decoded in other ways. That is, the possible greedy algorithm's use in the Liber Abaci (as noted by Sylvester in 1891 in the last of its Egyptian fraction methods) only included the use of a second subtraction step, and not an n-step algorithm. In other words, by placing algorithms, and other none scribal arithmetic suggestions (like false position) aside, the central outline of 2,000 BCE scribal arithmetic come into view. 

Three of the four scribal arithmetic operations look much like our own modern arithmetic operations. The older duplation multiplication operation was unique to Egyptian mathematics.  Aspects of remainder arithmetic may have been disliked by Greeks, and Arabs. By the time of the Liber Abaci (1202 AD), Greek and Arab lattice multiplication came into dominance, thereby replacing the multiplication method Yet, the older Egyptian arithmetic's use of addition, subtraction and division operations looked nearly the same in 1650 BCE, as they did in 1202 AD, withing three arithmetic operations.

Generally ancient scribes wrote rational numbers as exact unit fraction series in optimal ways. The scribal methods for converting rational numbers has been a murky subject, in several respects. Hence few modern scholars have ventured into the deeper aspects of all four of the ancient scribal arithmetic topics, as they relate to the 4,000 year time period, since their first appearance. In other words, this summary is intended to high-light a few of the murky aspects of this longer arithmetic subject reporting for the first time a unified definition of the older Egyptian fractions.  

Returning to a broader Egyptian fraction decoding topic, the first chapter stresses that the Egyptian  Mathematical Leather roll and its conversion of 1/p and 1/pq unit fractions to Egyptian fractions was an ancient teaching tool for anyone wishing to become a scribe. The EMLR student raised its simple set of unit fractions to multiples of 2, 3, 4, 5, 7, and 25, as needed, and then parsed his/her denominators by multiples of the denominator. For example 1/3 was raised 2/2 = 2/6, allowed 1/3 + 1/3, a non-Egyptian fraction looking definition to be stated. Next 1/4, one of the binary numbers was raised to 4/4= 3/12, allowing (2 + 1)/12 to write 1/6 + 1/12. In total, the EMLR converted 26 lines of  Egyptian fractions, several repeated 1/p or 1/pq unit fractions, converted by a different multiple of 2, 3, 4, 5, 7 and 25, and finding not-so-elegant Egyptian fraction series.

The second decoding chapter begins with elegant two-term Egyptian fraction series for 2/pq vulgar fractions. Ahmes is reported as using an optimal \PMlinkexternal{RMP 2/n table}{http://en.wikipedia.org/wiki/RMP_2/n_table} method.

CONCLUSION: Egyptian and medieval unit fractions texts report general conversions of rational numbers to optimal and not-so-elegant unit fraction series. Ahmes, for example, wrote exact unit fraction series by an abstract scaling method. Seven rational number conversions methods, in three notations, were summarized in the 1202 AD Liber Abaci. Four Liber Abaci methods hint at 2,800 year older 2/n tables. Scribal methods in 1202 AD scaled rational numbers by a subtraction method. The 1650 BCE and older method scaled rational numbers by a multiplication method. 

Scribal methods used in 1650 BCE partitioned a volume unit named the hekat to hin, dja, ro and other sub-units. Scribal weights and measures system scaled quotients and remainders in interesting ways, one was square root. Pharaohs were interested in controlling beer, bread, grain and other vital national inventories by scaling every rational number remainder, if possible. Egyptian fraction controls recorded in the Akhmim Wooden Tablet used a remainder arithmetic system that double checked rational number results. Hekat units included the hin, oipe, dja and ro. Hekat units partitioned vulgar fractions in weights and measures applications. Several classes of remainder arithmetic units were cited in Egyptian fraction texts. One class was recorded in the Reisner Papyri. Several classes were recorded in the RMP. The Reisner and RMP reported a common production rate system that scaled to units of 10 to measure production outputs of workers days, likely in 10 hour days.

Scholars often consider three math windows to parse Egyptian fraction texts. Modern and scribal algebra offers one Egyptian fraction window. Scribal 2/n tables and optimized (but not optimal) least common multiple methods offer a second window. Remainder arithmetic applications used within weights and measures offers a third window. Considering the three math windows, scholars have debate Egyptian fraction methods for about 150 years. Happily, the Egyptian fraction debate is winding down. Agreements are being reached concerning theoretical and abstract aspects of the oldest Egyptian fraction arithmetic. A central element of the abstract arithmetic was  remainder arithmetic. 

\begin{thebibliography}{8}

\bibitem{1} Georges Daressy, \emph{"Calculs Egyptiens du Moyan Empire¢?, Recueil de Travaux Relatifs  De La  Phioogie et al Archaelogie Egyptiennes Et Assyriennes XXVIII, 1906, 62¢72}, Paris, 1906.
\bibitem{2} Milo Gardner, \emph{The Egyptian Mathematical Leather Roll Attested Short Term and Long Term, History of Mathematical Sciences}, Hindustan Book Company, 2004.
\bibitem{3} Milo Gardner, \emph{An Ancient Egyptian Problem and its Innovative Solution, Ganita Bharati}, MD Publications Pvt Ltd, 2006.
\bibitem{4}Richard Gillings, \emph{Mathematics in the Time of the Pharaohs}, Dover Books, 1992.
\bibitem{5} T.E. Peet, \emph{Arithmetic in the Middle Kingdom}, Journal Egyptian Archeology, 1923.
\bibitem{6} Tanja Pommerening, \emph{"Altagyptische Holmasse Metrologish neu Interpretiert" and relevant phramaceutical and medical knowledge, an abstract,  Phillips-Universtat, Marburg, 8-11-2004, taken from "Die Altagyptschen Hohlmass}, Buske-Verlag, 2005.
\bibitem{7} L.E. Sigler, \emph{Fibonacci's Liber Abaci: Leonardo Pisano's Book of Calculation}, Springer, 2002.
\bibitem{8} Hana Vymazalova, \emph{The Wooden Tablets from Cairo:The Use of the Grain Unit HK3T in Ancient Egypt, Archiv Orientalai}, Charles U Prague, 2002.
\end{thebibliography}

%%%%%
%%%%%
\end{document}
