\documentclass[12pt]{article}
\usepackage{pmmeta}
\pmcanonicalname{NUMB3RS}
\pmcreated{2013-03-22 16:38:05}
\pmmodified{2013-03-22 16:38:05}
\pmowner{PrimeFan}{13766}
\pmmodifier{PrimeFan}{13766}
\pmtitle{NUMB3RS}
\pmrecord{9}{38836}
\pmprivacy{1}
\pmauthor{PrimeFan}{13766}
\pmtype{Feature}
\pmcomment{trigger rebuild}
\pmclassification{msc}{01A65}
\pmclassification{msc}{01A61}
\pmclassification{msc}{00A06}

% this is the default PlanetMath preamble.  as your knowledge
% of TeX increases, you will probably want to edit this, but
% it should be fine as is for beginners.

% almost certainly you want these
\usepackage{amssymb}
\usepackage{amsmath}
\usepackage{amsfonts}

% used for TeXing text within eps files
%\usepackage{psfrag}
% need this for including graphics (\includegraphics)
%\usepackage{graphicx}
% for neatly defining theorems and propositions
%\usepackage{amsthm}
% making logically defined graphics
%%%\usepackage{xypic}

% there are many more packages, add them here as you need them

% define commands here

\begin{document}
{\em NUMB3RS} (sometimes listed as {\em Numbers}) is a television police drama airing weekly on CBS since January 2005. Created by Nick Falacci and Cheryl Heuton, the show stars Rob Morrow and David Krumholtz.

The protagonist is Don Eppes, an FBI agent who solves cases with the help of his brother Charlie, a mathematical physicist. When Charlie describes a given case, formulas are often displayed on the screen over scenes depicting the criminals or their deeds. Some episodes of the show have been used by math educators in the classroom.

The writers of the show are helped by consultants from Hollywood Math and Science Film Consulting, which claims to have suggested to the producers that they ``work with the National Council for Teachers of Mathematics to create homework assignments related to the mathematical and scientific topics discussed in each episode.''

Alice Silverberg, a cryptanalist and unpaid mathematics consultant for the show, says that ``getting the math right and getting it to fit with the plot are not priorities of the NUMB3RS team'' and says that Cheryl Heuton, one of the creators of the show, ``points out that few viewers will know the difference.''

Silverberg gives as an example her CEILIDH cryptosystem mentioned in one episode, which she says had nothing to do with the plot of that week's episode. She describes the consulting process as consisting of receiving the draft script from the writers, replacing ``jargon that makes us cringe a lot with jargon that makes us cringe a little,'' and then sending it back to the writers. Another example Silverberg gives is an episode in which Charlie is trying to decipher a coded message. The original draft mentioned full frequency analysis, Vignere deconstruction and ``{\em a} Lucas sequence,'' so Silverberg suggested that be changed to saying the message is not long enough to be a Vigen\`ere cypher and that if it was then ``we could try a Kasiski test or an index-of-coincidence analysis,'' making no mention of the Lucas sequence which is unlikely to have cryptographic applications.

In 2007, the National Science Foundation honored the show's creators with a Public Service Award for ``their contributions toward increasing scientific and mathematical literacy on a broad scale.''

\begin{thebibliography}{1}
\bibitem{AMS} \PMlinkexternal{AMS News 2007}{http://www.ams.org/dynamic_archive/home-news.html#numbers-nsb}
\bibitem{HMSFC} \PMlinkexternal{Hollywood Math and Science Film Consulting home page}{http://www.hollywoodmath.com/projects.htm}
\bibitem{NUMB3RSIMDB} \PMlinkexternal{Home page}{http://www.imdb.com/title/tt0433309/} for the show on IMDB.
\bibitem{as} A. Silverberg, ``Alice in NUMB3Rland'' {\it FOCUS} {\bf 26} 8 (2006): 12 - 13
\end{thebibliography}
%%%%%
%%%%%
\end{document}
