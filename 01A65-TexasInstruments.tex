\documentclass[12pt]{article}
\usepackage{pmmeta}
\pmcanonicalname{TexasInstruments}
\pmcreated{2013-03-22 16:47:10}
\pmmodified{2013-03-22 16:47:10}
\pmowner{PrimeFan}{13766}
\pmmodifier{PrimeFan}{13766}
\pmtitle{Texas Instruments}
\pmrecord{10}{39018}
\pmprivacy{1}
\pmauthor{PrimeFan}{13766}
\pmtype{Definition}
\pmcomment{trigger rebuild}
\pmclassification{msc}{01A65}
\pmclassification{msc}{01A61}
\pmclassification{msc}{01A60}

\endmetadata

% this is the default PlanetMath preamble.  as your knowledge
% of TeX increases, you will probably want to edit this, but
% it should be fine as is for beginners.

% almost certainly you want these
\usepackage{amssymb}
\usepackage{amsmath}
\usepackage{amsfonts}

% used for TeXing text within eps files
%\usepackage{psfrag}
% need this for including graphics (\includegraphics)
%\usepackage{graphicx}
% for neatly defining theorems and propositions
%\usepackage{amsthm}
% making logically defined graphics
%%%\usepackage{xypic}

% there are many more packages, add them here as you need them

% define commands here

\begin{document}
{\em Texas Instruments} (stock symbol TXN in the New York Stock Exchange) is the best known manufacturer of scientific calculators and programmable calculators. The company also produces semiconductors and microchips for cellular telephones.

\subsection*{Calculator models}

% Incomplete listing, intended to have every existing TI-??
\begin{tabular}{|c|l|}
\hline
TI-10 & A basic calculator that can also handle fractions and test less than and greater than \\
& inequalities. Intended for use in kindergarten classrooms. \\
\hline
TI-15 & Not quite a scientific calculator, but definitely more advanced. \\
& Intended for elementary school classrooms. \\
\hline
TI-30 & The company's oldest model of scientific calculator, variants are still produced today, \\
& like the TI-30Xa and the TI-30X IIS. All variants are capable of handling fractions, \\
& with the exception of a version designed specifically for certain standardized tests. \\
& None of them \PMlinkescapetext{supports} binary, octal or hexadecimal display. \\
\hline
TI-81 & The company's oldest model of graphing calculator. It uses TI-BASIC, which is a variant \\
& of \PMlinkname{BASIC}{BASICProgrammingLanguage}. \\
\hline
TI-89 & A graphing calculator permitted for use in some standardized tests. \\
\hline
TI-92 & A programmable graphing calculator, not allowed for use in standardized tests. \\
\hline
\end{tabular}
%%%%%
%%%%%
\end{document}
