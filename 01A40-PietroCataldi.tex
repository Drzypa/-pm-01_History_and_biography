\documentclass[12pt]{article}
\usepackage{pmmeta}
\pmcanonicalname{PietroCataldi}
\pmcreated{2013-03-22 18:05:07}
\pmmodified{2013-03-22 18:05:07}
\pmowner{Mravinci}{12996}
\pmmodifier{Mravinci}{12996}
\pmtitle{Pietro Cataldi}
\pmrecord{4}{40623}
\pmprivacy{1}
\pmauthor{Mravinci}{12996}
\pmtype{Biography}
\pmcomment{trigger rebuild}
\pmclassification{msc}{01A40}
\pmclassification{msc}{01A45}

% this is the default PlanetMath preamble.  as your knowledge
% of TeX increases, you will probably want to edit this, but
% it should be fine as is for beginners.

% almost certainly you want these
\usepackage{amssymb}
\usepackage{amsmath}
\usepackage{amsfonts}

% used for TeXing text within eps files
%\usepackage{psfrag}
% need this for including graphics (\includegraphics)
%\usepackage{graphicx}
% for neatly defining theorems and propositions
%\usepackage{amsthm}
% making logically defined graphics
%%%\usepackage{xypic}

% there are many more packages, add them here as you need them

% define commands here

\begin{document}
\emph{Pietro Antonio Cataldi} (1552 - 1626) Italian mathematician, astronomer, author and educator.

Still a teenager, Cataldi started his teaching career in 1567, and contrary to the custom of the day, he conducted class in the Bolognese dialect of Italian rather than in Latin. In addition to teaching, Cataldi also consulted with the military, his work on the calculation of roots, using continued fractions, was intended for practical artillery applications. After teaching in Florence and Perugia, Cataldi returned to Bologna in 1584, where he then taught astronomy in addition to mathematics, until his death. In 1588, Cataldi discovered two consecutive Mersenne primes, 131071 and 524287, and was for almost two centuries the discoverer of the largest known prime number.

In addition to his own books on arithmetic, number theory, and geometry, Cataldi also worked on an Italian edition of Euclid's {\it Elements}.
%%%%%
%%%%%
\end{document}
