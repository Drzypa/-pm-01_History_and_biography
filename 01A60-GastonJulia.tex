\documentclass[12pt]{article}
\usepackage{pmmeta}
\pmcanonicalname{GastonJulia}
\pmcreated{2013-03-22 16:58:08}
\pmmodified{2013-03-22 16:58:08}
\pmowner{Mravinci}{12996}
\pmmodifier{Mravinci}{12996}
\pmtitle{Gaston Julia}
\pmrecord{7}{39243}
\pmprivacy{1}
\pmauthor{Mravinci}{12996}
\pmtype{Biography}
\pmcomment{trigger rebuild}
\pmclassification{msc}{01A60}
\pmsynonym{Gaston Maurice Julia}{GastonJulia}

% this is the default PlanetMath preamble.  as your knowledge
% of TeX increases, you will probably want to edit this, but
% it should be fine as is for beginners.

% almost certainly you want these
\usepackage{amssymb}
\usepackage{amsmath}
\usepackage{amsfonts}

% used for TeXing text within eps files
%\usepackage{psfrag}
% need this for including graphics (\includegraphics)
%\usepackage{graphicx}
% for neatly defining theorems and propositions
%\usepackage{amsthm}
% making logically defined graphics
%%%\usepackage{xypic}

% there are many more packages, add them here as you need them

% define commands here

\begin{document}
\emph{Gaston Maurice Julia} (1893 - 1978) French mathematician best known for Julia sets.

In his 20s, Julia served in the French Army in World War I, losing his nose in the conflict. Coming home from the war, Julia attracted the attention of mathematicians for his paper on iterated rational functions. Even in those days before computer graphics, he was able to visualize which \PMlinkescapetext{Julia sets} are connected and which are not. During World War II, he taught at the \'Ecole Polytechnique. One of his students there, Beno\^it Mandelbrot, would grow up to popularize his work on fractals.
%%%%%
%%%%%
\end{document}
