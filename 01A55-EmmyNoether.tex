\documentclass[12pt]{article}
\usepackage{pmmeta}
\pmcanonicalname{EmmyNoether}
\pmcreated{2013-03-22 17:21:17}
\pmmodified{2013-03-22 17:21:17}
\pmowner{PrimeFan}{13766}
\pmmodifier{PrimeFan}{13766}
\pmtitle{Emmy Noether}
\pmrecord{7}{39713}
\pmprivacy{1}
\pmauthor{PrimeFan}{13766}
\pmtype{Biography}
\pmcomment{trigger rebuild}
\pmclassification{msc}{01A55}
\pmclassification{msc}{01A60}

% this is the default PlanetMath preamble.  as your knowledge
% of TeX increases, you will probably want to edit this, but
% it should be fine as is for beginners.

% almost certainly you want these
\usepackage{amssymb}
\usepackage{amsmath}
\usepackage{amsfonts}

% used for TeXing text within eps files
%\usepackage{psfrag}
% need this for including graphics (\includegraphics)
%\usepackage{graphicx}
% for neatly defining theorems and propositions
%\usepackage{amsthm}
% making logically defined graphics
%%%\usepackage{xypic}

% there are many more packages, add them here as you need them

% define commands here

\begin{document}
\PMlinkescapeword{child}
\PMlinkescapeword{regular}
\PMlinkescapeword{class}
\PMlinkescapeword{level}
\PMlinkescapeword{power}
\PMlinkescapeword{union}
\PMlinkescapeword{permanent}
\PMlinkescapeword{operation}
\PMlinkescapeword{body}

\emph{Amalie Emmy Noether} (1882 - 1935) German Jewish mathematician, best known for her work on invariants. Noetherian rings are named after her.

Born to a wealthy Jewish family, of mathematician Max Noether and socialite Ida Kaufmann, young Emmy was the oldest child of four, including brother Fritz who would also grow up to be a mathematician. Emmy Noether completed training to be a teacher of English and French, but instead decided to audit college courses at G\"ottingen and Erlangen. When the university at Erlangen, where her father taught, started admitting women as regular students, Emmy Noether enrolled as the only woman in a mathematics class of 47 students. with her dissertation ``On complete systems of invariants for ternary biquadratic forms'' with Paul Gordan as her advisor, Noether graduated summa cum laude but this wasn't enough to allow her to teach mathematics at the university level. However, as her father's health declined, she often substituted for him. In 1915, David Hilbert had the idea of letting Noether teach de facto classes which were on paper he taught de jure. Hilbert, Albert Einstein, and others were impressed by Nother's 1921 paper on ideals, but she modestly deflected compliments saying Richard Dedekind had already discovered it all. As Adolf Hitler rose to power in Germany, Noether used her wealth to help less fortunate Jewish mathematicians deal with the loss of their jobs. But Jewish mathematicians soon realized they had to escape. Noether herself, because of her communist leanings, seriously considered emigrating to the Soviet Union. Pavel Alexandrov suggested Noether come to Moscow, but the Moscow University officials dragged their feet. Emmy Noether instead went to the United States and, on the invitation of department head Anna Pell Wheeler, Noether started teaching at Bryn Mawr College in Pennsylvania as a guest professor. After two years of being well liked by faculty and students alike, both of whom had looked forward to her arrival with great anticipation, and despite her eccentric teaching methods, the college decided to offer Noether a more permanent position. Unfortunately, she died unexpectedly after an otherwise routine tumor removal operation. Her body was cremated and the ashes were buried on the Bryn Mawr campus.

One of Noether's students at G\"ottingen was the Dutch mathematician Bartel Leendert van der Waerden, who in his popularization of Noether's ideas created the ``New Math'' method of teaching mathematics.

In 1960, a street in Erlangen was named after her, and on the centennial of her birth, her old high school was also named after her.

\begin{thebibliography}{5}
\bibitem{hh} H. Henderson ``Emmy Noether'' in {\it Modern Mathematicians} New York: Facts On File, Inc. (1996): 47 - 57
\bibitem{ij} I. James ``Emmy Noether'' in {\it Remarkable Mathematicians: From Euler to von Neumann} Cambridge: Mathematical Association of America \& Cambridge University Press (2002): 321 - 326
\bibitem{mm} M. A. M. Murray {\it Women Becoming Mathematicians: Creating a Professional Identity in Post-World War II America} Cambridge: MIT Press (2000): 59 - 60
\bibitem{gn} G. E. Noether ``Emmy Noether'' in {\it Women of Mathematics: A Bibliographic Sourcebook} L. Grinstein, P. Cambpell, ed.s New York: Greenwood Press (1987): 165 - 170
\bibitem{ss} S. Segal, {\it Mathematicians Under The Nazis} Princeton: Princeton University Press (2003): 59 - 60
\end{thebibliography}
%%%%%
%%%%%
\end{document}
