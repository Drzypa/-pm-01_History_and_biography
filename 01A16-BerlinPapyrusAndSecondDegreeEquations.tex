\documentclass[12pt]{article}
\usepackage{pmmeta}
\pmcanonicalname{BerlinPapyrusAndSecondDegreeEquations}
\pmcreated{2016-04-13 6:24:50}
\pmmodified{2016-04-13 6:24:50}
\pmowner{milogardner}{13112}
\pmmodifier{milogardner}{13112}
\pmtitle{Berlin Papyrus and second degree equations}
\pmrecord{47}{42108}
\pmprivacy{1}
\pmauthor{milogardner}{13112}
\pmtype{Definition}
\pmcomment{trigger rebuild}
\pmclassification{msc}{01A16}

% this is the default PlanetMath preamble.  as your knowledge
% of TeX increases, you will probably want to edit this, but
% it should be fine as is for beginners.

% almost certainly you want these
\usepackage{amssymb}
\usepackage{amsmath}
\usepackage{amsfonts}

% used for TeXing text within eps files
%\usepackage{psfrag}
% need this for including graphics (\includegraphics)
%\usepackage{graphicx}
% for neatly defining theorems and propositions
%\usepackage{amsthm}
% making logically defined graphics
%%%\usepackage{xypic}

% there are many more packages, add them here as you need them

% define commands here

\begin{document}
Abstract: The Berlin Papyrus was written in the 1900 BCE era. An Egyptian Middle Kingdom scribe left the hieratic papyrus at Saqqara, Egypt. A brief analysis of the math contents were published in 1862, the first Egyptian fraction mathematical topic discussed in the modern era. 

A more complete analysis of Egyptian fraction aspects of the text was published in 1900. The papyrus disclosed two aspects of ancient Egyptian mathematical knowledge. The text also included descriptions of an ancient pregnancy test procedure and other Middle Kingdom medical information. The two main math aspects included solutions to two second degree equations that applied two approaches that stressed an inverse proportion named pesu.

ANANYSIS: The first aspect solved an algebraic "area of a square of 100 equal to two smaller squares,  one was 1/2 + 1/4 the side of the other". The question does not suggest knowledge of the Pythagorean theorem. This aspect demonstrate that \PMlinkexternal{square root}{http://planetmath.org/encyclopedia/EgyptianAndGreekSquareRoot.html} was understood in one and two variables. The same imagery was used to write the hieratic word  pesu,  an arithmetic proportion. 

The second aspect solved two second degree equations considered variables stated as one unknown  was incompletely reported by Scott Williams (U. of Buffalo) per:

"100 square cubits is equal to that of two smaller squares, the side of one square is 1/2 + 1/4 of the other. 

What are the sides of the two unknown squares?

In modern era many would express the two sides this math text as 

1. x2 + y2 = 100 

and 

2. x = (3/4)y. 

In other words: What were x and y?

A modern solution might report 

((3/4)y)2 + y2 = 100 

implies (1 + 9/16)y2 = (25/16)y2 = 100 

implies y2 =(16/25)100 = 64 

implies y=8 and x= (3/4)8 = 6.

The 'ab initio" scribal method was more interesting. 

The scribe began with 4x = 3y.  The sum of two squares problem was solved by an inverse proportion, a one variable (pesu) method that was generalized by Gillings Pesu problems. 

Gillings called the method Aha problems, not disclosing a scribal link to the Berlin Papyrus. 

The Middle Kingdom single variable pesu method was recorded in \PMlinkexternal{RMP 69}{http://planetmath.org/encyclopedia/RMP69AndTheBerlinPaprusProportionMethod.html}, 70, 71, 72, 73, 74, 75, 76, 77, 78,  and the Kahun Papyrus. A parallel connected the BP, RMP 69-78, and the KP was recognized by Schack-Schackenberg in 1900. The pesu fact footnoted by Clagett in 1999, but misunderstood by Clagett in the narrative. Clagett misreported the inverse proportion's division operation and basic calculations as 'single false position' rather than the two-sided single variable method that Schack-Schackenberg reported. 

Gillings confused the BP scribal solution by reporting the Egyptian scribal inform as simultaneous equations are solved today:

   x2 + y2 = 100,   4x - 3y = 0,  what are x and y?

Clagett's single false position suggestion was borrowed from 1920s attempts to read closely related Rhind Mathematical Papyrus problems and methods, an approach that not involved in the Berlin Papyrus either. 

The 1900 BCE Berlin Papyrus solution was reported by Schack-Schackenberg in 1900 AD as Ahmes reported his 1650 BCE solution in RMP 69. 

Assume the square of the first side (y) to be 1 cubit. 

Then the other side (x) will be 1/2 + 1/4. 

Then y2 = 1, and using Egyptian multiplication we determine 

x2 = (1/2 + 1/4 + 1/2)* (1/4 + 1/8 1/4)* (1/8 + 1/16 1/2 + 1/4 1/4 + 1/8 + 1/8 + 1/16) 

= 1/2 + 1/16
 
Thus, x2 + y2 = 1 + 1/2 + 1/16. 

Now (1 + 1/2 + 1/16)1/2 = 1 + 1/4 and (100)1/2 = 10. 

Divide 10 by 1 + 1/4 and you get 8.

Ahmes' pesu, an inverse proportional valuation of commodities, was finitely informed by \PMlinkexternal{Egyptian square root}{ http://planetmath.org/encyclopedia/EgyptianAndGreekSquareRoot.html}. The pesu used in the RMP 69 shed light on the Berlin Papyrus in other ways. The Middle Kingdom method was not consistently parsed by scholars in the 20th century related to confusion over 'single false position and other issues. Gillings and Clagett missed MK scribal arithmetic details by reporting personalized versions of the text's math. By reading RMP 69, the 10 by 10 cubit was broken into two squares in the ratio of 1: 3/4 to one another in the Berlin Papyrus. Following Schack-Schackenburg, properly footnoted by Clagett, the pesu method offered a direct proof that abstract mathematics solved two second 1900 BCE second degree equations within hekat and loaf conversions to Pesu units.

Clagett was not alone in reporting the Berlin Papyrus method contained single false position division operation. Raw transliterated hieratic data shows that Ahmes obtained 5/4 from the pesu step, and not from single false position. The simplest version of the data says that 10 was divided by 5/4 and solved by 10 times 4/5 = 8, as we do today.

So we get x = 8.

The Berlin Papyrus reported

y2 = 100 - 64

y = 6 

obtained y = 6, using modern arithmetic steps such that:

64 + 36 = 100

was proven.

Berlin Papyrus Problem 2. You are told the area of a square of 400 square cubits was equal to that of two smaller squares, the side of one square is 1/2 + 1/4 of the other, reported by the scribe as 2: 3/2 of one another. What are the sides of the two unknown squares?

This is analogous to problem 1, ... except that the Berlin Papyrus scribe's pesu calculation used 2: 3/2, an analysis that placed the pesu calculation in a different logical step than the first problem. Subtle differences are important as historians report scribal shorthand notes as the mathematics was recorded.

The second Berlin Papyrus problem solved for x, and y

within x squared plus y squared equaled 400, by considering

2x = (3/2)y

without solving for x before applying the arithmetic proportion method.

The x value was found after the proportion step by first finding

2x = 20 x 2/5

x = 20 x 4/5 = 16,

y2 = 400 - 256 = 144

y = 12

correctly found

256 + 144 = 400

CONCLUSION: It is recommended that both Berlin Papyrus second degree equation problems be studied in the context of RMP 69-78, and the Kahun Papyrus pesu proportion method. Compare scribal conversions of hekat and loaves of bread to a pesu unit in the context of RMP 69-78, the Kahun Papyrus, and the proportional solution of two Berlin Papyrus second degree equations. Several scholars in the 21st century, 
cluding that Middle Kingdom mathematics wastheoretically based, recorded in hard-to-ecode shorthand, reporting anumber of unifying arithmetic aspects, two being a second definition of scribal multiplication allow a clear use of scribal division as inverse to the second multiplication operation/ in the 19th and 20th centuries. Scribal math was likely
unfied by a rational number double-check relationship to practical measurements recorded in double-entry accounting records, and scribal shorthand notes. The Berlin Papyrus, read within the context of RMP 69, offers two short pages of a long Middle Kingdom math story.

\begin{thebibliography}{9}
\bibitem{1}  A.B. Chace, Bull, L, Manning, H.P., Archibald, R.C., \emph{The Rhind Mathematical Papyrus}, Mathematical Association of America, Vol I, 1927. NCTM reprints available.
\bibitem{2}Marshall Clagett \emph{Ancient Egyptian Science, Volume III}, American Philosophical Society, Philadelphia, 1999.
\bibitem{3} Milo Gardner, \emph{An Ancient Egyptian Problem and its Innovative Solution, Ganita Bharati}, MD Publications Pvt Ltd, 2006.
\bibitem{4}Richard Gillings, \emph{Mathematics in the Time of the Pharaohs}, Dover Books, 1992.
\bibitem{5} H. Schack-Schackenburg, \emph{"Der Berliner Papyreys 6619", Zeitscrift fur Agypyische Sprache} , Vol 38 (1900), pp. 135-140 and Vol. 40 (1902), p. 65f.
\bibitem{6} T.E. Peet, \emph{Arithmetic in the Middle Kingdom}, Journal Egyptian Archeology, 1923.
\bibitem{7} Tanja Pommerening, \emph{"Altagyptische Holmasse Metrologish neu Interpretiert" and relevant phramaceutical and medical knowledge, an abstract,  Phillips-Universtat, Marburg, 8-11-2004, taken from "Die Altagyptschen Hohlmass}, Buske-Verlag, 2005.
\bibitem{8} Gay Robins, and Charles Shute \emph{Rhind Mathematical Papyrus}, British Museum Press, Dover reprint, 1987.
\bibitem{9} Hana Vymazalova, \emph{The Wooden Tablets from Cairo:The Use of the Grain Unit HK3T in Ancient Egypt, Archiv Orientalai}, Charles U Prague, 2002.
\end{thebibliography}

%%%%%
%%%%%
\end{document}
