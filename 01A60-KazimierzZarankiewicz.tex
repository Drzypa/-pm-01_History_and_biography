\documentclass[12pt]{article}
\usepackage{pmmeta}
\pmcanonicalname{KazimierzZarankiewicz}
\pmcreated{2013-03-22 19:05:30}
\pmmodified{2013-03-22 19:05:30}
\pmowner{ZZZTopologist}{24722}
\pmmodifier{ZZZTopologist}{24722}
\pmtitle{Kazimierz Zarankiewicz}
\pmrecord{4}{41982}
\pmprivacy{1}
\pmauthor{ZZZTopologist}{24722}
\pmtype{Biography}
\pmcomment{trigger rebuild}
\pmclassification{msc}{01A60}

\endmetadata

% this is the default PlanetMath preamble.  as your knowledge
% of TeX increases, you will probably want to edit this, but
% it should be fine as is for beginners.

% almost certainly you want these
\usepackage{amssymb}
\usepackage{amsmath}
\usepackage{amsfonts}

% used for TeXing text within eps files
%\usepackage{psfrag}
% need this for including graphics (\includegraphics)
%\usepackage{graphicx}
% for neatly defining theorems and propositions
%\usepackage{amsthm}
% making logically defined graphics
%%%\usepackage{xypic}

% there are many more packages, add them here as you need them

% define commands here

\begin{document}
Kazimierz Zarankiewicz (1902-1959) Polish topologist. He taught at the Society for Scientific Courses in Warsaw its first year in 1906. His work confirmed a result by Urysohn regarding hereditarily locally connected continua. The Zarankiewicz problem concerns the edges of bipartite graphs.

L'Europe mathématique: histoires, mythes, identités
 By Cathérine Goldstein, Jeremy Gray, Jim Ritter
  p. 294

Handbook of the history of general topology, Volume 2
 By Charles E. Aull, Robert Lowen
  p. 719
%%%%%
%%%%%
\end{document}
