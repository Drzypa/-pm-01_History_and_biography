\documentclass[12pt]{article}
\usepackage{pmmeta}
\pmcanonicalname{IshangoBone}
\pmcreated{2013-03-22 17:20:26}
\pmmodified{2013-03-22 17:20:26}
\pmowner{PrimeFan}{13766}
\pmmodifier{PrimeFan}{13766}
\pmtitle{Ishango bone}
\pmrecord{6}{39695}
\pmprivacy{1}
\pmauthor{PrimeFan}{13766}
\pmtype{Definition}
\pmcomment{trigger rebuild}
\pmclassification{msc}{01A10}

\endmetadata

% this is the default PlanetMath preamble.  as your knowledge
% of TeX increases, you will probably want to edit this, but
% it should be fine as is for beginners.

% almost certainly you want these
\usepackage{amssymb}
\usepackage{amsmath}
\usepackage{amsfonts}

% used for TeXing text within eps files
%\usepackage{psfrag}
% need this for including graphics (\includegraphics)
%\usepackage{graphicx}
% for neatly defining theorems and propositions
%\usepackage{amsthm}
% making logically defined graphics
%%%\usepackage{xypic}

% there are many more packages, add them here as you need them

% define commands here

\begin{document}
The {\em Ishango bone} is an ancient baboon bone with numerical markings on it which was unearthed in a bank of Edward Lake in the Congo by geologist Jean de Heinzelin de Braucourt. Carbon dating shows the bone is at least 20,000 years old. Now on exhibit at the museum of the Royal Belgian Institute of Natural Sciences, it is characterized as ``the oldest mathematical artifact.''

The bone has three columns of numbers, the middle column reads: 3, 6, 4, 8, 10, 5, 5, 7 (this sequence is A100000 in Sloane's OEIS). The other two columns read: 11, 13, 17, 19 (a prime quadruplet); and 11, 21, 19, 9. The columns add up to 60, 48 and 60.

Mathematicians and scientists have speculated on the meaning of the numbers on the bone. Alexander Marshack believes the bone might be a lunar calendar, while Claudia Zaslavsky speculates the author was a woman tracking her menstrual cycle. Number theorists note that 48 and 60 are both multiples of 12 and cite this as evidence of the early humans' ability to multiply.

\begin{thebibliography}{4}
\bibitem{dh} D. Huylebrouck, ``L'Afrique, berceau des mathematiques'', in {\it Mathematiques exotiques Dossier No. 47, Pour La Science} (2005) 46 - 50 
\bibitem{gj} G. G. Joseph, {\it The Crest of the Peacock: Non-European Roots of Mathematics} London: Penguin Books (1992). 
\bibitem{am} A. Marshack, {\it The Roots of Civilization} Mount Kisco (1991)
\bibitem{do} D. Olivastro, ``The First Etches'' {\it Ancient Puzzles} New York: Bantam Books (1993): 7 - 30 
\bibitem{vp} V. Pletser \& D. Huylebrouck, ``The Ishango artifact: the missing base 12 link'', {\it Proc. Katachi Univ. Symmetry Congress (KUS2), Paper C11}, (1999) 339 - 346. 
\bibitem{cz} Claudia Zaslavsky, {\it Africa Counts} New York: Lawrence Hill Books (1973) 
\end{thebibliography}
%%%%%
%%%%%
\end{document}
