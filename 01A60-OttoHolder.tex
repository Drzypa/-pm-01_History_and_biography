\documentclass[12pt]{article}
\usepackage{pmmeta}
\pmcanonicalname{OttoHolder}
\pmcreated{2013-03-22 16:14:22}
\pmmodified{2013-03-22 16:14:22}
\pmowner{Mravinci}{12996}
\pmmodifier{Mravinci}{12996}
\pmtitle{Otto H\"older}
\pmrecord{11}{38341}
\pmprivacy{1}
\pmauthor{Mravinci}{12996}
\pmtype{Definition}
\pmcomment{trigger rebuild}
\pmclassification{msc}{01A60}
\pmsynonym{Otto Holder}{OttoHolder}
\pmsynonym{Otto Hoelder}{OttoHolder}

% this is the default PlanetMath preamble.  as your knowledge
% of TeX increases, you will probably want to edit this, but
% it should be fine as is for beginners.

% almost certainly you want these
\usepackage{amssymb}
\usepackage{amsmath}
\usepackage{amsfonts}

% used for TeXing text within eps files
%\usepackage{psfrag}
% need this for including graphics (\includegraphics)
%\usepackage{graphicx}
% for neatly defining theorems and propositions
%\usepackage{amsthm}
% making logically defined graphics
%%%\usepackage{xypic}

% there are many more packages, add them here as you need them

% define commands here

\begin{document}
\emph{Otto H\"older} German mathematician (1859--1937) best known for H\"older's inequality and the \PMlinkname{Jordan-H\"older theorem}{JordanHolderDecompositionTheorem}.  H\"older also made many essential first steps in the then emerging study of groups.  Some of his results include a complete classification of the groups of \PMlinkname{order}{OrderGroup} $p^2,\,p^3,\,p^4$, $pq$ and $p^2q$ ($p$ and $q$ different primes) and are an early example of the power of the Sylow theorems for finite group theory.

He studied at the University of Berlin, got his doctorate from the University of T\"ubingen, and taught at the University of Leipzig.

The generalized mean is sometimes called the ``H\"older mean''.
%%%%%
%%%%%
\end{document}
