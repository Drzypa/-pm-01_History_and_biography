\documentclass[12pt]{article}
\usepackage{pmmeta}
\pmcanonicalname{PaulEhrenfest}
\pmcreated{2013-03-22 16:50:07}
\pmmodified{2013-03-22 16:50:07}
\pmowner{PrimeFan}{13766}
\pmmodifier{PrimeFan}{13766}
\pmtitle{Paul Ehrenfest}
\pmrecord{4}{39078}
\pmprivacy{1}
\pmauthor{PrimeFan}{13766}
\pmtype{Biography}
\pmcomment{trigger rebuild}
\pmclassification{msc}{01A55}
\pmclassification{msc}{01A60}

\endmetadata

% this is the default PlanetMath preamble.  as your knowledge
% of TeX increases, you will probably want to edit this, but
% it should be fine as is for beginners.

% almost certainly you want these
\usepackage{amssymb}
\usepackage{amsmath}
\usepackage{amsfonts}

% used for TeXing text within eps files
%\usepackage{psfrag}
% need this for including graphics (\includegraphics)
%\usepackage{graphicx}
% for neatly defining theorems and propositions
%\usepackage{amsthm}
% making logically defined graphics
%%%\usepackage{xypic}

% there are many more packages, add them here as you need them

% define commands here

\begin{document}
\emph{Paul Ehrenfest} (1880 - 1933) Jewish Dutch mathematician and physicist of Austrian birth, mainly concerned with the mathematics of quantum mechanics, best known for the Ehrenfest theorem. Husband of Tatyana Alexeyevna Afanasyeva and father of Tatyana Pavlovna Ehrenfest.

Born in Vienna, the son of grocers, young Paul did not always do well in school. Going to university, he at first focused on chemistry, but gradually his interests shifted to theoretical physics. At G\"ottingen, he met Tatyana Afanasyeva. Soon, in 1905, she bore him a daughter, Tatyana Pavlovna, whom they home-schooled at first with a solid background in physics and mathematics. His Ph.D. dissertation was on the motion of rigid bodies in fluids. In 1907, Ehrenfest succeeded Hendrik Lorentz as professor of physics at the University of Leiden, where he invited Albert Einstein to teach there. In 1933, Ehrenfest committed suicide.
%%%%%
%%%%%
\end{document}
