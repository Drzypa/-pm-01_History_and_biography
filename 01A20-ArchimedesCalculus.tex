\documentclass[12pt]{article}
\usepackage{pmmeta}
\pmcanonicalname{ArchimedesCalculus}
\pmcreated{2015-01-20 11:57:59}
\pmmodified{2015-01-20 11:57:59}
\pmowner{milogardner}{13112}
\pmmodifier{milogardner}{13112}
\pmtitle{Archimedes' calculus}
\pmrecord{117}{40788}
\pmprivacy{1}
\pmauthor{milogardner}{13112}
\pmtype{Definition}
\pmcomment{trigger rebuild}
\pmclassification{msc}{01A20}
\pmsynonym{differential calculus}{ArchimedesCalculus}

\endmetadata

% this is the default PlanetMath preamble.  as your knowledge
% of TeX increases, you will probably want to edit this, but
% it should be fine as is for beginners.

% almost certainly you want these
\usepackage{amssymb}
\usepackage{amsmath}
\usepackage{amsfonts}

% used for TeXing text within eps files
%\usepackage{psfrag}
% need this for including graphics (\includegraphics)
%\usepackage{graphicx}
% for neatly defining theorems and propositions
%\usepackage{amsthm}
% making logically defined graphics
%%%\usepackage{xypic}

% there are many more packages, add them here as you need them

% define commands here

\begin{document}
Archimedes is properly given credit for publishing the first \PMlinkexternal{calculus}{ http://planetmath.org/encyclopedia/NonNewtonianCalculus2.html}. The best known of today's \PMlinkexternal{calculus}{http://en.wikipedia.org/wiki/Fundamental_theorem_of_calculus} was published by Newton. Newton's notation was improved to dx/dy notatiopn by Leibnitz. Modern differential calculus is based on the \PMlinkexternal{mean value theorem}{http://en.wikipedia.org/wiki/Mean_Value_Theorem}.

Heiberg's 1906 translation of the fragmented vellum text directly showed Archimedes recorded two methods in the 300 BCE Classical Greek era. The oldest calculus converted rational numbers to \PMlinkexternal{Egyptian unit fraction series}{http://www.mathpages.com/home/kmath340/kmath340.htm}. The first method scaled rational numbers to a 1/4 geometric series algorithm followed a tradition established by Eudoxus, and one phase of the Egyptian Eye of Horus notation. The first unit fraction notation was not in use in Classical Greece. The infinite series converted rational number 4/3 was intended to be summed to an area of a parabola. The infinite series may have reported a historical "exhaustion of exhaustion" calculus method. 

Equally, or more likely the infinite series may have been a statement of a problem, and thus, not Archimedes' primary calculus method. 

The second method reported 4/3 as a concise finite series data that may have been intended by Archimedes as a primarily calculation. The second method's data was written in the standard Greek arithmetic notation used in the 300 BCE Hibeh Papyrus and \PMlinkexternal{Plato's "Republic"}{http://planetmath.org/encyclopedia/PlatosMathematics.html}.  

That is, which unit fraction data set was primary to Archimedes? Combining both data sets into one historical thread, was a statement of a problem and a finite calculation reported by Archimedes'. 

In other words, did the calculus work of Archimedes passed down to Byzantines arrive with clear explanations? 


ANCIENT DISCUSSION

The traditional calculus story says that Archimedes only used a "method of exhaustion " that defined the area of a parabola on an erasable parchment (palimpsest). The original intent of the data is not clear. The parchment's numerical information was not recorded in Archimedes' handwriting. Worse, the parchment's information was copied over hundreds of years, and erased in 1,100 AD by Byzantine priests. Byzantines used the vellum parchment to write religious texts. 

In 1906 J.L. Heiberg translated portions of the hard-to-read text and showed that the first method exactly summed the area of a parabola to an infinite 1/4 geometric series,

4A/3 = A + A/4 + A/16 + A/64 + ...

method of exhaustion data that was at least a statement of a problem to be solved.  Or, did the data represent one of two equally important methods? 

In other words, Archimedes may have asked. "How can an infinite series be written as an exact finite series"?

The second method wrote out a finite Egyptian fraction series, exactly pointed out the same answer in 3-terms,

4A/3 = A + A/4 + A/12 

Considering both methods, Archimedes' calculus may have stated a 1/4 geometric series (algorithm) problem that was required to be solved by a finite unit fraction series. 

MODERN DISCUSSION

A. To introduce Classical Greek accuracy of Archimedes' rational number system a solution to x^2 = 3 offers a limit to  an irrational number x that resides in the range

265/156 < x < 1351/780

The problem was documented by \PMlinkexternal{Kevin Brown}{http://planetmath.org/squarerootof3567and29} who quotes several math historians. The problem was solved in 2012 in an \PMlinkexternal{Archimedes' square root}{http://planetmath.org/squarerootof3567and29} study by:

(1) step 1. guess (1 + 2/3)^2 = 1 + 4/3 + 4/9 = (2 + 3/9 + 4/9) = 2 + 7/9 = error 2/9

(2) step 2 reduce error 2/9 (3/10) = 1/15 (divided 2/9 by [2(1 + 2/3) = 10/3],

and added 1/15 such that

(3) (1 + 2/3 + 1/15)^2, error (1/15)^2 = 1/225, meant (1 + 11/15) = 26/15

step 3 reduced error 1/225 (15/52) = 1/15(52) = 1/780 (same as step 2)

(4) The lower limit 265/153 modified step 2, used
  
1/17 rather than 1/15, (1+ 2/3 + 1/17) = (1 + 37/51)

such that (1 + 111/153)changed to (1 + 112/153) = 265/153

(26/15 + 1/780)^2 = (1353/780)^2 in modern fractions

B. Rational aspects of the first calculus story line were reported by Stanford University researchers. The researchers stressed the infinite series algorithm side of the document as Archimedes' solution without mentioning the historical context in which exact finite data were not possible in solving x^2 = 3 and x^2 = p problems. 

Oddly, the finite Egyptian fraction information published by Heiberg in 1906 and Dijksterhuis in 1987 was ignored by Stanford researchers. The omission skipped over Archimedes' rational number system. Archimedes relied upon exactly solving the area of a parabola that did not sum the infinite series.

\PMlinkexternal{E.J. Dijksterhuis}{http://mathforum.org/kb/message.jspa?messageID=5847373&tstart=90} included Heiberg's view in the 1987 "Archimedes" biography published by Princeton Press, The discussion begins with an Archimedes Lemma: In Quadrature of the Parabola Archimedes proves the following proposition on the sum of a geometrical progression with a common ratio of 1/4. 

Given a series of magnitudes, each of which is equal to four times the order of the next, all of the magnitudes and one-third of the least added together will exceed the greatest by one-third.

Let the magnitudes A, B, C, D, E be given such that

A + B + C + D + E + 1/3E = (4/3)A

Dijksterhuis wrote out the 1/4 geometric infinite series:

4A/3 = A + A/4 + A/16 + A/64 + ...

an infinite series.

Heiberg  published a finite Egyptian fraction series side of the discussion, as Dijksterhuis wrote as:

4A/3 = A + A/4 + A/12

that proved the accuracy of a finite 1/4 geometric series method that followed Eudoxus that used the same tradition.

The palimpsest document came on the open market a few years ago. It was auctioned for 2,000,000 dollars. NOVA reported a revised analysis of the text that was suggested by its new owners. The NOVA program did not include Heiberg and Dijksterhuis' 1/4 geometric series method written as a finite series in its review. 
Stanford University investigators}{http://www.archimedespalimpsest.org/mediacenter_presskit.html} only published the infinite series (1/4 geometric series) side of the document without discussing the equally important Egyptian fraction series side connecting: 

1/3 = 1/4 + 1/16 + 1/64 + ... + 1/4n + ... 

which is one-half phase of the Horus-Eye series:

1 = 1/2 + 1/4 + 1/8 + 1/16 + 1/32 + 1/64 + ... +  1/2n + ... 

CONCLUSION: Archimedes is proposed as creating the first calculus by stating the problem (finding the area of parabola) as an infinite series. The finite series was recorded in the standard Greek arithmetic notation. Considering the infinite and finite series as one data set, Archimedes stated a rational number problem by a 1/4 geometric infinite series, and calculated a finite series in the standard Greek notation.  

References from Wikipedia: 

1. \PMlinkexternal{Parabola}{http://en.wikipedia.org/wiki/Parabola}
2. \PMlinkexternal{Quadrature of a parabola}{http://en.wikipedia.org/wiki/The_Quadrature_of_the_Parabola}}

%%%%%
%%%%%
\end{document}
