\documentclass[12pt]{article}
\usepackage{pmmeta}
\pmcanonicalname{JohnHortonConway}
\pmcreated{2013-03-22 16:35:23}
\pmmodified{2013-03-22 16:35:23}
\pmowner{Mravinci}{12996}
\pmmodifier{Mravinci}{12996}
\pmtitle{John Horton Conway}
\pmrecord{4}{38784}
\pmprivacy{1}
\pmauthor{Mravinci}{12996}
\pmtype{Biography}
\pmcomment{trigger rebuild}
\pmclassification{msc}{01A60}
\pmclassification{msc}{01A61}
\pmclassification{msc}{01A65}
\pmsynonym{John Conway}{JohnHortonConway}

\endmetadata

% this is the default PlanetMath preamble.  as your knowledge
% of TeX increases, you will probably want to edit this, but
% it should be fine as is for beginners.

% almost certainly you want these
\usepackage{amssymb}
\usepackage{amsmath}
\usepackage{amsfonts}

% used for TeXing text within eps files
%\usepackage{psfrag}
% need this for including graphics (\includegraphics)
%\usepackage{graphicx}
% for neatly defining theorems and propositions
%\usepackage{amsthm}
% making logically defined graphics
%%%\usepackage{xypic}

% there are many more packages, add them here as you need them

% define commands here

\begin{document}
\emph{John Horton Conway} (1937 - ) British mathematician, best known for his work in number theory and game theory. Conway invented the Game of Life and chained arrow notation for large numbers, among other things.

In 1979, Conway published a paper with Croft, Erd\H{o}s and Guy ``On the Distribution of Values of Angles Determined by Coplanar Points'' in the {\it Journal of the London Mathematical Society}, giving him Erd\H{o}s number 1.
%%%%%
%%%%%
\end{document}
