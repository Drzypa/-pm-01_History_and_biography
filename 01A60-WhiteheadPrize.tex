\documentclass[12pt]{article}
\usepackage{pmmeta}
\pmcanonicalname{WhiteheadPrize}
\pmcreated{2013-03-22 16:48:54}
\pmmodified{2013-03-22 16:48:54}
\pmowner{PrimeFan}{13766}
\pmmodifier{PrimeFan}{13766}
\pmtitle{Whitehead Prize}
\pmrecord{4}{39051}
\pmprivacy{1}
\pmauthor{PrimeFan}{13766}
\pmtype{Definition}
\pmcomment{trigger rebuild}
\pmclassification{msc}{01A60}
\pmclassification{msc}{01A61}
\pmclassification{msc}{01A65}
\pmdefines{Senior Whitehead Prize}

% this is the default PlanetMath preamble.  as your knowledge
% of TeX increases, you will probably want to edit this, but
% it should be fine as is for beginners.

% almost certainly you want these
\usepackage{amssymb}
\usepackage{amsmath}
\usepackage{amsfonts}

% used for TeXing text within eps files
%\usepackage{psfrag}
% need this for including graphics (\includegraphics)
%\usepackage{graphicx}
% for neatly defining theorems and propositions
%\usepackage{amsthm}
% making logically defined graphics
%%%\usepackage{xypic}

% there are many more packages, add them here as you need them

% define commands here

\begin{document}
The {\em Whitehead Prize} is a prize awarded by the London Mathematical Society to a mathematician working in the United Kingdom who is at an early stage of their career and has not received any other prizes from the Society. More specifically, the person being considered for the award must be resident in the United Kingdom (England, Scotland, Canada, Australia, etc.) on January 1 of the award year and have less than 15 years of work at the postdoctorate level. Since the inception of the prize, no more than two could be awarded per year, but in 1999 this was increased to 4 ``to allow for the award of prizes across the whole of mathematics, including applied mathematics, mathematical physics, and mathematical aspects of computer science.'' The {\em Senior Whitehead Prize} has similar residence requirements and rules concerning prior prizes, but is intended to recognize more experienced mathematicians. These prizes were instituted in memory of homotopy theory pioneer John Henry Constantine Whitehead.

\subsection{External links}
\PMlinkexternal{Official Whitehead Prize rules}{http://www.lms.ac.uk/activities/prizes_com/prizes.html}
%%%%%
%%%%%
\end{document}
