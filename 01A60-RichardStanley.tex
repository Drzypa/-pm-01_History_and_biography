\documentclass[12pt]{article}
\usepackage{pmmeta}
\pmcanonicalname{RichardStanley}
\pmcreated{2013-03-22 17:01:02}
\pmmodified{2013-03-22 17:01:02}
\pmowner{PrimeFan}{13766}
\pmmodifier{PrimeFan}{13766}
\pmtitle{Richard Stanley}
\pmrecord{7}{39301}
\pmprivacy{1}
\pmauthor{PrimeFan}{13766}
\pmtype{Biography}
\pmcomment{trigger rebuild}
\pmclassification{msc}{01A60}
\pmclassification{msc}{01A61}
\pmclassification{msc}{01A65}
\pmsynonym{Richard Peter Stanley}{RichardStanley}

% this is the default PlanetMath preamble.  as your knowledge
% of TeX increases, you will probably want to edit this, but
% it should be fine as is for beginners.

% almost certainly you want these
\usepackage{amssymb}
\usepackage{amsmath}
\usepackage{amsfonts}

% used for TeXing text within eps files
%\usepackage{psfrag}
% need this for including graphics (\includegraphics)
%\usepackage{graphicx}
% for neatly defining theorems and propositions
%\usepackage{amsthm}
% making logically defined graphics
%%%\usepackage{xypic}

% there are many more packages, add them here as you need them

% define commands here

\begin{document}
\emph{Richard Peter Stanley} (1944 - ) is an American mathematician.
He is a student of Gian-Carlo Rota.

A Harvard graduate, Stanley went on to teach at MIT. His book {\it
Enumerative Combinatorics} is seen as a landmark introduction to
combinatorics.  The book is infamous for its exercise which presents
66 objects counted by the Catalan numbers, then challenges the reader
to give a bijective proof of that fact.  On his website, he maintains
the ``Catalan addendum'', which as of 26 February 2007 includes 147
different combinatorial interpretations for the Catalan numbers.

Stanley was awarded the Schock prize in 2003 for his contributions to
combinatorics, in particular his proof of the necessity of McMullen's
conditions for a tuple $(f_0,\dots,f_{d-1})$ to be the f-vector of a
simplicial polytope (one half of the g-theorem), and for his advancement
of graduate-level mathematical exposition.

In 1978, Stanley co-authored a paper on the ``Enumeration of power
sums modulo a prime'' in {\it J. Number Theory} {\bf 10} with Andrew
Odlyzko, giving Stanley an \PMlinkname{Erd\H{o}s number}{ErdHosNumber}
of 2.

%%%%%
%%%%%
\end{document}
