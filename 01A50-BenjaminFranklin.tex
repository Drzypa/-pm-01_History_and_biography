\documentclass[12pt]{article}
\usepackage{pmmeta}
\pmcanonicalname{BenjaminFranklin}
\pmcreated{2013-03-22 17:46:16}
\pmmodified{2013-03-22 17:46:16}
\pmowner{PrimeFan}{13766}
\pmmodifier{PrimeFan}{13766}
\pmtitle{Benjamin Franklin}
\pmrecord{5}{40227}
\pmprivacy{1}
\pmauthor{PrimeFan}{13766}
\pmtype{Biography}
\pmcomment{trigger rebuild}
\pmclassification{msc}{01A50}
\pmclassification{msc}{01A45}
\pmsynonym{Ben Franklin}{BenjaminFranklin}

% this is the default PlanetMath preamble.  as your knowledge
% of TeX increases, you will probably want to edit this, but
% it should be fine as is for beginners.

% almost certainly you want these
\usepackage{amssymb}
\usepackage{amsmath}
\usepackage{amsfonts}

% used for TeXing text within eps files
%\usepackage{psfrag}
% need this for including graphics (\includegraphics)
%\usepackage{graphicx}
% for neatly defining theorems and propositions
%\usepackage{amsthm}
% making logically defined graphics
%%%\usepackage{xypic}

% there are many more packages, add them here as you need them

% define commands here

\begin{document}
{\em Benjamin Franklin} (born January 17 (by the Gregorian calendar), 1706 in Boston, died April 17, 1790 in Philadelphia) was one of the Founding Fathers of the United States of America. His key role during the War of Independence of the American Colonies from the British Empire, when he was appointed as an Ambassador to France, was crucial in arranging for prompt financial and military support from France during the war.  Today he is also remembered for his other many achievements, such as being the first to study electricity, coming up with Daylight Savings Time, public libraries, bifocals, odometers, refining the Americal postal service, etc. His likeness appears on the American \$100 bill.

Many well-known proverbs and sayings come from his {\it Poor Richard's Almanack}, which in addition to the usual calculations on sunrises and sunsets, ebbs and tides, also contained population statistics.

Franklin has sometimes been described as ``a polymath who did everything but mathematics.'' However, in a recent book, Paul Pasles argues that Franklin did a lot more in mathematics than has been previously acknowledged. Besides the Franklin magic square, Franklin also tackled the problem of circling the square, statistical projections, etc. Somewhat of an anti-war activist and an abolitionist, Franklin argued against war and slavery using arguments pertaining to economics rather than morality.

\begin{thebibliography}{1}
\bibitem{pp} Paul Pasles, {\it Benjamin Franklin's numbers : an unsung mathematical odyssey}. Princeton: Princeton University Press (2008)
\end{thebibliography}
%%%%%
%%%%%
\end{document}
