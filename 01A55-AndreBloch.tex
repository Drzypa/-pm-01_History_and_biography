\documentclass[12pt]{article}
\usepackage{pmmeta}
\pmcanonicalname{AndreBloch}
\pmcreated{2013-03-22 16:23:58}
\pmmodified{2013-03-22 16:23:58}
\pmowner{PrimeFan}{13766}
\pmmodifier{PrimeFan}{13766}
\pmtitle{Andr\'e Bloch}
\pmrecord{4}{38547}
\pmprivacy{1}
\pmauthor{PrimeFan}{13766}
\pmtype{Biography}
\pmcomment{trigger rebuild}
\pmclassification{msc}{01A55}
\pmclassification{msc}{01A60}
\pmsynonym{Andre Bloch}{AndreBloch}

\endmetadata

% this is the default PlanetMath preamble.  as your knowledge
% of TeX increases, you will probably want to edit this, but
% it should be fine as is for beginners.

% almost certainly you want these
\usepackage{amssymb}
\usepackage{amsmath}
\usepackage{amsfonts}

% used for TeXing text within eps files
%\usepackage{psfrag}
% need this for including graphics (\includegraphics)
%\usepackage{graphicx}
% for neatly defining theorems and propositions
%\usepackage{amsthm}
% making logically defined graphics
%%%\usepackage{xypic}

% there are many more packages, add them here as you need them

% define commands here

\begin{document}
\emph{Andr\'e Bloch} (1893 - 1948) French mathematician perhaps best known for Bloch's theorem on holomorphic disks (not to be confused with Felix Bloch's theorem on eigenfunctions of periodic particle wavefunctions).

In 1914, Andr\'e Bloch briefly served in the French Army during World War I. Three years later, Bloch killed three relatives and was confined to a mental institution. From there he collaborated by mail with important mathematicians such as Jacques Hadamard and George P\'olya. In 1948, Bloch was awarded the Becquerel Prize.
%%%%%
%%%%%
\end{document}
