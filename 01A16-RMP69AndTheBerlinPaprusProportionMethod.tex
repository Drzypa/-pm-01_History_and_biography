\documentclass[12pt]{article}
\usepackage{pmmeta}
\pmcanonicalname{RMP69AndTheBerlinPaprusProportionMethod}
\pmcreated{2013-05-26 13:29:24}
\pmmodified{2013-05-26 13:29:24}
\pmowner{milogardner}{13112}
\pmmodifier{milogardner}{13112}
\pmtitle{RMP 69 and the Berlin Paprus proportion method}
\pmrecord{18}{42107}
\pmprivacy{1}
\pmauthor{milogardner}{13112}
\pmtype{Definition}
\pmcomment{trigger rebuild}
\pmclassification{msc}{01A16}

% this is the default PlanetMath preamble.  as your knowledge
% of TeX increases, you will probably want to edit this, but
% it should be fine as is for beginners.

% almost certainly you want these
\usepackage{amssymb}
\usepackage{amsmath}
\usepackage{amsfonts}

% used for TeXing text within eps files
%\usepackage{psfrag}
% need this for including graphics (\includegraphics)
%\usepackage{graphicx}
% for neatly defining theorems and propositions
%\usepackage{amsthm}
% making logically defined graphics
%%%\usepackage{xypic}

% there are many more packages, add them here as you need them

% define commands here

\begin{document}
RMP 69 and the Berlin Papyrus: An Inverse Proportion Method

This entry outlines scribal shorthand data by translating raw unit fraction data into modern arithmetic.  The scribal shorthand data was made readable by adding back missing data. The restored info created accurate scribal longhand data recorded in modern arithmetic statements.

There are three reasons for modern students of Egyptian mathematics to study the 1650 BCE Rhind Mathematical Papyrus(RMP) problem 69. All three reasons solve two Berlin Papyrus second degree equations in scribal algebra.  

To outline the math shorthand used by Ahmes, the RMP scribe, a three phase proof will be analyzed. 

1. In RMP 69 Ahmes converted 3 1/2 hekats of grain that made 80 loaves of bread into a pesu unit. Scholars Schack-Schackenburg, early on, commented on a proportion method that a \PMlinkexternal{Berlin Papyrus}{http://planetmath.org/encyclopedia/BerlinPapyrusAndSecondDegreeEquations.html} scribe, and Ahmes, the RMP scribe, utilized as a common mathematical tool. 

Initially 3 1/2 hekats of grain meal, that produced 80 loaves of bread, were combined into pesu units for distribution purposes. Ahmes' first phase calculated the pesu as a rational rational number by applying a proportion.

Ahmes conversion of 7/2 hekat, making 80 loaves of bread, to 22 18/21 pesu was achieved by inverting 7/2 to 2/7 and multiplying by 80, reporting:

80 times 2/7 equaled (160/7) pesu and (22 + 2/3 + 1/7 + 1/21) pesu

by applying the Old Kingdom duplation multiplication method.

Modern conversions of hekat, loaf, and pesu date to modern rational numbers discussed the same proportion method used in the Berlin Papyrus. In modern fractions the Berlin scribe found two squared areas, one 10 cubit by 10 cubits, and the second 20 cubits by 20 cubits, by considering the proportions: 1: 3/4, and 2: 1/3.The two Berlin Papyrus problems solved second degree equations

2. Ahmes proved the answer by returning its pesu unit fractions to 80 loaves of bread. This was done by:

(22 + 18/21) pesu times 7/2 equals 80 loaves of bread,

by applying the Old Kingdom duplation method.

3. The second phase of Ahmes' discussion links two Berlin Papyrus solutions to two squares equal to 100 and 400 cubits squared, with x proportional to y by 1: 3/4.

Gillings recorded the proportional relationship as

4x + 3y = 0

 The BP scribe avoided modern thinking by reducing:

a. 4x = 3y

in the A = 100 problem to:

b. x = (3/4)y

. for the A = 400 problem considered

 2x = (3/2)y

The BP scribe proved that either x or 2x proportional solutions were valid by applying the well known proportional method, a direct analogy to the pesu method.

3. The third phase of RMP 69 multiplied 14 ro times 80 to shows that one loaf of bread's ro value was properly stated to 80 loaves in hekat units. Ahmes did that when he wrote:

a. 14 ro as 1/32 hekat + 4 ro

b. 28 ro as 1/16 1/64 hekat + 3 ro

c. 56 ro as 1/8 1/32 1/64 hekat + 1ro

d. 112 ro as 1/4 1/16 1/32 hekat + 2 ro

e. 224 ro as 1/2 1/8 1/16 hekat + 4 ro

f  448 ro as 1 1/4 1/8 1/64 hekat + 3 ro

g. 896 ro as 2 1/2 1/4 1/32 hekat + 1 ro

h. 1220 ro as 3 1/2 hekat

The above reports the shorthand data, Scholars have analyzed this data over the years.

Aspects of meta level were not reported until 2002. 

After 2002 the hekat unity (64/64) have been increasingly reported as divisions by n in terms of

(64/64)/n

a method reported in the Akhmim Wooden Tablet that set n = 3, 7, 10, 11 and 13.

What was n at every stage of the RMP 69 duplation proof?

Begin with 14 ro = 14/320 = 7/160

that meant that multiplier 7/160 was understood by Ahmes to be the divisor 160/7, or

a. n = 160/7

b. n = 320/7

c. n = 640/7

d. n = 1280/7

e. n = 2560/7

f n = 5120/7

g. n = 10240/7

h. n = 12800/7

CONCLUSION: Ahmes followed the 1825 BCE Akhmim Wooden Papyrus scribe and the Berlin Paprus scribal methods that later solved RMP 69. The meta level of ancient math can be seen as reported expected (theoretical) and actual (practical) calculations, details that are reported in RMP 69, and the Berlin Papyrus. Middle Kingdom Egyptian scribes also scaled the hekat to 320 ro units, 1/320th of a hekat.


%%%%%
%%%%%
\end{document}
