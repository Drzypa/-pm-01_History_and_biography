\documentclass[12pt]{article}
\usepackage{pmmeta}
\pmcanonicalname{OlgaLadyzhenskaya}
\pmcreated{2013-03-22 17:16:59}
\pmmodified{2013-03-22 17:16:59}
\pmowner{Mravinci}{12996}
\pmmodifier{Mravinci}{12996}
\pmtitle{Olga Ladyzhenskaya}
\pmrecord{5}{39624}
\pmprivacy{1}
\pmauthor{Mravinci}{12996}
\pmtype{Biography}
\pmcomment{trigger rebuild}
\pmclassification{msc}{01A61}
\pmclassification{msc}{01A60}
\pmsynonym{Olga Alexandrowna Ladyzhenskaya}{OlgaLadyzhenskaya}
\pmsynonym{Olga Alexandrovna Ladyzhenskaya}{OlgaLadyzhenskaya}
\pmsynonym{Ol'ga Alexandrowna Ladyzhenskaya}{OlgaLadyzhenskaya}
\pmsynonym{Ol'ga Alexandrovna Ladyzhenskaya}{OlgaLadyzhenskaya}

% this is the default PlanetMath preamble.  as your knowledge
% of TeX increases, you will probably want to edit this, but
% it should be fine as is for beginners.

% almost certainly you want these
\usepackage{amssymb}
\usepackage{amsmath}
\usepackage{amsfonts}

% used for TeXing text within eps files
%\usepackage{psfrag}
% need this for including graphics (\includegraphics)
%\usepackage{graphicx}
% for neatly defining theorems and propositions
%\usepackage{amsthm}
% making logically defined graphics
%%%\usepackage{xypic}

% there are many more packages, add them here as you need them

% define commands here

\begin{document}
\PMlinkescapeword{degree}
\PMlinkescapeword{even}
\PMlinkescapeword{union}

\emph{Olga Alexandrowna Ladyzhenskaya} (1922 - 2004) Russian mathematician, best known for her work on Hilbert's 19th problem and the Navier-Stokes equation.

Her father was Alexander Ivanovich Ladyzhenski, a high school math teacher who ignored warnings of a midnight arrest. Young Olga was able to finish high school but found many roadblocks on her way to earning a college degree. After Joseph Stalin died in 1953, Ladyzhenskaya presented her doctoral thesis and was given the degree she had long before earned. She went on to teach at the university in Leningrad and at the Steklov Institute, staying in Russia even after the collapse of the Soviet Union and the rapid salary deflation for professors.

In 2002, she was awarded the Lomonosov Gold Medal. Ladyzhenskaya has \PMlinkname{Erd\H{o}s number}{ErdHosNumber} 3: she co-authored a paper on measures for the Navier-Stokes equation with Anatoliuy Vershik in a Soviet journal, while Vershik co-authored a paper on random partitions of integers with Gregory Freiman, who with Erd\H{o}s wrote a paper ``On two additive problems'' in the {\it Journal of Number Theory}.
%%%%%
%%%%%
\end{document}
