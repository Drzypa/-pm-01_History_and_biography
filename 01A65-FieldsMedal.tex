\documentclass[12pt]{article}
\usepackage{pmmeta}
\pmcanonicalname{FieldsMedal}
\pmcreated{2013-03-22 16:11:54}
\pmmodified{2013-03-22 16:11:54}
\pmowner{Mravinci}{12996}
\pmmodifier{Mravinci}{12996}
\pmtitle{Fields medal}
\pmrecord{5}{38291}
\pmprivacy{1}
\pmauthor{Mravinci}{12996}
\pmtype{Definition}
\pmcomment{trigger rebuild}
\pmclassification{msc}{01A65}
\pmclassification{msc}{01A61}
\pmclassification{msc}{01A60}
\pmrelated{NoncommutativeGeometry}

% this is the default PlanetMath preamble.  as your knowledge
% of TeX increases, you will probably want to edit this, but
% it should be fine as is for beginners.

% almost certainly you want these
\usepackage{amssymb}
\usepackage{amsmath}
\usepackage{amsfonts}

% used for TeXing text within eps files
%\usepackage{psfrag}
% need this for including graphics (\includegraphics)
%\usepackage{graphicx}
% for neatly defining theorems and propositions
%\usepackage{amsthm}
% making logically defined graphics
%%%\usepackage{xypic}

% there are many more packages, add them here as you need them

% define commands here

\begin{document}
A prize awarded every four years to two, three, or four young mathematicians (under forty years of age) at each International Congress of the International Mathematical Union. The cash value of the award in 2006 was over $US\$13000$.

Canadian mathematician John Charles Fields founded the award in his testament, and the first recipients received the award in 1936. From 1950 the award has been periodic.

The Fields Medal is to mathematics what the Nobel prize in physics is to physics.
%%%%%
%%%%%
\end{document}
