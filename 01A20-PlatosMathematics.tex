\documentclass[12pt]{article}
\usepackage{pmmeta}
\pmcanonicalname{PlatosMathematics}
\pmcreated{2015-03-17 18:25:21}
\pmmodified{2015-03-17 18:25:21}
\pmowner{milogardner}{13112}
\pmmodifier{milogardner}{13112}
\pmtitle{Plato's mathematics}
\pmrecord{66}{41089}
\pmprivacy{1}
\pmauthor{milogardner}{13112}
\pmtype{Definition}
\pmcomment{trigger rebuild}
\pmclassification{msc}{01A20}
\pmsynonym{platonism}{PlatosMathematics}
\pmrelated{MathematicalPlatonism}
\pmrelated{FormalLogicsAndMetaMathematics}
\pmrelated{GWLeibnizSQuote}

% this is the default PlanetMath preamble.  as your knowledge
% of TeX increases, you will probably want to edit this, but
% it should be fine as is for beginners.

% almost certainly you want these
\usepackage{amssymb}
\usepackage{amsmath}
\usepackage{amsfonts}

% used for TeXing text within eps files
%\usepackage{psfrag}
% need this for including graphics (\includegraphics)
%\usepackage{graphicx}
% for neatly defining theorems and propositions
%\usepackage{amsthm}
% making logically defined graphics
%%%\usepackage{xypic}

% there are many more packages, add them here as you need them

% define commands here

\begin{document}
Plato's Mathematics (\PMlinkexternal{Platonism}{http://en.wikipedia.org/wiki/Philosophy_of_mathematics}) from Wikipedia

INTRODUCTION: Ancient math studies must parse available ancient texts as originally written. Concerning Plato's cave and Platonism texts adds back missing shorthand steps, must point out initial number theory facts, and include intermediate and final number theory calculations.

As background, it is well known that Plato's math ideas were influenced by Pythagoreans and Egyptian algebraic arithmetic. Greek and Egyptian number theory commonly scaled rational numbers to concise unit fraction series by multiplication in almost every quotient and remainder detail. For example,Greek arithmetic proofs defined division as an inverse multiplication operation, a relationship copied from Egyptian proofs.. 
    
In "The Republic" Plato placed Greek math building blocks in historical perspective. The term Platonism offers a parallel to Plato's belief in a World of Ideas typified by Allegory of the cave: the everyday world was imperfectly approximate to an unchanging, ultimate reality. To many modern scholars, Platonism suggests that ancient Greek mathematical used abstract entities with no causal properties. The ancient entities may have been eternal and unchanging to everyone, one view of modern Platonism pondered by modern scholars.

Modern Platonic paradigms have been claimed by historians that decode Classical Greek texts. The same is true for scholars that decode Hellene and medieval math era texts. That is, Classical Greek math texts offers a historical context that decodes abstract mathematics that may answer ultimate reality questions, like are abstract numbers eternal?

Another modern context of mathematical Platonism suggests where and how did the mathematical entities exist and how do we know about them in Classical Greece? 

For example, was the Greek world separate from our physical world which were occupied by the mathematical entities?  In other words, did the theoretical realm of numbers control Classical Greece thinkers including Plato? 

PLATO'S "REPUBLIC": How can any one gain access to two abstract worlds, one theoretical and one practical, and discover truths about math entities? One answer might ultimately ensemble a theory that postulates structures that exist mathematically also exist physicaly. To see aspects of the world of numbers through ancient eyes read Plato's Republic:

Plato spoke of the ancient mathematical world by asking in the "Republic", How do you mean?

"I mean, as I was saying, that arithmetic has a very great and elevating effect, compelling the soul to reason about abstract number, and rebelling against the introduction of visible or tangible objects into the argument. You know how steadily the masters of the art repel and ridicule any one who attempts to divide absolute unity when he is calculating, and if you divide, they multiply, taking care that one shall continue one and not become lost in fractions.

That is very true.

Now, suppose a person were to say to them: O my friends, what are these wonderful numbers about which you are reasoning, in which, as you say, there is a unity such as you demand, and each unit is equal, invariable, indivisible, --what would they answer? "

from Chapter 7. "The Republic" (Jowell translation).

In context, chapter 8, H.D.P. Lee translation, reports the education of a philosopher containing five mathematical disciplines:

1. arithmetic, written in unit fraction 'parts' using theoretical unities and abstract numbers;

2. plane geometry, and, 

3. solid geometry consider the line to be segmented into rational and irrational unit 'parts';

4. astronomy;

5. harmonics, that include music.

OTHER CONSIDERATIONS: Translators of Plato's works at various times rebelled against practical versions of classical practical mathematics. Plato himself and Greeks copied several Egyptian fraction abstract unity ideas at least 1,500 years older. 

For example, a hekat unity (64/64) was well defined by 1950 BCE as an initial fact. The hekat unity beginning point allowed divisions by 3, 7, 10, 11, 13  and multiplications by 1/3, 1/7, 1/10, 1/11 and 1/13 in the Akhmim Wooden Tablet. 

Ahmes 300 years later divided a larger hekat unity. Ahmes thought of 100 hekat written as (6400/64) allowed divisions by 10, 20, 30, 40, 50, 60, 70, 80, 90 and 100 in RMP 47 that were written as (6400)/64) times 1/10, 1/20, 1/30, 1/40, 1/50, 1/60, 1/70, 1/80, 1/90 and 1/100 in 10 separate problems. 

Ahmes used additional hekat unities:  320/320 and 320 ro in RMP 35-38. In RMP 38, hekat unity 320 ro was multiplied by 7/22, with the answer 101 + 9/11 ro multiplied by 22/7 that exactly returned 320 ro as a proof (ro = 1/320 of a hekat).

Ahmes defined another class of unity: 

53/53 = 2/53 + 3/53 + 5/53 + 15/53 + 28/53 

in RMP 36. 

The multiple hekat unity methods allowed \PMlinkexternal{Egyptian}{http://www.nytimes.com/2010/12/07/science/07first.html?_r=1&ref=science}, Greek, Arab and medieval scribes as late as \PMlinkexternal{Galileo}{http://www.ams.org/samplings/feature-column/fc-2013-05}  to not get lost in unit fraction calculations. 

Taking a long historical view, Egyptian, Greek, Arab and medieval weights and measures defined region-wide economic units within localized economic systems that spanned across 3,700 years of the Ancient East East. Middle Kingdom Egyptians used scaling methods, such as:

4/13 by LCM 4 to 16/52 = (13 + 2 + 1)/52 = 1/4 + 1/26 + 1/52. 

with 13 + 2 + 1 recorded in red.

Arabs, and Fibonacci scaled difficult rational numbers, like 4/13 by two LCMs, considering 

4/13 to 1/4, with remainder 3/52 scaled by 1/18 to obtain a final 

1/4 + 1/18 + 1/468 

series, thereby maintained the unit fraction system a few more years.  

The oldest unit fraction system formally began in 2050 BCE, and ended in Europe in 1454 AD when the Liber Abaci fell out of favor. The unit fraction system completely ended in 1637 AD when the Arab world introduced modern Arabic script, destroying linguistic connections to the very old economic system. 

Godel's platonism postulated a mathematical intuition that allowed perceptions of mathematical objects, but not the precise mathematical language that describes the object. This view resemblances things Husserl said about mathematics, and supports Kant's proposed idea that mathematics can be analytic-synthetic distinction: conceptual containment (synthetic), A priori, and a posteriori (philosophy). Philip J. Davis and Reuben Hersh suggest in 'The Mathematical Experience' that most mathematicians act as Platonists, even though, if pressed to defend the position carefully, they may retreat from this formalism taken from the philosophy of mathematics.

Mathematicians may infer opinions that amount to nuanced versions of Platonism. These ideas are best described as neo-platonism.

CONCLUSION Modern neo-platonic points of view provide unclear templates to decode ancient Greek and \PMlinkexternal{Egyptian arithmetic}{http://ahmespapyrus.blogspot.com/2009/01/ahmes-papyrus-new-and-old.html} texts. To decode Greek number theory, arithmetic, algebra, and geometry the Greek word multitude must be understood as ancient scribes reported finite least common multiples as a scaling idea. To Greek arithmetic that followed the older Egyptian arithmetic scaled rational numbers n/p by LCM m to mn/mp. The best divisors of mp were pondered in red numbers that summed to numerator mn in the RMP, Kahun Papyrus 2/n tables that reported 2-term, 3-term, 4-term and 5-term series To \PMlinkexternal{post-800 AD Arabs and medieval scribes}{http://planetmath.org/encyclopedia/ArabicNumerals.html} multitude m scaled rational numbers n/p by LCM m to (n/p - 1/m) = (mn - p)/mp, before setting (mn-p) = 1 whenever possible to 2-term series, and when not possible selected a second m that calculated 3-term series.   

\end{document}
