\documentclass[12pt]{article}
\usepackage{pmmeta}
\pmcanonicalname{Internet}
\pmcreated{2013-03-22 17:06:16}
\pmmodified{2013-03-22 17:06:16}
\pmowner{PrimeFan}{13766}
\pmmodifier{PrimeFan}{13766}
\pmtitle{Internet}
\pmrecord{5}{39402}
\pmprivacy{1}
\pmauthor{PrimeFan}{13766}
\pmtype{Definition}
\pmcomment{trigger rebuild}
\pmclassification{msc}{01A65}
\pmclassification{msc}{01A61}
\pmclassification{msc}{01A60}
\pmsynonym{information superhighway}{Internet}
\pmdefines{World Wide Web}

\endmetadata

% this is the default PlanetMath preamble.  as your knowledge
% of TeX increases, you will probably want to edit this, but
% it should be fine as is for beginners.

% almost certainly you want these
\usepackage{amssymb}
\usepackage{amsmath}
\usepackage{amsfonts}

% used for TeXing text within eps files
%\usepackage{psfrag}
% need this for including graphics (\includegraphics)
%\usepackage{graphicx}
% for neatly defining theorems and propositions
%\usepackage{amsthm}
% making logically defined graphics
%%%\usepackage{xypic}

% there are many more packages, add them here as you need them

% define commands here

\begin{document}
The {\em Internet} is a network of computer terminals and servers for accessing documents and media from the {\em World Wide Web}.

The Internet grew out of the radar network of the Defense Advanced Research Projects Agency of the United States. Collaboration between Harvard, MIT, DARPA and the National Science Foundation led to the creation of NSFNet in 1983, connecting several universities and government agencies. In the early 1990s, Tim Berners-Lee of CERN created the Hypertext Transfer Protocol (HTTP) and the Hypertext Markup Language (HTML). By the late 1990s, with media coverage of the {\em information superhighway}, the Internet became very popular. By the early 2000s, a large portion of the Internet was devoted to pornography, something which was spoofed in an episode of {\it The Simpsons} in which a video of Homer dancing around was described by Lenny as ``the number 1 non-porn download on the Internet, making you one trillionth overall.''

Even so, there are many important resources on the Internet for the sciences and for mathematics. The On-Line Encyclopedia of Integer Sequences, run by Neil Sloane, is a significant resource not only for number theory but for all mathematics. PlanetMath and MathWorld are significant encyclopedias of mathematical terms, and distributed computing projects such as the Great Internet Mersenne Prime Search and Seventeen or Bust have discovered increasingly larger numbers of specific kinds. Wolfram Research makes a version of Mathematica called webMathematica for the purpose of spicing up Web pages with ``interactive calculations.''
%%%%%
%%%%%
\end{document}
