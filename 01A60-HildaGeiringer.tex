\documentclass[12pt]{article}
\usepackage{pmmeta}
\pmcanonicalname{HildaGeiringer}
\pmcreated{2013-03-22 16:53:00}
\pmmodified{2013-03-22 16:53:00}
\pmowner{Mravinci}{12996}
\pmmodifier{Mravinci}{12996}
\pmtitle{Hilda Geiringer}
\pmrecord{5}{39137}
\pmprivacy{1}
\pmauthor{Mravinci}{12996}
\pmtype{Biography}
\pmcomment{trigger rebuild}
\pmclassification{msc}{01A60}
\pmclassification{msc}{01A55}
\pmsynonym{Hilda von Mises}{HildaGeiringer}
\pmsynonym{Hilda Geiringer von Mises}{HildaGeiringer}

\endmetadata

% this is the default PlanetMath preamble.  as your knowledge
% of TeX increases, you will probably want to edit this, but
% it should be fine as is for beginners.

% almost certainly you want these
\usepackage{amssymb}
\usepackage{amsmath}
\usepackage{amsfonts}

% used for TeXing text within eps files
%\usepackage{psfrag}
% need this for including graphics (\includegraphics)
%\usepackage{graphicx}
% for neatly defining theorems and propositions
%\usepackage{amsthm}
% making logically defined graphics
%%%\usepackage{xypic}

% there are many more packages, add them here as you need them

% define commands here

\begin{document}
\PMlinkescapeword{parents}
\PMlinkescapeword{even}
\PMlinkescapeword{child}
\PMlinkescapeword{interests}
\PMlinkescapeword{Ph}
\PMlinkescapeword{power}
\PMlinkescapeword{support}

\emph{Hilda Geiringer} sometimes referred to as {Hilda von Mises} (1893 - 1973) Austrian mathematician and educator who applied probabilistic methods to the study of genetics.

Born in Vienna of a Jewish Hungarian father and an Austrian mother, her parents helped her pay tuition at Vienna University. Even as a child, she displayed remarkable memory for numbers and formulas, which she is said to have inherited from her father. In 1917, she earned a Ph.D. for her thesis on Fourier series, after this she assisted Leon Lichtenstein, editor of a mathematics journal and one of her teachers ``in the most essential sense.'' Moving to Berlin in 1921, Geiringer married the electrical engineer Felix Pollaczek and her interests shifted to applied mathematics. She bore Felix a child, Magda, of whom she was given full custody after the divorce.

Around that time Geiringer started corresponding with Albert Einstein, who foreseeing the gathering threat of Nazism, hoped to get Geiringer to move to the United States. If Hitler hadn't come to power, it is speculated that Geiringer would've become a full professor at the Berlin University. While in Berlin, she met Richard Edler von Mises, whom she followed to Turkey when he was appointed professor of statistics. Though they lived in a German enclave there, they both had to learn to speak Turkish.

In 1938, Geiringer finally came to the United States, to teach at the Bryn Mawr Women's College and help in the scientific effort in support of the Allies in the war. After finally marrying von Mises in 1943, she went on to teach at Wheaton Women's College (where she was half the math faculty) while applying to more prestigious universities like Princeton and Yale and being passed over just because of her gender. Her husband began teaching at Harvard in 1944, and after his death in 1953, she was offered a position there but never a full professorship. She edited her husband's lectures into books.

In her spare time, Geiringer climbed mountains and went to orchestra concerts. She died in California.

\begin{thebibliography}{1}
\bibitem{jr} J. L. Richards ``Hilda Geiringer von Mises'' in {\it Women of Mathematics: A Bibliographic Sourcebook} L. Grinstein, P. Cambpell, ed.s New York: Greenwood Press (1987): 41 - 46
\end{thebibliography}
%%%%%
%%%%%
\end{document}
