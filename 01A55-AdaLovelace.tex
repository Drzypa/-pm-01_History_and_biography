\documentclass[12pt]{article}
\usepackage{pmmeta}
\pmcanonicalname{AdaLovelace}
\pmcreated{2013-03-22 17:15:45}
\pmmodified{2013-03-22 17:15:45}
\pmowner{Mravinci}{12996}
\pmmodifier{Mravinci}{12996}
\pmtitle{Ada Lovelace}
\pmrecord{5}{39600}
\pmprivacy{1}
\pmauthor{Mravinci}{12996}
\pmtype{Definition}
\pmcomment{trigger rebuild}
\pmclassification{msc}{01A55}
\pmsynonym{Augusta Ada Lovelace}{AdaLovelace}
\pmsynonym{Countess of Lovelace}{AdaLovelace}

\endmetadata

% this is the default PlanetMath preamble.  as your knowledge
% of TeX increases, you will probably want to edit this, but
% it should be fine as is for beginners.

% almost certainly you want these
\usepackage{amssymb}
\usepackage{amsmath}
\usepackage{amsfonts}

% used for TeXing text within eps files
%\usepackage{psfrag}
% need this for including graphics (\includegraphics)
%\usepackage{graphicx}
% for neatly defining theorems and propositions
%\usepackage{amsthm}
% making logically defined graphics
%%%\usepackage{xypic}

% there are many more packages, add them here as you need them

% define commands here

\begin{document}
Countess \emph{Augusta Ada Lovelace} n\'ee \emph{Augusta Ada Byron} (1815 - 1852) English mathematician, women's \PMlinkescapetext{rights} activist, novelist and playwright.

Born in England of the poet Lord Byron and housewife Anne Isabella Milbanke. Taught by Mary Somerville, Ada later worked with Charles Babbage and married William King, the 1st Earl of Lovelace. More than a hundred years after her death, the Ada programming language was named in her honor.

\begin{thebibliography}{1}
\bibitem{ak} K. D. Rappaport ``Augusta Ada Lovelace'' in {\it Women of Mathematics: A Bibliographic Sourcebook} L. Grinstein, P. Cambpell, ed.s New York: Greenwood Press (1987): 135 - 139
\end{thebibliography}
%%%%%
%%%%%
\end{document}
