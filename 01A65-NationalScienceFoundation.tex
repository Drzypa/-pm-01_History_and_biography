\documentclass[12pt]{article}
\usepackage{pmmeta}
\pmcanonicalname{NationalScienceFoundation}
\pmcreated{2013-03-22 16:47:16}
\pmmodified{2013-03-22 16:47:16}
\pmowner{PrimeFan}{13766}
\pmmodifier{PrimeFan}{13766}
\pmtitle{National Science Foundation}
\pmrecord{5}{39020}
\pmprivacy{1}
\pmauthor{PrimeFan}{13766}
\pmtype{Definition}
\pmcomment{trigger rebuild}
\pmclassification{msc}{01A65}
\pmclassification{msc}{01A61}
\pmclassification{msc}{01A60}
\pmsynonym{NSF}{NationalScienceFoundation}

\endmetadata

% this is the default PlanetMath preamble.  as your knowledge
% of TeX increases, you will probably want to edit this, but
% it should be fine as is for beginners.

% almost certainly you want these
\usepackage{amssymb}
\usepackage{amsmath}
\usepackage{amsfonts}

% used for TeXing text within eps files
%\usepackage{psfrag}
% need this for including graphics (\includegraphics)
%\usepackage{graphicx}
% for neatly defining theorems and propositions
%\usepackage{amsthm}
% making logically defined graphics
%%%\usepackage{xypic}

% there are many more packages, add them here as you need them

% define commands here

\begin{document}
\PMlinkescapeword{development}

The {\em National Science Foundation} (NSF for short) is an agency of the United States government which primarily promotes research in engineering and computer science (playing a pivotal r\^ole in the development of the Internet, for example), but also gives grants to professional mathematicians for research in pure and applied mathematics. The involvement of the NSF in mathematics educations has had its share of controversy, however.

\subsection{External links}
\PMlinkexternal{National Science Foundation Official Website}{http://www.nsf.gov/index.jsp}

\PMlinkexternal{NSF Mathematical \& Physical Sciences Research Directorate}{http://www.nsf.gov/dir/index.jsp?org=MPS}
%%%%%
%%%%%
\end{document}
