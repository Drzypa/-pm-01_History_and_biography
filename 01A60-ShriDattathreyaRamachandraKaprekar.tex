\documentclass[12pt]{article}
\usepackage{pmmeta}
\pmcanonicalname{ShriDattathreyaRamachandraKaprekar}
\pmcreated{2013-03-22 16:16:27}
\pmmodified{2013-03-22 16:16:27}
\pmowner{Mravinci}{12996}
\pmmodifier{Mravinci}{12996}
\pmtitle{Shri Dattathreya Ramachandra Kaprekar}
\pmrecord{4}{38385}
\pmprivacy{1}
\pmauthor{Mravinci}{12996}
\pmtype{Biography}
\pmcomment{trigger rebuild}
\pmclassification{msc}{01A60}
\pmsynonym{D. R. Kaprekar}{ShriDattathreyaRamachandraKaprekar}

\endmetadata

% this is the default PlanetMath preamble.  as your knowledge
% of TeX increases, you will probably want to edit this, but
% it should be fine as is for beginners.

% almost certainly you want these
\usepackage{amssymb}
\usepackage{amsmath}
\usepackage{amsfonts}

% used for TeXing text within eps files
%\usepackage{psfrag}
% need this for including graphics (\includegraphics)
%\usepackage{graphicx}
% for neatly defining theorems and propositions
%\usepackage{amsthm}
% making logically defined graphics
%%%\usepackage{xypic}

% there are many more packages, add them here as you need them

% define commands here

\begin{document}
{\em Shri Dattathreya Ramachandra Kaprekar} (1905 - 1986) Indian mathematician and mathematics educator mostly concerned with radix representation and other digital problems in recreational number theory. He is best known for being among the first (if not the first) to study Harshad numbers, self numbers, Kaprekar numbers and the Kaprekar constant. He also studied magic squares.

Kaprekar attended the university of his native Mumbai, receiving his bachelor's degree in 1929. From 1930 until his retirement in 1962, he worked as a teacher in Devlali.
%%%%%
%%%%%
\end{document}
