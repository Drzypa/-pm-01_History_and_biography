\documentclass[12pt]{article}
\usepackage{pmmeta}
\pmcanonicalname{EvelynNelson}
\pmcreated{2013-03-22 17:22:34}
\pmmodified{2013-03-22 17:22:34}
\pmowner{Mravinci}{12996}
\pmmodifier{Mravinci}{12996}
\pmtitle{Evelyn Nelson}
\pmrecord{5}{39741}
\pmprivacy{1}
\pmauthor{Mravinci}{12996}
\pmtype{Biography}
\pmcomment{trigger rebuild}
\pmclassification{msc}{01A60}
\pmsynonym{Evelyn Merle Nelson}{EvelynNelson}
\pmsynonym{Evelyn Merle Roden}{EvelynNelson}

\endmetadata

% this is the default PlanetMath preamble.  as your knowledge
% of TeX increases, you will probably want to edit this, but
% it should be fine as is for beginners.

% almost certainly you want these
\usepackage{amssymb}
\usepackage{amsmath}
\usepackage{amsfonts}

% used for TeXing text within eps files
%\usepackage{psfrag}
% need this for including graphics (\includegraphics)
%\usepackage{graphicx}
% for neatly defining theorems and propositions
%\usepackage{amsthm}
% making logically defined graphics
%%%\usepackage{xypic}

% there are many more packages, add them here as you need them

% define commands here

\begin{document}
\PMlinkescapeword{Ph}

\emph{Evelyn Merle Nelson} n\'ee \emph{Evelyn Merle Roden} (1943 - 1987) Canadian mathematician, best known for her work in theoretical computer science.

Born in Ontario of Russian immigrants, young Evelyn's education began at a high school in her native town and continued at the University of Toronto, then returned to her home town to study at McMaster University, where she began researching after earning her \PMlinkescapetext{Ph}.D. and later taught in the computer science department.

Nelson has \PMlinkname{Erd\H{o}s number}{ErdHosNumber} 2. With Saharon Shelah and Alan Mekler she published an article on ``A variety with solvable, but not uniformly solvable, word problem'' in the {\it Proceedings of the London Mathematical Society} of 1993, while Shelah co-authored with Erd\H{o}s a paper on the ``Separability properties of almost-disjoint families of sets'' in the {\it Israel Journal of Mathematics} {\bf 12}.

%%%%%
%%%%%
\end{document}
