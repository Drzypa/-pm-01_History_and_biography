\documentclass[12pt]{article}
\usepackage{pmmeta}
\pmcanonicalname{AkhmimWoodenTablet}
\pmcreated{2013-03-22 19:19:55}
\pmmodified{2013-03-22 19:19:55}
\pmowner{CompositeFan}{12809}
\pmmodifier{CompositeFan}{12809}
\pmtitle{Akhmim Wooden Tablet}
\pmrecord{5}{42278}
\pmprivacy{1}
\pmauthor{CompositeFan}{12809}
\pmtype{Definition}
\pmcomment{trigger rebuild}
\pmclassification{msc}{01A16}
\pmsynonym{Egyptian math}{AkhmimWoodenTablet}
%\pmkeywords{hekat unity}
\pmdefines{hekat division}

\endmetadata

% this is the default PlanetMath preamble.  as your knowledge
% of TeX increases, you will probably want to edit this, but
% it should be fine as is for beginners.

% almost certainly you want these
\usepackage{amssymb}
\usepackage{amsmath}
\usepackage{amsfonts}

% used for TeXing text within eps files
%\usepackage{psfrag}
% need this for including graphics (\includegraphics)
%\usepackage{graphicx}
% for neatly defining theorems and propositions
%\usepackage{amsthm}
% making logically defined graphics
%%%\usepackage{xypic}

% there are many more packages, add them here as you need them

% define commands here

\begin{document}
\section{Introduction}

The Akhmim Wooden Tablet may date to 2,000 BCE, 12th dynasty, or as late as 15th dynasty. The tablet is housed in the Cairo, Egypt Museum. It is 46.5 x 26n cm in size and mentions 27 servant names, an unknown king's name (citing the 8th year of his reign), five division calculations, one of which was repeated four times and five proofs. The document was reported in 1901 and analyzed and published in 1906 by Georges Daressy. Daressy indicated five divisions by 3, 7, 10, 11 and 13, and wrote out exact 1/p unit fraction series, and validated three of the five proofs.

Daressy discussed the AWT in terms of binary fractions and minimized aspects of the five Egyptian fraction series. Typos and other errors muddled the 1/11 and 1/13th multiplications. Exactness was not identified in the scribal proof for the 1/11 and 1/13 cases. Daressy cited the cubit-cubit rather than the hekat, the actual AWT context (Peet's main complaint).

However, Daressy's cubit view was consistent with a binary fraction remainder arithmetic, and exact partitions. Peet did not identify several scribal arithmetic facts. Daressy's approach properly analyzed AWT data that included binary fractions and scaled remainders.

A small number of scholars worked on the 1/10th of a hekat (a volume unit) named hin, hinu, or henu scaled a linear cubit to a cubit-cubit-cubit within a hekat unity (64/64) such that 1/320 of a hekat was named ro

(64/64)/10 hekat = (6/64 + 4/640)hekat =

(4 + 2)/64 hekat + 20/10 ro =

(1/16 + 1/32)hekat + 2ro =

1 hin

Ahmes in Rhind Mathematical Papyrus (RMP) 81 used 29 binary hekat quotients + scaled ro remainders by following the theoretical statement

(64/64)/n = Q/64 + (5R/n)ro

with Q = Quotient, R = Remainder, and

n limited to the range 1/64 < n < 64. The AWT and RMP two-part statements used one part statements. For example: 10/n hin simply meant a hekat was scaled to a 1/10 unit named hin was the limit scholarly discussions. Peet in 1923 incompletely discussed the AWT's binary fractions statements that conflicted with Daressy's earlier work. Peet reported 1/3, 1/7, 1/10, 1/11, 1/13 multiplication aspects of the problems and only stressed the 1/320 ro unit. Peet had not reported the meta (64/64)/n division that reported five binary quotient plus a 1/320 remainder answers and proofs. Peet under reported the scribal context and modern translation of the AWT n = 11 and 13 cases, correctly reported by: a. (64/64)/11 hekat = (5/64 + 9/704)hekat = (4 + 1)/64 hekat + (45/11)ro = (1/16 + 1/64)hekat + (4 + 1/11)ro b. (64/64)/13 hekat = (4/64 + 12/832)hekat = 4/64 hekat + (60/13)ro = 1/16 hekat + (4 + 8/13)ro with 8/13 scaled by LCM 2 to 16/26 = (13 + 2 + 1)/26 = 1/2 + 1/13 + 1/26 recorded as: (64/64)/13 hekat = 1/16 hekat + (4 + 1/2 + 1/13 + 1/26)ro It 80 years for Vymazalova to correctly report the proof context of the AWT arithmetic story. Hana Vymazalova reported the proof side of the five two-part statements that returned (64/64) five times when multiplied by the initial divisors. Vymazalova did not challenge Peet's calculation views of the five AWT binary quotient and 1/320 remainder answers, points that were corrected in 2006.

Daressy's 1906 review of the AWT's data garbled the n = 11 and n = 13 proofs thereby confusing Peet and later researchers. In 2002 by Hana Vymazalova corrected Daressy's two proof errors. Vymazalova's corrections and other meta points were published in 2006 and 2011.
The AWT reported a well-defined system of weights and measures arithmetic from an Old Kingdom inexact system to an exact Middle Kingdom rational number system.

The meta context in which Peet properly identified the 1/320 ro aspect of the AWT was missed until Hana Vyamazalova's 2002 paper. Scholars for almost 100 years did not connect the AWT partition method to RMP 81 and 29 data points, and over 30 additional two-part partitions of a (64/64) hekat unity discussed in the RMP.

\section{Contents of the tablet}

Translated to our modern base 10, the AWT simply states that unity (64/64th) was divided by 3, 7, 10, 11 and 13 following a general rule of division, as is clearly read by:

(64/64)/n = Q/64 + (5R/n)ro

with Q = quotient, and R = Remainder

writing the initial problem in modern base 10 fractions.

Middle Kingdom scribes used this pattern, by easily writing the quotient term into a Horus-Eye series, for example (64/64)/3 = 21/64 (Q). Several scholars have seen this portion, but become foggy with respect to the remainder, one (1) in the case of n = 3.

The 2nd portion, the handling of the remainder, has been grossly confused by scholars. One reason can be excused since R/(n*64) was 'encoded' by scribes replacing 1/64th with equivalent 5/320 remainders. This allowed student scribes to add Q and R values as one number.

To the average scholar seeing an Egyptian fraction series, actually (5R/n) followed by ro (long known to be 1/320) did not 'feel' like a remainder component. However, Ro, clearly used as a common divisor, can also have been seen by Ahmes as n LCM, or even a GCD. Whatever ro's meaning to Ahmes, its details was left to modern code breakers.

Only a few scholars opnenly 'scratched' their respective heads when seeing AWT and RMP two-part data. Two recent scholars: Robins-Shute saw (6400/64)/70 one of 10 RMP 47 problems. Robins-Shute did not report the two-part Q/64 + (5R/70)ro expression in a 1987 RMP book. Robins-Shute fairly reported products and remainders contained in the data, a form of scratching of their respective heads.

An earlier motivated scholar was Chace. In a 1927 RMP book, RMP 83 reported three data sets for (64/64) divided by n = 6, 20, and 40. The bird-feeding rates made little sense to him, suggesting that Ahmes had garbled the data. Chace mentioned that Ahmes left no clues to on this matter, a personal point that is obviously incorrect when reading the AWT in its broader context.

Factually, Ahmes and the AWT scribes created

Q/64 +(5R/n)(1/320)statements, over 60 times following the same theoretical style.

For none mathematicians reading, it may be best to refer to the modern base 10 version of the 4,000 year old arithmetic notation (shorthand), and then, say a few days later, actually attempt write in the ancient 2-part notation. That is, clearly think in our modern base 10 for a few days, before trying to think and write as a 4,000 year old scribe.

There are four trees being discussed here, each suggesting confusion, unless care is taken. They are: (1) the Horus-Eye notation, followed by a hekat, written in the first half of the expression, (2) Hieratic Egyptian fraction notion, followed by ro in the second half of the expression, (3) the Egyptian fraction series represented only the rational number (5*R/n), noting n in the denominator, allowing it to grow to any size, thereby allowing an exact computation, every time, and (4)the word ro,
as 1/320th was factored from the remainder term.

Clearly ro was a minor term, in the expression, possibly only a common divisor factor, used to add the Horus-Eye and Egyptian fraction series together, allowing a proof to be quickly performed, as was listed five times in the AWT (and not at all in the RMP).

It is therefore recommended that novice readers of this blog do not allow one or two of the different types of trees that you run into, to be confusing. Look for the forest of each type of notation, stated as clearly as your education allows. Then and only then try to read and work with the inner workings of the AWT, such as an individual tree. The work on a complete (64/64)/n division problem as the 2,000 BC student scribe was trained.

Have patience, each of the ideas are simple, seen separately. Taking in the set of anjcient ideas at once causes problems for many people. Please avoid shortcuts, especially the ones that Ahmes himself practiced, until you discover the foundations of Ahmes' arithmetic, such as the ancient methods that created 2/n tables.

Peet, Gillings and other may have taken a couple of modern shortcuts, missing an ancient tree or two.

Begin at the beginning of each of the AWT problems, and compare your beginning, middle and end work with the same type of problem written out in the RMP (\#47, 81, 83 are the best examples). Then work to the end of each AWT problem, doing your own work, every step of the way.

You will be rewarded. Spend the necessary time to work through more than one ancient problem as scribes solved it, using all of the old tools.

\section{Background}

Thomas E. Peet, 1923, partially repeated an analysis of the AWT by showing several connections between Egyptian math and the practical experiences of an ancient Egyptian scribe that used two numeration systems, Horus-Eye and the Egyptian fractions (cited in the RMP). Peet muddled where one system ended and where the other system began by only detailing additive aspects of the two numeration systems, missing the exact Egyptian division features using 5ro as a partitioning idea, using numerators and denominators = 320 in an interesting way. Peet also prematurely concluded that the Egyptian division was only an inverse of the Egyptian multiplication operation.

Peet did not directly discuss Egyptian division, as a general operation, as confirmed by the AWT examples. However, contrary to Gillings and Robins-Shute, Peet did seem to compute with 5ro, 4ro, 3ro, 2 ro and ro, but only from a limited view of the AWT student. Peet was slightly myopic, asking few meta questions, such as: were all of the student's divisions required to be exact? More importantly, no comparisons of Peet's view of the AWT were made to the RMP and its 84 problems. At least ten RMP problems, 36-43, and 81-82, have been misread with respect to ro, suggesting it was a weights and measures unit. Ro was actually connected to a generalized partitioning role, as closely related to other exact partitioning methods cited in the RMP, and other Middle Kingdom mathematical texts.

Scholars are, of course, free to explore these issues on their own, commenting on the actual mix of Egyptian mathematics that meets a few of the standards that are deduced from the AWT, and its interesting set of division methodologies.

Peet, and two later scholars, Gillings, 1972, and Robin-Shute, 1987, show that all the three scholars prematurely concluded in independent analyses that MK ro data (from the RMP and AWT) only meant 1/320 of a hekat, and no more. Not one of the three scholars grasped Ahmes central fact, that quotients and scaled ro remainders defined a weights and measures unit by beginning with a hekat unity (64/64), and dividing by any rational number less than 64.

Returning to Peet, and his analysis will be shown in the next few paragraphs. He apparently made serious errors with respect to ro and its relationship to Egyptian division, as was vividly declared in the AWT hekat and 1/64 divided by 1/n context. Peet did not see ro's actual association with remainders, though he mentioned remainders from time to time. It is clear that 64 times 5, an early form of mod 5 in the R/3 term, included the use of numerators and denominators, or, by example, let the divisor of 64/64 be n, then

Q + R/3

appeared in two shorthand forms, the first being

(1) (64/64)/n = Q/64 + R/(n*64).

(2) (64/64)/n = Q/64 + (5*R/n)* 1/320, with ro = 1/320

Ahmes used the second form. The first form is the manner in which modern mathematicians read this type of information.

One of the most exciting aspects of the AWT is that the rational number (5*R/n) was easily converted to an Egyptian fraction series. It not known if this was the
first generalized used of Egyptian fractions.

It is plausible that Silverman's 1975 point, that Egyptian fractions were found in the Old Kingdom, may mean that the balance beam problem was solved by an Old Kingdom scribe, noting:

R/(n*64)

with the Egyptian fractions series being either R/n or R/(n*64).

Research is continuing in a broader context in which Egyptian remainder arithmetic was found. Thanks to the scribes for leaving red flags raised by the AWT, and other clues, so that the simple five division problems of the AWT are fully decoded.

Early on, Daressy in 1906 discussed the Egyptian fraction series as remainder based. He saw the exact divisions of some unit, which he labeled in cubit units. Adding in a broader view of binary fractions to the discussion, going beyond Daressy's analysis, a clue to the Old Kingdom manner of partitioning a cubit, hekat, or any unit by remainder arithmetic has been provided.

In addition a required meta view of the 2,000 BCE Egyptian economy adds absentee landlords and Pharaoh using the MK finite Egyptian fraction units.


%%%%%
%%%%%
\end{document}
