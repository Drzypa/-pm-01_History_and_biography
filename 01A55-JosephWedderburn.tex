\documentclass[12pt]{article}
\usepackage{pmmeta}
\pmcanonicalname{JosephWedderburn}
\pmcreated{2013-03-22 16:17:05}
\pmmodified{2013-03-22 16:17:05}
\pmowner{Mravinci}{12996}
\pmmodifier{Mravinci}{12996}
\pmtitle{Joseph Wedderburn}
\pmrecord{4}{38399}
\pmprivacy{1}
\pmauthor{Mravinci}{12996}
\pmtype{Biography}
\pmcomment{trigger rebuild}
\pmclassification{msc}{01A55}
\pmclassification{msc}{01A60}
\pmsynonym{Joseph Henry Maclagen Wedderburn}{JosephWedderburn}
\pmsynonym{Joe Wedderburn}{JosephWedderburn}
\pmrelated{WedderburnsTheorem}

\endmetadata

% this is the default PlanetMath preamble.  as your knowledge
% of TeX increases, you will probably want to edit this, but
% it should be fine as is for beginners.

% almost certainly you want these
\usepackage{amssymb}
\usepackage{amsmath}
\usepackage{amsfonts}

% used for TeXing text within eps files
%\usepackage{psfrag}
% need this for including graphics (\includegraphics)
%\usepackage{graphicx}
% for neatly defining theorems and propositions
%\usepackage{amsthm}
% making logically defined graphics
%%%\usepackage{xypic}

% there are many more packages, add them here as you need them

% define commands here

\begin{document}
\emph{Joseph Henry Maclagen Wedderburn} (1882 - 1948) Scottish mathematician, perhaps best known for the Wedderburn-Artin theorem and his paper on hypercomplex numbers. He also worked on graph theory and the Wedderburn-Etherington numbers are named after him.

Trained mainly in Scotland, Wedderburn also studied briefly in Germany and America, where he worked with Leonard Dickson. Wedderburn began teaching at Princeton University before World War I, during which time he served in the British Army. After the war, he resumed teaching at Princeton and began editing {\it Annals of Mathematics}.


%%%%%
%%%%%
\end{document}
