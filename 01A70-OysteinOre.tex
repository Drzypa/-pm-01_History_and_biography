\documentclass[12pt]{article}
\usepackage{pmmeta}
\pmcanonicalname{OysteinOre}
\pmcreated{2013-03-22 15:59:43}
\pmmodified{2013-03-22 15:59:43}
\pmowner{Mravinci}{12996}
\pmmodifier{Mravinci}{12996}
\pmtitle{{\O}ystein Ore}
\pmrecord{9}{38020}
\pmprivacy{1}
\pmauthor{Mravinci}{12996}
\pmtype{Definition}
\pmcomment{trigger rebuild}
\pmclassification{msc}{01A70}
\pmclassification{msc}{01A60}
\pmsynonym{Oystein Ore}{OysteinOre}

% this is the default PlanetMath preamble.  as your knowledge
% of TeX increases, you will probably want to edit this, but
% it should be fine as is for beginners.

% almost certainly you want these
\usepackage{amssymb}
\usepackage{amsmath}
\usepackage{amsfonts}

% used for TeXing text within eps files
%\usepackage{psfrag}
% need this for including graphics (\includegraphics)
%\usepackage{graphicx}
% for neatly defining theorems and propositions
%\usepackage{amsthm}
% making logically defined graphics
%%%\usepackage{xypic}

% there are many more packages, add them here as you need them

% define commands here

\begin{document}
Norwegian mathematician (7 October 1899 - 13 August 1968).

A graduate of Oslo University, he later taught at Yale. He mostly worked on combinatorics and ring theory.

Among the concepts named after him are Ore numbers and Ore's theorem. Dover continues to republish his most famous book, {\it Number Theory and its History}, McGraw-Hill, 1948. Ore's writing style energetically discusses the central problems and solutions in the history of number theory, with Gaussian congruences playing a central role in the 1800's. Earlier, Ore details astronomical problems and solutions from India (Brahmagupta) and China related to the Chinese Remainder Theorem, and the indeterminate equations used by Diophantus and Fibonacci. 
%%%%%
%%%%%
\end{document}
