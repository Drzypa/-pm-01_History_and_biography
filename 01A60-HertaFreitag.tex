\documentclass[12pt]{article}
\usepackage{pmmeta}
\pmcanonicalname{HertaFreitag}
\pmcreated{2013-03-22 16:51:51}
\pmmodified{2013-03-22 16:51:51}
\pmowner{PrimeFan}{13766}
\pmmodifier{PrimeFan}{13766}
\pmtitle{Herta Freitag}
\pmrecord{5}{39113}
\pmprivacy{1}
\pmauthor{PrimeFan}{13766}
\pmtype{Biography}
\pmcomment{trigger rebuild}
\pmclassification{msc}{01A60}
\pmsynonym{Herta Tau\ss{}ig Freitag}{HertaFreitag}
\pmsynonym{Herta Taussig Freitag}{HertaFreitag}

% this is the default PlanetMath preamble.  as your knowledge
% of TeX increases, you will probably want to edit this, but
% it should be fine as is for beginners.

% almost certainly you want these
\usepackage{amssymb}
\usepackage{amsmath}
\usepackage{amsfonts}

% used for TeXing text within eps files
%\usepackage{psfrag}
% need this for including graphics (\includegraphics)
%\usepackage{graphicx}
% for neatly defining theorems and propositions
%\usepackage{amsthm}
% making logically defined graphics
%%%\usepackage{xypic}

% there are many more packages, add them here as you need them

% define commands here

\begin{document}
\emph{Herta Taußig Freitag} (1908 - 2000) Austrian mathematician, best known for her work on the Fibonacci numbers.

Freitag has \PMlinkname{Erd\H{o}s number}{ErdHosNumber} 3. With Daniel Fielder she published an article in {\it Applications of Fibonacci Numbers} {\bf 8}, while Fielder co-authored with Hoggatt an article in {\it Fibonacci Quarterly} {\bf 11}, Hoggatt in turn wrote with Erd\H{o}s an article ``On additive partitions of integers'' in {\it Discrete Mathematics} {\bf 22}.
%%%%%
%%%%%
\end{document}
