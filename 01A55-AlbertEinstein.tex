\documentclass[12pt]{article}
\usepackage{pmmeta}
\pmcanonicalname{AlbertEinstein}
\pmcreated{2013-03-22 16:38:38}
\pmmodified{2013-03-22 16:38:38}
\pmowner{PrimeFan}{13766}
\pmmodifier{PrimeFan}{13766}
\pmtitle{Albert Einstein}
\pmrecord{96}{38847}
\pmprivacy{1}
\pmauthor{PrimeFan}{13766}
\pmtype{Biography}
\pmcomment{trigger rebuild}
\pmclassification{msc}{01A55}
\pmclassification{msc}{01A60}
%\pmkeywords{general relativity}
%\pmkeywords{spacetime}
%\pmkeywords{special relativity}
%\pmkeywords{biography}
%\pmkeywords{A-bombs}
%\pmkeywords{first letter to FDR}
%\pmkeywords{nuclear fission chain reaction}
%\pmkeywords{nuclear fission bombs}
%\pmkeywords{high-energy physics}
%\pmkeywords{photoelectric effect theory}
%\pmkeywords{Noble Laureate}
%\pmkeywords{gravitational fields}
%\pmkeywords{cosmological const}
\pmrelated{AlexanderGrothendieck}
\pmdefines{Einstein's fundamental energy equation E= mc^2}

% this is the default PlanetMath preamble.  as your knowledge
% of TeX increases, you will probably want to edit this, but
% it should be fine as is for beginners.

% almost certainly you want these
\usepackage{amssymb}
\usepackage{amsmath}
\usepackage{amsfonts}

% used for TeXing text within eps files
%\usepackage{psfrag}
% need this for including graphics (\includegraphics)
%\usepackage{graphicx}
% for neatly defining theorems and propositions
%\usepackage{amsthm}
% making logically defined graphics
%%%\usepackage{xypic}

% there are many more packages, add them here as you need them

% define commands here

\begin{document}
\emph{Albert Einstein} (1879--1955) was a German-born physicist (of German -Jewish parents)--with both Swiss and German citizenships until 1932.  Following his taking up permanent residence as a Professor at Princeton, USA, in 1933 he became a top American physicist best known for the special and general theories of relativity; he became a naturalized US citizen in 1940. He also reported the first correct quantum interpretation of the photoelectric effect, for which he was awarded the 1921 Nobel Prize in Physics. He published a total of about 450 physics articles, including also several books. 

His equation \, $$E = mc^2$$\ that relates the energy $E$ of a (quantum) particle to its mass, $m$, and the speed of light $c$, (also called the ``mass-energy equivalence'') is unchallenged even today as a fundamental equation in quantum theory and mathematical physics. Regretably, however, even according to Einstein himself, the latter has also lead towards the end of WWII to the successful development, testing at Alamogordo, and the deployment of the first (so-called) `atom' bombs (or A-bombs, that are in fact \textbf{nuclear fission} bombs). Thus, the famous Princeton Professor Albert Einstein--at the strong prompting and insistence of his close nuclear physicist friend Dr.Leo Szilard (a Hungarian- Jewish refug\'ee from Horthy's Nazis in Hungary)-- wrote a two-page (first) letter to the 
\PMlinkexternal{thirty-second US President Franklin D. Roosevelt}{http://en.wikipedia.org/wiki/Franklin_D._Roosevelt}, that initiated the  fatal `chain reaction' leading to the design, construction and testing of the first A-bombs, and many more afterwards, including the Hydrogen bomb. His fundamental energy equation $E = mc^2$ is, and was, the basis for the huge energy release calculated for the nuclear fission chain reaction in which a relatively small change in mass of the nuclear `explosive' is `converted' extremely rapidly, and thus, explosively into a very large amount of energy in the form of a huge number of gamma rays, X-rays, photons and infrared radiation that raises local temperatures to peak values in excess of a few tens of million degrees C for an A-bomb, and on the order of a hundred million degrees C in the case of an H-bomb (`Hydrogen' bomb, that in fact also employs an A-bomb to detonate). 

 Most unfortunately, a powerful enough H-bomb can, in principle,  `burn up' the entire atmosphere of our planet Earth.
 
 The `benefits' of the A- and H- bomb development to the high-energy physics and mathematical physics community have been during the second half of the last century in the form of huge, ever increasing amounts of funding available for ever more powerful particle accelerators and `fundamental physics and mathematics research' during the Cold War, thus including advanced mathematics relevant to quantum physics.   

 Einstein was a strong pacifist and he was not himself involved in any way in the direct development of any A-bomb; however, other top US mathematical physicists such as notably Richard Feynman, J. Wheeler, and Oppenheimer (who was officially placed in charge of organizing the scientific side of the Manhattan project by former President Roosevelt) made possible the design and construction of the A-bomb believing, they said, that it would never be dropped on any human population during or after WWII. 

\textbf{Notes:}

1. A fact little known is that the unreported first experimental findings of nuclear fission were made by the Curies in France before WWII, but that they deliberately refrained on moral grounds from publishing their observations in the hope of avoiding the design and development of nuclear fission weapons or bombs by either the French or the Germans for use in the upcoming war. The first reports of nuclear fission observations were however published by Einstein's old friend Otto Hahn at the Kaiser Wilhelm Institute in Berlin during the last weeks of 1938. The interpretation of Hahn's nuclear fission observations was then published early in 1939 by Lise Meitner and her nephew Otto Frisch.

2. \PMlinkexternal{Max (Karl Ernst Ludwig) Planck}{http://nobelprize.org/nobel_prizes/physics/laureates/1918/planck-bio.html} himself considered the possibility of the `latent energy of the atom', also based on
the interpretation of Einstein's equation $E = mc^2$, and said in 1908 that ``though the actual production of such a
`radical' process might have appeared extremely small only a decade ago, it is now in the range of the possible...''
In spite of this hypothetical, theoretical possibility, Einstein did not seem to have considered the practical possibility of an A-bomb before 1936. 

3. Dr. Leo Szilard filed in the spring of 1934 a patent application that was approved which described the laws
governing a nuclear chain reaction and the design of a (hypothetical) nuclear fission reactor. He assigned his patent
to the British Admiralty of Great Britain because at that time a patent could be kept secret in Britain only if it was assigned to the British government. Although he approached in 1934 both the British War Office and the Admiralty, neither were interested at that time in following through with the possible military applications of Szilard's patent. The opposite happened however in the USA in 1939, following Einstein's first letter to President Franklin D. Roosevelt (inspired by Dr. Szilard) which is reproduced below, and more importantly, at the insistence of the British war allies. Einstein signed and arranged for the letter to be presented to FDR in spite of previous warnings from Max Born not to get involved in war work of this nature; much too late in 1945, Einstein, as well as Szilard, regreted his action: \emph{``I made one great mistake in my life--when I signed the letter to president Roosevelt recommending that atom bombs be made''} (reportedly to have been said to Nobel Laureate Linus Pauling, and also repeated in a short filmed (B/W) interview with Einstein, re-played several times on TV in the USA). 

 However, the first nuclear fission reactor was built and operated by Enrico Fermi's team in Chicago in 1942 as part of the Manhattan project; claims were only recently made that the Japanese may have also built working nuclear fission reactors for military purposes both in Tokyo and occupied Korea towards the end of WWII. Certain sources provide
documentary evidence that the Manhattan project began in earnest only when the British allies became convinced of the
practical possibility of making an A-bomb, at least in part as a result of the interpretation and somewhat over--optimistic computations of Otto Frisch in Great Britain in 1936-1939. Thus, Einstein's first letter has been claimed
to have had only a lukewarm reception by members of FDR's administration until the British government sudden revival
of interest in having the A-bomb built in the USA, as it was not contemplated to have it dropped in Europe if it were
developed.  

 \emph{Princeton University Professor Albert Einstein's first letter to President Franklin D. Roosevelt:}
(\PMlinkexternal{photocopy available on line through this weblink}{http://hypertextbook.com/eworld/einstein.shtml#first})


"Albert Einstein

 Old Grove Rd.

 Nassau Point

 Peconic, Long Island


                                                         

                                                       August 2nd 1939




F.D. Roosevelt

 President of the United States

 White House

 Washington, D.C.



Sir:

      Some recent work by E.Fermi and L. Szilard, which has been com-

municated to me in manuscript, leads me to expect that the element uran-

ium may be turned into a new and important source of energy in the im-

mediate future. Certain aspects of the situation which has arisen seem

to call for watchfulness and, if necessary, quick action on the part

of the Administration. I believe therefore that it is my duty to bring

to your attention the following facts and recommendations:

      In the course of the last four months it has been made probable -

through the work of Joliot in France as well as Fermi and Szilard in

America - that it may become possible to set up a nuclear chain reaction

in a large mass of uranium,by which vast amounts of power and large quant-

ities of new radium-like elements would be generated. Now it appears

almost certain that this could be achieved in the immediate future.

      This new phenomenon would also lead to the construction of bombs,

and it is conceivable - though much less certain - that extremely power-

ful bombs of a new type may thus be constructed. A single bomb of this

type, carried by boat and exploded in a port, might very well destroy

the whole port together with some of the surrounding territory. However,

such bombs might very well prove to be too heavy for transportation by

air. 

 

                                 -2-

      The United States has only very poor ores of uranium in moderate

quantities. There is some good ore in Canada and the former Czechoslovakia.

while the most important source of uranium is Belgian Congo.

      In view of the situation you may think it desirable to have more

permanent contact maintained between the Administration and the group

of physicists  working on chain reactions in America. One possible way

of achieving this might be for you to entrust with this task a person

who has your confidence and who could perhaps serve in an inofficial

capacity. His task might comprise the following:

      a) to approach Government Departments, keep them informed of the

further development, and put forward recommendations for Government action,

giving particular attention to the problem of securing a supply of uran-

ium ore for the United States;

      b) to speed up the experimental work,which is at present being car-

ried on within the limits of the budgets of University laboratories, by

providing funds, if such funds be required, through his contacts with y

private persons who are willing to make contributions for this cause,

and perhaps also by obtaining the co-operation of industrial laboratories

which have the necessary equipment.

      I understand that Germany has actually stopped the sale of uranium

from the Czechoslovakian mines which she has taken over. That she should

have taken such early action might perhaps be understood on the ground

that the son of the German Under-Secretary of State, von Weizsäcker, is

attached to the Kaiser-Wilhelm-Institut in Berlin where some of the

American work on uranium is now being repeated.

                                            Yours very truly,
                                             
                                            (Albert Einstein) "
 

Einstein's work on special relativity theory was published only shortly after that of Poincar\'e, albeit in a complete form, unlike Poincar\'e 's publication that was incomplete. His published reports and book on General Relativity (GR) theory surpassed special relativity in Minkowsky 4D spacetime but may not be conceptually consistent in Einstein's formulations with standard quantum mechanics, as pointed out by Einstein himself who considered quantum mechanics of his days to be an `incomplete', and thus, a transient theory. Subsequent developments in quantum physics and elementary particles/high-energy physics seem to have however disproved the Einstein's viewpoint of quantum mechanics. 

After Marcel Grossmann presented to Einstein the advantages of Riemannian geometry for the formulation of General Relativity, he began to ponder on the geometry of time in \PMlinkescapetext{relation} to space.\, Einstein's innovations have had great influence not only on physics but also on mathematics, and many mathematicians have pondered on the mathematical implications of Einstein's work. Einstein died on April 18, 1955, still trying to find an unified field theory; nowadays, numerous theoretical and mathematical attempts are still being made at consolidating quantum field theories (QFT) with General Relativity into a  single quantum gravity theory, or a theory of eveything (TOE).

Einstein has \PMlinkname{Erd\H{o}s number}{ErdHosNumber} 2. With Ernst Straus he published an article on gravitational fields for a physics journal, and Straus co-authored an article on mass formulas and mass inequalities with Coleman and Daniel Kleitman. Kleitman co-authored with Erd\H{o}s a paper ``On linear independence of sequences in a Banach space'' in the {\it Pacific Journal of Mathematics} in 1953.

One can find certain similarities between Albert Einstein-- a top theoretical/mathematical physicist-- and Alexander
Grothendieck, a top mathematician. Both are German born, and of a German-Jewish mother; both suffered because of the Nazis. Both have exhibited the ability of an entirely original, and very creative thinking, as well as the ability to
create new paradigms in fundamental science. Interestingly, neither of the two liked wearing socks. Much more significantly, both scientists were determined pacifists and also idealists that had a major influence on modern culture. Unlike Grothendieck, however, Einstein signed the fateful letter that marked his (later regreted) involvement in the sequence of events which initiated the A-bomb project in the US. 

Perhaps, a moral to be derived from Albert Einstein's own experience with the fundamental equation $E= mc^2$, and the idea on the A-bomb based on it, is that fundamental findings in mathematics and mathematical physics can have very profound effects on the entire world, and that such effects can be either very good or very bad, depending on how such fundamental results are put into practice, and to what ends they are being used. `Platonic' results that may appear quite innocent and remote from the `real' world can have indeed great impact on the latter.



\begin{thebibliography}{99}

\bibitem{AE05}
A. Einstein. On the Electrodynamics of Moving Bodies, Annalen der Physik, 17:891, June 30, 1905 (English translation by W. Perrett and G.B. Jeffery); also reprint in Hendrik Antoon Lorentz, Albert Einstein, H. Minkowski, Hermann Weyl (1952). \emph{The Principle of Relativity}. Courier Dover Publications. 

\bibitem{WR77}
Wolfgang Rindler (1977). Essential Relativity. Birkh\''auser. 
 
\bibitem{AE2k1}
Albert Einstein (2001). Relativity: The Special and the General Theory (Reprint of 1920 translation by Robert W. Lawson ed.). Routledge. 

\bibitem{RF98}
Richard Phillips Feynman (1998). Six Not-so-easy Pieces: Einstein's relativity, symmetry, and space-time (Reprint of 1995 edition ed.). Basic Books.

\bibitem{AE49}
A. Einstein, Autobiographical Notes, 1949.
 
\bibitem{AELMWH52}
A. Einstein, Lorentz, H. A., Minkowski, H., and Weyl, H. (1952). The Principle of Relativity: a collection of original memoirs on the special and general theory of relativity. Courier Dover Publications.
\bibitem{AE1907-21}
A. Einstein's fundamental papers:
``On the Relativity Principle and the Conclusions Drawn from It'', 1907; 
``The Principle of Relativity and Its Consequences in Modern Physics, 1910; 
``The Theory of Relativity'', 1911; 
``Manuscript on the Special Theory of Relativity'', 1912; 
``Theory of Relativity", 1913; 
``Einstein, Relativity, the Special and General Theory'', 1916; 
``The Principle Ideas of the Theory of Relativity'', 1916; 
``What Is The Theory of Relativity ?'', 1919; 
``Fundamental Ideas and Methods of the Theory of Relativity'', 1920) 
``The Principle of Relativity'' (Princeton Lectures), 1921;
``Physics and Reality'', 1936; 
``The Theory of Relativity'', 1949. 

\end{thebibliography}
%%%%%
%%%%%
\end{document}
