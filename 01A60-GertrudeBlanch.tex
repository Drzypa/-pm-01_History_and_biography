\documentclass[12pt]{article}
\usepackage{pmmeta}
\pmcanonicalname{GertrudeBlanch}
\pmcreated{2013-03-22 16:41:50}
\pmmodified{2013-03-22 16:41:50}
\pmowner{Mravinci}{12996}
\pmmodifier{Mravinci}{12996}
\pmtitle{Gertrude Blanch}
\pmrecord{5}{38910}
\pmprivacy{1}
\pmauthor{Mravinci}{12996}
\pmtype{Biography}
\pmcomment{trigger rebuild}
\pmclassification{msc}{01A60}
\pmsynonym{Gittel Kaimowitz}{GertrudeBlanch}

% this is the default PlanetMath preamble.  as your knowledge
% of TeX increases, you will probably want to edit this, but
% it should be fine as is for beginners.

% almost certainly you want these
\usepackage{amssymb}
\usepackage{amsmath}
\usepackage{amsfonts}

% used for TeXing text within eps files
%\usepackage{psfrag}
% need this for including graphics (\includegraphics)
%\usepackage{graphicx}
% for neatly defining theorems and propositions
%\usepackage{amsthm}
% making logically defined graphics
%%%\usepackage{xypic}

% there are many more packages, add them here as you need them

% define commands here

\begin{document}
\emph{Gertrude Blanch} n\'ee \emph{Gittel Kaimowitz} (1897? - 1996) Polish-born American mathematician and educator, considered one of the pioneers of numerical analysis and computation.

When she was barely 10-years-old, her family moved to the United States. She attended a high school in New York and worked full-time in order to afford to go to college. In the 1930s she earned degrees from New York University and Cornell University in mathematics and phsyics. In 1938 she was director of the Mathematical Tables Project which computed ballistics for the Army and the Navy. The FBI, however, questioned her patriotism and fingered her as a communist. A hearing cleared her name.

After retirement in 1967, Blanch devoted the rest of her life to studying the Matthieu functions by means of continued fractions.
%%%%%
%%%%%
\end{document}
