\documentclass[12pt]{article}
\usepackage{pmmeta}
\pmcanonicalname{vStefanZnam}
\pmcreated{2013-03-22 15:59:41}
\pmmodified{2013-03-22 15:59:41}
\pmowner{Mravinci}{12996}
\pmmodifier{Mravinci}{12996}
\pmtitle{\v{S}tefan Zn\'am}
\pmrecord{6}{38019}
\pmprivacy{1}
\pmauthor{Mravinci}{12996}
\pmtype{Definition}
\pmcomment{trigger rebuild}
\pmclassification{msc}{01A70}
\pmclassification{msc}{01A60}
\pmsynonym{Stefan Zn\'am}{vStefanZnam}
\pmsynonym{Stefan Znam}{vStefanZnam}

\endmetadata

% this is the default PlanetMath preamble.  as your knowledge
% of TeX increases, you will probably want to edit this, but
% it should be fine as is for beginners.

% almost certainly you want these
\usepackage{amssymb}
\usepackage{amsmath}
\usepackage{amsfonts}

% used for TeXing text within eps files
%\usepackage{psfrag}
% need this for including graphics (\includegraphics)
%\usepackage{graphicx}
% for neatly defining theorems and propositions
%\usepackage{amsthm}
% making logically defined graphics
%%%\usepackage{xypic}

% there are many more packages, add them here as you need them

% define commands here

\begin{document}
Slovak mathematician and mathematics professor, (February 9, 1936 - July 17, 1993), posed Zn\'am's problem.

His Erdős number is 2 by at least six different paths. For example, in 1968, Znám wrote a paper on "Strongly geodetic graphs" with Juraj Bosák in the {\it Journal of Combinatorial Theory}. Three years later, Bosák co-authored "Decompositions of complete graphs into factors with diameter two" with Erdős and Alexander Rosa in {\it Mathematics Casopis Slovene Akademie Vied}.
%%%%%
%%%%%
\end{document}
