\documentclass[12pt]{article}
\usepackage{pmmeta}
\pmcanonicalname{PierreDeFermat}
\pmcreated{2013-03-22 16:54:07}
\pmmodified{2013-03-22 16:54:07}
\pmowner{PrimeFan}{13766}
\pmmodifier{PrimeFan}{13766}
\pmtitle{Pierre de Fermat}
\pmrecord{4}{39159}
\pmprivacy{1}
\pmauthor{PrimeFan}{13766}
\pmtype{Biography}
\pmcomment{trigger rebuild}
\pmclassification{msc}{01A45}

% this is the default PlanetMath preamble.  as your knowledge
% of TeX increases, you will probably want to edit this, but
% it should be fine as is for beginners.

% almost certainly you want these
\usepackage{amssymb}
\usepackage{amsmath}
\usepackage{amsfonts}

% used for TeXing text within eps files
%\usepackage{psfrag}
% need this for including graphics (\includegraphics)
%\usepackage{graphicx}
% for neatly defining theorems and propositions
%\usepackage{amsthm}
% making logically defined graphics
%%%\usepackage{xypic}

% there are many more packages, add them here as you need them

% define commands here

\begin{document}
{\em Pierre de Fermat} (August 17, 1601 - January 12, 1665) was a French lawyer at the Parlement of Toulouse, France, and a mathematician who is given credit for early developments that led to modern calculus. In particular, he is recognised for his discovery of an original method of finding the greatest and the smallest ordinates of curved lines, which is analogous to that of the then unknown differential calculus, as well as his research into the theory of numbers. He also made notable contributions to analytic geometry and probability.

With his gift for number relations (perhaps the most outstanding since Diophantus) and his ability to find proofs for his theorems, Fermat essentially created the modern theory of numbers. The quality of his work can be gauged by the fact that many of his results were not proved for over a century after his death, and one of them, his Last Theorem, took more than three centuries to prove. It was the convention among mathematicians in his day to challenge each other, often not publishing them to retain an advantage in the competition.

Although he carefully studied and drew inspiration from Diophantus, Fermat began a different tradition. Diophantus was content to find a single solution to his equations, even if it was an undesired fraction. Fermat was interested only in integer solutions to his diophantine equations, and he looked for all solutions of the equation. He also often proved that certain equations had no solution, which baffled his contemporaries.

He studied Pell's equation, Fermat numbers, perfect number, and amicable numbers. It was while researching perfect numbers that he discovered Fermat's theorem.

He invented the proof technique of infinite descent, and Fermat's factorization method.

He also developed two-square theorem, and the polygonal number theorem, which states that each number is a sum of three triangular numbers, four square numbers, five pentagonal numbers, and so on.

He was the first to evaluate the integral of general power functions. Using an ingenious trick, he was able to reduce this evaluation to the sum of geometric series. The resulting formula was helpful to Newton and then Leibniz when they, independently, developed the fundamental theorems of calculus.

Although Fermat claimed to be able to prove all his arithmetic results, few records of his proofs have survived. Many mathematicians, including Gauss, doubted his claim, especially given the difficulty of some of the problems, and the limited mathematical tools available to Fermat.

Together with René Descartes, Fermat was one of the two leading mathematicians of the first half of the 17th century. Independently of Descartes, he discovered the fundamental principles of analytic geometry. With Blaise Pascal, he was a founder of the theory of probability.

Fermat was known to be secretive and was a recluse. His only contact with the wider mathematical community aside from a brief exchange of letters with Pascal, was Marin Mersenne. However as Mersenne operated a correspondence network of sorts with other European thinkers, Fermat's results became widely distributed through him.

{\it This entry was adapted from the Wikipedia article \PMlinkexternal{Pierre de Fermat}{http://en.wikipedia.org/wiki/Pierre de Fermat} as of April 7, 2007.}
%%%%%
%%%%%
\end{document}
