\documentclass[12pt]{article}
\usepackage{pmmeta}
\pmcanonicalname{JuliaBowmanRobinson}
\pmcreated{2013-03-22 18:06:42}
\pmmodified{2013-03-22 18:06:42}
\pmowner{Mravinci}{12996}
\pmmodifier{Mravinci}{12996}
\pmtitle{Julia Bowman Robinson}
\pmrecord{4}{40657}
\pmprivacy{1}
\pmauthor{Mravinci}{12996}
\pmtype{Biography}
\pmcomment{trigger rebuild}
\pmclassification{msc}{01A60}
\pmsynonym{Julia Bowman}{JuliaBowmanRobinson}
\pmsynonym{Julia Robinson}{JuliaBowmanRobinson}

% this is the default PlanetMath preamble.  as your knowledge
% of TeX increases, you will probably want to edit this, but
% it should be fine as is for beginners.

% almost certainly you want these
\usepackage{amssymb}
\usepackage{amsmath}
\usepackage{amsfonts}

% used for TeXing text within eps files
%\usepackage{psfrag}
% need this for including graphics (\includegraphics)
%\usepackage{graphicx}
% for neatly defining theorems and propositions
%\usepackage{amsthm}
% making logically defined graphics
%%%\usepackage{xypic}

% there are many more packages, add them here as you need them

% define commands here

\begin{document}
\PMlinkescapeword{children}

\emph{Julia Bowman Robinson} (n\'ee \emph{Julia Bowman}) (1919 - 1985) American mathematician and author.

The daughter of a machine tool salesman and a housewife, Julia was born in Missouri but after the death of her mother two years later the family moved to Arizona. There, she enjoyed ``arranging pebbles in the shadow of a giant saguaro on the Arizona desert,'' something which became her earliest memory (Reid \& Robinson, 1987). Then the family moved to San Diego, staying there for decades. In the high school there, at the time, girls had the option to drop math from their studies, something they all did with the exception of Julia Bowman. She continued her study of mathematics at San Diego State College despite trouble paying tuition due to the Great Depression. Transferring to the University of California, she was taught number theory by Raphael M. Robinson, who introduced her to the theories of Kurt G\"odel.

In 1942, Julia married Raphael and took his name. With the United States now drawn into World War II, Julia Robinson worked at the Berkeley Statistical Laboratory, which supported military operations. In 1948, supervised by Alfred Tarski, Robinson wrote a thesis on decision problems, earning her Ph.D. Next she worked on a problem suggested by Tarski, related to the tenth of Hilbert's problems. Though she was unable to solve the problem, when Yuri Matiyasevi\v{c} disproved it in 1970, he recognized Robinson's contributions. And it was her later collaborations with Matiyasevi\v{c} which earned her \PMlinkname{Erd\H{o}s number}{ErdHosNumber} 3 (since Matiyasevi\v{c} wrote with Richard K. Guy a paper on ``A new formula for $\pi$'' for {\it American Mathematical Monthly} and Guy had written with Erd\H{o}s a paper on lattice point distances in {\it Elementary Mathematics}).

In 1982, she became the first woman elected president of the American Mathematical Society. By now her health had deteriorated due to the same heart problem which back in 1945 had prompted doctors to advise her not to have children. But it was leukemia, diagnosed in 1984, which led to her death in 1985.

\begin{thebibliography}{1}
\bibitem{cr} C. Reid \& R. M. Robinson ``Julia Bowman Robinson'' in {\it Women of Mathematics: A Bibliographic Sourcebook} L. Grinstein, P. Cambpell, ed.s New York: Greenwood Press (1987): 182 - 189
\end{thebibliography}

%%%%%
%%%%%
\end{document}
