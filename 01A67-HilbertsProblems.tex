\documentclass[12pt]{article}
\usepackage{pmmeta}
\pmcanonicalname{HilbertsProblems}
\pmcreated{2013-03-22 16:05:40}
\pmmodified{2013-03-22 16:05:40}
\pmowner{Daume}{40}
\pmmodifier{Daume}{40}
\pmtitle{Hilbert's problems}
\pmrecord{18}{38156}
\pmprivacy{1}
\pmauthor{Daume}{40}
\pmtype{Feature}
\pmcomment{trigger rebuild}
\pmclassification{msc}{01A67}
\pmclassification{msc}{01A60}
\pmrelated{DehnsTheorem}

\endmetadata

% this is the default PlanetMath preamble.  as your knowledge
% of TeX increases, you will probably want to edit this, but
% it should be fine as is for beginners.

% almost certainly you want these
\usepackage{amssymb}
\usepackage{amsmath}
\usepackage{amsfonts}
\usepackage{amsthm}

% used for TeXing text within eps files
%\usepackage{psfrag}
% need this for including graphics (\includegraphics)
%\usepackage{graphicx}
% making logically defined graphics
%%%\usepackage{xypic} 

% there are many more packages, add them here as you need them

% define commands here

% The below lines should work as the command
% \renewcommand{\bibname}{References}
% without creating havoc when rendering an entry in
% the page-image mode.
\makeatletter
\@ifundefined{bibname}{}{\renewcommand{\bibname}{References}}
\makeatother

\newtheorem{thm}{Theorem}
\newtheorem{defn}{Definition}
\newtheorem{prop}{Proposition}
\newtheorem{lemma}{Lemma}
\newtheorem{cor}{Corollary}
\begin{document}
\PMlinkescapeword{open}
\PMlinkescapeword{class}
\PMlinkescapeword{topology}
\PMlinkescapeword{algebraic}
\PMlinkescapeword{curves}
\PMlinkescapeword{surfaces}
\PMlinkescapeword{entire}

On the morning of the $8^{th}$ of August 1900 at the second International Congress of Mathematicians in Paris, David Hilbert 
gave a talk on `The Problems of Mathematics in the Future' (`Sur les probl\`emes futures des math\'ematiques').\cite{GGI}  He was invited to give a lecture and
gave 10 problems (from the 23 known Hilbert's problems) they were (1,2,6,8,12,13,16,19,21,22).\cite{GGI}  The entire 23 problems where published after the conference in \emph{Archiv der Mathematik und Physik}.
Hermann Weyl, one of Hilbert's students, later on stated that any one who solved one of the 23 problems
would be part of the honours class of mathematicians.\cite{GJ} 

\textbf{The 23 problems:}\\
\begin{tabular}{|c|l|l|}
\hline
Hilbert's problem & short description of problem & status\\
\hline
1. &\PMlinkname{Cantor's continuum hypothesis}{ContinuumHypothesis} &? \\
2. & Consistency of arithmetic axioms & \\ 
3. & Polyhedral assembly from polyhedron of equal volume & Solved \\ 
4. & Constructibility of metrics by geodesics & \\
5. & Existence of topological groups as manifolds that are not \PMlinkname{differential groups}{LieGroup} & Solved \\
6. & Axiomatization of physics & In progress--AQFT*,TQFT\\
7. & Genfold-Schneider theorem & \\
8. & Riemann hypothesis & \\
9. & Algebraic number field reciprocity theorem & \\
10. & Matiyasevich's theorem & Solved \\
11. & Quadratic form solution with algebraic numerical coefficients & \\
12. & Extension of Kronecker's theorem to other number fields & \\
13. & Solution of 7th degree equations with 2-parameter functions & \\
14. & Proof of finiteness of complete systems of functions & \\ 
15. & Schubert's enumerative calculus & \\
16. & \PMlinkname{Problem of the topology of algebraic curves and surfaces}{HilbertsSixteenthProblem}& Open\\
17. & \PMlinkname{Problem related to quadratic forms}{TheoremsOnSumsOfSquares} & Solved \\
18. & Existence of space-filling polyhedron and densest sphere packing & \\
19. & Existence of Lagrangian solution that is not analytic & \\
20. & Solvability of variational problems with boundary conditions & \\
21. & Existence of linear differential equations with monodromic group & \\
22. & Uniformization of analytic relations & \\
23. & Calculus of variations & \\
\hline
\end{tabular}\\

\textbf{See also:}
\begin{itemize}
\item David Hilbert, \PMlinkexternal{Mathematische Probleme}{http://www.mathematik.uni-bielefeld.de/~kersten/hilbert/rede.html}
\item David Hilbert, \PMlinkexternal{Mathematical Problems}{http://aleph0.clarku.edu/~djoyce/hilbert/problems.html}
\item Wikipedia, \PMlinkexternal{Hilbert's problems}{http://en.wikipedia.org/wiki/Hilbert_problems}
\end{itemize}

\begin{thebibliography}{10}
\bibitem[GGI]{GGI}
{\scshape Ivor Grattan-Guinness}, \emph{A Sideways Look at Hilbert's
Twenty-three Problems of 1900}, Notices of the AMS, Vol 47, 7, 2000.

\bibitem[GJ]{GJ}
{\scshape Jeremy Gray}, \emph{The Hilbert problems}, European Mathematical Society, Newsletter 36, 10-12, 2000.

\bibitem[BF]{BF}
{\scshape Felix E. Browder (ed.)}, \emph{Mathematical Problems Arising from Hilbert problems}, Proceedings of Symposia in Pure Mathematics Vol. XXVII, Part I and Part II, American Mathematical Society, 1976.

\bibitem[YB]{YB}
{\scshape Benjamin H. Yandell}, \emph{The Honors Class: Hilbert's problems and their solvers}, A K Peters, 2002.
\end{thebibliography}


\textbf{Notes:}\\
This entry is under construction please feel free to add information as it editable by anyone who is a member.  Please reference what is added, thank you.
The idea, is maybe:
\begin{itemize} 
\item have a good introduction,
\item have a small discription of each problem, and as attached entry have more details on each problem separately,
\item have a good bibliography.
\item *AQFT = Algebraic, or Axiomatic Quantum Field Theory
\end{itemize}
Also I think we should not CC wikipedia.  This note can be removed once
the entry is complete.



%%%%%
%%%%%
\end{document}
