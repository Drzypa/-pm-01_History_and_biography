\documentclass[12pt]{article}
\usepackage{pmmeta}
\pmcanonicalname{TerenceTao}
\pmcreated{2013-03-22 16:18:37}
\pmmodified{2013-03-22 16:18:37}
\pmowner{PrimeFan}{13766}
\pmmodifier{PrimeFan}{13766}
\pmtitle{Terence Tao}
\pmrecord{7}{38433}
\pmprivacy{1}
\pmauthor{PrimeFan}{13766}
\pmtype{Definition}
\pmcomment{trigger rebuild}
\pmclassification{msc}{01A61}
\pmclassification{msc}{01A60}
\pmsynonym{Terence C. Tao}{TerenceTao}

\endmetadata

% this is the default PlanetMath preamble.  as your knowledge
% of TeX increases, you will probably want to edit this, but
% it should be fine as is for beginners.

% almost certainly you want these
\usepackage{amssymb}
\usepackage{amsmath}
\usepackage{amsfonts}

% used for TeXing text within eps files
%\usepackage{psfrag}
% need this for including graphics (\includegraphics)
%\usepackage{graphicx}
% for neatly defining theorems and propositions
%\usepackage{amsthm}
% making logically defined graphics
%%%\usepackage{xypic}

% there are many more packages, add them here as you need them

% define commands here

\begin{document}
\emph{Terence Tao} (1975 - ) Australian mathematician, son of Hong Kong immigrants. As a child, Tao won medals at the  International Mathematical Olympiads. Charles Fefferman calls Tao a mathematical ``Mr. \PMlinkescapetext{Fix} It.'' His 2004 paper with Ben Green on arithmetic progressions of primes was hailed by Neil Sloane as a major mathematical landmark. Two years later, Tao became the first Australian to win the Fields medal.
%%%%%
%%%%%
\end{document}
