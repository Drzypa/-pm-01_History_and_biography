\documentclass[12pt]{article}
\usepackage{pmmeta}
\pmcanonicalname{WolfPrize}
\pmcreated{2013-03-22 16:59:52}
\pmmodified{2013-03-22 16:59:52}
\pmowner{PrimeFan}{13766}
\pmmodifier{PrimeFan}{13766}
\pmtitle{Wolf Prize}
\pmrecord{4}{39279}
\pmprivacy{1}
\pmauthor{PrimeFan}{13766}
\pmtype{Definition}
\pmcomment{trigger rebuild}
\pmclassification{msc}{01A60}
\pmclassification{msc}{01A61}
\pmclassification{msc}{01A65}

% this is the default PlanetMath preamble.  as your knowledge
% of TeX increases, you will probably want to edit this, but
% it should be fine as is for beginners.

% almost certainly you want these
\usepackage{amssymb}
\usepackage{amsmath}
\usepackage{amsfonts}

% used for TeXing text within eps files
%\usepackage{psfrag}
% need this for including graphics (\includegraphics)
%\usepackage{graphicx}
% for neatly defining theorems and propositions
%\usepackage{amsthm}
% making logically defined graphics
%%%\usepackage{xypic}

% there are many more packages, add them here as you need them

% define commands here

\begin{document}
The {\em Wolf Prize} is a prize awarded by the Wolf \PMlinkescapetext{Foundation} to ``outstanding scientists and artists - irrespective of nationality, race, color, religion, sex or political views - for achievements in the interest of mankind and friendly relations among peoples.'' The six categories of the Wolf Prizes are: agriculture, chemistry, mathematics, medicine, physics and the arts. Nominations for a Wolf Prize in Mathematics are only accepted from presidents of mathematical societies, directors of universities or research institutions, dean of math faculty at an university, or from past Wolf Prize in Mathematics recipients.

\subsection{External links}
\PMlinkexternal{Official Wolf Foundation website}{http://www.wolffund.org.il/main.asp}
%%%%%
%%%%%
\end{document}
