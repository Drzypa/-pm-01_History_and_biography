\documentclass[12pt]{article}
\usepackage{pmmeta}
\pmcanonicalname{Aristotle}
\pmcreated{2013-03-22 14:21:35}
\pmmodified{2013-03-22 14:21:35}
\pmowner{Daume}{40}
\pmmodifier{Daume}{40}
\pmtitle{Aristotle}
\pmrecord{9}{35840}
\pmprivacy{1}
\pmauthor{Daume}{40}
\pmtype{Biography}
\pmcomment{trigger rebuild}
\pmclassification{msc}{01A20}

\endmetadata

% this is the default PlanetMath preamble.  as your knowledge
% of TeX increases, you will probably want to edit this, but
% it should be fine as is for beginners.

% almost certainly you want these
\usepackage{amssymb}
\usepackage{amsmath}
\usepackage{amsfonts}

% used for TeXing text within eps files
%\usepackage{psfrag}
% need this for including graphics (\includegraphics)
\usepackage{graphicx}
% for neatly defining theorems and propositions
%\usepackage{amsthm}
% making logically defined graphics
%%%\usepackage{xypic} 

% there are many more packages, add them here as you need them

% define commands here

% The below lines should work as the command
% \renewcommand{\bibname}{References}
% without creating havoc when rendering an entry in
% the page-image mode.
%\makeatletter
%\@ifundefined{bibname}{}{\renewcommand{\bibname}{References}}
%\makeatother
\begin{document}
\PMlinkescapeword{children}
\PMlinkescapeword{implication}
\PMlinkescapeword{logic}
\PMlinkescapeword{real}
\PMlinkescapeword{sources}
\PMlinkescapeword{successor}
\PMlinkescapeword{support}
\PMlinkescapeword{syntax}

\begin{center}
\includegraphics[scale=1]{aristotle.eps}\\
\small Aristotle \textit{(384-322 BC)}\\
\textit{(picture from \cite{WA})}
\end{center}

Aristotle was born in 384 BC in Stageira.\cite{BI}  Stageira was a sea port and Greek colony in Macedonia.\cite{WA}.  He was the son of Nicomachus who was a medical doctor and Phaestis.\cite{OR}
 
Nicomachus was the personal physical physician of the King of Macedonia.  This is how Aristotle had his first encounter with the Macedonian court of the King and his son Philip.  Nicomachus taught Aristotle biology since it was customary that doctors' skills would be passed down to their children.  At the age of ten, Aristotle's father died leaving him orphaned \textit{(his mother had died previously)}.  Therefore his uncle Proxenus of Atarneus became his guardian.  He taught Aristotle Greek, rhetoric, and poetry.\cite{OR}  
At the age of seventeen, Proxenus sent him to Athens.  Here he became the student of Plato at the school which he had founded, the Academy.\cite{IEP}  Aristotle studied at the Academy for several years and then became a teacher.  Early on, Aristotle's views were largely in support of Plato's, hence the views of the Academy. Eventually Plato died and Aristotle was expected to become the next head of the Academy, but by that time his views had diverged too much from those of Plato.\cite{OR}  Therefore Plato's nephew was chosen to take over the Academy.  Plato is often referred as an idealist or rationalist, while Aristotle, contrary to Plato, believed in acquiring knowledge from the senses and therefore is categorized as an empiricist.\cite{WA}

After Aristotle failed to become the head of the Academy, he and Xenocrates another member of the Academy, travelled to the court of the King Hermias.\cite{WA}  There Aristotle stayed for a few years and married the King's niece and adopted daughter.\cite{IEP}  Soon the Persians attacked the city and executed the King, so Aristotle fled to the island of Lesbos where he continued his research in the area of biology.\cite{OR}

Continuing his travel he ended back in Macedonia in the court of the King Philip who he had grown up with.  He stayed there for seven years and during that time might have been in contact with Alexander the Great.  Aristotle is often said to have been the tutor of Alexander the Great but some sources deny this fact.\cite{OR}

At this time, Plato's nephew who was still at the head of the Academy died.  King Philip believed that Aristotle would become his successor, but for the second time the position was denied him, Xenocrates was elected instead.

When returning to Athens, Plato's school was still thriving and Aristotle was inspired to start his own school to share his ideas.  He called his school the Lyceum.\cite{IEP}  He regularly gave lectures in philosophy in the gymnasium \textit{(lecture hall)} which led him to be known as the father of what is called the Peripatetic school.\cite{BI}  Called Peripatetic to describe Aristotle's walking about the gymnasium while lecturing \textit{(pateo is the Greek work for walk)}.\cite{IEP}

After the death of Alexander there was a great anti-Macedonian movement and Aristotle's life was at risk for being close to the Macedonian court.  He left Athens saying that he would not give the Athenians a chance to sin a third time against philosophy \textit{(referring to the charges of impiety brought against Anaxagoras and Socrates)}.\cite{WA}

He then moved to Chalci, Greece where one year later he got a stomach illness and died in 322 BC.\cite{IEP}

\section*{Before Aristotle's Logic}

Aristotle ``says that `on the subject of reasoning' he `had nothing else on an earlier date to speak about'''\cite{BI}.  But Plato reports that syntax was thought of before him, by Prodikos of Keos who was concerned by the \PMlinkescapetext{right} use of \PMlinkescapetext{words.}\cite{BI}  Logic seems to have emerged from dialectics, the earlier philosophers used concepts like reductio ad absurdum as rule when discussing, but never understood its logical implications.\cite{BI}  \PMlinkescapetext{Even} Aristotle's teacher, Plato, had difficulties with logic although he had the idea of constructing a system for deduction.  He was never able to construct one, since he relied on his dialectic which was a confusion \PMlinkescapetext{between} different sciences and methods.\cite{BI}  Plato thought that deduction would simply follow from premises, so he focused on having good premises so that the conclusion would follow.  Later on, Plato realised that a method for obtaining the conclusion would be beneficial.  Plato never obtained such a method, his best attempt was published in his book \emph{Sophist} where he introduced his division method.\cite{RL}

\section*{Aristotle's Logic}

See Aristotelian logic.

\section*{Aristotle's Modal Logic}

In addition to the Aristotelian logic that will be discussed in this paper, Aristotle is also the creator of syllogisms with modalities \textit{(modal logic)}.  The word modal refers to the word `modes', explaining the fact that modal logic deals with the modes of truth.  Aristotle introduced the qualification of \emph{necessarily} and \emph{possibly} premises.  He constructed a logic which helped in the evaluation of truth but which was very difficult to interpret.\cite{SR}

\section*{After Aristotle's Logic}

Aristotle's Logic was not always popular, during the Hellenistic era, Stoic logic was predominant with the work of Chrysipus \textit{(none of his work has survived)}.\cite{SR}  Immanuel Kant(1724--1804) thought that there was nothing else to invent after the work of Aristotle and a famous logic historian called Carl Prantl(1820-1888) claimed that any logician who said anything new about logic was ``confused, stupid or perverse''.\cite{SR}  These examples illustrate the general tendency to accept without question the work of Aristotle.  He had already become known by the Scholastics (medieval Christian scholars) as `The Philosopher'.  The dogmatism created by the Scholastics in favor of Aristotle took a long time to disappear.\cite{WAL}  As a consequence no real progress was made in logic until the twentieth century.  Although some controversy about Aristotle's logic had started \PMlinkescapetext{even} earlier, when some believed that logic was part of philosophy\textit{(they were known as the Stoic)} and others believed that it was simply a tool to study philosophy, Aristotle thought that.\cite{SR}  

Aristotelian logic has lost most of its reputation as the one only correct logic.  Gottlob Frege and Bertrand Russell criticized the work of Aristotle and showed its many limitations.  They helped remove the positive prejudice associated with the work of Aristotle.  Today logicians who study modern logic respect the Aristotelian logic in the sense of its great early accomplishment.\cite{SR}

\section*{See also}

\begin{itemize}
\item
The Internet Encyclopedia of Philosophy: \PMlinkexternal{Aristotle}{http://www.utm.edu/research/iep/a/aristotl.htm}
\item
The MacTutor History of Mathematics Archive: \PMlinkexternal{Aristotle}{http://www-gap.dcs.st-and.ac.uk/~history/}
\item
Stanford Encyclopedia of Philosophy: \PMlinkexternal{Aristotle's Logic}{http://plato.stanford.edu/entries/aristotle-logic/}
\item
Wikipedia: \PMlinkexternal{Aristotle}{http://en.wikipedia.org/Aristotle}
\end{itemize}

\begin{thebibliography}{99}
\bibitem[BI]{BI} Boche\'nski, I.M. (1951), \textit{Ancient Formal Logic}, North-Holland Publishing Company, Amsterdam.
\bibitem[IEP]{IEP} ``Aristotle", The Internet Encyclopedia of Philosophy [online], \PMlinkexternal{http://www.utm.edu/research/iep/a/aristotl.htm}{http://www.utm.edu/research/iep/a/aristotl.htm} , 2004.
\bibitem[OR]{OR} O'Connor, J. John, and Robertson, F. Edmund (2004), ``Aristotle", The MacTutor History of Mathematics Archive [online], \PMlinkexternal{http://www-gap.dcs.st-and.ac.uk/~history/}{http://www-gap.dcs.st-and.ac.uk/~history/}.
\bibitem[RL]{RL} Rose, Lynn E. (1968), ``Aristotle's Syllogistic", Charles C. Thomas Publisher, Springfield. 
\bibitem[SR]{SR} Smith, Robin, ``Aristotle's Logic", Stanford Encyclopedia of Philosophy [online] \PMlinkexternal{http://plato.stanford.edu/entries/aristotle-logic/}{http://plato.stanford.edu/entries/aristotle-logic/}
\bibitem[AET]{AET} Taylor, A.E. (1919/1955), \textit{Aristotle}, revised edition, 1919.  Reprinted, 1955, Dover Publications, Mineola, NY, 1955.
\bibitem[WA]{WA} Wikipedia, ``Aristotle" [online], \PMlinkexternal{http://en.wikipedia.org/Aristotle}{http://en.wikipedia.org/Aristotle}, 2004.
\bibitem[WAL]{WAL} Wikipedia, ``Aristotelian Logic" [online], \PMlinkexternal{http://en.wikipedia.org/Aristotelian_logic}{http://en.wikipedia.org/Aristotelian\_logic} , 2004.
\end{thebibliography}

%%%%%
%%%%%
\end{document}
