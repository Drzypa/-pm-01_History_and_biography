\documentclass[12pt]{article}
\usepackage{pmmeta}
\pmcanonicalname{MichaelGolomb}
\pmcreated{2013-03-22 16:59:43}
\pmmodified{2013-03-22 16:59:43}
\pmowner{PrimeFan}{13766}
\pmmodifier{PrimeFan}{13766}
\pmtitle{Michael Golomb}
\pmrecord{4}{39276}
\pmprivacy{1}
\pmauthor{PrimeFan}{13766}
\pmtype{Biography}
\pmcomment{trigger rebuild}
\pmclassification{msc}{01A60}
\pmclassification{msc}{01A61}
\pmclassification{msc}{01A65}

% this is the default PlanetMath preamble.  as your knowledge
% of TeX increases, you will probably want to edit this, but
% it should be fine as is for beginners.

% almost certainly you want these
\usepackage{amssymb}
\usepackage{amsmath}
\usepackage{amsfonts}

% used for TeXing text within eps files
%\usepackage{psfrag}
% need this for including graphics (\includegraphics)
%\usepackage{graphicx}
% for neatly defining theorems and propositions
%\usepackage{amsthm}
% making logically defined graphics
%%%\usepackage{xypic}

% there are many more packages, add them here as you need them

% define commands here

\begin{document}
\emph{Michael Golomb} (1909 - ) Jewish German-born American mathematician, one of the pioneers in the study of normed vector spaces using numerical analysis.

With Erhard Schmidt as his teacher, Golomb earned a doctorate from the University of Berlin in 1933. The next year he fleed to Yugoslavia to escape Nazism, four years later he came to the United States. In 1942 he started teaching at Purdue, decades later earning the title of Professor Emeritus. In 1955, he co-authored with Erd\H{o}s a paper on functions which are symmetric about several points in a Dutch journal, giving him an \PMlinkname{Erd\H{o}s number}{ErdHosNumber} of 1. When the International Congress of Mathematicians chose Berlin for its annual meeting in 1998, the city honored the achievements of Golomb and other Jewish mathematicians who left before World War II.
%%%%%
%%%%%
\end{document}
