\documentclass[12pt]{article}
\usepackage{pmmeta}
\pmcanonicalname{MariaGaetanaAgnesi}
\pmcreated{2013-03-22 16:35:50}
\pmmodified{2013-03-22 16:35:50}
\pmowner{Mravinci}{12996}
\pmmodifier{Mravinci}{12996}
\pmtitle{Maria Gaetana Agnesi}
\pmrecord{6}{38793}
\pmprivacy{1}
\pmauthor{Mravinci}{12996}
\pmtype{Biography}
\pmcomment{trigger rebuild}
\pmclassification{msc}{01A50}

% this is the default PlanetMath preamble.  as your knowledge
% of TeX increases, you will probably want to edit this, but
% it should be fine as is for beginners.

% almost certainly you want these
\usepackage{amssymb}
\usepackage{amsmath}
\usepackage{amsfonts}

% used for TeXing text within eps files
%\usepackage{psfrag}
% need this for including graphics (\includegraphics)
%\usepackage{graphicx}
% for neatly defining theorems and propositions
%\usepackage{amsthm}
% making logically defined graphics
%%%\usepackage{xypic}

% there are many more packages, add them here as you need them

% define commands here

\begin{document}
\PMlinkescapeword{rights}
\PMlinkescapeword{translation}

\emph{Maria Gaetana Agnesi} (1718 - 1799) Italian linguist, mathematician and women's rights activist, best known for her study of the versiera curve, nowadays known as the ``witch of Agnesi'' or Agnesi curve. Her {\it Instituzioni Analitiche} was one of the first mathematics books printed in Italian (although she could have as easily written it in Latin or Greek) and was of great use to many other mathematicians, Joseph-Louis Lagrange (in a French translation) among them. After the death of her father, she devoted herself to theology and charitable work.

\begin{thebibliography}{1}
\bibitem{hk} H. Kennedy ``Maria Gaetana Agnesi'' in {\it Women of Mathematics: A Bibliographic Sourcebook} L. Grinstein, P. Cambpell, ed.s New York: Greenwood Press (1987): 1 - 5
\end{thebibliography}
%%%%%
%%%%%
\end{document}
