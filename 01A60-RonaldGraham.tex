\documentclass[12pt]{article}
\usepackage{pmmeta}
\pmcanonicalname{RonaldGraham}
\pmcreated{2013-03-22 16:50:02}
\pmmodified{2013-03-22 16:50:02}
\pmowner{Mravinci}{12996}
\pmmodifier{Mravinci}{12996}
\pmtitle{Ronald Graham}
\pmrecord{4}{39076}
\pmprivacy{1}
\pmauthor{Mravinci}{12996}
\pmtype{Biography}
\pmcomment{trigger rebuild}
\pmclassification{msc}{01A60}
\pmclassification{msc}{01A61}
\pmclassification{msc}{01A65}
\pmsynonym{Ronald Lewis Graham}{RonaldGraham}
\pmsynonym{Ron Graham}{RonaldGraham}

\endmetadata

% this is the default PlanetMath preamble.  as your knowledge
% of TeX increases, you will probably want to edit this, but
% it should be fine as is for beginners.

% almost certainly you want these
\usepackage{amssymb}
\usepackage{amsmath}
\usepackage{amsfonts}

% used for TeXing text within eps files
%\usepackage{psfrag}
% need this for including graphics (\includegraphics)
%\usepackage{graphicx}
% for neatly defining theorems and propositions
%\usepackage{amsthm}
% making logically defined graphics
%%%\usepackage{xypic}

% there are many more packages, add them here as you need them

% define commands here

\begin{document}
\emph{Ronald Lewis Graham} (1935 - ) American mathematician and juggler, perhaps best known for Graham's number, husband of Fan Chung.

In 1962, Graham earned a Ph.D at the University of California-Berkeley. In 1972, he coauthored with Erd\H{o}s a paper on the sums of terms of the Fibonacci sequence in the {\it Fibonacci Quarterly}, giving him an \PMlinkname{Erd\H{o}s number}{ErdHosNumber} of 1. His most famous paper, however, came in 1977. Concerning itself with Ramsey theory, the paper gave a spectacularly large upper bound on the solution to one of the problems, which the {\it Guinness Book of World Records} acknowledged the next year as the largest number ever used in a professional mathematician's paper, which came to be known as Graham's number. (According to Wells, the solution could be as small as 6). Graham was president of the International Jugglers Union for a year.

\begin{thebibliography}{1}
\bibitem{dw} Wells, D. . {\it The Penguin Dictionary of Curious and Interesting Numbers} London: Penguin Group. (1987): 255 - 256
\end{thebibliography}
%%%%%
%%%%%
\end{document}
