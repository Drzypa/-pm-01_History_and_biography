\documentclass[12pt]{article}
\usepackage{pmmeta}
\pmcanonicalname{JohnCharlesFields}
\pmcreated{2013-03-22 16:12:10}
\pmmodified{2013-03-22 16:12:10}
\pmowner{Mravinci}{12996}
\pmmodifier{Mravinci}{12996}
\pmtitle{John Charles Fields}
\pmrecord{7}{38297}
\pmprivacy{1}
\pmauthor{Mravinci}{12996}
\pmtype{Biography}
\pmcomment{trigger rebuild}
\pmclassification{msc}{01A60}
\pmsynonym{John C. Fields}{JohnCharlesFields}
\pmsynonym{J. Charles Fields}{JohnCharlesFields}
\pmsynonym{J. C. Fields}{JohnCharlesFields}

% this is the default PlanetMath preamble.  as your knowledge
% of TeX increases, you will probably want to edit this, but
% it should be fine as is for beginners.

% almost certainly you want these
\usepackage{amssymb}
\usepackage{amsmath}
\usepackage{amsfonts}

% used for TeXing text within eps files
%\usepackage{psfrag}
% need this for including graphics (\includegraphics)
%\usepackage{graphicx}
% for neatly defining theorems and propositions
%\usepackage{amsthm}
% making logically defined graphics
%%%\usepackage{xypic}

% there are many more packages, add them here as you need them

% define commands here

\begin{document}
John Charles Fields was a Canadian mathematician who was best known as the founder of the Fields Medal. He was born in Hamilton, Ontario on May 14, 1863 and died on August 9, 1932.

\PMlinkescapetext{Fields} graduated from Hamilton Collegiate Institute in 1880 and the University of Toronto in 1884, then studied at Johns Hopkins University in Baltimore, Maryland. \PMlinkescapetext{Fields} received his \PMlinkescapetext{Ph}.D. in 1887 with the thesis {\it \PMlinkescapetext{Symbolic Finite Solutions and Solutions by Definite Integrals of the Equation}} $\displaystyle {{d^n y} \over {dx^n}} = x^m y$. He returned to Canada in 1902 to lecture at the University of Toronto.

\PMlinkescapetext{Fields} was elected fellow of the Royal Society of Canada in 1907 and fellow of the Royal Society of London in 1913.
%%%%%
%%%%%
\end{document}
