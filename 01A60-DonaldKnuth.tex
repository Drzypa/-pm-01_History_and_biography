\documentclass[12pt]{article}
\usepackage{pmmeta}
\pmcanonicalname{DonaldKnuth}
\pmcreated{2013-03-22 17:07:38}
\pmmodified{2013-03-22 17:07:38}
\pmowner{PrimeFan}{13766}
\pmmodifier{PrimeFan}{13766}
\pmtitle{Donald Knuth}
\pmrecord{6}{39430}
\pmprivacy{1}
\pmauthor{PrimeFan}{13766}
\pmtype{Biography}
\pmcomment{trigger rebuild}
\pmclassification{msc}{01A60}
\pmclassification{msc}{01A65}
\pmclassification{msc}{01A61}
\pmsynonym{Donald Ervin Knuth}{DonaldKnuth}
\pmsynonym{Gao Dena}{DonaldKnuth}
\pmsynonym{G\bar{a}o D\'en\`a}{DonaldKnuth}

\endmetadata

% this is the default PlanetMath preamble.  as your knowledge
% of TeX increases, you will probably want to edit this, but
% it should be fine as is for beginners.

% almost certainly you want these
\usepackage{amssymb}
\usepackage{amsmath}
\usepackage{amsfonts}

% used for TeXing text within eps files
%\usepackage{psfrag}
% need this for including graphics (\includegraphics)
%\usepackage{graphicx}
% for neatly defining theorems and propositions
%\usepackage{amsthm}
% making logically defined graphics
%%%\usepackage{xypic}

% there are many more packages, add them here as you need them

% define commands here

\begin{document}
{\em Donald Ervin Knuth} (1938 - ) American mathematician and computer programmer, best known for creating \TeX{} and writing {\it The Art of Computer Programming}. He is also famous for giving reward checks that increase by \PMlinkname{powers}{Power} of 2 to those who find mistakes in his creations, and for giving his software version numbers that gradually converge to an important irrational constant such as $e$ or $\pi$. Knuth is one of the few mathematicians to have been published in {\it MAD} magazine.

Born in Wisconsin, Knuth earned mathematics bachelor's and master's \PMlinkescapetext{degrees} from Case Western Reserve University (Case Institute of Technology at the time) in 1960. In 1968 he started teaching at Stanford. He now has the title of Professor Emeritus of The Art of Computer Programming. Other honors Knuth has received include the Turing Award, the John von Neumann Medal, the Kyoto Prize, and a Chinese ateje given him by Frances Yao.
%%%%%
%%%%%
\end{document}
