\documentclass[12pt]{article}
\usepackage{pmmeta}
\pmcanonicalname{FrancescoFaaDiBruno}
\pmcreated{2013-03-22 16:31:51}
\pmmodified{2013-03-22 16:31:51}
\pmowner{PrimeFan}{13766}
\pmmodifier{PrimeFan}{13766}
\pmtitle{Francesco Fa\`a di Bruno}
\pmrecord{4}{38711}
\pmprivacy{1}
\pmauthor{PrimeFan}{13766}
\pmtype{Biography}
\pmcomment{trigger rebuild}
\pmclassification{msc}{01A50}
\pmsynonym{Francesco Faa di Bruno}{FrancescoFaaDiBruno}

% this is the default PlanetMath preamble.  as your knowledge
% of TeX increases, you will probably want to edit this, but
% it should be fine as is for beginners.

% almost certainly you want these
\usepackage{amssymb}
\usepackage{amsmath}
\usepackage{amsfonts}

% used for TeXing text within eps files
%\usepackage{psfrag}
% need this for including graphics (\includegraphics)
%\usepackage{graphicx}
% for neatly defining theorems and propositions
%\usepackage{amsthm}
% making logically defined graphics
%%%\usepackage{xypic}

% there are many more packages, add them here as you need them

% define commands here

\begin{document}
\emph{Francesco Fa\`a di Bruno} (1825 - 1888) Italian mathematician and saint, best known for the Fa\`a di Bruno formula. He was a captain in an Italian militia, but gave up the post to become a Catholic priest. Instead of saying Mass, however, Bruno dedicated himself to teaching and studying mathematics, publishing dozens of articles in important mathematical journals of the day. He began what would have been a multi-volume book about elliptic functions but did not complete the work.

\section{Bibliography}

P. H. Linehan, "Francesco Faa di Bruno" in {\it The Catholic Encyclopedia} 1909. New York: Robert Appleton Company
%%%%%
%%%%%
\end{document}
