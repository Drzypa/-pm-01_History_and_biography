\documentclass[12pt]{article}
\usepackage{pmmeta}
\pmcanonicalname{SimonStevin}
\pmcreated{2013-03-22 18:02:57}
\pmmodified{2013-03-22 18:02:57}
\pmowner{PrimeFan}{13766}
\pmmodifier{PrimeFan}{13766}
\pmtitle{Simon Stevin}
\pmrecord{8}{40574}
\pmprivacy{1}
\pmauthor{PrimeFan}{13766}
\pmtype{Biography}
\pmcomment{trigger rebuild}
\pmclassification{msc}{01A40}
\pmrelated{TriangleMidSegmentTheorem}

% this is the default PlanetMath preamble.  as your knowledge
% of TeX increases, you will probably want to edit this, but
% it should be fine as is for beginners.

% almost certainly you want these
\usepackage{amssymb}
\usepackage{amsmath}
\usepackage{amsfonts}

% used for TeXing text within eps files
%\usepackage{psfrag}
% need this for including graphics (\includegraphics)
%\usepackage{graphicx}
% for neatly defining theorems and propositions
%\usepackage{amsthm}
% making logically defined graphics
%%%\usepackage{xypic}

% there are many more packages, add them here as you need them

% define commands here

\begin{document}
{\em Simon Stevin} (1548 - 1620) was a Flemish mathematician and engineer. He was active in a great many areas of science and engineering, both theoretical and practical. He also translated various mathematical terms into Dutch, making it one of the few European languages in which the word for mathematics,  ``wiskunde'', was not derived from Greek (via Latin).

Stevin was born in Bruges, Flanders (now Belgium) in the year 1548 by Antheunis Stevin and Cathelyne van der Poort. Very little has been recorded about his life. Even the exact date of birth and the date and place of his death (The Hague or Leiden) are uncertain. It is known that he left a widow with two children; and one or two hints scattered throughout his works inform us that he began life as a merchant's clerk in Antwerp, that he travelled in Poland, Denmark and other parts of northern Europe. After his travels, he became advisor and teacher of Prince Maurice of Nassau, who asked his advice on many occasions, and made him a public officer, at first director of the so-called ``waterstaet'' (the government authority for public works), and later quartermaster-general.

His claims to fame are varied, encompassing physics, optics, astronomy, engineering, music theory, accounting and mathematics.

Stevin was the first to show how to model regular and semiregular polyhedra by delineating their frames in a plane. Stevin also distinguished stable from unstable equilibria. He proved the law of the equilibrium on an inclined plane, using an ingenious and intuitive diagram showing a rope containing evenly spaced beads draped over an inclined plane. The diagram is said to have been inscribed on his tombstone, leading the physicist Richard Feynman to remark to his students, ``If you get an inscription like that on your tombstone, you are doing fine!''

He demonstrated the resolution of forces before Pierre Varignon, which had not been remarked previously, even though it is a simple consequence of the law of their composition. 

Bookkeeping by double entry may have been known to Stevin, as he was a clerk in Antwerp in his younger years, either practically or through the medium of the works of Italian authors such as Luca Pacioli and Gerolamo Cardano. However, Stevin was the first to recommend the use of impersonal accounts in the national household. He brought it into practice for Prince Maurice, and recommended it to the French statesman Maximilien de Béthune, duc de Sully.

Stevin wrote a 36 page booklet called {\it De Thiende} (``the tenth''), first published in Dutch in 1585, though the French translation "Disme" subtitle: "Teaching how all computations that are met in business may be performed by integers without the use of fractions" doesn't exceed seven pages. Fractions to Stevin meant unit fractions or Egyptian fractions. 

Decimal fractions had been employed for the extraction of square roots some five centuries before his time, but nobody established their daily use before Stevin. He felt that this innovation was so significant, that he declared the universal introduction of decimal coinage, measures and weights to be merely a question of time.

His notation, however, was rather unwieldy. The point separating the integers from the decimal fractions seems to be the invention of Bartholomaeus Pitiscus, in whose trigonometrical tables (1612) it occurs and it was accepted by John Napier in his logarithmic papers (1614 and 1619). 

Stevin printed little circles around the exponents of the different powers of one-tenth. That Stevin intended these encircled numerals to denote mere exponents is clear from the fact that he employed the very same symbol for powers of algebraic quantities. He didn't avoid fractional exponents; only negative exponents don't appear in his work.

There are two complete editions in French of his works, both printed in Leiden, one in 1608, the other in 1634.

Stevin thought the Dutch language to be excellent for scientific writing, and he translated a lot of the mathematical terms to Dutch. As a result, Dutch is one of the few Western European languages that have a lot of mathematical terms that do not stem from Latin, such as the aforementioned ``Wiskunde'' (mathematics).

Some of the words he invented evolved after his death: ``aftrekken'' (subtract) and ``delen'' (divide) stayed the same, but over time ``menigvuldigen'' became ``vermenigvuldigen'' (multiply, the added ``ver'' has no meaning). ``Vergaderen'' became ``optellen'' (add). Another example is the Dutch word for diameter: ``middellijn''', literally meaning ``line through the middle.''

The word ``zomenigmaal'' (quotient, literally ``that many times'') has become the perhaps less poetic ``quoti\"ent'' in modern day Dutch. Other terms did not make it into modern day mathematical Dutch, like ``teerling'' (dice, although still being used in the meaning as die), instead of cube.

There was a mathematical journal called {\it Simon Stevin} which published quarterly on pure and applied mathematical topics until 1993.

{\it This entry was adapted from the Wikipedia article \PMlinkexternal{Simon Stevin}{http://en.wikipedia.org/wiki/Simon_Stevin} as of May 9, 2008.}
and by Number Theory and its History, Oystein Ore, 1948,  page 313-314
%%%%%
%%%%%
\end{document}
