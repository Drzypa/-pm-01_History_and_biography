\documentclass[12pt]{article}
\usepackage{pmmeta}
\pmcanonicalname{JohnPollard}
\pmcreated{2013-03-22 16:44:05}
\pmmodified{2013-03-22 16:44:05}
\pmowner{PrimeFan}{13766}
\pmmodifier{PrimeFan}{13766}
\pmtitle{John Pollard}
\pmrecord{6}{38957}
\pmprivacy{1}
\pmauthor{PrimeFan}{13766}
\pmtype{Biography}
\pmcomment{trigger rebuild}
\pmclassification{msc}{01A61}
\pmclassification{msc}{01A60}
\pmclassification{msc}{01A65}

\endmetadata

% this is the default PlanetMath preamble.  as your knowledge
% of TeX increases, you will probably want to edit this, but
% it should be fine as is for beginners.

% almost certainly you want these
\usepackage{amssymb}
\usepackage{amsmath}
\usepackage{amsfonts}

% used for TeXing text within eps files
%\usepackage{psfrag}
% need this for including graphics (\includegraphics)
%\usepackage{graphicx}
% for neatly defining theorems and propositions
%\usepackage{amsthm}
% making logically defined graphics
%%%\usepackage{xypic}

% there are many more packages, add them here as you need them

% define commands here

\begin{document}
\emph{John M. Pollard} (? - ) British mathematician, best known for devising various integer factorization algorithms, such as \PMlinkname{Pollard's $p - 1$ method}{PollardsP1Algorithm} and Pollard's $\rho$ method. Pollard has \PMlinkname{Erd\H{o}s number}{ErdHosNumber} 3: he coauthored with Hendrik W. Lenstra Jr. (and others) a paper of the number field sieve in {\it Lecture Notes in Math.}, {\bf 1554}, who collaborated with Jeffrey Shallit on a paper on continued fractions for {\it Math. Comp.} {\bf 61}, who co-authored with Erd\H{o}s ``New bounds on the length of finite Pierce and Engel series'' in {\it Sém. Théor. Nombres Bordeaux} {\bf 2} 3.
%%%%%
%%%%%
\end{document}
