\documentclass[12pt]{article}
\usepackage{pmmeta}
\pmcanonicalname{FriedrichBessel}
\pmcreated{2013-03-22 16:33:12}
\pmmodified{2013-03-22 16:33:12}
\pmowner{PrimeFan}{13766}
\pmmodifier{PrimeFan}{13766}
\pmtitle{Friedrich Bessel}
\pmrecord{4}{38738}
\pmprivacy{1}
\pmauthor{PrimeFan}{13766}
\pmtype{Biography}
\pmcomment{trigger rebuild}
\pmclassification{msc}{01A55}
\pmsynonym{Friedrich Wilhelm Bessel}{FriedrichBessel}

\endmetadata

% this is the default PlanetMath preamble.  as your knowledge
% of TeX increases, you will probably want to edit this, but
% it should be fine as is for beginners.

% almost certainly you want these
\usepackage{amssymb}
\usepackage{amsmath}
\usepackage{amsfonts}

% used for TeXing text within eps files
%\usepackage{psfrag}
% need this for including graphics (\includegraphics)
%\usepackage{graphicx}
% for neatly defining theorems and propositions
%\usepackage{amsthm}
% making logically defined graphics
%%%\usepackage{xypic}

% there are many more packages, add them here as you need them

% define commands here

\begin{document}
\emph{Friedrich Wilhelm Bessel} (1784 - 1846) German astronomer and mathematician, perhaps best known for the Bessel function.

As a teenager, he was apprenticed to a merchant and soon became his accountant. As the merchant did a lot of business with ships, Bessel developed an interest in astronomy. After publishing tables of atmospheric refraction, Bessel came up with the idea of using the parallax effect to measure the distance from the Earth to a star on the sky. In recognition of his achievements in astronomy, an asteroid was named after him.
%%%%%
%%%%%
\end{document}
