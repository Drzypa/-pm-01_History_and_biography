\documentclass[12pt]{article}
\usepackage{pmmeta}
\pmcanonicalname{ElizavetaLitvinova}
\pmcreated{2013-03-22 17:15:41}
\pmmodified{2013-03-22 17:15:41}
\pmowner{Mravinci}{12996}
\pmmodifier{Mravinci}{12996}
\pmtitle{Elizaveta Litvinova}
\pmrecord{5}{39599}
\pmprivacy{1}
\pmauthor{Mravinci}{12996}
\pmtype{Definition}
\pmcomment{trigger rebuild}
\pmclassification{msc}{01A60}
\pmclassification{msc}{01A55}
\pmsynonym{Elizaveta Fedorovna Litvinova}{ElizavetaLitvinova}
\pmsynonym{Elizaveta Fedorovna Ivanshkina}{ElizavetaLitvinova}
\pmsynonym{Elizaveta Ivanshkina}{ElizavetaLitvinova}

% this is the default PlanetMath preamble.  as your knowledge
% of TeX increases, you will probably want to edit this, but
% it should be fine as is for beginners.

% almost certainly you want these
\usepackage{amssymb}
\usepackage{amsmath}
\usepackage{amsfonts}

% used for TeXing text within eps files
%\usepackage{psfrag}
% need this for including graphics (\includegraphics)
%\usepackage{graphicx}
% for neatly defining theorems and propositions
%\usepackage{amsthm}
% making logically defined graphics
%%%\usepackage{xypic}

% there are many more packages, add them here as you need them

% define commands here

\begin{document}
\emph{Elizaveta Fedorovna Litvinova} n\'ee \emph{Elizaveta Fedorovna Ivanshkina} (1845 - 1919?) Russian mathematician, biographer and women's \PMlinkescapetext{rights} activist.

Born in czarist Russia, young Elizaveta's early education was at a women's high school in St. Petersburg. In 1866 Elizaveta married Viktor Litvinov and took his last name. Unlike Vladimir Kovalevskii (Sofia Kovalevskaya's husband), Litvinov would not allow his wife to travel to Europe to study at the universities there. Litvinova studied with Strannoliubskii (who had also privately tutored Kovalevskaya). As soon as her husband died, Litvinova went to Z\"urich and enrolled at a polytechnic institute. In 1873 the Russian czar decreed all Russian women studying in Z\"urich had to return to Russia or \PMlinkescapetext{face} the consequences. Litvinova was one of the few to ignore the decree and continue her studies in Z\"urich. When Litvinova returned to Russia, she was denied university appointments because she had defied the 1873 recall. She taught at a women's high school and supplemented her \PMlinkescapetext{meager} income by writing biographies of more famous mathematicians such as Kovalevskaya and Aristotle. After retiring, it is believed that Litvinova died during the Russian Revolution in 1919.

\begin{thebibliography}{1}
\bibitem{ak} A. H. Koblitz ``Sofia Vasilevna Kovalevskaia'' in {\it Women of Mathematics: A Bibliographic Sourcebook} L. Grinstein, P. Cambpell, ed.s New York: Greenwood Press (1987): 129 - 134
\end{thebibliography}
%%%%%
%%%%%
\end{document}
