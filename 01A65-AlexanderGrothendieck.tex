\documentclass[12pt]{article}
\usepackage{pmmeta}
\pmcanonicalname{AlexanderGrothendieck}
\pmcreated{2013-03-22 16:37:37}
\pmmodified{2013-03-22 16:37:37}
\pmowner{PrimeFan}{13766}
\pmmodifier{PrimeFan}{13766}
\pmtitle{Alexander Grothendieck}
\pmrecord{20}{38826}
\pmprivacy{1}
\pmauthor{PrimeFan}{13766}
\pmtype{Biography}
\pmcomment{trigger rebuild}
\pmclassification{msc}{01A65}
\pmclassification{msc}{01A61}
\pmclassification{msc}{01A60}
\pmrelated{AlexanderGrothendiecksAvailableSeminarsAndBooks}
\pmrelated{AlexanderGrothendieckABiographyOf}
\pmrelated{AlbertEinstein}

% this is the default PlanetMath preamble.  as your knowledge
% of TeX increases, you will probably want to edit this, but
% it should be fine as is for beginners.

% almost certainly you want these
\usepackage{amssymb}
\usepackage{amsmath}
\usepackage{amsfonts}

% used for TeXing text within eps files
%\usepackage{psfrag}
% need this for including graphics (\includegraphics)
%\usepackage{graphicx}
% for neatly defining theorems and propositions
%\usepackage{amsthm}
% making logically defined graphics
%%%\usepackage{xypic}

% there are many more packages, add them here as you need them

% define commands here

\begin{document}
\emph{Alexander Grothendieck} (1928 - ) German-born, French mathematician, one of the pioneers of topos theory, and the 
`new algebraic geometry and number theory'. In 1966, he was awarded the Fields Medal for fundamental contributions to mathematics (an algebraic proof of one of the Riemann-Roch theorems, previously conjectured), but boycotted the ceremony held in Moscow (USSR), (as further explained in the text). Two decades later, he declined the Crafoord Prize that was awarded to him and his student Pierre Deligne, because he didn't want the money and because the award was in recognition of work he had done much earlier in his career. Some concepts named after him include the Grothendieck group, the Grothendieck topology, the \PMlinkname{Grothendieck category}{GrothendieckCategory} and the Grothendieck universe.

In 1949, Grothendieck worked on functional analysis with Jean Dieudonn\'e at the University of Nancy in France; he was one of the `Nicolas Bourbaki' group of mathematicians that included at various times: Andr\'e Weil, Henri Cartan, Charles Ehresmann and J. Dieudonn\'e; Alexandre Grothendieck's doctoral thesis, supervised by his advisor Laurent Schwartz, was entitled ``Produits tensoriels topologiques et espaces nucl\'eaires''.

From 1953 to 1955, Grothendieck was visiting at the University of S\~ao Paulo, supported by the Centre National de la Recherche Scientifique. He returned to France in 1956, to resume at the Centre National de la R\'echerche Scientifique.

In 1960, he visited at the University of Kansas in the United States working on topology and geometry, supported by the Centre National de la R\'echerche Scientifique beginning with 1956.

From 1959 to 1970 he acted as the Chair and effective leader of the newly formed Institut des Hautes \'Etudes Scientifiques (IHES); the IHES years have been referred to as his `Golden Age', when an entire new school of abstract mathematics flourished under Grothendieck's leadership; thus, Grothendieck's S\'eminaire de G\'eom\'etrie Alg\`ebrique \cite{ALEXsem1, Alexsem2} established IHES as the World's Center of algebraic geometry during the 1960s, with Grothendieck as its driving force. He travelled widely across Europe, including Eastern Europe (such as the invited visit he made in the Summer of 1968 when he delivered a lecture at the School of Mathematics in Bucharest at the invitation c/o Acad. Prof. Dr. Miron Nicolescu of the Romanian Academy ({\em supported after 1866 by Prince Charles von Hohenzollern-Sigmaringen--later in 1881--King Carol I of Romania}), and across the World. He is, and was, a very strong pacifist with very high ideals and goals, of real honesty and also extreme modesty; Alex campaigned against the military built-up of the 1960s, which built-up almost ended up in total annihilation of the planet during the Cuban missile crisis, and this is the reason for which he declined to go to Moscow to collect his Fields Medal in 1966 as a protest against the very aggressive, rapid, and indeed extremely dangerous, build-up of nuclear missiles by the Soviets.

 Alexander Grothendieck's work during the `Golden Age' period established unifying themes in: algebraic geometry, number theory, topology and functional/complex analysis. Grothendieck introduced his own `theory of schemes' in the 1960's which allowed two of Andr\'e Weil's number theory conjectures to be solved by Alex, but in a much more general context
than originally envisaged by Andr\'e Weil. One of his former French students, then proved the third Weil conjecture by a mathematical `bypass' and his former student collected the Crafoord prize instead of Alex who publicly declined the prize in a letter that was only partially  published in the French magazine ``Le Monde" (The World). His stated
reasons were many, but in essence, he wanted to draw the attention of the entire world to the fact that mathematical
work was used in peacetime for war-related purposes which he very strongly disapproved of, and also that the mathematical profession and community that he knew had serious internal problems related to how mathematicians' work and results were both evaluated and rewarded (mostly in France of the late 1970's and 80's!).  

 He worked on the theory of topoi/toposes that are relevant not only to mathematical logic and category theory, but also to algebraic geometry, number theory, computer software/programming and institutional ontology classification and bioinformatics. He provided an algebraic proof of one of the Riemann-Roch theorems, algebraic definition of the fundamental group of a curve, the definition of the fundamental functor for a categorical Galois theory, the re-definition of Abelian categories, (as for example in the case of $A b5$ categories that carry his name-the Grothendieck and local Grothendieck categories), he outlined the 
\PMlinkexternal{`Dessins d' Enfants'}{http://planetmath.org/?op=getobj&from=lec&id=78} 
combinatorial topology theory and much, much more. His S\'eminaires de G\'eometrie alg\`ebriques alone are several thousands of pages in (typewritten) printed length, or close to 500 Mb in electronic format.

In the early 1970s he was a Visiting Professor at the Coll\`ege de France, then at Orsay and Montpellier.

He retired in 1988 as an Emeritus Professor from the (French) University of Montpellier, and currently lives in France.


\begin{thebibliography}{99}

\bibitem{Alex2}
Alexander Grothendieck. 1957, Sur quelque point d-alg\`{e}bre homologique. , \emph{Tohoku Math. J.}, \textbf{9:} 119-121.

\bibitem{Alex3}
Alexander Grothendieck and J. Dieudon\'{e}.: 1960, El\'{e}ments de geometrie alg\`{e}brique., \emph{Publ. Inst. des Hautes Etudes de Science}, \textbf{4}.

\bibitem{ALEXsem1}
Alexander Grothendieck et al.,1971. S\'eminaire de G\'eom\'etrie Alg\`ebrique du Bois-Marie, Vol. 1--7, Berlin: Springer-Verlag.

\bibitem{Alexsem2}
Alexander Grothendieck. 1962. S\'eminaire de G\'eom\'etrie Alg\`ebrique du Bois-Marie, Vol. 2 - Cohomologie Locale des Faisceaux Coh\'erents et Th\'eor\'mes de Lefschetz Locaux et Globaux. , pp.287. (with an additional contributed expos\'e by Mme. Michele Raynaud).
\PMlinkexternal{Typewritten manuscript available in French}{http://modular.fas.harvard.edu/sga/sga/2/index.html};
\PMlinkexternal{see also a brief summary in English}{http://planetmath.org/?op=getobj&from=books&id=78}


\bibitem{AlexEsqP84}
Alexander Grothendieck, 1984. ``Esquisse d'un Programme'', (1984 manuscript), finally published in ``Geometric Galois Actions'', L. Schneps, P. Lochak, eds., London Math. Soc. Lecture Notes 242, Cambridge University Press, 1997, pp.5-48; English transl., ibid., pp. 243-283. MR 99c:14034 . 

\end{thebibliography}
%%%%%
%%%%%
\end{document}
