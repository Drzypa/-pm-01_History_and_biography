\documentclass[12pt]{article}
\usepackage{pmmeta}
\pmcanonicalname{HjalmarMellin}
\pmcreated{2015-02-17 14:49:42}
\pmmodified{2015-02-17 14:49:42}
\pmowner{pahio}{2872}
\pmmodifier{pahio}{2872}
\pmtitle{Hjalmar Mellin}
\pmrecord{8}{40201}
\pmprivacy{1}
\pmauthor{pahio}{2872}
\pmtype{Biography}
\pmcomment{trigger rebuild}
\pmclassification{msc}{01A55}
\pmclassification{msc}{01A60}
\pmsynonym{Mellin}{HjalmarMellin}
%\pmkeywords{function theory}
\pmrelated{MellinsInverseFormula}

\endmetadata

% this is the default PlanetMath preamble.  as your knowledge
% of TeX increases, you will probably want to edit this, but
% it should be fine as is for beginners.

% almost certainly you want these
\usepackage{amssymb}
\usepackage{amsmath}
\usepackage{amsfonts}

% used for TeXing text within eps files
%\usepackage{psfrag}
% need this for including graphics (\includegraphics)
%\usepackage{graphicx}
% for neatly defining theorems and propositions
 \usepackage{amsthm}
% making logically defined graphics
%%%\usepackage{xypic}

% there are many more packages, add them here as you need them

% define commands here

\theoremstyle{definition}
\newtheorem*{thmplain}{Theorem}

\begin{document}
\PMlinkescapeword{inverse}
Robert Hjalmar Mellin (1854--1933), a Finnish function-theorist.  He studied in Helsinki University under G\"osta Mittag-Leffler, in Berlin under Karl Weierstrass and Leopold Kronecker.  He worked as professor of mathematics in the Helsinki Polytechnic Institute (later the Technical University of Finland).

Mellin is best known for his integral transform, the 
{\it \PMlinkname{Mellin transform}{MellinTransform}}
$$F(s) := \int_0^\infty t^{s-1}f(t)\,dt,$$
which he utilised in study of gamma function, hypergeometric functions, Dirichlet series, Riemann zeta function and related number-theoretic functions.  Mellin's transform and its inverse transform
$$f(t) = \frac{1}{2\pi i}\int_{a-i\infty}^{a+i\infty}t^{-z}F(z)\,dz$$
are much used also in physics.  Mellin himself applied his transformations for solving partial differential equations and the inverse transformations for forming asymptotic series expansions.

Mellin has addressed \PMlinkescapetext{strict criticism to the foundations of Einstein's theory} of relativity.

%%%%%
%%%%%
\end{document}
