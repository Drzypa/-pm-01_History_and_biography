\documentclass[12pt]{article}
\usepackage{pmmeta}
\pmcanonicalname{RhindMathematicalPapyrus}
\pmcreated{2014-05-06 6:03:20}
\pmmodified{2014-05-06 6:03:20}
\pmowner{milogardner}{13112}
\pmmodifier{milogardner}{13112}
\pmtitle{Rhind Mathematical Papyrus}
\pmrecord{81}{41040}
\pmprivacy{1}
\pmauthor{milogardner}{13112}
\pmtype{Definition}
\pmcomment{trigger rebuild}
\pmclassification{msc}{01A16}
\pmsynonym{Egyptian math}{RhindMathematicalPapyrus}
%\pmkeywords{number theory}
\pmdefines{Egyptian fractions}

\endmetadata

% this is the default PlanetMath preamble.  as your knowledge
% of TeX increases, you will probably want to edit this, but
% it should be fine as is for beginners.

% almost certainly you want these
\usepackage{amssymb}
\usepackage{amsmath}
\usepackage{amsfonts}

% used for TeXing text within eps files
%\usepackage{psfrag}
% need this for including graphics (\includegraphics)
%\usepackage{graphicx}
% for neatly defining theorems and propositions
%\usepackage{amsthm}
% making logically defined graphics
%%%\usepackage{xypic}

% there are many more packages, add them here as you need them

% define commands here

\begin{document}
\PMlinkexternal{Rhind Mathematical Papyrus}{http://en.wikipedia.org/wiki/Rhind_Mathematical_Papyrus}, 
From Wikipedia

The Rhind Mathematical Papyrus (RMP) has been housed in the British Museum since 1864, and first published in 1879. Germans had visited the British Museum and published the text without permission, the first of many controversies associated with decoding the scribal contents in an attested historical context. 

The papyrus is named after Alexander Henry Rhind a modern era owner.  Rhind, a Scottish antiquarian, purchased the papyrus in Luxor , Egypt in 1858. The papyrus was apparently found during an illegal excavation near the Ramesseum. The British Museum(BM) acquired the RMP in 1864 along with the Egyptian Mathematical Leather Roll (EMLR). Both documents were gifted to the BM after the untimely death of Henry Rhind. There are fragments of the RMP held by the Brooklyn Museum in New York. 

The RMP is one of two well-known Middle Kingdom mathematical papyri. The second is the Moscow Mathematical Papyrus (MMP). The RMP contains 87 problems. The MMP contains 25 problems. [1] A third lesser known text, the Akhmim Wooden Tablet (AWT) added an exact hekat unity scaling issue, 1 hekat = (64/64) hekat, and its encoding method was used by Ahmes over 60 times. The hekat scaling method used in the RMP was not fully decoded until 2006.

The RMP dates to the Second Intermediate Period of Egypt and is the best example of Egyptian mathematics. The RMP was partly copied by its scribe Ahmes  (Ahmose) from  a now-lost text likely written during the reign of king Amenemhat III (12th dynasty). The hieratic papyrus is 33 cm tall and over 5 meters long has been fairly transliterated in the 19th, 20th and 21st centuries. Translations from the 19th, 20th and 21st centuries differ in subtle ways.
The document dates to Year 33 of the Hyksos king Apophis and contains a later date, Year 11 on its verso likely from his successor, Khamudi.[2].

In the opening paragraphs of the papyrus, Ahmes presents the papyrus as giving - Accurate reckoning for inquiring into things, and the knowledge of all things, mysteries…all secrets.

Contents: 1 Mathematical problems 2. Mathematical knowledge 3. Conclusion.

1. Mathematical problems

The papyrus was written on both sides. The text began with a RMP 2/n table, followed by 87 problems. Taking one third of the manuscript, the RMP 2/n table expressed 2 divided by the odd numbers from 5 to 101 in terms of unit fraction series, written in increasingly larger denominators. Unit fractions were not repeated. Ahmes conversion method was parsed three fragmented ways by 20th century and earlier researchers. One way was identified by F. Hultsch in 1895, and confirmed by E.M. Bruins in 1945. Today the aliquot part method is called the Hultsch-Bruins (H-B) method. A second way was red auxiliary numbers noted by least common multipliers (by Gillings and others). Two of the LCM ways are implied by the student- based Egyptian Mathematical Leather Roll, selected single and double LCMs per rational number conversion. A third way, considered in 2001, factored 2/95 = (2/19)(1/5) before the Hultsch-Bruins Method and red auxiliary numbers were applied. Improved translations of the 2/n table and its ancient scribal meanings are being proposed after 2008.

Ahmes central rational number conversion method used red auxiliary numbers, a subtle decoding change fully published in the 21st century. A fresh translation of RMP 36, the EMLR, and the Akhmim Wooden Tablet all scaled n/p by the best LCM (m/m) used in the 2/n table. 
Ahmes can now be read as scaling n/p and 2/n vulgar fractions to concise, but not always optimal, Egyptian fraction series by selecting the best LCM m. The best LCM m scaled n/p to mn/mp and 2/n to 2n/nm within red auxiliary shorthand notes. The best divisors of denominators summed numerators mn and 2n that were written in red. The red auxiliary numbers allowed exact unit fraction series to be recorded. By 2008, 2009, and 2011, it was clear that Ahmes’ red auxiliary number method converted 2/95 by selecting 12 as a LCM 12 such that:

(2/95)(12/12) = 24/(12)(95) = (19 + 3 + 2)/1140 = 1/60 + 1/380 + 1/570

Confirmation of the 2008 analysis included AWT scalings of the hekat by 64/ 64 and 320 ro was confirmed in 2009 data from RMP 36 and RMP 81 data. The 2008 and 2009 analyses showed that Ahmes frequently omitted aspects of the three of the intermediate steps within an unclear shorthand notation. In 2011 the Ahmes’ shorthand omissions were corrected by publishing ‘ab initio’ longhand statements of rational numbers in all RMP problems and over 60 examples of hekat scaled by (64/64) and 320 ro. 
Concerning geometry, RMP 41, RMP 42 and RMP 43 applied algebra to validate three geometric formulas, one for the area of a circle and two for volumes of circular granaries.

Confusions existed for over 100 years related to fully decoding Ahmes’ arithmetic operations in a validated historical context. This confusion caused 20th century math historians to improperly propose incorrect ’ab initio’ arithmetic operations like ‘false position’. For example, Gillings in 1972 reported valid LCM multipliers by suggesting ‘false position’ defined scribal division.  Oddly, the majority of 20th century historians agreed that ‘false position’was used by Ahmes and other hieratic math texts, likely the reason that Gillings repeated the misleading suggestion. 

Scholars in the 1920s had guessed at Ahmes’ calculation steps, often by citing three steps,  all of which were not made clear until the AWT, the EMLR and EMLR was fully decoded in 2008, without using ‘false position’. Scribal division followed a modern rule, invert and multiple.

Appying Ahmes’ three-step red reference number method, the RMP’s 87 problems began with six division by 10 problems, the subject of the Reisner Papyrus. RMP 1-6 selected optimized red reference numbers to calculate quotient and exact remainder answers. Confusingly, Ahmes’ shorthand notes only outlined duplation proofs, the context in which 1920s scholars suggested dominated scribal methods. Ahmes, however, showed that n/10 divisions were returned to unity by multiplying answers by its divisor (a point mentioned by Robins-Shute in 1987, but minimized by 1920s math historians). There were 15 problems that dealt with addition, and 18 algebra problems. There were 15 algebra problems of the same type. The algebra problems asked the reader to find x and a fraction of x such that their sum equals an integer.
RMP 33 solved

x +(2/3 + 1/2+ 1/7)x = 37

x + (55/42)x = 37

(97/42)x = 37

x = 1554/97

as modern arithmetic operations require today, with division defined as the inverse to modern multiplication method, without one use of scholarly ‘false position’.

Ahmes further considered

x = (16 + 2/97) = (quotient + remainder)

with quotient 16 and remainder 2/97 converted to an Egyptian fraction series as the 2/n table selected LCM (56/56) to reach (112/5432), and the best divisors  (97 + 8 + 7) = 112 so that the entire longhand remainder states:
remainder 

= (2/97)(56/56) = (112/54342) = (97 + 8 + 7)/(5432) = 1/56 + 1/679 + 1/77 meant

(quotient + remainder) = 

16 + 1/56 + 1/679 + 1/776

was Ahmes’  answer. 

In other words, Ahmes’ shorthand notes often began with answers within binary (duplation) procedures that led to definitions of RMP problems. But what did Ahmes arithmetic write, outside of duplation, translated into modern base 10 decimal arithmetic state?

Ahmes shorthand note are now decoded by adding back missing arithmetic facts to reach exact longhand ’ab initio’ translations.

Each of Ahmes 14 algebra shorthand notes converted vulgar fraction answers to exact unit fraction series. All 14 algebraic problem steps can be corrected to ‘ab initio’ longhand translations, as mentioned above, to better grasp Ahmes’ primal thinking process.

Generally, Ahmes unit fraction answers were converted to concise Egyptian fraction series by selecting red reference numbers that scaled rational numbers by LCMs that sometimes are were not cited in shorthand notes.  Ahmes often discussed incomplete shorthand duplation statements and  proofs. Scribal statements and proofs varied in completeness. Ahmes shorthand omitted important arithmetical facts that can be decoded into readable base 10 decimal statements by preparing corrected longhand translations.

In geometry, Ahmes’  problems 41 and 42 modified the area (A) of a circle formula by replacing radius (R) with diameter (D/2), with  pi estimated by 256/81 that  added volume (V) by multiplying by height (H) such that

Sqrt(A) =(8/9)(D) cubits squared (algebraic geometry formula 1.0)

reported in the MMP 300 years earlier, and

V = (3/2)[(8/9)(D)(8/9)(D)] cubits cubed (algebraic geometry formula 1.1)

RMP 43 scaled algebraic geometry formula 1.1 by 3/2 considering

(3/2)Sqrt(A) = (3/2)(8/9)(D) = (4/3)D

and

V = [(2/3)H(4/3)(D)(4/3)(D)]khar (algebraic geometry formula 1.2)

a formula reported in the Kahun Papyrus 250 years earlier was implied by RMP 47 and the heka .
Two arithmetical progressions (A.P.) were solved in RMP 40 and RMP 64. A common method was defined and applied in the Kahun Papyrus(KP). RMP 64 problem solved the A.P. sharing 10 hekats of barley, between 10 men, with a difference of 1/8th of a hekat implied 1 + 7/16 as the largest term.
The second A.P. problem was RMP 40. The problem divided 100 loaves of bread between five men such that the smallest two shares were 1/7 of the largest three shares’ sum (87 1/2). Ahmes found the shares for each man, which he did without finding the difference (9 1/6) or the largest term (38 1/3). All five shares 38 1/3, 29 1/6, 20, 10 2/3 1/6, and 1 1/3) were calculated by first finding the five terms from a proportional A.P. that summed to 60. The median and the smallest term, x1, were used to find the differential and each term. Ahmes then multiplied each term by 1 2/3 to obtain the sum to 100 A.P. terms. 
                                                                            
                                                                           X + 7x  =  60
                                                                           
In RMP 81, 29 examples of Akhmim Wooden Tablet partitioned a hekat unity (64/64) by rational numbers 1/64 ¡ n ¡ 64. In addition 15 RMP problems were similar to Moscow Mathematical Papyrus problems. 

Finally, 23 RMP weights and measures problems often discussed the hekat, with three recreational diversion problems following, the last being RMP 84, the famous multiple of 7 riddle, passed down to the Medieval era as As I was going to St. Ives from other sources.

The Rhind Mathematical Papyrus also contains geometry related problems:[3]
”If a pyramid is 250 cubits high and the side of its base 360 cubits long, what is its seked?”

The solution to the problem is given as the ratio of half the side of the base of the pyramid to its height, or the run-to-rise ratio of its face. In other words, the quantity he found for the seked is the cotangent of the angle to the base of the pyramid and its face.[3]

2. Mathematical knowledge

Upon closer inspection, modern-day mathematical analysis of Ahmes’ problem-solving strategies reveal a basic awareness of composite and prime numbers;[4] arithmetic, geometric and harmonic means;[4] a simplistic understanding of the Sieve of Eratosthenes[4], and perfect numbers.[4][5]

The papyrus also demonstrates knowledge of solving first order linear equations[5] and summing arithmetic and geometric series.[5]

The papyrus formally calculated Pi as (8/9) squared, 256/81 in RMP 41, 42, 43 and 44. However, earlier in RMP 38 Ahmes informally corrected granary losses by a Pi estimate of  22/7.

Other problems in the Rhind papyrus demonstrate knowledge of number theory, arithmetic progressions, algebra and geometry.

The papyrus also demonstrated knowledge of weights and measures, business, and recreational diversions.

3. Conclusion

Ahmes’ 2/n table, which took up 1/3 of the entire papyrus, was the fundamental numerical foundation of the Middle Kingdom. Ahmes used the system to scale rational numbers (2/n) and (n/p) by LCMs (m/m) by multiplication to reach (mn/mp) such that red auxiliary numbers, the best devisors of (mp), were summed to numerator (mn) calculated concise unit fraction series. Within the RMP’s 87 problems 1920s scholars suggested scribal ‘false position’ was the scribal division operation, which it was not. Only a modern division rule, an inverse multiplication operation, corrects scribal shorthand, exactly adds back missing steps, to translate scribal shorthand omissions to scribal longhand arithmetic statements that are double checked against other Middle Kingdom math documents in the 21st century.

The RMP document is slowly disclosing its subtle number theory, and deeper math secrets as one of the main sources of ancient Middle Kingdom Egyptian mathematics. To confirm the historical contents and scribal methods used by Ahmes, the EMLR, MMP, KP and the AP must be consulted in ways that 21st century scholars are learning to apply


\begin{thebibliography}{9}
\bibitem{1} Milo Gardner, \emph{The Egyptian Mathematical Leather Roll Attested Short Term and Long Term, History of Mathematical Sciences}, Hindustan Book Company, 2002.
\bibitem{2} Milo Gardner, \emph{An Ancient Egyptian Problem and its Innovative Solution, Ganita Bharati}, MD Publications Pvt Ltd, 2006.
\bibitem{3}Richard Gillings, \emph{Mathematics in the Time of the Pharaohs}, Dover Books, 1992.
\bibitem{4} Oystein Ore, \emph{Number Theory and its History}, McGraw-Hill Books, 1948, Dover reprints available.
\bibitem{5} T.E. Peet, \emph{Arithmetic in the Middle Kingdom}, Journal Egyptian Archeology, 1923.
\bibitem{6} Tanja Pommerening, \emph{"Altagyptische Holmasse Metrologish neu Interpretiert" and relevant phramaceutical and medical knowledge, an abstract,  Phillips-Universtat, Marburg, 8-11-2004, taken from "Die Altagyptschen Hohlmass}, Buske-Verlag, 2005.
\bibitem{7} Gay Robins, and Charles Shute \emph{Rhind Mathematical Papyrus}, British Museum Press, Dover reprint, 1987.
\bibitem{8} L.E. Sigler, \emph{Fibonacci's Liber Abaci: Leonardo Pisano's Book of Calculation}, Springer, 2002.
\bibitem{9} Hana Vymazalova, \emph{The Wooden Tablets from Cairo:The Use of the Grain Unit HK3T in Ancient Egypt, Archiv Orientalai}, Charles U Prague, 2002.
\end{thebibliography}




%%%%%
%%%%%
\end{document}
