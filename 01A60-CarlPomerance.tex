\documentclass[12pt]{article}
\usepackage{pmmeta}
\pmcanonicalname{CarlPomerance}
\pmcreated{2013-03-22 16:56:00}
\pmmodified{2013-03-22 16:56:00}
\pmowner{PrimeFan}{13766}
\pmmodifier{PrimeFan}{13766}
\pmtitle{Carl Pomerance}
\pmrecord{4}{39199}
\pmprivacy{1}
\pmauthor{PrimeFan}{13766}
\pmtype{Biography}
\pmcomment{trigger rebuild}
\pmclassification{msc}{01A60}
\pmclassification{msc}{01A61}
\pmclassification{msc}{01A65}

% this is the default PlanetMath preamble.  as your knowledge
% of TeX increases, you will probably want to edit this, but
% it should be fine as is for beginners.

% almost certainly you want these
\usepackage{amssymb}
\usepackage{amsmath}
\usepackage{amsfonts}

% used for TeXing text within eps files
%\usepackage{psfrag}
% need this for including graphics (\includegraphics)
%\usepackage{graphicx}
% for neatly defining theorems and propositions
%\usepackage{amsthm}
% making logically defined graphics
%%%\usepackage{xypic}

% there are many more packages, add them here as you need them

% define commands here

\begin{document}
\emph{Carl Pomerance} (1944 - ) American mathematician, discoverer of the quadratic sieve method for integer factorization, co-author with Richard Crandall of the seminal book {\it Prime Numbers: A Computational Perspective}.

Born in Missouri, Pomerance studied at Brown and later Harvard. After graduation in 1972, he taught at the University of Georgia for the rest of the century. He worked for four years at Lucent Technologies (now Alcatel-Lucent), then joined the math faculty of Dartmouth College at about the same time he was appointed a Fellow of the American Association for the Advancement of Science. Early in his teaching career, Pomerance was recognized with the Chauvenet Prize.

In 1978, Pomerance co-authored with Erd\H{o}s a paper on the largest prime factors of $n$ and $n + 1$ in {\it Aequationes Math.} {\bf 17}, giving him \PMlinkname{Erd\H{o}s number}{ErdHosNumber} 1. (In 1969, when fellow Erd\H{o}s number 1 collaborator Hank Aaron broke Babe Ruth's homerun record with 715, Pomerance began studying Ruth-Aaron pairs). The greatest recognition for Pomerance, however, might be that the use of his quadratic sieve method cracked RSA-129.
%%%%%
%%%%%
\end{document}
