\documentclass[12pt]{article}
\usepackage{pmmeta}
\pmcanonicalname{JessieMacWilliams}
\pmcreated{2013-03-22 17:18:50}
\pmmodified{2013-03-22 17:18:50}
\pmowner{PrimeFan}{13766}
\pmmodifier{PrimeFan}{13766}
\pmtitle{Jessie MacWilliams}
\pmrecord{4}{39663}
\pmprivacy{1}
\pmauthor{PrimeFan}{13766}
\pmtype{Biography}
\pmcomment{trigger rebuild}
\pmclassification{msc}{01A60}

% this is the default PlanetMath preamble.  as your knowledge
% of TeX increases, you will probably want to edit this, but
% it should be fine as is for beginners.

% almost certainly you want these
\usepackage{amssymb}
\usepackage{amsmath}
\usepackage{amsfonts}

% used for TeXing text within eps files
%\usepackage{psfrag}
% need this for including graphics (\includegraphics)
%\usepackage{graphicx}
% for neatly defining theorems and propositions
%\usepackage{amsthm}
% making logically defined graphics
%%%\usepackage{xypic}

% there are many more packages, add them here as you need them

% define commands here

\begin{document}
\PMlinkescapeword{degree}
\PMlinkescapeword{even}
\PMlinkescapeword{path}

\emph{Jessie MacWilliams} (1917 - 1990) British mathematician, best known for her work on error-correcting codes with Neil Sloane at the Bell Labs.

Born in Stoke-on-Trent, MacWilliams earned a bachelor's and a master's degree from Cambridge, then moved to the United States to continue her education at Harvard. After Harvard she began working at Bell Labs, where she assisted Sloane and co-authored with him {\it The Theory of Error-Correcting Codes}.

Because of her work with Sloane, one might guess MacWilliams has \PMlinkname{Erd\H{o}s number}{ErdHosNumber} 3, but there's actually an even shorter path: With Jacobus Hendricus van Lint, MacWilliams wrote a paper on ``Generalized quadratic residue codes'' for {\it IEEE Trans. Inform. Theory} {\bf 24}. Lint in turn wrote with Erd\H{o}s ``On the number of positive integers $\leq x$ and free of prime factors $> y$'' in {\it Simon Stevin} {\bf 40}.
%%%%%
%%%%%
\end{document}
