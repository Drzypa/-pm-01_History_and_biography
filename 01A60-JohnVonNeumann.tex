\documentclass[12pt]{article}
\usepackage{pmmeta}
\pmcanonicalname{JohnVonNeumann}
\pmcreated{2013-03-22 16:32:57}
\pmmodified{2013-03-22 16:32:57}
\pmowner{PrimeFan}{13766}
\pmmodifier{PrimeFan}{13766}
\pmtitle{John von Neumann}
\pmrecord{7}{38732}
\pmprivacy{1}
\pmauthor{PrimeFan}{13766}
\pmtype{Biography}
\pmcomment{trigger rebuild}
\pmclassification{msc}{01A60}
\pmsynonym{Neumann J\'anos Lajos}{JohnVonNeumann}
\pmsynonym{Neumann Janos Lajos}{JohnVonNeumann}
\pmsynonym{J\'anos Lajos Neumann}{JohnVonNeumann}
\pmsynonym{Janos Lajos Neumann}{JohnVonNeumann}
\pmsynonym{Johann von Neumann}{JohnVonNeumann}

\endmetadata

% this is the default PlanetMath preamble.  as your knowledge
% of TeX increases, you will probably want to edit this, but
% it should be fine as is for beginners.

% almost certainly you want these
\usepackage{amssymb}
\usepackage{amsmath}
\usepackage{amsfonts}

% used for TeXing text within eps files
%\usepackage{psfrag}
% need this for including graphics (\includegraphics)
%\usepackage{graphicx}
% for neatly defining theorems and propositions
%\usepackage{amsthm}
% making logically defined graphics
%%%\usepackage{xypic}

% there are many more packages, add them here as you need them

% define commands here

\begin{document}
\PMlinkescapeword{child}
\PMlinkescapeword{power}
\PMlinkescapeword{development}
\PMlinkescapeword{unit}
\PMlinkescapeword{interests}
\PMlinkescapeword{structure}

\emph{John von Neumann}, born \emph{Neumann J\'anos Lajos} (1903 - 1957) Hungarian American mathematician and physicist who made substantial advances in quantum physics, computer science, game theory, economics, etc. Possessing instantaneous eidetic memory and myriad interests, young J\'anos learned calculus as a child. As Hitler gained power in the 1930s, the Neumann family fled to the United States.

von Neumann was part of the Manhattan project, and was involved at Los Alamos on the team led by J. Robert Oppenheimer that resulted in the development of the first fission bomb. He later had a key idea relating to the structure of the hydrogen, or fusion, bomb.

Perhaps the invention for which von Neumann is most widely known today is the development of what is now called the \emph{von Neumann architecture} for digital computers, consisting of the familiar stored-program computer in which the data and program are separate from the computational unit.

In 1957, Neumann died of bone cancer. In 2005, the United States Postal Service honored von Neumann with a stamp (along with Richard Feynman, Josiah Willard Gibbs and Barbara McClintock).
%%%%%
%%%%%
\end{document}
