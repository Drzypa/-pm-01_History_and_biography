\documentclass[12pt]{article}
\usepackage{pmmeta}
\pmcanonicalname{GeorgFrobenius}
\pmcreated{2013-03-22 17:24:45}
\pmmodified{2013-03-22 17:24:45}
\pmowner{PrimeFan}{13766}
\pmmodifier{PrimeFan}{13766}
\pmtitle{Georg Frobenius}
\pmrecord{7}{39785}
\pmprivacy{1}
\pmauthor{PrimeFan}{13766}
\pmtype{Biography}
\pmcomment{trigger rebuild}
\pmclassification{msc}{01A55}
\pmclassification{msc}{01A60}
\pmsynonym{Ferdinand Georg Frobenius}{GeorgFrobenius}

\endmetadata

% this is the default PlanetMath preamble.  as your knowledge
% of TeX increases, you will probably want to edit this, but
% it should be fine as is for beginners.

% almost certainly you want these
\usepackage{amssymb}
\usepackage{amsmath}
\usepackage{amsfonts}

% used for TeXing text within eps files
%\usepackage{psfrag}
% need this for including graphics (\includegraphics)
%\usepackage{graphicx}
% for neatly defining theorems and propositions
%\usepackage{amsthm}
% making logically defined graphics
%%%\usepackage{xypic}

% there are many more packages, add them here as you need them

% define commands here

\begin{document}
\emph{Ferdinand Georg Frobenius} (1849 - 1917) German mathematician, best known for being the first to prove the Cayley-Hamilton theorem. Several different mathematical concepts now bear his name, including the Frobenius algebra, various Frobenius morphisms, the Frobenius map, the Frobenius group, the Frobenius inequality, etc.

Born in Charlottenburg, Frobenius earned a doctorate at the University of Berlin for his thesis on differential equations with \PMlinkname{Karl Weierstra{\ss}}{KarlWeierstrass} as his advisor. After that he taught mathematics in Berlin and Z\"urich, and researched group theory (proving the existence of Sylow groups) and number theory. In the lead-up to World War I, Frobenius began investigating matrices. He died in Berlin in 1917 and was buried there.
%%%%%
%%%%%
\end{document}
