\documentclass[12pt]{article}
\usepackage{pmmeta}
\pmcanonicalname{BourbakiNicolas}
\pmcreated{2013-03-22 13:33:23}
\pmmodified{2013-03-22 13:33:23}
\pmowner{Daume}{40}
\pmmodifier{Daume}{40}
\pmtitle{Bourbaki, Nicolas}
\pmrecord{17}{34161}
\pmprivacy{1}
\pmauthor{Daume}{40}
\pmtype{Biography}
\pmcomment{trigger rebuild}
\pmclassification{msc}{01A60}
\pmsynonym{Bourbaki}{BourbakiNicolas}
\pmrelated{AlexanderGrothendieckABiographyOf}

% this is the default PlanetMath preamble.  as your knowledge
% of TeX increases, you will probably want to edit this, but
% it should be fine as is for beginners.

% almost certainly you want these
\usepackage{amssymb}
\usepackage{amsmath}
\usepackage{amsfonts}

% used for TeXing text within eps files
%\usepackage{psfrag}
% need this for including graphics (\includegraphics)
\usepackage{graphicx}
% for neatly defining theorems and propositions
%\usepackage{amsthm}
% making logically defined graphics
%%%\usepackage{xypic} 

% there are many more packages, add them here as you need them

% define commands here
\begin{document}
\PMlinkescapeword{mean}
\PMlinkescapeword{integral}
\PMlinkescapeword{moment}
\PMlinkescapeword{perfect}
\PMlinkescapeword{name}
\PMlinkescapeword{series}
\PMlinkescapeword{complete}
\PMlinkescapeword{field}
\PMlinkescapeword{fixed}
\PMlinkescapeword{sources}
\PMlinkescapeword{structure}
\PMlinkescapeword{simple}
\PMlinkescapeword{presentation}
\PMlinkescapeword{presentations}
\PMlinkescapeword{open}
\PMlinkescapeword{body}
\PMlinkescapeword{group}
\PMlinkescapeword{groups}
\PMlinkescapeword{row}
\PMlinkescapeword{order}
\PMlinkescapeword{ball}
\PMlinkescapeword{names}
\small by \'Emilie Richer\\\normalsize
\section*{The Problem}\normalsize

The devastation of World War I presented a unique challenge to aspiring
mathematicians of the mid 1920's. Among the many casualties of the war 
were great numbers of scientists and mathematicians who would at this 
time have been serving as mentors to the young students. Whereas other 
countries such as Germany were sending their scholars to do scientific 
work, France was sending promising young students to the front. A war-time 
directory of the \'ecole Normale Sup\'erieure in Paris confirms that about 
2/3 of their student population was killed in the war.\cite{DJ} Young men 
studying after the war had no young teachers, they had no previous 
generation to rely on for guidance. What did this mean? According to Jean 
Dieudonn\'e, it meant that students like him were missing out on important 
discoveries and advances being made in mathematics at that time. He 
explained : ``I am not saying that they (the older professors) did not 
teach us excellent mathematics (...) But it is indubitable that a 50 year 
old mathematician knows the mathematics he learned at 20 or 30, but has 
only notions, often rather vague, of the mathematics of his epoch, i.e. 
the period of time when he is 50.'' He continued : ``I had graduated from 
the \'ecole Normale and I did not know what an ideal was! This gives you 
and idea of what a young French mathematician knew in 1930.''\cite{DJ} 
Henri Cartan, another student in Paris shortly after the war affirmed : 
``we were the first generation after the war. Before us there was a vide, 
a vacuum, and it was necessary to make everything new.''\cite{JA} This is 
exactly what a few young Parisian math students set out to do.\\


%==============================================[[The Beginnings begins here]]
\section*{The Beginnings}\normalsize

After graduation from the \'ecole Normale Sup\'erieure de Paris a group of 
about ten young mathematicians had maintained very close ties.\cite{WA} 
They had all begun their careers and  were scattered across France teaching 
in universities. Among them were Henri Cartan and Andr\'e Weil who were both 
in charge of teaching a course on differential and integral calculus at the 
University of Strasbourg. The standard textbook for this class at the time was 
``Trait\'e d'Analyse'' by E. Goursat which the young professors found to be 
inadequate in many ways.\cite{BA} According to Weil, his friend Cartan was 
constantly asking him questions about the best way to present a given topic to 
his class, so much so that Weil eventually nicknamed him ``the grand 
inquisitor''.\cite{WA} After months of persistent questioning, in the winter 
of 1934, Weil finally got the idea to gather friends (and former classmates) 
to settle their problem by rewriting the treatise for their course. It is at 
this moment that Bourbaki was conceived.\\


%===============================================[[The Project begins here]]
The suggestion of writing this treatise spread and very soon a loose circle 
of friends, including Henri Cartan, Andr\'e Weil, Jean Delsarte, Jean 
Dieudonn\'e and Claude Chevalley began meeting regularly at the Capoulade, 
a caf\'e in the Latin quarter of Paris to plan it . They called themselves 
the ``Committee on the Analysis Treatise''\cite{BL}. According to Chevalley 
the project was extremely naive. The idea was to simply write another textbook 
to replace Goursat's.\cite{GD} After many discussions over what to include in 
their treatise they finally came to the conclusion that they needed to start 
from scratch and present all of essential mathematics from beginning to end. 
With the idea that ``the work had to be primarily a tool, not usable in some 
small part of mathematics but in the greatest possible number of places''.
\cite{DJ} Gradually the young men realized that their meetings were not 
sufficient, and they decided they would dedicate a few weeks in the summer 
to their new project. The collaborators on this project were not aware of 
its enormity, but were soon to find out.\\


%==============================================[[Getting Started begins here]]
In July of 1935 the young men gathered for their first congress (as they would 
later call them) in Besse-en-Chandesse. The men believed that they would be 
able to draft the essentials of mathematics in about three years. They did not 
set out wanting to write something new, but to perfect everything already 
known.  Little did they know that their first chapter would not be completed 
until 4 years later. It was at one of their first meetings that the young men 
chose their name : Nicolas Bourbaki. The organization and its membership 
would go on to become one of the greatest enigmas of 20th century mathematics.
\begin{center}
\includegraphics[scale=0.5]{b1}\\
\begin{quote}
\footnotesize\itshape The first Bourbaki congress, July 1935. From left to 
right, back row: Henri Cartan, Ren\'e de Possel, Jean Dieudonn\'e, Andr\'e 
Weil, university lab technician, seated: Mirl\`es, Claude Chevalley, Szolem 
Mandelbrojt.
\end{quote}
\end{center}
Andr\'e Weil recounts many years later how they decided on this name. He and 
a few other Bourbaki collaborators had been attending the \'ecole Normale in 
Paris, when a notification was sent out to all first year science students : 
a guest speaker would be giving a lecture and attendance was highly 
recommended. As the story goes, the young students gathered to hear, 
(unbeknownst to them) an older student, Raoul Husson who had disguised himself 
with a fake beard and an unrecognizable accent. He gave what is said to be an 
incomprehensible, nonsensical lecture, with the young students trying 
desperately to follow him. All his results were wrong in a non-trivial way and 
he ended with his most extravagant : Bourbaki's Theorem. One student even 
claimed to have followed the lecture from beginning to end. Raoul had taken 
the name for his theorem from a general in the Franco-Prussian war. The 
committee was so amused by the story that they unanimously chose Bourbaki 
as their name. Weil's wife was present at the discussion about choosing a 
name and she became Bourbaki's godmother baptizing him Nicolas.\cite{WA}  
Thus was born Nicolas Bourbaki.  

Andr\'e Weil, Claude Chevalley, Jean Dieudonn\'e, Henri Cartan and Jean 
Delsarte were among the few present at these first meetings, they were all 
active members of Bourbaki until their retirements. Today they are 
considered by most to be the founding fathers of the Bourbaki group.  
According to a later member they were ``those who shaped Bourbaki and 
gave it much of their time and thought until they retired'' he also claims 
that some other early contributors were Szolem Mandelbrojt and Ren\'e de 
Possel.\cite{BA}\\


%==========================[[Reforming Mathematics: The Idea begins here]]
\section*{Reforming Mathematics : The Idea}\normalsize

Bourbaki members all believed that they had to completely rethink mathematics. 
They felt that older mathematicians were holding on to old practices and 
ignoring the new. That is why very early on Bourbaki established one its 
first and only rules : obligatory retirement at age 50. As explained by 
Dieudonn\'e ``if the mathematics set forth by Bourbaki no longer correspond 
to the trends of the period, the work is useless and has to be redone, this 
is why we decided that all Bourbaki collaborators would retire at age 50.''
\cite{DJ} Bourbaki wanted to create a work that would be an essential tool 
for all mathematicians. Their aim was to create something logically ordered, 
starting with a strong foundation and building continuously on it. The 
foundation that they chose was set theory which would be the first book in a 
series of 6 that they named ``\'el\'ements de math\'ematique''(with the 's' 
dropped from math\'ematique to represent their underlying belief in the unity 
of mathematics). Bourbaki felt that the old mathematical divisions were no 
longer valid comparing them to ancient zoological divisions. The ancient 
zoologist would classify animals based on some basic superficial similarities 
such as ``all these animals live in the ocean''. Eventually they realized that 
more complexity was required to classify these animals. Past mathematicians 
had apparently made similar mistakes : ``the order in which we (Bourbaki) 
arranged our subjects was decided according to a logical and rational scheme. 
If that does not agree with what was done previously, well, it means that 
what was done previously has to be thrown overboard.''\cite{DJ} After many 
heated discussions, Bourbaki eventually settled on the topics for 
``\'el\'ements de math\'ematique'' they would be, in order:
\begin{quote}
I Set theory\\
II Algebra\\
III Topology\\
IV Functions of one real variable\\
V Topological vector spaces\\
VI Integration
\end{quote}
They now felt that they had eliminated all secondary mathematics, that 
according to them ``did not lead to anything of proved importance.''\cite{DJ} 
The following table summarizes Bourbaki's choices.
\begin{center}
\scriptsize\begin{tabular}{|l|l|} \hline
\textbf{What remains after cutting the loose threads} & \textbf{What is 
excluded\textsl{(the loose threads)}} \\ \hline
Linear and multilinear algebra & Theory of ordinals and cardinals \\
A little general topology \textsl{the least possible} & Lattices \\
Topological vector Spaces & Most general topology \\
Homological algebra & Most of group theory \textsl{finite groups} \\
Commutative algebra & Most of number theory \\
Non-commutative algebra & Trigonometrical series \\
Lie groups & Interpolation \\
Integration & Series of polynomials \\
Differentiable manifolds & Applied mathematics \\
Riemannian geometry &  \\ \hline 
\end{tabular}
\end{center}
\begin{quote}
\footnotesize\emph{Dieudonn\'e's metaphorical ball of yarn:} ``here is 
my picture of mathematics now. It is a ball of wool, a tangled hank where 
all mathematics react upon another in an almost unpredictable way. And 
then in this ball of wool, there are a certain number of threads coming 
out in all directions and not connecting with anything else. Well the 
Bourbaki method is very simple-we cut the threads.''\cite{DJ}  
\end{quote}
\normalsize


%==========================[[Reforming Mathematics: The Process begins here]]
\section*{Reforming Mathematics : The Process}\normalsize

It didn't take long for Bourbaki to become aware of the size of their project. 
They were now meeting three times a year (twice for one week and once for two 
weeks) for Bourbaki ``congresses'' to work on their books. Their main rule was 
unanimity on every point. Any member had the right to veto anything he felt 
was inadequate or imperfect. Once Bourbaki had agreed on a topic for a chapter 
the job of writing up the first draft was given to any member who wanted it. 
He would write his version and when it was complete it would be presented at 
the next Bourbaki congress. It would be read aloud line by line. According 
to Dieudonn\'e ``each proof was examined point by point and criticized 
pitilessly. He goes on ``one has to see a Bourbaki congress to realize the 
virulence of this criticism and how it surpasses by far any outside attack.''
\cite{DJ} Weil recalls a first draft written by Cartan (who has unable to attend 
the congress where it would being presented). Bourbaki sent him a telegram 
summarizing the congress, it read : ``union intersection partie produit tu 
es d\'emembr\'e foutu Bourbaki'' (union intersection subset product you are 
dismembered screwed Bourbaki).\cite{WA} During a congress any member was 
allowed to interrupt to criticize, comment or ask questions at any time. 
Apparently Bourbaki believed it could get better results from confrontation 
than from orderly discussion.\cite{BA} Armand Borel, summarized his first 
congress as ``two or three monologues shouted at top voice, seemingly 
independent of one another''.\cite{BA} 
\begin{center}
\includegraphics[scale=0.5]{b2}\\
\footnotesize\itshape Bourbaki congress 1951.
\end{center}
After a first draft had been completely reduced to pieces it was the job of 
a new collaborator to write up a second draft. This second collaborator 
would use all the suggestions and changes that the group had put forward 
during the congress. Any member had to be able to take on this task because 
one of Bourbaki's mottoes  was ``the control of the specialists by the 
non-specialists''\cite{BA} i.e. a member had to be able to write a chapter 
in a field that was not his specialty. This second writer would set out on 
his assignment knowing that by the time he was ready to present his draft 
the views of the congress would have changed and his draft would also be 
torn apart despite its adherence to the congress' earlier suggestions. 
The same chapter might appear up to ten times before it would finally be 
unanimously approved for publishing. There was an average of 8 to 12 years 
from the time a chapter was approved to the time it appeared on a bookshelf.
\cite{DJ} Bourbaki proceeded this way for over twenty years, (surprisingly) 
publishing a great number of volumes.
\begin{center}
\includegraphics[scale=0.5]{b3}\\
\footnotesize\itshape Bourbaki congress 1951.
\end{center}

%==============================[[Recruitment and Membership begins here]]
\section*{Recruitment and Membership}\normalsize

During these years, most Bourbaki members held permanent positions at 
universities across France. There, they could recruit for Bourbaki, students 
showing great promise in mathematics. Members would never be replaced formally 
nor was there ever a fixed number of members. However when it felt the need, 
Bourbaki would invite a student or colleague to a congress as a ``cobaye'' 
(guinea pig). To be accepted, not only would the guinea pig have to understand 
everything, but he would have to actively participate. He also had to show 
broad interests and an ability to adapt to the Bourbaki style. If he was 
silent he would not be invited again.(A challenging task considering he 
would be in the presence of some of the strongest mathematical minds of the 
time) Bourbaki described the reaction of certain guinea pigs invited to a 
congress : ``they would come out with the impression that it was a gathering 
of madmen. They could not imagine how these people, shouting -sometimes three 
or four at a time- about mathematics, could ever come up with something 
intelligent.''\cite{DJ} If a new recruit was showing promise, he would continue 
to be invited and would gradually become a member of Bourbaki without any 
formal announcement. Although impossible to have complete anonymity, Bourbaki 
was never discussed with the outside world. It was many years before Bourbaki 
members agreed to speak publicly about their story. The following table gives 
the names of some of Bourbaki's collaborators.
\begin{center}
\scriptsize
\begin{tabular}{|l|l|l|} \hline
\textbf{$1^{st}$ generation \textsl{(founding fathers)}} & \textbf{$2^{nd}$ 
generation \textsl{(invited after WWII)}} & \textbf{$3^{rd}$ generation} \\ 
\hline
H. Cartan & J. Dixmier & A. Borel \\
C. Chevalley & R. Godement & F. Bruhat \\
J. Delsarte & S. Eilenberg & P. Cartier \\
J. Dieudonn\'e & J.L. Koszul & A. Grothendieck \\
A. Weil & P. Samuel & S. Lang \\
& J.P Serre & J. Tate \\
& L. Shwartz & \\ \hline  
\end{tabular}
\end{center}
\begin{quote}
\footnotesize\emph{3 Generations of Bourbaki} \textsl{(membership according 
to Pierre Cartier)}\cite{SM}. Note: There have been a great number of Bourbaki 
contributors, some lasting longer than others, this table gives the members 
listed by Pierre Cartier. Different sources list different ``official 
members'' in fact the Bourbaki website lists J.Coulomb, C.Ehresmann, 
R.de Possel and S. Mandelbrojt as $1^{st}$ generation members.\cite{BW}
\end{quote}
\begin{center}
\includegraphics[scale=0.5]{b5}\\
\begin{quote}
\footnotesize\itshape Bourbaki congress 1938, from left to right: S. Weil, C. 
Pisot, A. Weil, J. Dieudonn\'e, C. Chabauty, C. Ehresmann, J. Delsarte.
\end{quote}
\end{center}
\normalsize


%===============================[[Recruitment and Membership begins here]]
\section*{The Books}\normalsize

The Bourbaki books were the first to have such a tight organization, the first 
to use an axiomatic presentation. They tried as often as possible to start from 
the general and work towards the particular. Working with the belief that 
mathematics are fundamentally simple and for each mathematical question there 
is an optimal way of answering it. This required extremely rigid structure 
and notation. In fact the first six books of ``\'el\'ements de math\'ematique'' 
use a completely linearly-ordered reference system. That is, any reference at 
a given spot can only be to something earlier in the text or in an earlier 
book. This did not please all of its readers as Borel elaborates : ``I was 
rather put off by the very dry style, without any concession to the reader, 
the apparent striving for the utmost generality, the inflexible system of 
internal references and the total absence of outside ones''. However, 
Bourbaki's style was in fact so efficient that a lot of its notation and 
vocabulary is still in current usage. Weil recalls that his granddaughter 
was impressed when she learned that he had been personally responsible for 
the symbol $\emptyset$ for the empty set,\cite{WA} and Chevalley explains 
that to ``bourbakise'' now means to take a text that is considered screwed 
up and to arrange it and improve it. Concluding that ``it is the notion of 
structure which is truly bourbakique''.\cite{GD}

As well as $\emptyset$, Bourbaki is responsible for the introduction of 
the $\Rightarrow$ (the implication arrow), $\mathbb{N}$, $\mathbb{R}$, 
$\mathbb{C}$, $\mathbb{Q}$ and $\mathbb{Z}$ (respectively the natural, 
real, complex, rational numbers and the integers) $C_A$ (complement of 
a set A), as well as the words bijective, surjective and injective.
\cite{DR} 


%=========================================[[The Decline begins here]]
\section*{The Decline}\normalsize

Once Bourbaki had finally finished its first six books, the obvious 
question was ``what next?''. The founding members who (not intentionally) 
had often carried most of the weight were now approaching mandatory 
retirement age. The group had to start looking at more specialized 
topics, having covered the basics in their first books. But was the 
highly structured Bourbaki style the best way to approach these topics? 
The motto ``everyone must be interested in everything'' was becoming much 
more difficult to enforce. (It was easy for the first six books whose 
contents are considered essential knowledge of most mathematicians) 
Pierre Cartier was working with Bourbaki at this point. He says ``in 
the forties you can say that Bourbaki know where to go: his goal was 
to provide the foundation for mathematics''.[12] It seemed now that 
they did not know where to go. Nevertheless, Bourbaki kept publishing. 
Its second series (falling short of Dieudonn\'e's plan of 27 books 
encompassing most of modern mathematics \cite{BA}) consisted of two 
very successful books :
\begin{quote}
Book VII Commutative algebra\\
Book VIII Lie Groups
\end{quote}
However Cartier claims that by the end of the seventies, Bourbaki's 
method was understood, and many textbooks were being written in its 
style : ``Bourbaki was left without a task. (...) With their rigid 
format they were finding it extremely difficult to incorporate new 
mathematical developments''\cite{SM} To add to its difficulties, Bourbaki was 
now becoming involved in a battle with its publishing company over royalties 
and translation rights. The matter was settled in 1980 after a ``long and 
unpleasant'' legal process, where, as one Bourbaki member put it ``both 
parties lost and the lawyer got rich''\cite{SM}. In 1983 Bourbaki published 
its last volume : IX Spectral Theory.

By that time Cartier says Bourbaki was a dinosaur, the head too far away 
from the tail. Explaining : ``when Dieudonn\'e was the ``scribe of Bourbaki'' 
every printed word came from his pen. With his fantastic memory he knew every 
single word. You could say ``Dieudonn\'e what is the result about so and so?'' 
and he would go to the shelf and take down the book and open it to the right 
page. After Dieudonn\'e retired no one was able to do this. So Bourbaki lost 
awareness of his own body, the 40 published volumes.''\cite{SM} Now after 
almost twenty years without a significant publication is it safe to say the 
dinosaur has become extinct?\footnote{Today what remains is  ``L'Association 
des Collaborateurs de Nicolas Bourbaki'' who organize Bourbaki seminars three 
times a year. These are international conferences, hosting over 200 
mathematicians who come to listen to presentations on topics chosen by 
Bourbaki (or the A.C.N.B). Their last publication was in 1998, chapter 
10 of book VI commutative algebra.} But since Nicolas Bourbaki never in 
fact existed, and was nothing but a clever teaching and research ploy, could 
he ever be said to be extinct?



%====================================[[Bibliography begins here]]
\begin{thebibliography}{99}
\bibitem[BL]{BL} L. BEAULIEU: A Parisian Caf\'e and Ten Proto-Bourbaki Meetings (1934-1935), The Mathematical Intelligencer Vol.15 No.1 1993, pp 27-35.
\bibitem[BCCC]{BCCC} A. BOREL, P.CARTIER, K. CHANDRASKHARAN, S. CHERN, S. IYANAGA: Andr\'e Weil (1906-1998), Notices of the AMS Vol.46 No.4 1999, pp 440-447.
\bibitem[BA]{BA} A. BOREL: Twenty-Five Years with Nicolas Bourbaki, 1949-1973, Notices of the AMS Vol.45 No.3 1998, pp 373-380.
\bibitem[BN]{BN} N. BOURBAKI: Th\'eorie des Ensembles, de la collection \'el\'ements de Math\'ematique, Hermann, Paris 1970.
\bibitem[BW]{BW} Bourbaki website: [online] at \PMlinkexternal{www.bourbaki.ens.fr}{www.bourbaki.ens.fr}.
\bibitem[CH]{CH} H. CARTAN: Andr\'e Weil:Memories of a Long Friendship, Notices of the AMS Vol.46 No.6 1999, pp 633-636.
\bibitem[DR]{DR} R. D\'eCAMPS: Qui est Nicolas Bourbaki?, [online] at \PMlinkexternal{http://faq.maths.free.fr}{http://web.archive.org/web/*/http://faq.maths.free.fr/html-old/faq34.php3}.
\bibitem[DJ]{DJ} J. DIEUDONN\'e: The Work of Nicholas Bourbaki, American Math. Monthly 77,1970, pp134-145.
\bibitem[EY]{EY} Encylop\'edie Yahoo: Nicolas Bourbaki, [online] at  \PMlinkexternal{http://fr.encylopedia.yahoo.com}{http://fr.encylopedia.yahoo.com}.
\bibitem[GD]{GD} D. GUEDJ: Nicholas Bourbaki, Collective Mathematician: An Interview with Claude Chevalley, The Mathematical Intelligencer Vol.7 No.2 1985, pp18-22.
\bibitem[JA]{JA} A. JACKSON: Interview with Henri Cartan, Notices of the AMS Vol.46 No.7 1999, pp782-788.
\bibitem[SM]{SM} M. SENECHAL: The Continuing Silence of Bourbaki- An Interview with Pierre Cartier, The Mathematical Intelligencer, No.1 1998, pp 22-28.
\bibitem[WA]{WA} A. WEIL: The Apprenticeship of a Mathematician, Birkh\"auser Verlag 1992, pp 93-122.
\end{thebibliography}
%%%%%
%%%%%
\end{document}
