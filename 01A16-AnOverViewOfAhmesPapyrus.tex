\documentclass[12pt]{article}
\usepackage{pmmeta}
\pmcanonicalname{AnOverViewOfAhmesPapyrus}
\pmcreated{2014-12-03 17:44:44}
\pmmodified{2014-12-03 17:44:44}
\pmowner{milogardner}{13112}
\pmmodifier{milogardner}{13112}
\pmtitle{An over view of Ahmes Papyrus}
\pmrecord{201}{41500}
\pmprivacy{1}
\pmauthor{milogardner}{13112}
\pmtype{Definition}
\pmcomment{trigger rebuild}
\pmclassification{msc}{01A16}

% this is the default PlanetMath preamble.  as your knowledge
% of TeX increases, you will probably want to edit this, but
% it should be fine as is for beginners.

% almost certainly you want these
\usepackage{amssymb}
\usepackage{amsmath}
\usepackage{amsfonts}

% used for TeXing text within eps files
%\usepackage{psfrag}
% need this for including graphics (\includegraphics)
%\usepackage{graphicx}
% for neatly defining theorems and propositions
%\usepackage{amsthm}
% making logically defined graphics
%%%\usepackage{xypic}

% there are many more packages, add them here as you need them

% define commands here

\begin{document}
I \PMlinkexternal{ABSTRACT}{http://ahmespapyrus.blogspot.com/2009/01/ahmes-papyrus-new-and-old.html}

Old Kingdom (OK) numerals were written in hieroglyphic many-to-one symbols. The OK numeration system followed a cursive binary algorithm that rounded off, throwing away a 1/64 unit, in an infinite series. The system included an uncorrected binary system before 2350 BCE. 

After 2050 BCE a fully developed hieratic numeration and math system corrected Old Kingdom round-off errors in Ahmes Papyrus and other texts. The corrected numeration system encoded rational numbers by \PMlinkexternal{LCM}{http://planetmath.org/redauxiliarynumbersthefirstlcmmethod}s in 2/n tables and every day math problems in finite ways. 

By 1900 BCE the Akhmim Wooden Tablet (AWT), 1825 BCE in the Moscow Mathematical Papyrus(MMP), 1800 BCE in the Egyptian Mathematical Leather Roll (EMLR), 1800 BCE in the Kahun Papyrus (KP) and 1650 BCE in the Ahmes Papyrus (better known as the 1650 BCE Rhind Mathematical Papyrus (RMP)) applied the improved hieratic numeration system that solved a wide range business problems.

For example, Ahmes Papyrus recorded exact unit fraction data in a finite arihmetic notation. Scholars in the 20th century incompletely considered scribal shorthand facts to unfairly report scribal longhand methods.  

In 2006 AD Ahmes shorthand was decoded in RMP 36 AND RMP 37 longhand by adding back missing 2/n table data. The long hand information allowed 'Ab initio' aspects of RMP 2/n tables, mentioned by Peet and 1920s scholars as needed, to be formally added-back to once hard-to-read shorthand data within four 2/n table construction rules:

1) 2/p = (2/p)(m/m) = (2m/mp), with the best divisors of mp to sum to 2m and the best best fraction series 

2) n/p = (n/p)(m/m) = mn/mp, with the best divisors of mp summed to mn and the best unit fraction series.

3) n/p = (2/p) + (n -2)/p scaled by (m1/m1) and (m2/m2)

4) when rules 1, 2 or 3 could not find (m/m), (m1/m1) and/or (m2/m2) identity statements like 

1 = 53/53 = 2/53(30/30) + 3/53(20/20) + 5/53(12/12) + 15/53(4/4) + 28/53(2/2) were solved

The four rules  added back 'ab initio' reconstructed facts to least common multiple (LCM) m arithmetic statements scaled to unit fraction series. Scribal longhand included uses of red auxiliary numbers, and other 'ab initio' facts. Considering these new facts assists in better decoding Egyptian fraction data recorded in the RMP, EMLR, MMP, the KP and other Middle Kingdom and later Greek, Arab and medieval texts.

That is, RMP 36 and RMP 37 rational numbers (n/p) were scaled by least common multiples (LCM) m were thought of as (m/m) to (mn/mp). The best divisors of denominator (mp) were written in red (red auxiliary numbers) a misunderstood shorthand unit fraction series calculation step. 

Ahmes' longhand rational number arithmetic steps reports finite arithmetic within exact unit fraction series whenever possible. A small number of round container calculations reported in RMP 41-44 rounded-off by applying an estimate for pi of 256/81 (3.1605). RMP 38 mentioned 22/7 (3.1429) as a correction  that \PMlinkexternal{Archimedes}{http://planetmath.org/squarerootof3567and29} reported (3 + 10/71) less pi less (3 + 10/70) 1500 years later.

II BACKGROUND ISSUES

After 2050 BCE hieratic numerals and numbers were mapped one-to-one onto sound symbols (for the first time in the Western Tradition) within finite arithmetic statements. Scribes demonstrated that Middle Kingdom rational numbers. written in finite arithmetic, were exact whenever possible. The finite arithmetic used unit fraction statements that applied prime numbers that were equivalent to our modern fundamental theorem of arithmetic. The finite arithmetic contained algebra, algebraic geometry, arithmetic progressions, geometric progression and weights and measures unit calculations that focused on business transactions. The hieratic arithmetic defined "one of the essential skills of mathematics changed context, including notations, units, and a great deal more", a feature of modern mathematics. Statements of Middle Kingdom problems were often recorded in algorithms with solutions recorded in finite arithmetic statements. 

A student scribe scaled rational numbers in the Egyptian Mathematical Leather Roll (EMLR) to show that the system was easy to use. The EMLR student  scaled 26 rational numbers 1/p and 1/pq by multiples m that obtained m/pm and m/pqm to find concise and non-concise unit fraction series. Fully trained scribes (Ahmes and the Kahun Papyrus (KP) author) converted 2/n table rational numbers by up to 14 multiples m that obtained 2m/mp before concise unit fraction series were calculated in the RMP 2/n table and a shorter KP 2/n table. Rational numbers recorded as unit fraction series used algebraic formulas in the KP, the \PMlinkexternal{EMLR}{http://www.nytimes.com/2010/12/07/science/07first.html?_r=2&ref=science}, the Akhmim Wooden Tablet (AWT), and the Moscow Mathematical Papyrus (MMP) and other texts. The Ahmes Papyrus is also known as the RMP.

III Ahmes Papyrus, a  summary

Ahmes scaled rational fractions (n/p) by LCM m/m to mn/mp. The best aliquot parts of divisors (mp) were written in red ink that summed to numerator (mp). Final unit fraction series recorded concise 50 member 2/n table members and 87 problems written in the same notation. Scribal problems may have first been solved in an economic context. The RMP economic context scaled commodities like bread and beer in hekats of grain, a volume unit that was used to pay wages. RMP 36 and scaled solutions to 2/53, 3/53, 5/53, 15/53 and 30/53 exposed the scribal meaning of red auxiliary numbers that built concise 2/n tables. Scribal vulgar fraction (rational number arithmetic) exposed meta numbers and a window to decode ancient \PMlinkexternal{proto-number theory}{http://planetmath.org/encyclopedia/RMP35To38PlusRMP66.html}. 

Scribal meta arithmetic notations confused 19th and 20th century scholars. 20th century scholars suggested scribal division consisted of 'single false position', an idea that Arabs used to find roots to second degree equation around 800 AD. Clagett as late as 1999 stressed 'single false position' rather than following Schack-Schackenberg's 1900 decoding of RMP 69's pesu arithmetic operations, and a well defined RMP 36 decoding information. In RMP 69-78 a pesu calculation followed two \PMlinkexternal{Berlin Papyrus} {http://planetmath.org/encyclopedia/BerlinPapyrusAndSecondDegreeEquations.html} second degree equation solutions that were recorded 250 years earlier. 

The meta scribal arithmetic was 'cracked' in 21st century AD. Middle Kingdom rational numbers and arithmetic operations are being corrected, one problem at a time. Corrections add back missing scribal shorthand to expose modern-looking vulgar fraction and modern-looking arithmetic operations that recorded the following:

A. addition: (n/p + m/q) = (nq + mp)/pq
  1. common: (1/p + 1/q) = (p + q)/pq
  2. frequent: (1/p + 1/np) = (n + 1)/np  

B. subtraction: (n/p - m/q) = (nq - mp)/pq, 51 times, as a secondary 2/nth table method. 

C. multiplication: (n/p x m/q) = (nm/pq), and (n/p x (m/m) = nm/mp, with mp aliquot parts = nm, writing 51 exact unit fractions series in 2/n table. A second Old Kingdom duplation method was used to prove answers' correct value, by working the problem backwards (as RMP 1-28 detail). The second method followed a doubling pattern (duplation), while the first and second methods looked and acted as modern arithmetic definitions.

D. division: (nm/pq)/(m/q) = (nm/pq) x (q/m) = n/p, with division by m/q, an inverse q/m multiplication operation.    

In 2010 it became clear that scribal geometry was algebraic based. 

1. The area (A) of a circle formula, used in MMP 10, RMP 41, and 42 replaced radius (R) with 1/2 a diameter (D/2), and replaced pi with 256/81 in three steps (a, b, c) 

a. A = (256/81)(D/2)(D/2) 

b. A = (64/81)(D)(D)

c. A = (8/9)(8/9)(D)(D) cubits

2. RMP 42, adding height (H) to compute volume

a. V = (8/9)(8/9)(D)(D) cubits,

b. V = (3/2)(8/9)(8/9)(D)(D) khar (convert cubit units to khar units)

3. In RMP 43 and the Kahun Papyrus scribes scaled RMP 42 formula

V =(3/2)(8/9)(8/9)(D)(D)khar by 3/2 in two steps

a. (3/2) V = (3/2) (3/2)(8/9)(8/9)(D)(D)khar 

b. V = 2/3)(H)(4/3)(4/3)(D)(D) khar (directly calculate in khar units)

RMP 43 data d = 8 and H = 6

d. V =(2/3)(6)(4/3)(8)(4/3)(8)= (4)(32/3)(32/3) = 4096/9 = (455 + 1/9)khar

algebraic arithmetic scaled  geometry facts that Robins-Shute "RMP", 1987 (page 46) reported RMP 42 formuls V= "(8/9)(8/9)(D)D)cubit cubits = (3/2) x (8/9)(8/9)(D)(D)(H)khar = (32/27)(D)(D)(H) =  (2/3)(4/3)(4/3)(H)khar", and 20 khar/20 = (1) 400-hekat unit. Ahmes reported 4-hekat and 4-ro units in two-part Q/64 + (5R/n)ro answers, and divided 400-hekat, 100-hekat, 4-hekat and 1-hekat units by divisor n by substituting (6400/64) for 100-hekat, (64/64) or 320 ro for 1-hekat. 

IV CONCLUSIONS

Updated 21st century analysis of Ahmes' 87 problems strip away Middle Kingdom Egyptian fraction notations and expose rational numbers, algebra, algebraic geometry, arithmetic progression and arithmetic operations. Scholars in the 20th century transliterated hieratic numerical data within an additive notation, and did not publish complete translations of scribal shorthand information. To fairly translate scribal rational number arithmetic into modern arithmetic statements meta points of view have been developed.

Weights and measures units used in the Middle Kingdom were also poorly translated. Translation errors are corrected by developing meta theoretical and practical properties of ancient Egyptian fraction mathematics. Scribes scaled 2/n tables, algebra, geometry, and weights and measures statements by selecting the best LCM m and the best red auxiliary numbers.

Considering the 3,700 year mathematical life of Egyptian fraction arithmetic LCM m was applied two ways. In the Middle Kingdom n/p was scaled by LCM m to mn/mp, a multiplication context. Sometime before 800 AD, Greek, Arab and/or medieval scribes replaced the traditional multiplication context with a subtraction context: 

(n/p - 1/m) = (mn -p)/mp, with (mn-p) set to unity as often as possible.   

Fibonacci reported three versions of the unit fraction arithmetic system in the Liber Abaci. With the fall of the Byzantine Empire in 1454 AD, the Liber Abaci and the unit fraction arithmetic fell out of use in Europe with Galileo. The Arabs version (present in Ghobar script) was used as late as 1637 AD, was erased with the implementation of modern Arabic script. By 1600 AD Stevin's 1585 AD decimal system, written in an Arab algorithm, created base 10 decimal arithmetic, algebraic geometry, algebra and unit fraction weights and measures operations, began to trivialize the once bright Egyptian fraction arithmetic system.  

\begin{thebibliography}{10}
\bibitem{1}  A.B. Chace, Bull, L, Manning, H.P., Archibald, R.C., \emph{The Rhind Mathematical Papyrus}, Mathematical Association of Amnerica, Vol I, 1927. NCTM reprints available.
\bibitem{2} Edward Cherlin, \emph{ [Math 2.0]}, <echerlin@gmail.com>, mathfuture@googlegroups.com, Tue, March 1, 2011 
\bibitem{3} Milo Gardner, \emph{An Ancient Egyptian Problem and its Innovative Solution, Ganita Bharati}, MD Publications Pvt Ltd, 2006.
\bibitem{4}Richard Gillings, \emph{Mathematics in the Time of the Pharaohs}, Dover Books, 1992.
\bibitem{5} Oystein Ore, \emph{Number Theory and its History}, McGraw-Hill Books, 1948, Dover reprints available.
\bibitem{6} T.E. Peet, \emph{Arithmetic in the Middle Kingdom}, Journal Egyptian Archeology, 1923.
\bibitem{7} Tanja Pommerening, \emph{"Altagyptische Holmasse Metrologish neu Interpretiert" and relevant phramaceutical and medical knowledge, an abstract,  Phillips-Universtat, Marburg, 8-11-2004, taken from "Die Altagyptschen Hohlmass}, Buske-Verlag, 2005.
\bibitem{8} Gay Robins, and Charles Shute \emph{Rhind Mathematical Papyrus}, British Museum Press, Dover reprint, 1987.
\bibitem{9} L.E. Sigler, \emph{Fibonacci's Liber Abaci: Leonardo Pisano's Book of Calculation}, Springer, 2002.
\bibitem{10} Hana Vymazalova, \emph{The Wooden Tablets from Cairo:The Use of the Grain Unit HK3T in Ancient Egypt, Archiv Orientalai}, Charles U Prague, 2002.
\end{thebibliography}




%%%%%
%%%%%
\end{document}
