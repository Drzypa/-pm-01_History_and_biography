\documentclass[12pt]{article}
\usepackage{pmmeta}
\pmcanonicalname{SophieGermain}
\pmcreated{2013-03-22 16:17:43}
\pmmodified{2013-03-22 16:17:43}
\pmowner{Mravinci}{12996}
\pmmodifier{Mravinci}{12996}
\pmtitle{Sophie Germain}
\pmrecord{6}{38414}
\pmprivacy{1}
\pmauthor{Mravinci}{12996}
\pmtype{Biography}
\pmcomment{trigger rebuild}
\pmclassification{msc}{01A50}
\pmclassification{msc}{01A55}
\pmsynonym{Marie-Sophie Germain}{SophieGermain}
\pmsynonym{Monsieur Le Blanc}{SophieGermain}
\pmsynonym{Monsieur LeBlanc}{SophieGermain}

% this is the default PlanetMath preamble.  as your knowledge
% of TeX increases, you will probably want to edit this, but
% it should be fine as is for beginners.

% almost certainly you want these
\usepackage{amssymb}
\usepackage{amsmath}
\usepackage{amsfonts}

% used for TeXing text within eps files
%\usepackage{psfrag}
% need this for including graphics (\includegraphics)
%\usepackage{graphicx}
% for neatly defining theorems and propositions
%\usepackage{amsthm}
% making logically defined graphics
%%%\usepackage{xypic}

% there are many more packages, add them here as you need them

% define commands here

\begin{document}
\emph{Marie-Sophie Germain} (1776 - 1831) French mathematician. In her lifetime, she published some of her work as \emph{Monsieur Le Blanc} and also carried on a correspondence with Carl Friedrich Gauss under that name. One of her first papers that she wrote using this pseudonym was submitted to Joseph-Louis Lagrange. After meeting Germain in person, he encouraged her and introduced her to various French mathematicians and scientists. Acquiring contacts in this manner was crucial for Germain to be able to progress in mathematics. The only other \PMlinkescapetext{means} that Germain had for studying mathematics was by reading. She read mathematical books as well as lecture notes from various courses taught at \'{E}cole Polytechnique.

In 1816, Germain received the French Academy of Science Grand Prize.  In 1831, Gauss recommended that Germain receive an honorary doctorate from G\"{o}ttingen University, but she died before the \PMlinkescapetext{degree} could be conferred.

Germain made headway towards proving Fermat's last theorem. Today she is best known for the Sophie Germain primes.

\begin{thebibliography}{1}
\bibitem{jr} M. W. Gray ``Sophie Germain'' in {\it Women of Mathematics: A Bibliographic Sourcebook} L. Grinstein, P. Cambpell, ed.s New York: Greenwood Press (1987): 47 - 56
\end{thebibliography}
%%%%%
%%%%%
\end{document}
