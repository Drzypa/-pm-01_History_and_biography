\documentclass[12pt]{article}
\usepackage{pmmeta}
\pmcanonicalname{MoritzStern}
\pmcreated{2013-03-22 16:19:47}
\pmmodified{2013-03-22 16:19:47}
\pmowner{PrimeFan}{13766}
\pmmodifier{PrimeFan}{13766}
\pmtitle{Moritz Stern}
\pmrecord{4}{38459}
\pmprivacy{1}
\pmauthor{PrimeFan}{13766}
\pmtype{Biography}
\pmcomment{trigger rebuild}
\pmclassification{msc}{01A55}
\pmsynonym{Moritz Abraham Stern}{MoritzStern}

\endmetadata

% this is the default PlanetMath preamble.  as your knowledge
% of TeX increases, you will probably want to edit this, but
% it should be fine as is for beginners.

% almost certainly you want these
\usepackage{amssymb}
\usepackage{amsmath}
\usepackage{amsfonts}

% used for TeXing text within eps files
%\usepackage{psfrag}
% need this for including graphics (\includegraphics)
%\usepackage{graphicx}
% for neatly defining theorems and propositions
%\usepackage{amsthm}
% making logically defined graphics
%%%\usepackage{xypic}

% there are many more packages, add them here as you need them

% define commands here

\begin{document}
\emph{Moritz Abraham Stern} (1807 - 1894) German mathematician. Stern succeeded Carl Friedrich Gauss as Ordinarius at G\"ottingen University in 1858, becoming the first Jewish Ordinarius at a German university. He may also have been the first to take notice of Bernhard Riemann's talent for mathematics. Stern was also very helpful to Ferdinand Eisenstein in formulating a proof of the quadratic reciprocity theorem. He was interested in primes that can't be expressed as the sum of a prime and twice a square (now known as Stern primes).
%%%%%
%%%%%
\end{document}
