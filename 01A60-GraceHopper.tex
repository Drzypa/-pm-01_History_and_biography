\documentclass[12pt]{article}
\usepackage{pmmeta}
\pmcanonicalname{GraceHopper}
\pmcreated{2013-03-22 16:25:27}
\pmmodified{2013-03-22 16:25:27}
\pmowner{Mravinci}{12996}
\pmmodifier{Mravinci}{12996}
\pmtitle{Grace Hopper}
\pmrecord{7}{38575}
\pmprivacy{1}
\pmauthor{Mravinci}{12996}
\pmtype{Biography}
\pmcomment{trigger rebuild}
\pmclassification{msc}{01A60}
\pmsynonym{Grace Brewster Murray}{GraceHopper}
\pmsynonym{Grace Brewster Murray Hopper}{GraceHopper}

% this is the default PlanetMath preamble.  as your knowledge
% of TeX increases, you will probably want to edit this, but
% it should be fine as is for beginners.

% almost certainly you want these
\usepackage{amssymb}
\usepackage{amsmath}
\usepackage{amsfonts}

% used for TeXing text within eps files
%\usepackage{psfrag}
% need this for including graphics (\includegraphics)
%\usepackage{graphicx}
% for neatly defining theorems and propositions
%\usepackage{amsthm}
% making logically defined graphics
%%%\usepackage{xypic}

% there are many more packages, add them here as you need them

% define commands here

\begin{document}
\PMlinkescapeword{degrees}
\PMlinkescapeword{term}

\emph{Grace Hopper}, n\'ee {\em Grace Brewster Murray} (1906 - 1992) American mathematician, computer engineer and U. S. Navy Admiral.

After earning master's degrees in mathematics and physics from Yale, Hopper taught at Vassar  for a few years before joining the Navy, which assigned her to work on the Mark I computer. One day she extracted an insect from the machinery and helped popularize the preexisting term "bug" for computer malfunctions. Later on she worked on a compiler for COBOL.

\begin{thebibliography}{1}
\bibitem{ak} A. C. King \& Tina Schalch ``Grace Brewster Murray Hopper'' in {\it Women of Mathematics: A Bibliographic Sourcebook} L. Grinstein, P. Cambpell, ed.s New York: Greenwood Press (1987): 30 - 32
\end{thebibliography}
%%%%%
%%%%%
\end{document}
