\documentclass[12pt]{article}
\usepackage{pmmeta}
\pmcanonicalname{IrmgardFluggeLotz}
\pmcreated{2013-03-22 16:53:41}
\pmmodified{2013-03-22 16:53:41}
\pmowner{Mravinci}{12996}
\pmmodifier{Mravinci}{12996}
\pmtitle{Irmgard Fl\"ugge-Lotz}
\pmrecord{4}{39151}
\pmprivacy{1}
\pmauthor{Mravinci}{12996}
\pmtype{Biography}
\pmcomment{trigger rebuild}
\pmclassification{msc}{01A60}
\pmsynonym{Irmgard Flugge-Lotz}{IrmgardFluggeLotz}
\pmsynonym{Irmgard Fl\"ugge}{IrmgardFluggeLotz}
\pmsynonym{Irmgard Lotz}{IrmgardFluggeLotz}

\endmetadata

% this is the default PlanetMath preamble.  as your knowledge
% of TeX increases, you will probably want to edit this, but
% it should be fine as is for beginners.

% almost certainly you want these
\usepackage{amssymb}
\usepackage{amsmath}
\usepackage{amsfonts}

% used for TeXing text within eps files
%\usepackage{psfrag}
% need this for including graphics (\includegraphics)
%\usepackage{graphicx}
% for neatly defining theorems and propositions
%\usepackage{amsthm}
% making logically defined graphics
%%%\usepackage{xypic}

% there are many more packages, add them here as you need them

% define commands here

\begin{document}
\emph{Irmgard Fl\"ugge-Lotz} n\'ee \emph{Irmgard Lotz} (1903 - 1974) German mathematician and engineer, best known for her work on the mathematics of aerodynamics, the first female engineering professor at Stanford.

After her father was drafted for military service in World War I, the young Irmgard helped the family by becoming a math tutor. By high school and through college, she practically supported her family single-handedly. In college she was often the only woman in her class. In 1929 she earned a doctorate in engineering, but she had a tough time getting engineering jobs. Though limited, she had some choices available. Lotz went to work for the Aerodynamics Institute in G\"ottingen. After she solved a peculiarly thorny equation pertaining to wing lift distribution, Lotz was promoted to team leader. In 1938, she married the civil engineer Wilhelm Fl\"ugge and the pair moved to Berlin and later to the small town of Saulgau. After World War II, there was no need for the Fl\"ugges to undergo a denazification investigation, and they moved to France, and later to the United States, where they taught at Stanford. Fl\"ugge-Lotz started new engineering courses dealing with the mathematics of aerodynamics. In spite of suffering debilitating arthritis, Fl\"ugge-Lotz continued her engineering research even through retirement.

\begin{thebibliography}{1}
\bibitem{js} J. R. Spreiter \& W. Fl\"ugge ``Irmgard Fl\"ugge-Lotz'' in {\it Women of Mathematics: A Bibliographic Sourcebook} L. Grinstein, P. Cambpell, ed.s New York: Greenwood Press (1987): 33 - 40
\end{thebibliography}
%%%%%
%%%%%
\end{document}
