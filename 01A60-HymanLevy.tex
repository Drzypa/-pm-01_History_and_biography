\documentclass[12pt]{article}
\usepackage{pmmeta}
\pmcanonicalname{HymanLevy}
\pmcreated{2013-03-22 18:06:39}
\pmmodified{2013-03-22 18:06:39}
\pmowner{Mravinci}{12996}
\pmmodifier{Mravinci}{12996}
\pmtitle{Hyman Levy}
\pmrecord{4}{40656}
\pmprivacy{1}
\pmauthor{Mravinci}{12996}
\pmtype{Biography}
\pmcomment{trigger rebuild}
\pmclassification{msc}{01A60}
\pmclassification{msc}{01A55}

\endmetadata

% this is the default PlanetMath preamble.  as your knowledge
% of TeX increases, you will probably want to edit this, but
% it should be fine as is for beginners.

% almost certainly you want these
\usepackage{amssymb}
\usepackage{amsmath}
\usepackage{amsfonts}

% used for TeXing text within eps files
%\usepackage{psfrag}
% need this for including graphics (\includegraphics)
%\usepackage{graphicx}
% for neatly defining theorems and propositions
%\usepackage{amsthm}
% making logically defined graphics
%%%\usepackage{xypic}

% there are many more packages, add them here as you need them

% define commands here

\begin{document}
\PMlinkescapeword{children}

\emph{Hyman Levy} (1889 - 1975) Scottish mathematician and author.

The son of a Jewish art dealer in Edinburgh, Hyman was the third oldest of eight children. Thanks to scholarships, he was able to go to Germany to study at the University of G\"ottingen. But because of World War I, Levy had to return to the United Kingdom. Researching aeronautics at the National Physical Laboratory, Levy published papers and books on mathematical applications pertaining to aeronautics. He also wrote about differential equations and probability. 

In 1918, he married Marion Aitken, a Christian woman, despite the disapproval of his family. They had three children. Levy was in the Labour Party from 1920 to 1931, and then in 1931 he joined the British Communist Party. Despite his theoretical allegiance to the principles of communism, Levy became disappointed by the way the Russian communists treated Jews, and published on the topic, leading to his expulsion from the party in 1958.
%%%%%
%%%%%
\end{document}
