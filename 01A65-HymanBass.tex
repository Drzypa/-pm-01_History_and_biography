\documentclass[12pt]{article}
\usepackage{pmmeta}
\pmcanonicalname{HymanBass}
\pmcreated{2013-03-22 17:25:42}
\pmmodified{2013-03-22 17:25:42}
\pmowner{PrimeFan}{13766}
\pmmodifier{PrimeFan}{13766}
\pmtitle{Hyman Bass}
\pmrecord{5}{39803}
\pmprivacy{1}
\pmauthor{PrimeFan}{13766}
\pmtype{Biography}
\pmcomment{trigger rebuild}
\pmclassification{msc}{01A65}
\pmclassification{msc}{01A61}
\pmclassification{msc}{01A60}

\endmetadata

% this is the default PlanetMath preamble.  as your knowledge
% of TeX increases, you will probably want to edit this, but
% it should be fine as is for beginners.

% almost certainly you want these
\usepackage{amssymb}
\usepackage{amsmath}
\usepackage{amsfonts}

% used for TeXing text within eps files
%\usepackage{psfrag}
% need this for including graphics (\includegraphics)
%\usepackage{graphicx}
% for neatly defining theorems and propositions
%\usepackage{amsthm}
% making logically defined graphics
%%%\usepackage{xypic}

% there are many more packages, add them here as you need them

% define commands here

\begin{document}
\PMlinkescapeword{ph}

\emph{Hyman Bass} (1932 - ) American mathematician.

After earning a Ph.D. from the University of Chicago, Bass began teaching at Columbia University for almost 40 years, interspersed with various lecturing vacations around the world. After Columbia, he returned to his native state to teach at the University of Michigan. In 2001 he was elected president of the American Mathematical Society. In 2007 the United States awarded him the National Medal of Science.

In 1998, Bass co-authored with Henri Cartan and others a tribute to Samuel Eilenberg, while decades earlier Cartan co-authored with Jacques Dixmier and others a conference paper in French on problems of measure. Dixmier co-authored with Erd\H{o}s a paper on fundamental invariants of binary forms, also in French, giving Bass \PMlinkname{Erd\H{o}s number}{ErdHosNumber} 3.
%%%%%
%%%%%
\end{document}
