\documentclass[12pt]{article}
\usepackage{pmmeta}
\pmcanonicalname{VladimirFock}
\pmcreated{2013-03-22 16:42:47}
\pmmodified{2013-03-22 16:42:47}
\pmowner{PrimeFan}{13766}
\pmmodifier{PrimeFan}{13766}
\pmtitle{Vladimir Fock}
\pmrecord{6}{38929}
\pmprivacy{1}
\pmauthor{PrimeFan}{13766}
\pmtype{Biography}
\pmcomment{trigger rebuild}
\pmclassification{msc}{01A60}
\pmsynonym{Vladimir Fok}{VladimirFock}
\pmsynonym{Vladimir Aleksandrovich Fok}{VladimirFock}
\pmsynonym{Vladimir Alexandrovich Fok}{VladimirFock}
\pmsynonym{Vladimir Aleksandrovich Fock}{VladimirFock}
\pmsynonym{Vladimir Alexandrovich Fock}{VladimirFock}

\endmetadata

% this is the default PlanetMath preamble.  as your knowledge
% of TeX increases, you will probably want to edit this, but
% it should be fine as is for beginners.

% almost certainly you want these
\usepackage{amssymb}
\usepackage{amsmath}
\usepackage{amsfonts}

% used for TeXing text within eps files
%\usepackage{psfrag}
% need this for including graphics (\includegraphics)
%\usepackage{graphicx}
% for neatly defining theorems and propositions
%\usepackage{amsthm}
% making logically defined graphics
%%%\usepackage{xypic}

% there are many more packages, add them here as you need them

% define commands here

\begin{document}
\emph{Vladimir Aleksandrovich Fock} (or \emph{Vladimir Alexandrovich Fok} depending on the transliteration from Cyrillic) (1898 - 1974) Soviet physicist, one of the pioneers of quantum mechanics.

Just a few years after graduating from the university in Petrograd, he studied the Klein-Gordon equation and provided a generalization of it. For more than a dozen years he concentrated his work on optics, but around World War II his interest shifted to quantum mechanics. Today he is perhaps best known for the Bargmann-Fock space. Among theoretical physicists he's also well-known for the Hartree-Fock method still used today by chemists and molecular physicists.

\begin{thebibliography}{2}
\bibitem{cf} Charlotte Froese Fischer, {\it The Hartree-Fock Method for Atoms: A Numerical Approach}. New York: Wiley (1977)
\bibitem{pr} Peter Ring, Peter Schuck \& W. Beiglb\"ock, {\it The Nuclear Many-body Problem}. New York: Springer Verlag (1980): v
\end{thebibliography}
%%%%%
%%%%%
\end{document}
