\documentclass[12pt]{article}
\usepackage{pmmeta}
\pmcanonicalname{AhmesBirdfeedingRateMethod}
\pmcreated{2013-06-18 18:28:32}
\pmmodified{2013-06-18 18:28:32}
\pmowner{milogardner}{13112}
\pmmodifier{milogardner}{13112}
\pmtitle{Ahmes' bird-feeding rate method}
\pmrecord{155}{41301}
\pmprivacy{1}
\pmauthor{milogardner}{13112}
\pmtype{Definition}
\pmcomment{trigger rebuild}
\pmclassification{msc}{01A16}
\pmdefines{quotients and remainders}

\endmetadata

% this is the default PlanetMath preamble.  as your knowledge
% of TeX increases, you will probably want to edit this, but
% it should be fine as is for beginners.

% almost certainly you want these
\usepackage{amssymb}
\usepackage{amsmath}
\usepackage{amsfonts}

% used for TeXing text within eps files
%\usepackage{psfrag}
% need this for including graphics (\includegraphics)
%\usepackage{graphicx}
% for neatly defining theorems and propositions
%\usepackage{amsthm}
% making logically defined graphics
%%%\usepackage{xypic}

% there are many more packages, add them here as you need them

% define commands here

\begin{document}
STATEMENT

In 1927 \PMlinkexternal{A.B. Chace}{http://planetmath.org/AnOverViewOfAhmesPapyrus.html} reported additive aspects of three Egyptian Middle Kingdom bird-feeding rates. Chace's one part analysis unfairly introduced an incomplete two-part Middle Kingdom hekat weights and measures proposal. 

The hekat was about five gallons in modern units. The hekat system was applied in the 1650 BCE Rhind Mathematical Papyrus (RMP) over 60 times in 10 problems. In RMP #83, the bird-feeding problem, grain units were measured by 1/6, 1/20 and 1/40 portions of a hekat that fed seven birds by an unknown formula. Chace did not decode the complete formula. 

The complete formula, decoded in 2006, included an initial hekat unity (64/64) system  and multiplicative steps that parsed (64/64) by 1/6, 1/20 and 1/40. The scribe reported three exact two-part quotient and scaled remainder rates:

1. 2 geese, and a crane each ate (1/8 + 1/32) hekat + (3 + 1/3) ro 

2. a set-duck ate (1/32 + 1/64) hekat + 1 ro and,

3. a set-goose, dove, and quail each at (1/64) hekat + 3 ro

Ahmes, the scribe, asked how much grain did the seven birds eat in one day? Additionally, three other questions asked: how much grain was eaten in 10 days, 20 days and 30 days ob the seven birds?

The scribal commodity purpose for the three bird-feeding rates had been outlined by 200 and 300 year older texts. The 200 year older text was \PMlinkexternal{Kahun Papyrus}{http://planetmath.org/encyclopedia/KahunPapyrusAndArithmeticProgressions.html}(KP). The KP data computed relative values of 12 fowls brought to the MK commodity market in Middle Kingdom in set-duck units:

a. 3 re-geese unit value 8 set-ducks = 24

b. 3 terp-geese unit value 4 set-ducks = 12

c. 3 Dj. Cranes unit value 2 set-ducks = 6

d. 3 set-duck unit value 1 set-duck = 3

total value 45 set-ducks.

To decode one day of grain eaten by seven birds in historical context, Ahmes' intermediate hekat unity and 1/6, 1/20 and 1/40 portions are needed. Given that Ahmes omitted initial and intermediate steps a 300 year older Akhmim Wooden Tablet (AWT) was consulted. The AWT scribe multiplied the (64/64) unity by 1/3, 1/7, 1/10, 1/11 and 1/13 in steps that were missing in the RMP. The AWT scribe returned the same class of two-part hekat quotients and 1/320 of a hekat scaled remainders as the RMP.

In 2002 Hana Vyzamalova, a Charles U., Prague graduate student, wrote a paper on the AWT that opened a long closed (64/64) hekat unity door. Vymazalova showed that (64/64) was multiplied by 1/3, 1/7, 1/10, 1/11 and 1/13 and reported two-part answers, that include five proofs that multiplied by 3, 7, 10, 11 and 13, respectively, (64/64). But what were the intermediate steps that preceded the two-part quotient and remainder answers? 

Ahmes only reported 1/6 of a hekat (three times), 1/20 of a hekat (once) and and 1/40 of a hekat (three times). One set of missing intermediate steps included:

1. (3/6 + 1/20 + 3/40)hekat 

2. (20 + 2 + 3)/40 hekat 

3.(25/40)hekat = (5/8) hekat (of grain)

were eaten by the seven birds in one day. 

The 10 day answer was 50/8 = 6 1/4 hekat. 

The 20 day answer was 12 1/2 hekat. 

The 30 day, and commodity value of the seven fowls was 18 3/4 hekat.

Had Ahmes' shorthand included AWT intermediate and proof steps, Chace's scholarly error may not have taken place.

Conclusion:  Ahmes' shorthand calculations must be modified to include AWT scribal longhand and proof.

CONTEXTUAL PROOF CONSIDERATIONS

A. Ahmes also discussed theoretical aspects of the bird-feeding method in RMP 47. The details of RMP 47 discussed (6400/64) that scaled 100 hekat within 100-quadruple and 400-quadruple in 10 two-part statements. The 100-hekat calculations reported 100 hekat such that (6400/64)/n began the calculation rather than (64/64)/n began hekat divisions recorded in RMP 41, 42, 43, 44, 45 and 46. The 100-quadruple hekat and 400-quadruple RMP 47 narrative offered raw data calculations that confused additive math historians like A.B. Chace.

Historical bird-feeding rates problems requested by Ahmes first considered (64/64)/n and (6400/64)/n. Scribal partitions of hekat unities included the pesu. The pesu was an inverse proportion that valued geese, crane, set-ducks and other fowls detailed in the RMP and the Kahun Papyrus (KP) for use in commodities payments as wages to workers. 

The KP showed 200 years before the RMP that scribes easily calculated prices of birds scaled as commodities. The bird commodities were used as wage payments based on pesu inverse proportions of grain eaten by birds over 30 day standardized 'feeding periods'.  

B. The \PMlinkexternal{Akhmim Wooden Tablet (AWT)}{http://akhmimwoodentablet.blogspot.com/}was written around 1925 BCE. The scribe scaled a a volume unit (hekat) equivalent to 4800 ccm in modern metrics to (64/64) before sub-dividing to two-part units in the 1650 BCE Rhind Mathematical Papryus(RMP). The hekat unity (64/64) was multiplied by 1/3, 1/7, 1/10, 1/11 and 1/13 in the AWT. The scaled hekat unity method recorded five two-part binary quotient (Q) and 1/320 of a hekat remainders (R) before five proofs returns two-part answers to (64/64) that Ahmes, the RMP scribe, used over 60 times.

Seen in a modern context the scribal two-part units reported

1. [(64/64) times 1/n] = [Q/64 hekat + (5R/n times 1/320 of hekat] 

and

2. [(64/64)/n] = [Q/64 + 5R/n times ro]    

The word ro  meant 1/320 of a hekat

A two- part hekat-ro proof was published by Hana Vymazalova in 2002. Vymazalova did not fully report the the scribal partitioning method. Vymazalova cited without demonstrating that the (64/64) hekat unity method calculated five binary quotients (Q/64) and five (5R/n*ro) remainders in two-part answers. That is, the inverse initial divisors, 3, 7, 10, 11 and 13 divided (64/64) such that five two-part answers were reported by Vymazalova without offering other scribal details.

Vymazalova's paper explicitly corrected Daressy's 1906 incomplete analysis that had not fully parsed the scribal use of (64/64). Vymazalova's corrections showed that all five 64/64 times 1/3, 1/7, 1/10, 1/11 and 1/13 two-part answers were multiplied by 3, 7, 10, 11 and 13, respectively, and returned 64/64.    

C. The (64/64) hekat partitioning method was used in the \PMlinkexternal{Kahun Papyrus(KP)}{http://planetmath.org/encyclopedia/KahunPapyrusAndArithmeticProgressions.html} and the \PMlinkexternal{Rhind Mathematical Papyrus (RMP)}{http://www.nytimes.com/2010/12/07/science/07first.html?_r=1&ref=science} over 60 times. Other aspects of the (64/64) hekat unity method have been published in \PMlinkexternal{2006}{http://independent.academia.edu/MiloGardner/Papers/163573/The_Arithmetic_used_to_Solve_an_Ancient_Horus-Eye_Problem} and \PMlinkexternal{2011}{http://independent.academia.edu/MiloGardner/Papers/623827/Egyptian_Fractions_Unit_Fractions_Hekats_and_Wages_-_an_Update}.

The Kahun Papyrus(KP)reported bird valuations by taking pesu amounts of grain consumed by set-ducks and other fowl in 30 days. The KP problem stressed 100-quadruple hekat, which meant greater than 4-hekat units. Converting 100 water fowls into groups of set-ducks a 1/20 of a hekat meant the fowl ate 1/20 of a hekat every 30 day. The scaling method exposed a Middle Kingdom economic unit as well as a theoretical mathematical foundation.

The KP text ,for example, was transliterated by Griffith and published in \PMlinkexternal{Marshall Clagett, Ancient Egyptian Science, Vol III,1999}{http://books.google.com/books?id=8c10QYoGa4UC&pg=PA469&dq=Ancient+Egyptian+Science}. Clagett published a standard 1920s KP and RMP transliteration that improperly suggests certain incomplete bits information were complete. Complete initial and intermediate calculation steps were omitted by the Middle Kingdom scribes, a situation that Clagett and earlier transcribers did not know how correct.

To achieve complete translations, muddled transliterations have been corrected. The oldest text may be the 1950 BCE Akhmim Wooden Tablet (AWT). Hana Vymazalova began the correction process. Vymazalova proved an important aspect of AWT arithmetic in 2002. Ahmes bird-feeding method used the same arithmetic division aspect in 1650 BCE. The AWT and RMP scribes began with (64/64), a hekat unity, as Vymazalova showed by five proofs. 

Vymazalova's hekat unity (64/64) set of five proofs opened an important door. The method solved one level of Ahmes' bird feeding rate problem, and like problems reported in the Kahun Papyrus(KP). Vymazalova  corrected  Daressy's two 1906 "errors", divisor 11 and divisor 13 cases, and showed that all five divisions of hekat were proven by the initial  hekat unity (64/64) were exactly returned five times.

ADDITIONAL CONSIDERATIONS
 
The AWT hekat unity proofs validated that an 'ab initio' encoded method was also used by Ahmes 250 years later over 70 times. Ahmes, for example, demonstrated that tghe multiplication by 1/6 of a hekat unity by writing:

the 1/6 portion of the hekat steps looked like this:

(64/64) times 1/6 = 10/64 + 4/38

(8 + 2)64 + (20/6)ro

and rescaled the remainder 4/384 by (5/5) to 20/2220

but wrote (20/6)ro, since 1/320 was name ro that wrote (3 + 1/3)ro

as the remainder of the final unit fraction two-part statement:

(1/8 + 1/32)hekat + (3 + 1/3)ro

The non-additive aspects of the AWT and RMP division problems were parsed in a 2005 study and published in 2006. The study contrasted 40 hekat volume unit measures and found a generalized quotient and remainder hekat unity division method that limited n to the range of 1/64 < n < 64. The AWT and RMP method defined a hekat unity as (64/64) by Hana Vymazalova and her 2002 paper that divided the unity by several rational numbers n. Answers were written in binary quotients and scaled Egyptian fraction series when n was limited to the range 1/64 < n < 64. Scribal divisor n values a quotient and remainder answers were written (defining hekat subunits such that 10/n hin, 64/n dja (per Tanja Pemmerening's 2002 paper), and 320/n ro in the AWT, RMP, and the Ebers Papyrus.

Exact binary quotients and scaled ro remainders were used by Egyptian Middle Kingdom from 2000 BCE to 1500 BCE. RMP 83 and a division of a hekat unity method was oddly discussed by Chace, concluding "... the author (Ahmes) does not specify explicitly how he performed them". The broader context of the Old Kingdom Horus-Eye aspect of Ahmes' bird-feeding rate problem were jointly published in 2006 by Gardner, that exposed Ahmes' thinking.

Theoretically expected usage rates were calculated by Ahmes and Middle Kingdom scribes to control major commodity inventories. Knowing expected usages allowed decentralized managers double checked actual inventory usages, calculating profits, losses, paying taxes, and creating contracts of many types within a double entry accounting system. Overall, the economic context of Egyptian fractions contained theoretical elements, one being Ahmes' bird-feeding rate division method. In total, innovative Egyptian fraction mathematics facilitated centralized and decentralized business practices, including land rental contracts operated by absentee landlords.

With bird valuations reported by:

RMP 83 Ahmes considered RMP 83 within three Akhmim Wooden Tablet divisions of (64/64)/n (n =10, 7 and 3) within a  table of set-duck (rated at 1) valued at (1/32 + 1/64)hekat + 1 ro, which Ahmes recorded:

3/64 + 4/(20*64) = (64/64)/20

as well

1/40 a hekat = 1/64 hekat + (1/2 + 1/10)ro

Working the Akhmim Wooden Tablet method in reverse order Ahmes published an abbreviated table of fowl valuations* based on a set-duck consuming 1/20 of a hekat, compared to a small number of fowl consumption rates rated a 2, 4 and 8 times more valuable.

Ahmes' data compare binary quotients and scaled ro remainder information to RMP 47 data of the same type by:

Hekat/n recorded in binary quotients (hekat) and scaled ro remainders
=====================================================================

a. 1/40 (64/64)/40 = 1/64 hekat + [24/40(64) = (120/40(ro)]   

1/64 hekat + 3 ro

Ahmes' set-goose, dove and quail hekat consumption rate*

b. 1/20: (64/64)/20 = 3/64 + 20/20 ro =

(2 + 1)/64 + ro

Ahmes' valuation of a set-duck*

c. 1/16: (64/64)/16 = 4/64 = 1/16 hekat

1/10: (64/64)/10 = 6/64 + 20/20 =

[(4 + 2)/64 = 1/16 + 1/32] + 1 ro

(an AWT number ... and a djendjen consumption rate*)

d. 1/9: (64/64)/9 = 7/64 + 5/9 ro = ( 4 + 2 + 1)/64 + [5/9 = (10/18)ro = (9 + 1)/18)]

(1/16 + 1/32 + 1/64)hekat + (1/2 + 1/18)ro

e. 1/8: (64/64)/8 = 8/64 = 1/8 hekat

f. 1/7: (64/64)/7 = 9/64 + 5/7 ro =

(8 + 1)/64 + [(10/14)ro = (7 + 2 + 1) /14 ro] =

(1/8 + 1/64)hekat + (1/2 + 1/7 + 1/14)ro

(another AWT number)

g. 1/6: (64/64)/6 = 10/64 + 20/6 ro =

(8 + 2)/64 + (3 + 1/3)ro=

1/8 + 1/32) hekat + (3 + 1/3)ro

h. 1/5: (64/64)/5 = 12/64 + 20/6 ro =

(8 + 4)/64 hekat + (3 + 1/3)ro

(1/8 + 1/16)hekat + (3 + 1/3)ro

(Ahmes' terp-goose hekat consumption rate*)

i. 1/4: (64/64)/4 = 16/64 = 1/4 hekat

j. 1/3: (64/64)/3 = 21/64 + 5/3 ro =

(16 + 4 + 1)/64 + (1 + 2/3)ro =

(1/4 + 1/16 + 1/64 hekat + ( 3 + 2/3) ro

(another AWT number)

k. 1/(5/2): (64/64)/2 + (64/64)/8 =

(1/2 + 1/8)hekat

(Ahme' re-goose hekat consumption rate*)

l. 1/2: (64/64)/2 = 32/64 = 1/2 hekat

m. 1/1. (64/64)/1 = 64/64 = 1 hekat

Raw hekat consumption and division of a hekat by 40, 20, 10,  9, 8, 7, 6, 5, 4, 3, 5/2, 2, and 1 data has extrapolated Clagett's fowl valuations (in the KP) were based on fowl hekat consumption rates* in RMP 83. Consumption rates offer an important scribal economic context that Ahmes may explicitly have validated by several formulas, checking one economic unit, and calculation, against another economic unit, and related calculation.

*An unresolved aspect of hekat calculations is reported in RMP 43 and the Kahun Papyrus where the volume formula

V = (2/3)(H)[(4/3)(4/3)(D)(D)] (khar) 

divided khar info by 20 that reached units of 400-hekat, and not 100-hekat. RMP 41, 42, 43, 44, 45, 46 and 47 contained hieratic symbols for 400-hekat and 100-hekat units that seemed interchangeable, but are not. 

Archaeological studies fairly scale one hekat to 4800 ccm, 1/10 hekat (hin) to 480 ccm and 1/320 of a hekat (ro) to 15 ccm. 

An unsolved aspect from RMP 47 mixed two 400-hekat multiplications (by 1/10 and 1/20) found (10) 4-hekat and (5) 4-hekat, respectively, with eight 100-hekat multiplications (by 1/30, 1/40, 1/50, 1/60, 1/70, 1/80, 1/90 and 1/100) found 1-hekat quotients and 1-ro remainders. The scribal scaling method followed the AWT two-part number pattern, and two 400-hekat multiplications (by 1/10 and 1/20) found 4-hekat quotients (4-ro remainders were not cited, but would have been required if 1/30 was the multiplier).  

\begin{thebibliography}{9}

\bibitem{1}A.B. Chace, Bull, L., Manning, H.P. and Archibald, R.C. \emph{The Rhind Mathematical Papyrus}, Mathematical Association of America, Vol 1, 1927, vol 2, 1929, and reprint 1979 (NCTM).
\bibitem{2}A.B. Clagett, Marshall \emph{Ancient Egyptian Science, Vol 3}, American Philosophical Society, Philadelphia 1999.
\bibitem{3} Georges Daressy, \emph{"Calculs Egyptiens du Moyan Empire, Recueil de Travaux Relatifs  De La  Philologie et al Archaelogie Egyptiennes Et Assyriennes XXVIII}, Paris, 1906.
\bibitem{4} Milo Gardner, \emph{An Ancient Egyptian Problem and its Innovative Solution, Ganita Bharati}, MD Publications Pvt Ltd, 2006.
\bibitem{5} Milo Gardner, \emph{The Egyptian Mathematical Leather Roll, Attested Short Term and Long Term}, History of the Mathematical Sciences, Editors: Ivor Gratton-Guinness, and B.S. Yadav, Hindustan Book Agency, 119-134, 2004.
\bibitem{6}Richard Gillings, \emph{Mathematics in the Time of the Pharaohs}, Dover Books, 1992.
\bibitem{7} T.E. Peet, \emph{Arithmetic in the Middle Kingdom}, Journal Egyptian Archeology, 1923.
\bibitem{8} Tanja Pommerening, \emph{"Altagyptische Holmasse Metrologish neu Interpretiert" and relevant phramaceutical and medical knowledge, an abstract,  Phillips-Universtat, Marburg, 8-11-2004, taken from "Die Altagyptschen Hohlmass}, Buske-Verlag, 2005.
\bibitem{9} Hana Vymazalova, \emph{The Wooden Tablets from Cairo:The Use of the Grain Unit HK3T in Ancient Egypt, Archiv Orientalai}, Charles U Prague, 2002.
\end{thebibliography}



%%%%%
%%%%%
\end{document}
