\documentclass[12pt]{article}
\usepackage{pmmeta}
\pmcanonicalname{MayanSuperNumberArithmetic}
\pmcreated{2014-10-25 6:35:35}
\pmmodified{2014-10-25 6:35:35}
\pmowner{milogardner}{13112}
\pmmodifier{milogardner}{13112}
\pmtitle{Mayan Super number arithmetic}
\pmrecord{80}{42550}
\pmprivacy{1}
\pmauthor{milogardner}{13112}
\pmtype{Definition}
\pmcomment{trigger rebuild}
\pmclassification{msc}{01A12}
\pmsynonym{modular arithmetic}{MayanSuperNumberArithmetic}

\endmetadata

% this is the default PlanetMath preamble.  as your knowledge
% of TeX increases, you will probably want to edit this, but
% it should be fine as is for beginners.

% almost certainly you want these
\usepackage{amssymb}
\usepackage{amsmath}
\usepackage{amsfonts}

% used for TeXing text within eps files
%\usepackage{psfrag}
% need this for including graphics (\includegraphics)
%\usepackage{graphicx}
% for neatly defining theorems and propositions
%\usepackage{amsthm}
% making logically defined graphics
%%%\usepackage{xypic}

% there are many more packages, add them here as you need them

% define commands here

\begin{document}
INTRODUCTION Four Mayan super-numbers:  341640, 1195740, 1765140 and 2448420 were recorded on a wall near Tikal, Guatemala around 419 AD. All four highly composite long count distance numbers were multiples of 780, a nominal Mars cycle, one calendar round (18980) and three calendar rounds The four distance numbers were also multiples of other nominal planetary cycles, including Mercury (117), moon (260), Venus (584, 585), earth (360, 365) and 2340, 4680.

A. MATH ISSUES: Concerning Mars, outside of almanacs, the four 419 AD super-numbers divided by 780 created quotients 438, 1533,  2263 and 3139, respectively. The quotients describe annual retrogrades that can be further parsed into base 10 arithmetic by considering:

1 86400 sec/day divided by 90 = 960(780) = 748800 days to lose one full Mars day 

2. 748800 =6400(117) = 2880 (260) = 2080(360) = 1280(585) = 960(780)

Note 748800 is a composite number 160(4680) that is numerically  consistent with Mayan calender round super-number math. 

Mayan super numbers were reported by Aveni in 1992 "The Sky in Mayan Literature" per:

1. 2920 =8(365)= 5(584)

2. 4680 = 40(117) = 18(260) = 13(360)= 8(585)=6(780)

3. 18980 = 73(260) = 52(365) 

4. 37960 = 146(260) = 104(365) 

5. 56940 = 219(260) = 156(365) =  73(780)

6. 1366560 = 5256(260) = 1752(780) =3744(365)= 72(18980)

In 2012 Aveni, et al, data reported the four Mayan super-number

1. 341640 parsed by  2920(117) = (6)(219)(260) = 949(360) = (6)(156)(365)= 585(584) = (6)(73)(780)= 146(2340) = 73(4680)= 18(18980) = 9(37960) = 6(56940) appear over 28 times (still counting)  as LCMs of 260, 360,364, 365, 584, 585, and 780 taking 2 cycles, 3 cycles, 4, 5, 6 and 7 cycles at a time.

2. 1195740 = 10220(117) = (21)(219)(260) = 3285(364)= (21)(156)(365)= 2044(585)= (21)(73)(780) = 63(18940) = 21(56940) meant LCM (260, 364, 365, 585) = LCM (364, 365, 585,780) and two other LCMs.

3. 1765140 = (31)(219)(260) = (31)(156)(365) = (31)(73)(780)=(3)(31)(18980) = 31(56940)= LCM (260, 365, 780, 2263)

4. 2448420 = (43)(219)(260) = (43)(156)(365) = (43)(73)(780) = (3)(43)(18980)= 43(56940) =  LCM (260, 365, 780, 3139)

*validates LCM  quotients of 260, 365, 780 and other planets can be any integer.  Most LCMs lead to calendar rounds, the domain of Mayan math and astronomy.

The most basic Mayan almanac reported the cycles of Mercury (117) and Venus (585) were scaled to Mars (780) and lunar(260) on three levels. The Mercury 117 day nominal cycle was factored into 9(13). 

Second the nominal aspect of Mercury (117) and Venus (585) were extended to 2340, a least common multiple cited within the four-part Mayan culture by:

2340 = 20(117)= 4(585) planetary cycles that also equaled 9(260) and 3(780).

A second level of the Dresden Codex seasonal almanac scaled four lines of  quotients that summed to 91 by 20 = 1820 and added remainders that reached 1898, 1898, 1924 and 1911. The Madrid and Dresden contained calendar round texts. 

A third level of the \PMlinkexternal{Dresden Codex seasonal almanac}{http://planetmath.org/mayanseasonalalmanac} discussed earth cycles(360, 364, 365) in terms of the lunar (260), Venus (584, 585) and Mars(780) in 20 additional meta-cycles that summed to  2-calendar rounds 38960, 38960, 38960 + 260 and 38960 + 520. 

Scholars have suggested that the four stations of Mercury and Venus, periods of first and last visibility, as morning stars and evening stars, and periods of invisibility explain the nominal numbers. Mercury's four stations are symmetric averaging 35 days before the first conjunction, 5 days invisibility and 38 days as morning star and 38 days as the evening star. Venus' four stations are not symmetric, as Mayans (Dresden) and Mixtecs (Grolier) reported 8 days, 236 days, 90 days (morning star) and 250 days (evening star), by the Brickers (2011).

B. LANGUAGE \PMlinkexternal{ISSUES}{http://en.wikipedia.org/wiki/Mayan_languages}, stative and positional words: In Mayan languages, words were usually viewed as belonging to one of four classes: verbs, statives, adjectives, and nouns. Statives are a linguistic class of predicative words expressing a quality or state, whose syntactic properties fall in between those of verbs and adjectives in Indo-European languages. Verb and stative words are sometimes inflected for person but normally lack inflections for tense, aspect and other purely verbal categories. This is very similar to the so-called Japanese "adjectives". Stative words were  positional and numeral adjectives.

Positional words were a class of root words characteristic of, if not unique to, the Mayan languages, form stative adjectives and verbs (usually with the help of suffixes) with meanings related to the position or shape of an object or person. Mayan languages have between 250 and 500 distinct positional roots.

Mayans and regional cultures also stressed traditional Olmec four-part numbers in stative and positional language notations in animal and other images and added colors for emphasis.  Stative words were qualitative and positional. 

Quantitative words and numbers recorded greatest common divisors (GCD), least common divisors (LCM) and other number theory ideas in Mayan texts. At times Mayan quotients were red in color and remainders were black in color. It should be noted that the same color could be used for multiple stative purposes.    

After 1200 AD the Aztecs, a nearby  culture, considered (stative words) that represented east as an ideal aspect of life, south as a good aspect of life, weest as a bad aspect of life, and north as an evil aspect of life used symbols from the north that governed conquered ares by force to obtain annual tribute payments. 

Four-part Mayan aspects of life were active at all times. Mayans before and after 1200 AD limited meanings of four-part aspects of astronomy by scaling two planetary cycles in almanac LCMs and GCDs. Almanac extended practical four-part day names [two from 260 factored from 13(20) and two from 360 factored 18(20)] to political and local issues. Mayan scribes converted theoretical numerations base numbers to regional color based and other action phrase often pointing actual actual astronomical events. Red, black, white, and green, for example, were complementary aspects of four-parts of a given subject.

C. ANALYSIS: Recent scholarship suggests that Mayans selected two or more planetary cycles that converted theoretical almanac problems into day names and other ideas to report practical solutions. That is, Mayan created super-numbers (GCDs) and LCMs for purposes for almanac solution of no more than four astronomical cycles in practical named days, with the four directions encoded as trees in the Dresden Codex Mercury-Venus almanac, the later in ways that are not fully understood.

To explore Mayan two or more planetary cycles summed to super-numbers, Mars and other planets are discussed in the journal SCIENCE (May 2012). Four super-numbers were written on a wall near Tikal, Guatemala shortly after 419 AD. The smallest super-number 341640 can also be divided by 11960 (405-moons, a Mayan lunar calendar cycle) such that 28 + 26(260) offers another quotient and scaled remainder. The same class of 260 scaled remainder can be found by dividing other well known Mayan super-number with smaller Mayan super-numbers:

1. 37960/1820 = 20 + 6(260)

2. 37960/11960 = 3 + 8(260)

3. 11960/1820 = 20 + 6(260)

Mayan quotients and scaled remainders requires consistent numeration systems to translate historical data into modern base 10 arithmetic. Given that Mayan modular arithmetic, defined nominal (theoretical) planetary and lunar cycles, was used by Mayans to predict astronomical events. Studying Mayan almanacs, for example one parsed in \PMlinkexternal{1988}{http://adsabs.harvard.edu/full/1988JHAS...19....1B} and further discussed in \PMlinkexternal{2011}{http://www.amazon.com/Astronomy-Codices-Memoirs-American-Philosophical/dp/0871692651} may assist in pointing out additional Mars reelated super-numbers. 

Four Mayan super-numbers were mentioned 1,700 years ago. The intellectual effort that created the numbers may have modified 4,000 year old Egyptian rational numbers arithmetic that scaled n/p by LCM m/m to mn/mp in Mayan modular arithmetic. Of course, Mayans most likely, knew nothing of Egyptian math. 

Mayan rational number math was written in clock and modular arithmetic. Mayan inspected the cycles of our moon, Mercury, Venus, earth, Mars and possibly Saturn and Jupiter in nominal and actual way by considering 1,000 year older Olmec long count modular arithmetic methods.

The Olmec long count contained five digits: A.B.C.D.E The smallest digit E used 365 days (kin) as a base. The second smallest digit D used 20 days (uinal); the third smallest digit C used 360 (tun); the fourth B used 7200 (katun) and the largest A used 144,000. Mayans loved to create number systems of this type, some longer and some shorter.  

D. POSSIBLE CONNECTIONS TO EGYPT, GREECE AND MEDIEVAL MATH: Western astronomers created least common multiples (LCM) and other prime number based mathematics. Even though ancient Egyptian astronomical records have not been fully decoded by scholars, Egyptian math and 2/n tables were used by Greeks, Arabs and medieval scribes used LCMs to scale rational numbers before writing out concise unit fraction series. Egyptian, Greek, Hellene, Arab and medieval divisibility rules stressed prime number factoring of quotients and remainders. By 1650 BCE Ahmes created a 2/n table by scaling 2/n from 2/3 to 2/101 by an LCM m/m such that 2/n was converted to concise unit fraction series by solving 

2/p(m/m) = 2m/mp

Three examples

1. 2/7 times 4/4 = 8/28 = (7 + 1)/28 = 1/4 + 1/28

2. 2/41 times 24/24 = 48/984 = (41 + 4 + 3)/984 = 1/24 + 1/246 + 1/328

3. 2/43 times 42/42 = 84/1806 = (43 + 21 + 14 + 6)/1806 = 1/42 + 1/86 + 1/129 + 1/301

\PMlinkexternal{RMP 2/n table}{http://rmprectotable.blogspot.com/}


%%%%%
%%%%%
\end{document}
