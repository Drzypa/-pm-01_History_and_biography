\documentclass[12pt]{article}
\usepackage{pmmeta}
\pmcanonicalname{ReisnerPapyrus}
\pmcreated{2013-03-22 18:27:19}
\pmmodified{2013-03-22 18:27:19}
\pmowner{milogardner}{13112}
\pmmodifier{milogardner}{13112}
\pmtitle{Reisner Papyrus}
\pmrecord{35}{41118}
\pmprivacy{1}
\pmauthor{milogardner}{13112}
\pmtype{Definition}
\pmcomment{trigger rebuild}
\pmclassification{msc}{01A16}
%\pmkeywords{quotients}
%\pmkeywords{remainders}

% this is the default PlanetMath preamble.  as your knowledge
% of TeX increases, you will probably want to edit this, but
% it should be fine as is for beginners.

% almost certainly you want these
\usepackage{amssymb}
\usepackage{amsmath}
\usepackage{amsfonts}

% used for TeXing text within eps files
%\usepackage{psfrag}
% need this for including graphics (\includegraphics)
%\usepackage{graphicx}
% for neatly defining theorems and propositions
%\usepackage{amsthm}
% making logically defined graphics
%%%\usepackage{xypic}

% there are many more packages, add them here as you need them

% define commands here

\begin{document}
Reisner Papyrus, From Wikipedia

The Reisner Papyrus (RP) is one of the basic hieratic mathematical/economic texts. The text was found in 1904 by George Reisner and dates to the 1800 BCE period. The text was translated in the 1920s close to its historical arithmetic, omitting several basic economic implications. The document is housed in the Boston Museum of Fine Arts (MFA) along with supporting several documents. Gillings and other scholars reported out-dated 1920s views of the arithmetic and economic implications by accepting misleading translations. 

The text's tables 22.2 and 22.2 detail a scribal division by 10 method. The method also appears in the first six problems solved in the 1650 BCE Rhind Mathematical Papyrus(RMP). Egyptian fractions data followed a method defined in the Egyptian Mathematical Leather Roll(EMLR) and the \PMlinkexternal{RMP 2/n table}{http://en.wikipedia.org/wiki/Rhind_Mathematical_Papyrus_2/n_table}. Labor efficiencies and wage payments were included the Reisner Papyrus. Dockyard workshop labor was monitored in man-days by applying a division by 10 method. Wage payments were monitored by disks, pegs, cones and diamond shaped objects that could be placed around a worker's neck, named Terseset (trssr). This class of objects was mentioned in the Kahun Papyrus.  

The arithmetic computed a labor efficiency rate by answering how deep did 10 workmen dig (in building a chapel) in one day? Gillings in "Mathematics in the Time of the Pharaohs", repeated an incomplete view of the RP. Gillings analyzed lines G10, from table 22.3B, and line 17 from Table 22.2 on page 221, divided

39 by 10 = 4,

a poor rounded off approximation, as properly corrected by Gillings to:

39/10 = (30 + 9)/10 = 3 + 1/2 + 1/3 + 1/15

Yet, other RMP division by 10 answers were correctly stated as scribal quotients and remainders, facts that Gillings did not discuss in scribal economic and mathematical terms. Table 22.2  described the work done in the Eastern Chapel. Additional raw data was listed on lines G5, G6/H32, G14, G15, G16, G17/H33 and G18/H34, as follows:

12/10 = 1 + 1/5 (G5)

10/10 = 1 (G6 and H32)

8/10 = 1/2 + 1/4 + 1/20 (G14)

48/10 = 4 + 1/2 + 1/4 + 1/20 (G15)

16/10 = 1 + 1/2 + 1/10 (G16)

64/10 = 6 + 1/4 + 1/10 + 1/20 (G17 and H33)

36/10 = 3 + 1/2 + 1/10 (G18 and H34)

Peet in 1923 and Robins-Shute in 1987 noted the RP division by 10 method as also explained in the RMP. Peet and Robins-Shute muddled the quotients and remainders basis of the RP, and the RMP. Scholars from the 1920s had taken an additive position that muddled the reading of the first six problems of the RMP missing meta aspects of the 150 year older RP quotient and remainder method.

Gillings, Robins-Shute. and 1920's scholars (i.e. Peet) had not analyzed RMP data in a meta context. Early scholars had only reported transliteration aspects of  hieratic arithmetic. Meta mathematical aspects of the RP, for example, connect its arithmetic to the Akhmim Wooden Tablet (AWT), the RMP, and to other texts, had been omitted by scholarly reviews. That is, Gillings' review of the RP, and RMP, had scratched the surface of hieratic arithmetic. Had scholars dug a little deeper, looking for meta mathematics, they may have found 80 years ago that quotients and remainders had been present in the RP, correcting its 39/10 error, as well as structuring several hieratic mathematical texts.

The RP errors could have been resolved by Gillings in terms of a possible uses of well known scribal quotients and remainders. Ahmes and scribes used the RP method many times. Ahmes has been minimized by 1920s scholars to have not fully understood meta quotients and remainders. 

The RMP's first six problems were written in a meta division by 10 context that scales quotients and remainders. Gillings had forgotten to summarize his RP findings in a rigorous mathematical manner. Had he done several Middle Kingdom texts would have identified that used the same scaled rational numbers recorded to concise unit fraction quotients and remainders.

Seen on a meta level it is clear that the Reisner Papyrus scribe understood:

39/10 = (Q' + R)/10 with Q' = (Q*10), Q = 3 and R = 9

such that:

39/10 = 3 + 9/10 = 3 + 1/2 + 1/3 + 1/15

with the remainder written in Egyptian fractions. The remainder 9/10 was converted to a unit fraction series by following scaling rules set down in the AWT, the EMLR, and other hieratic texts. The RP was one of several texts that structured the 150 year younger RMP, that began with six RP-type labor valuation problems.

The RP remainder arithmetic has been directly confirmed by older and younger hieratic texts. The oldest text was the  2000 BCE to 1850 BCE AWT. The AWT defined scribal remainder arithmetic in term of another scaled rational quotient and remainder context, a hekat volume unit scaled to 64/64. The AWT wrote quotients as integers and binary fractions, and scaled (by a multiplier 5/5) remainders. Oddly, Gillings did not cite aspects of this vivid AWT arithmetic in "Mathematics in the Time of the Pharaohs", even though the AWT was included in the index of Gillings other-wise ecellent book. 

Gillings and 1920's scholars had missed several opportunities to point out likely uses of scribal remainder arithmetic built upon quotients and remainders. Gillings and the 1920s academic community that preceded him had inadvertently omitted a critical scaled rational number discussions of scribal remainder arithmetic. Remainder arithmetic had been used in other ancient Near East situations, two being solar and lunar calendars. Georges Daressy suggested a third use of AWT remainders within a discussion that was not revisited by Hana Vymazalova until 2002, a topic that was confirmed by a generalized approach in 2006.

Concerning labor wages, MFA 24 is a disk used to pay 2/3 hekat in wages over a 10 day period meant that daily wages were 1/15 hekat and monthly wages 2 hekat for a foreman of a work crew. Higher classifications of workers were paid upto to 8 hekat a month based on family size. Confirming information is included in the Hekanakaht Papers and the Abu Sir Papyri.

In summary, the RP was built upon a division of a hekat method described in the AWT, a text 200 years older than the RP. The RP method was adopted by Ahmes in the  150 year younger RMP. Concerning modern scholarship, the RP method follows Occam's Razor, the simplest method is the historical method. The RP remainder was written in terms of:

n/10 = Q + R/10

where Q, an integer or zero quotient, and R, a remainder.

The RP reports a 10 workmen digging rate using an exact quotient and remainder method. The digging rate method was reported in the RMP within its first six problems, thereby confirming the RP division method as a meta formula.

The wage payment side of the terseset(trsst) method has been confirmed by the Kahun Papyrus, the Hekanakht Papers and 
the Abu Sir Papyrus.
 
\begin{thebibliography}{6}
\bibitem{1} Milo Gardner, \emph{An Ancient Egyptian Problem and its Innovative Solution, Ganita Bharati}, MD Publications Pvt Ltd, 2006.
\bibitem{2}Richard Gillings, \emph{Mathematics in the Time of the Pharaohs}, Dover Books, 1992.
\bibitem{3} T.E. Peet, \emph{Arithmetic in the Middle Kingdom}, Journal Egyptian Archeology, 1923.
\bibitem{4} Gay Robins and Charles Shute, \emph{The Rhind Mathematical Papyrus}, British Museum Publications, 1987, Dover reprint available.
\bibitem{5} William Kelly Simpson \emph{The Reisner P}, JEA 59, 1973, pages 220-222.
\bibitem{6} Hana Vymazalova, \emph{The Wooden Tablets from Cairo:The Use of the Grain Unit HK3T in Ancient Egypt, Archiv Orientalai}, Charles U Prague, 2002.
\end{thebibliography}


%%%%%
%%%%%
\end{document}
