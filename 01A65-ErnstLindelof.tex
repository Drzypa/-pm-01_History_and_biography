\documentclass[12pt]{article}
\usepackage{pmmeta}
\pmcanonicalname{ErnstLindelof}
\pmcreated{2014-11-07 11:21:08}
\pmmodified{2014-11-07 11:21:08}
\pmowner{pahio}{2872}
\pmmodifier{pahio}{2872}
\pmtitle{Ernst Lindel\"of}
\pmrecord{14}{40168}
\pmprivacy{1}
\pmauthor{pahio}{2872}
\pmtype{Biography}
\pmcomment{trigger rebuild}
\pmclassification{msc}{01A65}
\pmclassification{msc}{01A60}
\pmsynonym{Ernst Leonard Lindel\"of}{ErnstLindelof}
\pmrelated{LarsAhlfors}
\pmrelated{RolfNevanlinna}
\pmrelated{Lindelof}
\pmrelated{EulersDerivationOfTheQuarticFormula}
\pmrelated{MethodOfIntegratingFactor}
\pmrelated{TaylorSeriesViaDivision}
\pmrelated{CurvatureDeterminesTheCurve}
\pmrelated{PowerTowerSequence}
\pmrelated{NormalCurvatures}
\pmrelated{NonExistenceOfUniversalSeriesConvergenceCriterion}
\pmrelated{Multiplic}

% this is the default PlanetMath preamble.  as your knowledge
% of TeX increases, you will probably want to edit this, but
% it should be fine as is for beginners.

% almost certainly you want these
\usepackage{amssymb}
\usepackage{amsmath}
\usepackage{amsfonts}

% used for TeXing text within eps files
%\usepackage{psfrag}
% need this for including graphics (\includegraphics)
%\usepackage{graphicx}
% for neatly defining theorems and propositions
 \usepackage{amsthm}
% making logically defined graphics
%%%\usepackage{xypic}

% there are many more packages, add them here as you need them

% define commands here

\theoremstyle{definition}
\newtheorem*{thmplain}{Theorem}

\begin{document}
Ernst Leonard Lindel\"of (1870-3-7 \`a 1946-6-3) was born in 
Helsinki, Finland.  Professor of mathematics in University of 
Helsinki 1903--1938.  \PMlinkescapetext{Lindel\"of} was the 
founder of the internationally noted Finnish school of function 
theory.  Rolf Nevanlinna and Lars Ahlfors were his pupils.  Most 
studies of \PMlinkescapetext{Lindel\"of} were on the complex 
analysis, theory of differential equations (e.g. 
\PMlinkname{Picard--Lindel\"of theorem}{picardstheorem-0}) and 
topology (see e.g. Lindel\"of space).  One of them, 
``{\em M\'emoire sur la th\'eorie des fonctions enti\`eres de 
genre fini}'' (1902) \PMlinkescapetext{complements} 
substantially the theory of entire functions; also 
``{\em M\'emoire sur certaines in\'egalit\'es dans la th\'eorie 
des fonctions monog\`enes} (1908), containing a new simple 
proof of \PMlinkname{Picard's theorem}{PicardsTheorem}, was 
important.  \PMlinkescapetext{Lindel\"of} has made excellent 
textbooks of mathematics: ``{\em Le calcul des r\'esidus et ses 
applications \`a la th\'eorie des fonctions}'' (1905), ``{\em  
Johdatus korkeampaan analyysiin}'' (`Introduction to the higher 
analysis'; 1912), ``{\em Differentiali- ja integralilasku ja 
sen sovellutukset I--IV}'' (`Differential and integral calculus 
with applications I--IV'; 1928--1946). 

%%%%%
%%%%%
\end{document}
