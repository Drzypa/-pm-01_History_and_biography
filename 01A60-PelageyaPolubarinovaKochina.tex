\documentclass[12pt]{article}
\usepackage{pmmeta}
\pmcanonicalname{PelageyaPolubarinovaKochina}
\pmcreated{2013-03-22 17:05:51}
\pmmodified{2013-03-22 17:05:51}
\pmowner{Mravinci}{12996}
\pmmodifier{Mravinci}{12996}
\pmtitle{Pelageya Polubarinova-Kochina}
\pmrecord{4}{39393}
\pmprivacy{1}
\pmauthor{Mravinci}{12996}
\pmtype{Biography}
\pmcomment{trigger rebuild}
\pmclassification{msc}{01A60}
\pmclassification{msc}{01A61}
\pmclassification{msc}{01A65}
\pmsynonym{Pelageya Yakovlevna Polubarinova-Kochina}{PelageyaPolubarinovaKochina}
\pmsynonym{Pelageya Yakovlevna Polubarinova}{PelageyaPolubarinovaKochina}
\pmsynonym{Pelageya Kochina}{PelageyaPolubarinovaKochina}

\endmetadata

% this is the default PlanetMath preamble.  as your knowledge
% of TeX increases, you will probably want to edit this, but
% it should be fine as is for beginners.

% almost certainly you want these
\usepackage{amssymb}
\usepackage{amsmath}
\usepackage{amsfonts}

% used for TeXing text within eps files
%\usepackage{psfrag}
% need this for including graphics (\includegraphics)
%\usepackage{graphicx}
% for neatly defining theorems and propositions
%\usepackage{amsthm}
% making logically defined graphics
%%%\usepackage{xypic}

% there are many more packages, add them here as you need them

% define commands here

\begin{document}
\emph{Pelageya Yakovlevna Polubarinova-Kochina} n\'ee \emph{Pelageya Yakovlevna Polubarinova} (1899 - ) Soviet Russian mathematician with a degree in pure mathematics best known for her work on the application of Fuchsian differential equations to hydrodynamics.

Born in czarist Russia of an accountant and a housewife, young Pelageya was the second youngest of four children, she studied at a women's high school in St. Petersburg and went on to Petrograd University (after the Revolution of 1917). Her father died in 1918 so she started to take care of the family by working at the laboratory of geophysics. There she met Nikolai Kochin, and they married in 1925 and she bore him two daughters. The two taught there until World War II, when Polubarinova-Kochina and their daughters were evacuated to Kazan and Kochin stayed in Moscow to work on aiding the military effort. He died before the war was over. After the war, she edited his lectures and continued to teach applied mathematics. She was given the Stalin Prize in 1946.

\begin{thebibliography}{1}
\bibitem{gw} G. W. Phillips ``Pelageya Yakovlevna Polubarinova-Kochina'' in {\it Women of Mathematics: A Bibliographic Sourcebook} L. Grinstein, P. Cambpell, ed.s New York: Greenwood Press (1987): 95 - 102
\end{thebibliography}
%%%%%
%%%%%
\end{document}
