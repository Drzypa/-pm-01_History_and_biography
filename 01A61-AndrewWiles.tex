\documentclass[12pt]{article}
\usepackage{pmmeta}
\pmcanonicalname{AndrewWiles}
\pmcreated{2013-03-22 16:32:12}
\pmmodified{2013-03-22 16:32:12}
\pmowner{PrimeFan}{13766}
\pmmodifier{PrimeFan}{13766}
\pmtitle{Andrew Wiles}
\pmrecord{9}{38718}
\pmprivacy{1}
\pmauthor{PrimeFan}{13766}
\pmtype{Biography}
\pmcomment{trigger rebuild}
\pmclassification{msc}{01A61}
\pmclassification{msc}{01A65}
\pmclassification{msc}{01A60}
\pmsynonym{Sir Andrew John Wiles}{AndrewWiles}
\pmsynonym{Andrew John Wiles}{AndrewWiles}

\endmetadata

% this is the default PlanetMath preamble.  as your knowledge
% of TeX increases, you will probably want to edit this, but
% it should be fine as is for beginners.

% almost certainly you want these
\usepackage{amssymb}
\usepackage{amsmath}
\usepackage{amsfonts}

% used for TeXing text within eps files
%\usepackage{psfrag}
% need this for including graphics (\includegraphics)
%\usepackage{graphicx}
% for neatly defining theorems and propositions
%\usepackage{amsthm}
% making logically defined graphics
%%%\usepackage{xypic}

% there are many more packages, add them here as you need them

% define commands here

\begin{document}
\PMlinkescapeword{incomplete}
\PMlinkescapeword{connection}
\PMlinkescapeword{focus}
\PMlinkescapeword{trek}

\emph{Sir Andrew John Wiles} (born 1953) is a British mathematician who proved Fermat's last theorem. It was Fermat's last theorem that inspired Wiles to go into mathematics at an early age. He studied at Oxford University, later Cambridge, and now teaches at Princeton.

After Kenneth Ribet suggested there might be a connection between the famous Fermat problem and the Taniyama-Shimura-Weil conjecture, Wiles decided to focus entirely on that connection.  A proof he presented in 1993 was incomplete, but Wiles and his former student Richard Taylor were able to fill in the gaps of the proof.  This was after Wiles turned 41, so he was ineligible for the Fields medal, but received plenty of other prizes in recognition, such as the Royal Medal and the Clay Research Award. He was knighted in 2000, and the Clay Mathematics Institute asked him to write the official problem description of the Birch and Swinnerton-Dyer conjecture for the Millennium Problems.

Wiles was mentioned in an episode of {\it Star Trek: Deep Space Nine}, retconning Picard's statement in {\it The Next Generation} that Fermat's last theorem was still unsolved as of the 24th century.

\begin{thebibliography}{1}
\bibitem{pf} P. Farrand {\it The Nitpicker's Guide for Next Generation Trekkers} 2. New York: Dell (1995): 178
\end{thebibliography}


%%%%%
%%%%%
\end{document}
