\documentclass[12pt]{article}
\usepackage{pmmeta}
\pmcanonicalname{EgyptianWeightsAndMeasuresHekatDivisions}
\pmcreated{2014-07-12 7:07:40}
\pmmodified{2014-07-12 7:07:40}
\pmowner{milogardner}{13112}
\pmmodifier{milogardner}{13112}
\pmtitle{Egyptian weights and measures, hekat divisions}
\pmrecord{58}{41728}
\pmprivacy{1}
\pmauthor{milogardner}{13112}
\pmtype{Definition}
\pmcomment{trigger rebuild}
\pmclassification{msc}{01A16}

\endmetadata

% this is the default PlanetMath preamble.  as your knowledge
% of TeX increases, you will probably want to edit this, but
% it should be fine as is for beginners.

% almost certainly you want these
\usepackage{amssymb}
\usepackage{amsmath}
\usepackage{amsfonts}

% used for TeXing text within eps files
%\usepackage{psfrag}
% need this for including graphics (\includegraphics)
%\usepackage{graphicx}
% for neatly defining theorems and propositions
%\usepackage{amsthm}
% making logically defined graphics
%%%\usepackage{xypic}

% there are many more packages, add them here as you need them

% define commands here

\begin{document}
From the Math history list:

The 1650 BCE Rhind Mathematical Papyrus reported RMP 47,82, and 83, three of twelve RMP hekat (unity) division problems. The same class of RMP problem was recorded 250 years earlier in the Akhmim Wooden Table (AWT). The AWT scaled one hekat (4800 ccm in modern metrics) to (64/64) and divided the unity by n six times created quotients (Q)/64 and scaled remainders (5R/n)(1/320 of a hekat = ro) in two-part statements. In 2005 42 hekat division of (64/64) problems were decoded in the two-part pattern. The information was published in 2006. In 2010 ten RMP 47 100-hekat problems were added that scaled 100-hekat to (6400/64) and divided (6400)/64)/64 by n in the two-part pattern. Ahmes actually introduced 400 hekat divided by by writing 4-(6400/64)/n = Q/64 + (5R/n)ro patterns.

The details of RMP 47 shows that Ahmes divided 100-hekat, written as (6400/64) by n in a manner that Griffith and Spalinger (1990) considered an open issue with respect to an alternate 400-hekat scaling. Ahmes, in my view, scaled 100-hekat to (6400/64) and divided the hekat unity by n = 10, 20, 30, 40, 50, 60, 70, 80, 90 and 100, recording Q/64 + (5R/n)ro in encoded two-part answers ten times. 

For example, in the n = 70 case Ahmes scaled the hekat quotient 91/64 to a binary (Horus-Eye) series considering:

(64 + 16 + 8 + 2 + 1)/64 = (1 + 1/4 + 1/8 + 1/32 + 1/64)hekat 

Ahmes encoded the remainder 30/64 term by scaling by 5/5 obtaining 1/320 ro units such that

(150/70)ro = [2 + [(1/7)(6/6)ro] = [2 + (6/42)ro] = [2+ (3 + 2 + 1)/42]ro = [2 + 1/14 + 1/21 + 1/42]ro 

Ahmes encoded (6400/64)/70 answer reported the complete AWT two-part answer by:

(1 + 1/4 + 1/8 + 1/32 + 1/64)hekat + (2 + 1/14 + 1/21 + 1/42)ro

Robins-Shute reported an unclear aspect of Ahmes' hard-to-read RMP 47 shorthand in 1987. Robins-Shute arithmetically found a correct answer, (2 + 1/7)ro. Robins-Shute seem to have garbled Ahmes' beginning and intermediate information by reporting Ahmes' remainder by not noticing Ahmes' conversion of 1/7 to a unit fraction series that considered:

1/7 = 6/42 = (3 + 2 + 1)/42 = 1/14 + 1/21 + 1/42 

Note the same conversion method was used by Ahmes in the 2/n table that converted 2/101 in the meta context:

2/101 = (2/101)(6/6) = 12/606 = (6 + 3 + 2 + 1)/606 = 1/101 + 1/202 + 1/303 + 1/606

the same method that an EMLR scribe used 250 years earlier that converted 1/101 by:

1/101 = (1/101)(6/6) = 6/606 = (3 + 2 + 1)/606 = 1/202 + 1/303 + 1/606 

RMP 82 listed 29 examples of one hekat scaled to (64/64), and divided by rational numbers n. The 29 divisors ranged from 1/64 < n < 64. The 29 problems reported binary quotients and scaled remainders to a 1/320 hekat unit as two-part answere that were preceded by ten RMP 47 examples.

The table of 29 answers also were converted to unscaled equivalent hin, 1/10 of a hekat unit showing that Ahmes generally reported hin units by writing 10/n hin statements.

Tanja Pemmerening pointed out corrected aspects of the unscaled 64/n dja and 320/n ro statements in 2002 and 2005.

RMP 83, the bird-Feeding rate problem, was not correctly read by Chace, nor by other 20th century scholars. Today the 1900 BCE Akhmim Wooden Tablet, and its six division problems, allows economic valuations of one hekat scaled to (64/64) divided by n. In RMP 83 divisors n equaled 6, 20 and 40. Division took place by multiplying 1/6, 1/20 and 1/40. Ahmes' answer calculated 5/8 of a hekat, the amount of grain eaten by six birds in one day.

Ahmes usually began hekat division discussions with a complex example, and proceeded to simpler examples. An exception was RMP 35-38. RMP 35 was the simplest example. It is important to note Egyptian weights and measures are decoded by stripping away \PMlinkexternal{Rhind Mathematical Papyrus 2/n table}{http://rmprectotable.blogspot.com/} to reveal modern rational numbers and modern arithmetic operations.

In \PMlinkexternal{RMP 35-38 and RMP 66}{http://ahmespapyrus.blogspot.com/2009/01/ahmes-papyrus-new-and-old.html}, hekat division replaced a hekat by 320 ro. Ahmes' unit fraction shorthand notations is translated to modern arithmetic as follows:

1. 320 ro was multiplied by 3/10 = 96 ro (RMP 35)

Proof: 3/10 + 6/10 + 1/10 = 1, and 96 ro + 192 ro + 32 ro = 320 ro = 1 hekat

2. 320 ro was multiplied by 1/90 = 3 + 1/2 + 1/18 = 64/18 (RMP 37)

Proof: 64/72 + 64/567 = 1 

3. 320 ro was divided by 7/22 = 101 9/11 (RMP 38)

was proven by 101 9/11 times 22/7 = 320 ro

It may be interesting to note that the initial divisor 7/22 was proven from binary steps yielding 35/11 times 1/10.

4. 10 hekats of fat, 3200 ro, was divided by 365 = 8 + 280/365  (RMP 66)

was proven by the quotient 8 created by the binary steps 1 - 365, 2 - 730, 4 - 1460 and 8 - 2920
and the remainder 280 (3200 less 2920) by the addition of 

246 1/3 + 36 1/2 +  1/6 = 320 ro = 1 hekat 

since the unit fractions contained in the answer (2/3 + 1/10 + 1/2190) were each multiplied by 365.

Summary: Egyptian volume weights and measures replaced a hekat by two equivalents, a unity (64/64), (6400/64), and  320 ro. Division of the hekat, be it (64/64)/n, (6400/64)/n  and (320/320)/n  Ahmes' division information anticipated modern multiplication and division as inverse operations. 

Scribal methods for finding areas of triangles and other shapes are reported in \PMlinkexternal{RMP 53-55}{http://rmp50-60.blogspot.com/} confirms connections to 3,650 year old Egyptian weights and measures topics, and scribal arithmetic operations. For example, RMP 41, 42, and 43 defined the volume of a hekat by a cylinder set pi = 256/81, and set a radius to semi-diameter (D/2), algebraic geometry facts and methods also reported in MMP -10 and the Kahun Papyrus. 

Footnote:

Aspects of hekat calculations are reported in RMP 43 and the Kahun Papyrus where the volume formula

V = (2/3)(H)[(4/3)(4/3)(D)(D)] (khar) 

Scribes divided khar data by 20 to reach 400-hekat and 100-hekat units. RMP 41, 42, 43, 44, 45, 46, and 47 mixed hieratic symbols for 400-hekat and 100-hekat, scribal facts cited on Wikipedia that muddle the ancient information per:

"Problem 47 gives a table with equivalent fractions for fractions of 100 quadruple hekat of grain. The quotients are expressed in terms of Horus eye fractions...

1/10 gives 10 quadruple hekat,
1/20 gives 5 quadruple hekat,
1/30 gives 3 1/4 1/16 1/64 (quadruple) hekat and (1 2/3 ro) (error),
1/40 gives 2 1/2 (quadruple) hekat,
1/50 gives 2 (quadruple) hekat,
1/60 gives 1 1/2 1/8 1/32 (quadruple) hekat (3 1/3) ro  (error),
1/70 gives 1 1/4 1/8 1/32 1/64 (quadruple) hekat (2 1/14 1/21) ro (error)*,
1/80 gives 1 1/4 (quadruple) hekat,  
1/90 gives 1 1/16 1/32 1/64 (quadruple) hekat 1/2 1/18 ro (error),
1/100 gives 1 (quadruple) hekat (error).


*Ahmes actually reported one (5R/n)ro remainder as 2 1/14 1/21 1/42 since (150/70)ro = (2 + 1/7)ro with 1/7 = 1/7(6/6) = 6/42 = (3 + 2 + 1)/42 = 1/14 + 1/21 + 1/42  (a 2/n table rule per 2/101, and an EMLR rule per 1/101).

Ahmes mixed initial 400-hekat multiplications by 1/10 and 1/20 reporting correct answers with 100-hekat (scaled to 6400/64) multiplications by 1/30, 1/40, 1/50, 1/60, 1/70, 1/80, 1/90 and 1/100 into Q/64 quotient and (5R/n) remainders report incorrect answers whenever 4-hekat quotients are added to 1-ro remainders. Correctly scaled RMP 47 answers report one of two mutually exclusive sets of balanced statements: 

A.  Table A reports 4-hekat quotient + 4-ro remainder answers: 

following 4 x (6400/64) x 1/n = (Q/64) 4-hekat + (5R/n)4-ro 

1/10 gives (10) 4-hekat,
1/20 gives (5) 4-hekat,
1/30 gives (3 1/4 1/16 1/64) 4-hekat + (1 2/3)4-ro,
1/40 gives (2 1/2) 4-hekat,
1/50 gives (2) 4-hekat,
1/60 gives (1 1/2 1/8 1/32) 4-hekat +  (3 1/3)4-ro,
1/70 gives (1 1/4 1/8 1/32 1/64) 4-hekat + (2 1/14 1/21 1/42) 4-ro,
1/80 gives (1 1/4) 4-hekat,
1/90 gives (1 1/16 1/32 1/64) 4-hekat (1/2 1/18)4-ro,
1/100 gives (1) 4-hekat, 


B. Table B reports with 1-hekat quotient + 1-ro remainder answers:

following 1 x (6400/64) x 1/n = (Q/64)1-hekat + (5R/n)1-ro 

1/10 gives (10) 1-hekat
1/20 gives (5) 1-hekat
1/30 gives (3 1/4 1/16 1/64) 1-hekat + (1 2/3)1-ro
1/40 gives (2 1/2) 1-hekat
1/50 gives (2) 1-hekat
1/60 gives (1 1/2 1/8 1/32) 1-hekat +  (3 1/3)1-ro
1/70 gives (1 1/4 1/8 1/32 1/64) 1-hekat + (2 1/14 1/21 1/42)1-ro
1/80 gives (1 1/4) 1-hekat
1/90 gives (1 1/16 1/32 1/64) 1-hekat (1/2 1/18)1-ro
1/100 gives (1) 1-hekat

C. The quadruple (400) hekat case is also made by 4-sack and 4-hekat economic shipping units recorded in Northumberland Papyri 1, 2 and 3 published by Barns and Gunn, 1948. Quadruple sack and hekat initial scaled values were monitored into individual bread contents of 4 hekat units of grain (and weights) thereby not mixing initial 4-hekat data with final 1-hekat quotients, and 1-ro remainders. Balanced 4-hekat, 4-ro or 1-hekat, 1-ro quotients and remainders was practiced by Ahmes.  Ahmes reported khar divided by 20 into 400 hekat units by two volume formulas. The 400 hekat and 100-hekat initial divisions byt rational numbers have been translated into one hekat into 4800 ccm, 1/10 of a hekat (hin) into 480 ccm, 4-ro into 60 ccm, and 1-ro = 15 ccm by scholars, often muddling scribal 4-hekat and 4-ro intermediate details.
 
One study reviews Ahmes' quail, dove, duck and geese daily feeding rates (RMP 83) by testing Tanja Pemmerening's 2005 480 ccm 1/10 hekat (hin) volume conclusion. Two duck and geese unaudited conclusions suggest a hin may have held about 250 ccm ... improved studies results are needed to introduce dove,quail, and other water fowl eating rates, hoping to find a larger pattern. Two of three ancient geese mentioned by Ahmes are extinct, reducing the study's sample size.

\begin{thebibliography}{10}

\bibitem{1}A.B. Chace, Bull, L., Manning, H.P. and Archibald, R.C. \emph{The Rhind Mathematical Papyrus}, Mathematical Association of America, Vol 1, 1927, vol 2, 1929, and reprint 1979 (NCTM).
\bibitem{2} Georges Daressy, \emph{"Calculs Egyptiens du Moyan Empire‚ Recueil de Travaux Relatifs  De La  Philologie et al Archaelogie Egyptiennes Et Assyriennes XXVIII, 1906, 62}, Paris, 1906.
\bibitem{3} Milo Gardner, \emph{An Ancient Egyptian Problem and its Innovative Solution, Ganita Bharati}, MD Publications Pvt Ltd, 2006.
\bibitem{4} Milo Gardner, \emph{The Egyptian Mathematical Leather Roll, Attested Short Term and Long Term}, History of the Mathematical Sciences, Editors: Ivor Gratton-Guinness, and B.S. Yadav, Hindustan Book Agency, 119-134, 2002.
\bibitem{5}Richard Gillings, \emph{Mathematics in the Time of the Pharaohs}, Dover Books, 1992.
\bibitem{6} T.E. Peet, \emph{Arithmetic in the Middle Kingdom}, Journal Egyptian Archeology, 1923.
\bibitem{7} Tanja Pommerening, \emph{"Altagyptische Holmasse Metrologish neu Interpretiert" and relevant phramaceutical and medical knowledge, an abstract,  Phillips-Universtat, Marburg, 8-11-2004, taken from "Die Altagyptschen Hohlmass}, Buske-Verlag, 2005.
\bibitem{8} Gay Robins, Charles Shute \emph{"Rhind Mathematical Papyrus", London, British Museum}, British Museum Press, 1987.
\bibitem{9} Anthony Spalinger \emph{"Rhind Mathematical Papyrus", SAK-17}, 1990.
\bibitem{10} Hana Vymazalova, \emph{The Wooden Tablets from Cairo:The Use of the Grain Unit HK3T in Ancient Egypt, Archiv Orientalai}, Charles U Prague, 2002.
\end{thebibliography}


%%%%%
%%%%%
\end{document}
