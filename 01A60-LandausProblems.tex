\documentclass[12pt]{article}
\usepackage{pmmeta}
\pmcanonicalname{LandausProblems}
\pmcreated{2013-03-22 16:38:25}
\pmmodified{2013-03-22 16:38:25}
\pmowner{PrimeFan}{13766}
\pmmodifier{PrimeFan}{13766}
\pmtitle{Landau's problems}
\pmrecord{4}{38843}
\pmprivacy{1}
\pmauthor{PrimeFan}{13766}
\pmtype{Definition}
\pmcomment{trigger rebuild}
\pmclassification{msc}{01A60}

\endmetadata

% this is the default PlanetMath preamble.  as your knowledge
% of TeX increases, you will probably want to edit this, but
% it should be fine as is for beginners.

% almost certainly you want these
\usepackage{amssymb}
\usepackage{amsmath}
\usepackage{amsfonts}

% used for TeXing text within eps files
%\usepackage{psfrag}
% need this for including graphics (\includegraphics)
%\usepackage{graphicx}
% for neatly defining theorems and propositions
%\usepackage{amsthm}
% making logically defined graphics
%%%\usepackage{xypic}

% there are many more packages, add them here as you need them

% define commands here

\begin{document}
{\em Landau's problems} are four conjectures about prime numbers which were unsolved at the time Edmund Landau presented on them at International Congress of Mathematicians of 1912. 

\begin{enumerate}
\item Goldbach's conjecture
\item The twin prime conjecture
\item Legendre's conjecture
\item The conjecture that there are infinitely many primes of the form $p = n^2 + 1$.
\end{enumerate}

Unlike similar collections of unsolved problems, this collection does not include the Riemann hypothesis.

Almost a hundred years later, all four of Landau's problems are still unsolved.
%%%%%
%%%%%
\end{document}
