\documentclass[12pt]{article}
\usepackage{pmmeta}
\pmcanonicalname{ListOfFieldsMedalists}
\pmcreated{2013-03-22 16:12:07}
\pmmodified{2013-03-22 16:12:07}
\pmowner{Mravinci}{12996}
\pmmodifier{Mravinci}{12996}
\pmtitle{list of Fields medalists}
\pmrecord{8}{38296}
\pmprivacy{1}
\pmauthor{Mravinci}{12996}
\pmtype{Example}
\pmcomment{trigger rebuild}
\pmclassification{msc}{01A61}
\pmclassification{msc}{01A65}
\pmclassification{msc}{01A60}

\endmetadata

% this is the default PlanetMath preamble.  as your knowledge
% of TeX increases, you will probably want to edit this, but
% it should be fine as is for beginners.

% almost certainly you want these
\usepackage{amssymb}
\usepackage{amsmath}
\usepackage{amsfonts}

% used for TeXing text within eps files
%\usepackage{psfrag}
% need this for including graphics (\includegraphics)
%\usepackage{graphicx}
% for neatly defining theorems and propositions
%\usepackage{amsthm}
% making logically defined graphics
%%%\usepackage{xypic}

% there are many more packages, add them here as you need them

% define commands here

\begin{document}
This is a list of mathematicians who've been awarded the Fields Medal, sorted by year.

1936: Lars Ahlfors (Finland), Jesse Douglas (U.S.) 

1950: Laurent Schwartz (France), Atle Selberg (Norway)

1954: Kunihiko Kodaira (Japan), Jean-Pierre Serre (France)

1958: Klaus Roth (UK), René Thom (France) 

1962: \PMlinkname{Lars Hörmander}{LarsHormander} (Sweden), John Milnor (U.S.) 

1966: Michael Atiyah (UK), Paul Joseph Cohen (U.S.), Alexander Grothendieck (France; boycotted ceremony), Stephen Smale (U.S.) 

1970: Alan Baker (UK), Heisuke Hironaka (Japan), Sergei Petrovich Novikov (USSR), John Griggs Thompson (U.S.) 

1974: Enrico Bombieri (Italy), David Mumford (U.S.) 

1978: Pierre Deligne (Belgium), Charles Fefferman (U.S.), Jacques Tits on behalf of Grigory Margulis (USSR), Daniel Quillen (U.S.) 

1982: Alain Connes (France), William Thurston (U.S.), Shing-Tung Yau (China) 

1986: Simon Donaldson (UK), Gerd Faltings (West Germany), Michael Freedman (U.S.) 

1990: Vladimir Drinfeld (USSR), Vaughan Frederick Randal Jones (New Zealand), Shigefumi Mori (Japan), Edward Witten (U.S.) 

1994: Efim Isakovich Zelmanov (Russia), Pierre-Louis Lions (France), Jean Bourgain (Belgium), Jean-Christophe Yoccoz (France) 

1998: Richard Ewen Borcherds (UK), William Timothy Gowers (UK), Maxim Kontsevich (Russia), Curtis T. McMullen (U.S.) 

2002: Laurent Lafforgue (France), Vladimir Voevodsky (Russia) 

2006: Andrei Okounkov (Russia), Grigori Perelman (Russia; refused award), Terence Tao (Australia), Wendelin Werner (France) 

\subsection{Fictional winners}

The backstory for MIT professor Gerald Lambeau in {\it Good Will Hunting} says he won the Fields Medal for his work in combinatorics.
%%%%%
%%%%%
\end{document}
