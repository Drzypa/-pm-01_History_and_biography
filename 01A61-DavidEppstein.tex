\documentclass[12pt]{article}
\usepackage{pmmeta}
\pmcanonicalname{DavidEppstein}
\pmcreated{2013-03-22 16:47:28}
\pmmodified{2013-03-22 16:47:28}
\pmowner{PrimeFan}{13766}
\pmmodifier{PrimeFan}{13766}
\pmtitle{David Eppstein}
\pmrecord{6}{39024}
\pmprivacy{1}
\pmauthor{PrimeFan}{13766}
\pmtype{Biography}
\pmcomment{trigger rebuild}
\pmclassification{msc}{01A61}
\pmclassification{msc}{01A60}
\pmclassification{msc}{01A65}

% this is the default PlanetMath preamble.  as your knowledge
% of TeX increases, you will probably want to edit this, but
% it should be fine as is for beginners.

% almost certainly you want these
\usepackage{amssymb}
\usepackage{amsmath}
\usepackage{amsfonts}

% used for TeXing text within eps files
%\usepackage{psfrag}
% need this for including graphics (\includegraphics)
%\usepackage{graphicx}
% for neatly defining theorems and propositions
%\usepackage{amsthm}
% making logically defined graphics
%%%\usepackage{xypic}

% there are many more packages, add them here as you need them

% define commands here

\begin{document}
{\em David Eppstein} (1963 - ) American computer programmer of English birth.

After earning a bachelor's degree in mathematics from Stanford in 1984 and a Ph.D. in computer science from Columbia university in 1989, Eppstein went on to work at the Palo Alto Research Center and teach computing at the University of California-Irvine. In 1991, he coauthored with Frances Yao and others a paper on horizon theorems for lines and polygons in ``Discrete and Computational Geometry: Papers from the DIMACS Special Year'', {\it DIMACS Ser. Discrete Math. and Theoretical Computer Science} {\bf 6}; since Yao coauthored with Fan Chung Graham, Ronald Graham, \PMlinkname{Stanis\l{}aw Ulam}{StanislawUlam} and Erd\H{o}s ``Minimal decompositions of two graphs into pairwise isomorphic subgraphs'' in {\it Proceedings of the Tenth Southeastern Conference on Combinatorics, Graph Theory and Computing } in 1979, Eppstein has \PMlinkname{Erd\H{o}s number}{ErdHosNumber} 2.

These days he edits Wikipedia articles on mathematical topics.
%%%%%
%%%%%
\end{document}
