\documentclass[12pt]{article}
\usepackage{pmmeta}
\pmcanonicalname{RolfNevanlinna}
\pmcreated{2013-03-22 17:43:22}
\pmmodified{2013-03-22 17:43:22}
\pmowner{pahio}{2872}
\pmmodifier{pahio}{2872}
\pmtitle{Rolf Nevanlinna}
\pmrecord{9}{40169}
\pmprivacy{1}
\pmauthor{pahio}{2872}
\pmtype{Biography}
\pmcomment{trigger rebuild}
\pmclassification{msc}{01A65}
\pmclassification{msc}{01A60}
\pmsynonym{Rolf Herman Nevanlinna}{RolfNevanlinna}
\pmrelated{ErnstLindelof}
\pmrelated{PeriodicFunctions}
\pmrelated{ZerosAndPolesOfRationalFunction}
\pmrelated{AnalyticContinuationOfGammaFunction}

\endmetadata

% this is the default PlanetMath preamble.  as your knowledge
% of TeX increases, you will probably want to edit this, but
% it should be fine as is for beginners.

% almost certainly you want these
\usepackage{amssymb}
\usepackage{amsmath}
\usepackage{amsfonts}

% used for TeXing text within eps files
%\usepackage{psfrag}
% need this for including graphics (\includegraphics)
%\usepackage{graphicx}
% for neatly defining theorems and propositions
 \usepackage{amsthm}
% making logically defined graphics
%%%\usepackage{xypic}

% there are many more packages, add them here as you need them

% define commands here

\theoremstyle{definition}
\newtheorem*{thmplain}{Theorem}

\begin{document}
Rolf Herman Nevanlinna (1895-10-22 \`a 1980-5-28), a Finnish matematician, was born in Joensuu, Finland.  He has acted as high school teacher 1918--1926, docent of mathematics 1922--1926, professor of mathematics in University of Helsinki 1926--1946.  He has teached also in the Technical university of Z\"urich and the University of G\"ottingen.  Later, Nevanlinna has worked as rector of the University of Helsinki and as chairman of \PMlinkexternal{International Mathematical Union}{http://www.mathunion.org/} (IMU).  In 1948, Nevanlinna was appointed to the Academy of Finland.  

Nevanlinna has acted on several \PMlinkescapetext{branches} of mathematics, but especially won international fame in the complex analysis, where his most known work is his value-distribution theory of holomorphic functions.  This wide, profound and in formal respects polished Nevanlinna theory was published in 1925.  Lars Ahlfors has said that after it, the function theory was not any more the same as before.  Hermann Weyl wrote in 1940s that Nevanlinna theory belongs to the great mathematical achievements of the twentieth century.  The point of departure of the theory is \PMlinkname{Picard's theorem}{PicardsTheorem} (1879), which had been already improved, among others, by Borel, but the work of Nevanlinna was entirely on a new \PMlinkescapetext{level}.  The essential contents of the theory may be described so that a meromorphic function attains all complex values ``approximately equally often''.  If some value appears less frequently than others among the values of function, then to counterbalance, the function comes nearer to that value ``more often'' than to the others.
%%%%%
%%%%%
\end{document}
