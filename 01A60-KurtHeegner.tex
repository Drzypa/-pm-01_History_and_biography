\documentclass[12pt]{article}
\usepackage{pmmeta}
\pmcanonicalname{KurtHeegner}
\pmcreated{2013-03-22 18:14:28}
\pmmodified{2013-03-22 18:14:28}
\pmowner{Mravinci}{12996}
\pmmodifier{Mravinci}{12996}
\pmtitle{Kurt Heegner}
\pmrecord{4}{40833}
\pmprivacy{1}
\pmauthor{Mravinci}{12996}
\pmtype{Biography}
\pmcomment{trigger rebuild}
\pmclassification{msc}{01A60}

\endmetadata

% this is the default PlanetMath preamble.  as your knowledge
% of TeX increases, you will probably want to edit this, but
% it should be fine as is for beginners.

% almost certainly you want these
\usepackage{amssymb}
\usepackage{amsmath}
\usepackage{amsfonts}

% used for TeXing text within eps files
%\usepackage{psfrag}
% need this for including graphics (\includegraphics)
%\usepackage{graphicx}
% for neatly defining theorems and propositions
%\usepackage{amsthm}
% making logically defined graphics
%%%\usepackage{xypic}

% there are many more packages, add them here as you need them

% define commands here

\begin{document}
\emph{Kurt Heegner} (1893 - 1965) was a German high school teacher and radio engineer from Berlin now famous for his mathematical discoveries.

In 1952 Heegner published what he claimed was the solution of a classic problem proposed by the great mathematician Gauss, the class number 1 problem, a significant and longstanding problem in number theory. Heegner's work was not accepted for years, due mainly to mistakes in the paper, though Harold Stark later showed these could be fixed. Heegner's proof was finally accepted as essentially correct in 1967 when Stark independently arrived at a similar proof, which he then showed was more or less equivalent to Heegner's.

{\it This entry was adapted from the Wikipedia article \PMlinkexternal{Kurt Heegner}{http://en.wikipedia.org/wiki/Kurt_Heegner} as of July 19, 2008.}
%%%%%
%%%%%
\end{document}
