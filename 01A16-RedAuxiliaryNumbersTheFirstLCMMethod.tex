\documentclass[12pt]{article}
\usepackage{pmmeta}
\pmcanonicalname{RedAuxiliaryNumbersTheFirstLCMMethod}
\pmcreated{2015-06-15 15:43:03}
\pmmodified{2015-06-15 15:43:03}
\pmowner{milogardner}{13112}
\pmmodifier{milogardner}{13112}
\pmtitle{Red Auxiliary numbers, the first LCM method}
\pmrecord{39}{41217}
\pmprivacy{1}
\pmauthor{milogardner}{13112}
\pmtype{Definition}
\pmcomment{trigger rebuild}
\pmclassification{msc}{01A16}
%\pmkeywords{LCM}

% this is the default PlanetMath preamble.  as your knowledge
% of TeX increases, you will probably want to edit this, but
% it should be fine as is for beginners.

% almost certainly you want these
\usepackage{amssymb}
\usepackage{amsmath}
\usepackage{amsfonts}

% used for TeXing text within eps files
%\usepackage{psfrag}
% need this for including graphics (\includegraphics)
%\usepackage{graphicx}
% for neatly defining theorems and propositions
%\usepackage{amsthm}
% making logically defined graphics
%%%\usepackage{xypic}

% there are many more packages, add them here as you need them

% define commands here

\begin{document}
Red auxiliary numbers, from Wikipedia

Ahmes practiced selections of a least common multiple (LCM) m in RMP 7-23. The majority of Rhind Mathematical Papyrus problems report red numbers that assisted conversions of  rational numbers n/p scaled by an LCM m to mn/mp. In RMP 36, and all other prolbems, the red auxiliary numbers were summed to numerator mn. Ahmes obtained an optimized unit fraction series. 

For examples, 3/53 was scaled by LCM 20 to 60/1060 that considered auxiliary numerator (53 + 4 + 2 + 1)/1060 was written out as 1/20 + 1/265 + 1/530 + 1/1060,  a component of solving the unity statement 53/53 = 2/53 + 3/53 + 5/53 + 15/53 + 28/53.
 
In RMP 37 two additional red number methods specified that any series of unit fractions could be individually inverted to a red number numerator. Ahmes showed that 1/4 and 1/8 could be scaled by LCM 72 playfully recalling that non-optimal unit fractional series were available as proofs of denominator. Ahmes recorded 1/4 = 72/288 = (9 + 18 + 24 +3 + 8 + 1 +8 + 1)/288 = 1/32 + 1/16 + 1/12 +1/96 + 1/36 + 1/288 + 1/36 + 1/288 in an out-of order EMLR-like series. Recording 1/8 = 72/576 = (8 + 36 + 18 + 9 + 1)/576 = 1/72 + 1/16 + 1/32 + 1/64 + 1/576, another EMLR-like series.

Red auxiliary numbers had been noted by historians for 130 years. Most historians muddled the historical purpose and applications of the red scribal data. An initial purpose allowed LCM m to be selected that converted 2/n rational numbers into a table rational numbers to optimized, but not always optimal, Egyptian fraction series. Math historians like F. Hultsch began to parse basic aspects of the red auxiliary number applications in the 19th century. Early math historians recognized that red numbers were connected to LCM, but the specific context was left vague until the 21st century.

It took over 120 years years to fully parse the RMPs three applications of red auxiliary numbers. As a review, Ahmes' red auxiliary problems was summarized by George G. Joseph, "Crest of the Peacock" in 1993, as fragmented details from which ancient scribal students learned to apply the method. On page 37, example 3.7 Joseph reports:

Complete 2/3 + 1/4 + 1/28 to 1.

This meant: solve $$2/3 + 1/4 + 1/28 + x = 1$$ (example 3.7)

The lowest common denominator (LCM) was not 28, or 84, but rather 42. Unskilled modern students would multiple 3 times 28 finding an LCM of 84. But 42 was sufficient for Egyptian scribes as noted by:

$$84/3 + 42/4 + 42/28 + 42x = 42$$ (example 3.7.1)

as written in fractions

$$28 + (10 + 1/2) + (1 + 1/2) + 2 = 42$$ (example 3.7.2)

with 42 marked in red. Modern algebra would have included 42x, solving

$$42x = 2$$

$$x = 2/42 = 1/21$$ (example 3.7.3)

meaning scribes would have a written final series in the form:

$$2/3 + 1/4 + 1/28 + 1/21 = 1$$ (example 3.7.4)

The RMP included three problems that asked Ahmes to complete a series of fractions adding to a given number. Two problems, RMP 21 and RMP 23, follow: 

RMP 21: Complete  $$2/3 + 1/15 + x = 1$$

use red auxiliary number 30 to find 

$$60/3 + 30/15 + 30x = 30$$
 
$$20 + 2 + 8 = 30$$

$$30x = 8$$

$$x = 8/30 = 4/15 = (3 + 1)/15 = 1/5 + 1/15$$

such that:

$$2/3 + 1/5 + 2/15 = 1$$

was written as:

$$2/3 + 1/5 + 1/10 + 1/30 = 1$$

RMP 23: Complete $$1/4 + 1/8 + 1/10 + 1/35 + 1/45 + x = 2/3$$

use red auxiliary 45 to compute $$x = 1/9 + 1/40$$

In addition, RMP 36 solved 2/53, 3/53, 5/53, 28/53 and 30/53 by finding red auxiliary numerators considering:

2/53*(30/30) = 60/1590 = (53 + 5 + 2)/1590 = 1/20 + 1/318 + 1/795

3/53*(20/20) = 60/1060 = (53 + 4 + 2 + 1)/1060 = 1/20 + 1/265 + 1/530 + 1/1060

5/53*(12/12) = 60/636 = (53 + 4 + 2 + 1)/636 = 1/12 + 1/159 + 1/318 + 1/636

28/53*(2/2) = 56/106 = (53 + 2 + 1)/106 = 1/2 + 1/53 + 1/106  

and,

30/53 = 2/53 + 28/53

with numerators (53 + 5 + 2), (53 + 4 + 2 + 1) and (53 + 2 + 1) written in red.

Note that 30/53 can not be solved by finding one LCM integer. In RMP 31, 28/97 also can not be solved by finding one LCM. In both cases the 2/n table allowed conversions of 30/53 and 28/97 to optimized unit fraction series to be found. In the 30/53 case, 30/53 became 2/53 + 28/53. In the 28/96 case, 28/97 was parsed to 2/97 + 26/97. That is, the 2/n table allowed almost any rational number, n/p and n/pq, to be converted to an optimized, but not always optimal unit fraction series, by first considerring n/p = 2/p + (n-2)/p, and n/pq = 2/pq + (n- 2)/pq.   

It turns out that the Egyptian Mathematical Leather Roll (EMLR) and the RMP 2/n table employed LCM's to convert rational numbers to Egyptian fraction series. Interestingly, the EMLR used non-optimal LCMs allowing students to select any LCM guess and work out an Egyptian fraction series. More importantly, the RMP began with 1/3 of the text reporting 51 2/n optimized, but not always optimal, Egyptian fraction series.

That is, red auxiliary numbers defined a core method that math historians pondered for 130 years, but did not see. Visibility began to appear around 2002 with publications of the EMLR, the Akhmim Wooden Tablet, the Ebers Papyrus, and other Egyptian fraction texts.

Understanding Ahmes' 'red auxiliary' numbers, and its proto-number theory properties, allows RMP arithmetic methods to come into focus. Recent journal papers report Egyptian fraction arithmetic in updated ways. One of the more important updates reports scribal multiplication and division of rational numbers as inverse operations. The ancient arithmetic operations had not followed duplation multiplication, as proposed for over 100 years, but the ancient operations in RMP 38, and other problems, looked and acted more like modern multiplication and division operations.

Three 2001 Russian math encyclopedia entries discuss this Egyptian fractions topic beginning with a modern definition of LCMs. The encyclopedia was published on-line by Springer, with one \PMlinkexternal{entry}{http://eom.springer.de/a/a011920.htm} suggesting ancient LCMs were known to Ahmes by writing:

$$3/11 = 1/6 + 1/11 + 1/66$$

and, converting 

$$3/11* (6/6) = (18/66) = (11 + 6+ 1)/66 = 1/6 + 1/11+ 1/66$$

The \PMlinkexternal{first entry}{http://eom.springer.de/F/f041200.htm} mentions modern LCMs, and did not attempt to take an ancient leap back in time. Had a formal hypothesis been offered a clear path from the present to a 4,000 year old ancient academic discussion would have be set in an interdisciplinary context.  Proof, or disproof, that Ahmes thought in a modern version LCMs could then be formally presented.

A \PMlinkexternal{third Russian entry}{http://eom.springer.de/A/a013260.htm} generally reports an ancient aliquot fraction or ratio idea (Russian terms for red auxiliary numbers) without identifying a hypothetical use of ancient LCMs.

Considering the three Russian entries, Ahmes' three LCM practice problems, and Ahmes' 2/n table patterns as one topic, a clear outline of a proof that Ahmes did use a modern LCM definition is provided.






%%%%%
%%%%%
\end{document}
