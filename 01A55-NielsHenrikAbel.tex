\documentclass[12pt]{article}
\usepackage{pmmeta}
\pmcanonicalname{NielsHenrikAbel}
\pmcreated{2013-03-22 18:57:53}
\pmmodified{2013-03-22 18:57:53}
\pmowner{pahio}{2872}
\pmmodifier{pahio}{2872}
\pmtitle{Niels Henrik Abel}
\pmrecord{12}{41823}
\pmprivacy{1}
\pmauthor{pahio}{2872}
\pmtype{Biography}
\pmcomment{trigger rebuild}
\pmclassification{msc}{01A55}
\pmsynonym{Abel}{NielsHenrikAbel}
\pmsynonym{N. H. Abel}{NielsHenrikAbel}
\pmsynonym{Niels Abel}{NielsHenrikAbel}
\pmrelated{AbelsMultiplicationRuleForSeries}
\pmrelated{AbelsTheoremOnPowerSeries}
\pmrelated{AbelsLimitTheorem}
\pmrelated{AbelSummability}

% this is the default PlanetMath preamble.  as your knowledge
% of TeX increases, you will probably want to edit this, but
% it should be fine as is for beginners.

% almost certainly you want these
\usepackage{amssymb}
\usepackage{amsmath}
\usepackage{amsfonts}

% used for TeXing text within eps files
%\usepackage{psfrag}
% need this for including graphics (\includegraphics)
%\usepackage{graphicx}
% for neatly defining theorems and propositions
 \usepackage{amsthm}
% making logically defined graphics
%%%\usepackage{xypic}

% there are many more packages, add them here as you need them

% define commands here

\theoremstyle{definition}
\newtheorem*{thmplain}{Theorem}

\begin{document}
\PMlinkescapeword{complete}

The Norwegian mathematician Niels Henrik Abel (1802--1829) was born in a family with many \PMlinkescapetext{children}.\, Their father provided all of them their basic education.\, In 1815, Abel got into the cathedral school of Kristiania (Oslo), where the mathematics teacher B. M. Holmboe soon saw his genius.\, Holmboe became a confidant and friend of Abel.\, In 1839, he also published the first complete edition of Abel's works.

Abel's father died in 1820, and Abel maintained himself with scholarships and by giving private lessons.\, In 1821, he enrolled at the University of Kristiania where he got his \PMlinkescapetext{MA} in 1822.\, The first mathematical texts of Abel date from 1823.\, In 1824, he printed the booklet \emph{M\'emoire sur les \'equations alg\'ebriques, o\`u l'on d\'emontre l'impossibilit\'e de la solution g\'en\'erale de l'\'equation du cinqui\`eme degr\'e} at his own expense.\, (The English translation of the title of this booklet is \emph{Account of algebraic equations in which the impossibility of the general solution of the fifth degree equation is proven}.)\, As the title indicates, this booklet contains a proof of the impossibility to solve the general quintic equation algebraically.

Abel was awarded a scholarship of Administration for studying two years abroad.\, In Berlin, he became acquainted with A. L. Crelle, who was starting the publishing of his later famed \emph{Journal f\"ur die reine und angewandte Mathematik}, or \emph{Crelle's Journal}.\, A continual collaboration with Abel regarding the new journal began.\, Abel's journey continued via Prague, Vienna and Italy to Paris; here he did not get the sympathetic response for which he had hoped from French mathematicians.\, Abel's great study on integrals of algebraic functions, the \emph{Abelian integrals}, was presented to the French academy of sciences in 1826, but it wasn't published until 1841.

After returning to Norway, where Abel only received offers of \PMlinkescapetext{minor} temporary posts, he did no more remarkable discoveries in his research.\, In April 1829, Crelle was able to offer Abel a post of professor at the University of Berlin, but the message did not reach Abel's home until two days after his death.

Although, since childhood, Abel had to fight against poverty, he was able to create works that had a fundamental significance on several \PMlinkescapetext{areas} of mathematics during his short life.\, For example, he almost singlehandedly founded the theory of elliptic functions.\, Abel's \PMlinkname{addition theorem}{AdditionTheorem} is important, too.\, The notion of algebraic number is due to Abel.\, He proved the \PMlinkname{two-periodicity}{DoublyPeriodic} of the elliptic functions.\, Abel also studied series theory; see e.g. non-existence of universal series convergence criterion, slower divergent series, summation by parts.\, In 1839, the Norwegian government had the greater part of Abel's works published in \emph{Crelle's Journal} (e.g. \PMlinkexternal{this}{http://www.digizeitschriften.de/no_cache/home/jkdigitools/loader/?tx_jkDigiTools_pi1[IDDOC]=512226&tx_jkDigiTools_pi1[pp]=69}), and a more complete \PMlinkescapetext{collection} by Ludwig Sylow and Sophus Lie was published in 1881.

%%%%%
%%%%%
\end{document}
