\documentclass[12pt]{article}
\usepackage{pmmeta}
\pmcanonicalname{DorothyBernstein}
\pmcreated{2013-03-22 16:40:26}
\pmmodified{2013-03-22 16:40:26}
\pmowner{Mravinci}{12996}
\pmmodifier{Mravinci}{12996}
\pmtitle{Dorothy Bernstein}
\pmrecord{6}{38881}
\pmprivacy{1}
\pmauthor{Mravinci}{12996}
\pmtype{Biography}
\pmcomment{trigger rebuild}
\pmclassification{msc}{01A60}
\pmsynonym{Dorothy Lewis Bernstein}{DorothyBernstein}

% this is the default PlanetMath preamble.  as your knowledge
% of TeX increases, you will probably want to edit this, but
% it should be fine as is for beginners.

% almost certainly you want these
\usepackage{amssymb}
\usepackage{amsmath}
\usepackage{amsfonts}

% used for TeXing text within eps files
%\usepackage{psfrag}
% need this for including graphics (\includegraphics)
%\usepackage{graphicx}
% for neatly defining theorems and propositions
%\usepackage{amsthm}
% making logically defined graphics
%%%\usepackage{xypic}

% there are many more packages, add them here as you need them

% define commands here

\begin{document}
\emph{Dorothy Lewis Bernstein} (1914 - 1988) American mathematician, computer programmer and educator. Member of the American Mathematical Society and first woman elected president of the Mathematical Association of America. Due in great part to Bernstein's ability to get grants from the National Science Foundation, Goucher College (where she taught for decades) was the first women's university to use computers in mathematics instruction in the 1960s. She and also made a great effort to give her students internship opportunities.

\begin{thebibliography}{1}
\bibitem{js} F. D. Fasanelli ``Dorothy Lewis Bernstein'' in {\it Women of Mathematics: A Bibliographic Sourcebook} L. Grinstein, P. Cambpell, ed.s New York: Greenwood Press (1987): 17 - 20
\end{thebibliography}
%%%%%
%%%%%
\end{document}
